% -------------------------------------------------------------------------
%  MusicFormats Library
%  Copyright (C) Jacques Menu 2016-2024

%  This Source Code Form is subject to the terms of the Mozilla Public
%  License, v. 2.0. If a copy of the MPL was not distributed with this
%  file, you can obtain one at http://mozilla.org/MPL/2.0/.

%  https://github.com/jacques-menu/musicformats
% -------------------------------------------------------------------------

% !TEX root = mfuserguide.tex

% -------------------------------------------------------------------------
\chapter{Music score representation}\label{Music score representation}
% -------------------------------------------------------------------------

It may be useful for \mf\ users to have an insight of what is behind the scenes.
The picture at \figureRef{The MSR music score representation}, shows how a music score is represented in \msrRepr.

\subimport{../mflatexlib}{mflatexlibMSRScoreRepresentation}

The background colors are used as follows:
\begin{itemize}
\item \colorbox{black}{\textcolor{green}{green}}: an element that is expected to be found in a score representation, such as \Main{Staff} and \Main{Chord};

\item \colorbox{black}{\textcolor{pink}{pink}}: a element needed internally in MSR to structure the representation, such as \Main{Segment} and \Main{Syllable}. Those are specific to \mf;

\item \colorbox{black}{\textcolor{yellow}{yellow}}: elements that can be used in other ones, such as a \Main{Voice} containing \Main{VoiceElement}s. Those are specific to \mf\ too.
\end{itemize}

The arrows colors have the following meaning:
\begin{itemize}
\item \textcolor{blue}{blue}: {\it can be contained in}. For example, an \Main{Tuplet} can be contained in a \Main{GraceNotesGroup}, which can be contained in (attached to) a \Main{Note}.

\item \textcolor{red}{red}: {\it is a particular case of}. For example, \Main{Part} is a particular case of \Main{PartGroupElement}, and a \Main{PartGroup} is also a particular case of \Main{PartGroupElement}, meaning that part groups can be \MainIt{nested}.
\end{itemize}

This representation is inspired by both \mxml\ and \lily. The specific \mf\ aspects are there due to implementation considerations, in particular \Main{Segment} and \Main{MultiMeasureRests}.
