% -------------------------------------------------------------------------
%  MusicFormats Library
%  Copyright (C) Jacques Menu 2016-2024

%  This Source Code Form is subject to the terms of the Mozilla Public
%  License, v. 2.0. If a copy of the MPL was not distributed with this
%  file, you can obtain one at http://mozilla.org/MPL/2.0/.

%  https://github.com/jacques-menu/musicformats
% -------------------------------------------------------------------------

% MusicFormats CLI user guide

\documentclass[11pt,a4paper]{report}


% -------------------------------------------------------------------------
% import common LaTeX settings
% -------------------------------------------------------------------------

\usepackage{import}

\subimport{../mflatexlib}{mflatexlibCommonSettings.tex}

\subimport{../mflatexlib}{mflatexlibDivisionsCommands.tex}

\subimport{../mflatexlib}{mflatexlibIndexing.tex}

\subimport{../mflatexlib}{mflatexlibFontsAndColors.tex}

\subimport{../mflatexlib}{mflatexlibReferencing.tex}

\subimport{../mflatexlib}{mflatexlibBoxes.tex}

\subimport{../mflatexlib}{mflatexlibGraphicsAndPictures.tex}

\subimport{../mflatexlib}{mflatexlibMusicFormatsCommands.tex}

\subimport{../mflatexlib}{mflatexlibMusicFormatsFilesAndFolders.tex}

\subimport{../mflatexlib}{mflatexlibMusicFormatsNames.tex}

\subimport{../mflatexlib}{mflatexlibShortcuts.tex}

\subimport{../mflatexlib}{mflatexlibListings.tex}

\subimport{../mflatexlib}{mflatexlibTablesAndLists.tex}

\subimport{../mflatexlib}{mflatexlibMusicNotation.tex}


% -------------------------------------------------------------------------
% indexing (specific to this document)
% -------------------------------------------------------------------------

\makeindex[name=Main, options= -s ../CommonLaTeXFiles/MusicFormats.ist, columns=2, title=Main index, intoc]

% JMI don't create too many indexes, this causes idxlayout not to produce any indexes at all!

\makeindex[name=Files, options= -s ../CommonLaTeXFiles/MusicFormats.ist, columns=2, title=Files index, intoc]
%
%\makeindex[name=Types, options= -s ../CommonLaTeXFiles/MusicFormats.ist, columns=2, title=Types index, intoc]
%
%\makeindex[name=MethodsAndFields, options= -s ../CommonLaTeXFiles/MusicFormats.ist, columns=2, title=Methods and fields index, intoc]
%
%\makeindex[name=ConstantsFunctionsAndVariables, options= -s ../CommonLaTeXFiles/MusicFormats.ist, columns=2, title={Constants, functions and variables index}, intoc]
%
\makeindex[name=Options, options= -s ../CommonLaTeXFiles/MusicFormats.ist, columns=2, title=Options, intoc]
%
\makeindex[name=MusicXML, options= -s ../CommonLaTeXFiles/MusicFormats.ist, columns=2, title=MusicXML index, intoc]

\indexsetup{level=\section}


%\usepackage{cleveref}
	% http://tug.ctan.org/tex-archive/macros/latex/contrib/cleveref/cleveref.pdf


% -------------------------------------------------------------------------
\begin{document}
% -------------------------------------------------------------------------

% defeat headwidth not being equal to \textwidth %%%JMI should not be necessary ???
\setlength{\headwidth}{\textwidth}

% use the lists fancyhead headers and footers
\useListsPagesHeadersAndFooters


%%%JMI

%\makeatletter
%\let\Chapter@Original=\chapter
%\renewcommand{\chapter}{%
%	{%
%		\tiny%
%  	\Chapter@Original%
%	}%
%	\normalsize%
%}
%\makeatother

%\makeatletter
%\renewcommand \thepart {\@Roman\c@part}
%\renewcommand \thechapter {\@arabic\c@chapter}
%\renewcommand \thesection {\thechapter.\@arabic\c@section}
%\renewcommand\thesubsection   {\thesection.\@arabic\c@subsection}
%\renewcommand\thesubsubsection{\thesubsection .\@arabic\c@subsubsection}
%\renewcommand\theparagraph    {\thesubsubsection.\@arabic\c@paragraph}
%\renewcommand\thesubparagraph {\theparagraph.\@arabic\c@subparagraph}
%\makeatother


\begin{titlepage}
  \begin{center}
    \vspace*{2cm}

    \textbf{
      \LARGE{\mf\ user guide} \\[10pt]
			\Large{\url{https://github.com/jacques-menu/musicformats}}
		}

    \vspace{0.25cm}

    \large{
    	\input{../../MusicFormatsVersionNumber.txt}%
			-- %
    	\input{../../MusicFormatsVersionDate.txt}
		}

    \vspace{0.75cm}

    \large{\textbf{Jacques Menu}}
  \end{center}

  \vspace{1cm}


\mf\ is about creating \MainIt{representations} of music scores and providing \MainIt{automatic conversions} between them. For example, to convert a \mxml\ file into a \lily\ file, one can use:

\begin{center}
\maxsizebox{0.975\linewidth}{5cm}{
\begin{minipage}{\linewidth}
\begin{lstlisting}[language=Terminal]
xml2ly MinimalScore.xml -obj MinimalScore.ly
\end{lstlisting}
\end{minipage}
}
\end{center}
There are many options such as \code{-obj} above to tailor the services to the user's needs. \\
\includegraphics[scale=.8]{../mfgraphics/mfgraphicsMinimalScore.png}

\mf\ is open source software, available with source code and documentation at \url{https://github.com/jacques-menu/musicformats}. It is written in \CPlusplus, and can be used from the \CLI\ on Linux, Windows and Mac OS. It can also be used within applications, including in \Web\ sites.

This document shows how to use the \mf\ library, both from the \CLI\ and from within applications.
It can be found at \url{https://github.com/jacques-menu/musicformats/blob/master/documentation/mfuserguide/mfuserguide.pdf}.

\begin{center}
\begin{turn}{1}
\maxsizebox{0.9125\linewidth}{5cm}{
\begin{minipage}{\linewidth}
\begin{lstlisting}[language=Terminal]
jacquesmenu@macmini > xml2ly -about
What xml2ly does:

    This multi-pass converter basically performs 5 passes:
        Pass 1:  reads the contents of MusicXMLFile or stdin ('-')
                 and converts it to a MusicXML tree;
        Pass 2a: converts that MusicXML tree into
                 a first Music Score Representation (MSR) skeleton;
        Pass 2b: populates the first MSR skeleton from the MusicXML tree
                 to get a full MSR;
        Pass 3:  converts the first MSR into a second MSR to apply options
        Pass 4:  converts the second MSR into a
                 LilyPond Score Representation (LPSR);
        Pass 5:  converts the LPSR to LilyPond code
                 and writes it to standard output.

    Other passes are performed according to the options, such as
    displaying views of the internal data or printing a summary of the score.

    The activity log and warning/error messages go to standard error.
\end{lstlisting} % no line break here
\end{minipage}
}
\end{turn}
\end{center}
\end{titlepage}

% -------------------------------------------------------------------------
{ % reduce vertical size of tables and lists
  \setlength {\parskip} {0.3ex plus \baselineskip minus 2pt}

  \listoffigures

  \tableofcontents
}


%% -------------------------------------------------------------------------
\part{Preamble}
%% -------------------------------------------------------------------------


% now we can switch to the regular headers and footers
\useRegularPagesHeadersAndFooters


\subimport{./}{acknowledgements}

\subimport{./}{about}


% -------------------------------------------------------------------------
\part{Discovering MusicFormats}
% -------------------------------------------------------------------------

\subimport{./}{architecture.tex}

\subimport{./}{aFirstExample.tex}

\subimport{./}{moreExamples.tex}

\subimport{./}{musicFormatsTerminology.tex}

\subimport{./}{multipleFilesConversion.tex}

\subimport{./}{repository.tex}

\subimport{./}{libraryComponents.tex}

\subimport{./}{musicScoreView.tex}

\subimport{./}{versionsNumbering.tex}


% -------------------------------------------------------------------------
\part{Shell basics}
% -------------------------------------------------------------------------

\subimport{./}{shellBasics.tex}


% -------------------------------------------------------------------------
\part{Installing MusicFormats}
% -------------------------------------------------------------------------

\subimport{./}{installing.tex}


% -------------------------------------------------------------------------
\part{Options and help (OAH)}
% -------------------------------------------------------------------------

\subimport{./}{optionsAndHelp.tex}

\subimport{./}{nonMusicalOptions.tex}

\subimport{./}{traceOptions.tex}

%\subimport{./}{inputAndOutput}%%%JMI


% -------------------------------------------------------------------------
\part{Warnings and errors (WAE)}
% -------------------------------------------------------------------------

\subimport{./}{warningsAndErrors.tex}


% -------------------------------------------------------------------------
\part{Multiple languages support}
% -------------------------------------------------------------------------

\subimport{./}{multipleLanguagesSupport.tex}


% -------------------------------------------------------------------------
\part{The MusicFormats Scripting Language (MFSL)}
% -------------------------------------------------------------------------

\subimport{./}{theMFSLLanguage.tex}


% -------------------------------------------------------------------------
\part{{\tt xml2ly}}
% -------------------------------------------------------------------------

\subimport{./}{xml2ly.tex}


% -------------------------------------------------------------------------
\part{{\tt xml2brl}}
% -------------------------------------------------------------------------

\subimport{./}{xml2brl.tex}


% -------------------------------------------------------------------------
\part{{\tt xml2xml}}
% -------------------------------------------------------------------------

\subimport{./}{xml2xml.tex}


% -------------------------------------------------------------------------
\part{{\tt xml2gmn}}
% -------------------------------------------------------------------------

\subimport{./}{xml2gmn.tex}


% -------------------------------------------------------------------------
% postamble
% -------------------------------------------------------------------------

% -------------------------------------------------------------------------
\part{Indexes}
% -------------------------------------------------------------------------

% back to the lists fancyhead headers and footers
\useListsPagesHeadersAndFooters

\printindex[Files]

\printindex[Options]

\printindex[MusicXML]

\printindex[Main]


% -------------------------------------------------------------------------
\end{document}
% -------------------------------------------------------------------------
