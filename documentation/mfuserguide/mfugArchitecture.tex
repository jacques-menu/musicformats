% -------------------------------------------------------------------------
%  MusicFormats Library
%  Copyright (C) Jacques Menu 2016-2024

%  This Source Code Form is subject to the terms of the Mozilla Public
%  License, v. 2.0. If a copy of the MPL was not distributed with this
%  file, you can obtain one at http://mozilla.org/MPL/2.0/.

%  https://github.com/jacques-menu/musicformats
% -------------------------------------------------------------------------

% !TEX root = mfuserguide.tex

% -------------------------------------------------------------------------
\chapter{The MusicFormats architecture}\label{The MusicFormats architecture}
% -------------------------------------------------------------------------

\subimport{../mflatexlib}{mflatexlibMusicFormatsArchitecturePicture}

The picture at \figureRef{Architecture}, shows how \mf\ is structured:
\begin{itemize}
\item central to \mf\ is \msrRepr, an internal fine-grained representation of the musical contents of music scores;

\item immediately around it, the round boxes are other (internal) representations used by various formats unitary conversions;

\item the outermost square boxes are the (external) formats that \mf\ supports;

\item the numbered arrows are conversion steps between formats and/or representations. The numbers indicate roughly the order in which they were added to the library.

Some conversions are two-way, such as that of \mxsrRepr\ to MSR\ and back. Others are one-way, such as the conversion of \lpsrRepr\ to LilyPond\ text;

\item the red arrows are conversions of a representation to the same one. These are meant to offer options to modify the contents of those representations;

\item the dimmed, dashed boxes and arrows indicate items not yet available or supported.

\end{itemize}

Decomposing the conversion work into successive steps has many advantages:
\begin{itemize}
\item each step concentrates on a subset of the tasks to be performed without interfering with the others. For example, converting \mxml\ text to \msrRepr\ has nothing to do with \lily;

\item development and debugging is therefore much easier than with a single, huge bulk of code;

\item most important still, this architecture allows the \MainIt{reuse} of the steps, which are combined to assemble the higher-level converters;

\item icing on the cake, the options and help associated with the various steps are combined to obtain the options and help for the converters and generators.
\end{itemize}

Technically, the conversion steps are called \MainIt{passes}, a term that comes from the \Main{compiler writing} field. We shall use it throughout this document.
