% -------------------------------------------------------------------------
%  MusicFormats Library
%  Copyright (C) Jacques Menu 2016-2025

%  This Source Code Form is subject to the terms of the Mozilla Public
%  License, v. 2.0. If a copy of the MPL was not distributed with this
%  file, you can obtain one at http://mozilla.org/MPL/2.0/.

%  https://github.com/jacques-menu/musicformats
% -------------------------------------------------------------------------

% !TEX root = mfuserguide.tex

% -------------------------------------------------------------------------
\chapter{\xmlToBrl\ }
% -------------------------------------------------------------------------

\subimport{./}{mfugXml2brlArchitecturePicture.tex}


% -------------------------------------------------------------------------
\section{Why \xmlToBrl?}
% -------------------------------------------------------------------------

After first creating \xmlToLy, the design goals for \xmlToBrl\ were:
\begin{itemize}
\item to experiment the re-use of \msrRepr\ for other needs than generating \lily\ code;
\item to provide a \mxml\ to \braille\ transalator that might prove useful.
\end{itemize}

The first goal has been reached, but the second one has not at the time of this writing: nearly none of the individuals and bodies this author contacted to ask whom might help him with technical details about the generation of braille files answered.\\
So this whole effort got \frozen\ at some point in time.

\xmlToBrl\ is incomplete in that is does not support, by far, the full range of Braille complexities. Anyone interested may take over if needed, though, which is why this part of \mf\ is presented in this document and detailed in the maintainance guide.


% -------------------------------------------------------------------------
\section{What \xmlToBrl\ does}
% -------------------------------------------------------------------------

\xmlToBrl\ performs the 5 steps from \mxml\ to LilyPond\ to translate the former into the latter, as shown in \figureRef{xmlToLyArchitecture}. Converting from MXSR to MSR\ is done in two sub-phases for implementation reasons.

The '{\tt -about}' option to \xmlToBrl\ details that somewhat:
\begin{lstlisting}[language=MusicXML]
jacquesmenu@macmini > xml2brl -about
What xml2brl does:

    This multi-pass converter basically performs 6 passes:
        Pass 1:  reads the contents of MusicXMLFile or stdin ('-')
                 and converts it to a MusicXML tree;
        Pass 2a: converts that MusicXML tree into
                 a first Music Score Representation (MSR) skeleton;
        Pass 2b: populates the MSR skeleton from the MusicXML tree
                 to get a full MSR;
        Pass 3:  converts the first MSR into a second MSR, to apply options
        Pass 4:  converts the second MSR into
                 a first Braille Score Representation (BSR)
                 containing one Braille page per MusicXML page;
        Pass 5:  converts the first BSR into a second BSR
                 with as many Braille pages as needed
                 to fit the line and page lengthes;
        Pass 6:  converts the BSR to Braille text
                 and writes it to standard output.

    In this preliminary version, pass 3 merely clones the MSR it receives.

    Other passes are performed according to the options, such as
    displaying views of the internal data or printing a summary of the score.

    The activity log and warning/error messages go to standard error.
\end{lstlisting}
