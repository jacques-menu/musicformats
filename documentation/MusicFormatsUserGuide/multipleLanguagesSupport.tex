% !TEX root = MusicFormatsUserGuide.tex

% -------------------------------------------------------------------------
\chapter{Multiple languages support}
% -------------------------------------------------------------------------

The \mf\ components support a number of languages, most of which being taken over from \mxml\ and \lily.

For example, \xmlToLy\ offers several languages options:
\begin{lstlisting}[language=Terminal]
jacquesmenu@macmini > xml2ly -find language
6 occurrences of std::string "language" have been found:
   1:
    -msr-pitches-language, -mplang LANGUAGE
    Use LANGUAGE to display note pitches in the MSR logs and text views.
                    The 13 MSR pitches languages available are:
                    arabic, catalan, deutsch, english, espanol, francais,
                    italiano, nederlands, norsk, portugues, suomi, svenska and vlaams.
                    The default is 'kQTPNederlands'.
   2:
    -lpsr-pitches-language, -lppl LANGUAGE
    Use LANGUAGE to display note pitches in the LPSR logs and views,
                    as well as in the generated LilyPond code.
                    The 13 LPSR pitches languages available are:
                    arabic, catalan, deutsch, english, espanol, francais,
                    italiano, nederlands, norsk, portugues, suomi, svenska and vlaams.
                    The default is 'kQTPNederlands'.
   3:
    -lpsr-chords-language, -lpcl LANGUAGE
    Use LANGUAGE to display chord names, their root and bass notes,
                    in the LPSR logs and views and the generated LilyPond code.
                    The 5 LPSR chords pitches languages available are:
                    french, german, ignatzek, italian and semiGerman.
                    'ignatzek' is Ignatzek's jazz-like, english naming used by LilyPond by default.
                    The default is 'kChordsIgnatzek'.
   4:
    -show-all-harmonies-contents, -sacc PITCH
    Write all harmonies contents for the given diatonic (semitones) PITCH,
                    supplied in the current language to standard output.
   5:
    -show-harmony-details, -scd HARMONY_SPEC
    Write the details of the harmony for the given diatonic (semitones) pitch
                    in the current language and the given harmony to standard output.
                    HARMONY_SPEC can be:
                    'ROOT_DIATONIC_PITCH HARMONY_NAME'
                    or
                    "ROOT_DIATONIC_PITCH = HARMONY_NAME"
                    Using double quotes allows for shell variables substitutions, as in:
                    HARMONY="maj7"
                    xml2ly -show-harmony-details "bes ${HARMONY}"
   6:
    -show-harmony-analysis, -sca HARMONY_SPEC
    Write an analysis of the harmony for the given diatonic (semitones) pitch
                    in the current language and the given harmony to standard output.
                    HARMONY_SPEC can be:
                    'ROOT_DIATONIC_PITCH HARMONY_NAME INVERSION'
                    or
                    "ROOT_DIATONIC_PITCH = HARMONY_NAME INVERSION"
                    Using double quotes allows for shell variables substitutions, as in:
                    HARMONY="maj7"
                    INVERSION=2
                    xml2ly -show-harmony-analysis "bes ${HARMONY} ${INVERSION}"
\end{lstlisting}
