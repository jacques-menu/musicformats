% !TEX root = MusicFormatsUserGuide.tex

% -------------------------------------------------------------------------
\chapter{MusicFormats terminology}
% -------------------------------------------------------------------------

The names used in music scores software have variations. In particular, \MainIt{measure} and \MainIt{bar} are sometimes synonyms.

Since \mf\ development started to create \xmlToLy\ in the first place, some names it uses had to be chosen when there are differences in \mxml\ and \lily.

Here is a summary of the essential such naming choices:
%\begin{adjustwidth}{-0.5cm}{-0.5cm}
\begin{center}
\small
\def \contentsWidth{0.6\textwidth}
\def \arraystretch{1.3}
%
%\begin{longtable}[t]{lp{\contentsWidth}}
%\begin{longtable}[t]{p{0.3\contentsWidth}llp{0.3\contentsWidth}}
\begin{longtable}[t]{lllp{4cm}}
{Concept} & {\mxml } & {\lily } & {\mf } \tabularnewline[0.5ex]
\hline\\[-3.0ex]
%
metronome and tempo indications & \musicXmlMarkup{metronome} & \lilypondCommand{tempo} & \className{msrTempo}
\tabularnewline

time signatures & \musicXmlMarkup{time}  & \lilypondCommand{time} & \className{msrTimeSignature}, \className{bsrTimeSignature}
\tabularnewline

rehearsal marks & \musicXmlMarkup{rehearsal} & \lilypondCommand{mark} & \className{msrRehearsalMark}
\tabularnewline

tuplets & \musicXmlMarkup{time-modification} & \lilypondCommand{tuplet} & \className{msrTuplet}
\tabularnewline

compressed empty measures & \musicXmlMarkup{multiple-rest} & \lilypondCommand{compressEmptyMeasures} & \className{msrMultipleFullBarRests}
\tabularnewline

crescendo/decrescendo & \musicXmlMarkup{wedge} & hairpin & \className{msrWedge}
\tabularnewline

\end{longtable}
\end{center}
