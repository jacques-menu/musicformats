% !TEX root = MusicFormatsUserGuide.tex

% -------------------------------------------------------------------------
\chapter{Non-musical options}
% -------------------------------------------------------------------------

\mf\ supplies options to obtain informations without inferering with the conversion activities in any way.


% -------------------------------------------------------------------------
\section{Timing measurements}
% -------------------------------------------------------------------------

There is a \option{cpu} option to see show much time is spent in the various translation activities. Note that the numbers obtained depend on the other activities on the machine. Also, on recent versions of \MacOS, the first run of an executable may be a bit slower that subsequent runs, because the \OS\ loads the code in a \cache\ for further use:
\begin{lstlisting}[language=MusicXML]
menu@macbookprojm > xml2ly -query cpu

--- Help for option 'cpu' in subgroup "CPU usage" of group "General" ---

General (-hg, -help-general):
--------------------------
  CPU usage (-hgcpu, -help-general-cpu-usage):

    -cpu, -display-cpu-usage
          Write information about CPU usage to standard error.
\end{lstlisting}

In practise, most of the time is spent in passes 1 and 2b. The \code{time} command is used to obtain the total run time, since \xmlToLy\ cannot account for input/output activities:
\begin{lstlisting}[language=MusicXML]
menu@macbookprojm > time xml2ly -aofn -display-cpu-usage xmlsamples3.1/ActorPreludeSample.xml
*** MusicXML warning *** xmlsamples3.1/ActorPreludeSample.xml:44: <system-distance /> is not supported yet by xml2ly
... ... ... ... ...
*** MusicXML warning *** xmlsamples3.1/ActorPreludeSample.xml:27761: <direction/> contains 2 <words/> markups
Warning message(s) were issued for input lines 44, 45, 46, 551, 584, 732, 1121, 1215, 4724, 27761

Timing information:

Activity                      Description       Kind  CPU (sec)
--------  -------------------------------  ---------  ---------

Pass 1    build xmlelement tree from file  mandatory  0.268994
Pass 2a   build the MSR skeleton           mandatory  0.076413
Pass 2b   build the MSR                    mandatory  0.276732
Pass 3    translate MSR to LPSR            mandatory  0.056381
Pass 4    translate LPSR to LilyPond       mandatory  0.082213

Total      Mandatory  Optional
-------    ---------  ---------
0.760733    0.760733   0


real	0m0.814s
user	0m0.751s
sys	0m0.058s
\end{lstlisting}

This compares favorably with \mxmlToLy\ measurements:
\begin{lstlisting}[language=MusicXML]
menu@macbookprojm > time musicxml2ly xmlsamples3.1/ActorPreludeSample.xml
musicxml2ly: Reading MusicXML from xmlsamples3.1/ActorPreludeSample.xml ...
musicxml2ly: Converting to LilyPond expressions...
... ... ... ... ...
musicxml2ly: Converting to LilyPond expressions...
musicxml2ly: Output to `ActorPreludeSample.ly'
musicxml2ly: Converting to current version (2.19.83) notations ...

real	0m4.113s
user	0m3.659s
sys	0m0.407s
\end{lstlisting}


% -------------------------------------------------------------------------
\section{Chords structure}
% -------------------------------------------------------------------------

In order to invert chords, as specified by the \musicXmlMarkup{inversion} element in \mxml\ data, \mxmlToLy\ knows the structure of many of them. This can be queried with the options in the \code{Extra} group:
\begin{lstlisting}[language=MusicXML]
menu@macbookprojm > xml2ly -help=extra

--- Help for group "Extra" ---

Extra (-he, -help-extra):
  These extra provide features not related to translation from MusicXML to other formats.
  In the text below:
    - ROOT_DIATONIC_PITCH should belong to the names available in
      the selected MSR pitches language, "nederlands" by default;
    - other languages can be chosen with the '-mpl, -msrPitchesLanguage' option;
    - HARMONY_NAME should be one of:
        MusicXML chords:
          "maj", "min", "aug", "dim", "dom",
          "maj7", "min7", "dim7", "aug7", "halfdim", "minmaj7",
          "maj6", "min6", "dom9", "maj9", "min9", "dom11", "maj11", "min11",
          "dom13", "maj13", "min13", "sus2", "sus4",
          "neapolitan", "italian", "french", "german"
        Jazz-specific chords:
          "pedal", "power", "tristan", "minmaj9", "domsus4", "domaug5",
          "dommin9", "domaug9dim5", "domaug9aug5", "domaug11", "maj7aug11"
  The single or double quotes are used to allow spaces in the names
  and around the '=' sign, otherwise they can be dispensed with.
--------------------------
  Chords structures    (-hecs, -help-extra-chord-structures):
    -scs, -show-chords-structures
          Write all known chords structures to standard output.
  Chords contents      (-hecc, -help-extra-chords-contents):
    -sacc, -show-all-chords-contents PITCH
          Write all chords contents for the given diatonic (semitones) PITCH,
          supplied in the current language to standard output.
  Chord details        (-hecd, -help-extra-chords-details):
    -scd, -show-chord-details CHORD_SPEC
          Write the details of the chord for the given diatonic (semitones) pitch
          in the current language and the given harmony to standard output.
          CHORD_SPEC can be:
          'ROOT_DIATONIC_PITCH HARMONY_NAME'
          or
          "ROOT_DIATONIC_PITCH = HARMONY_NAME"
          Using double quotes allows for shell variables substitutions, as in:
          HARMONY="maj7"
          xml2ly -show-chord-details "bes ${HARMONY}"
  Chord analysis       (-heca, -help-extra-chords-analysis):
    -sca, -show-chord-analysis CHORD_SPEC
          Write an analysis of the chord for the given diatonic (semitones) pitch
          in the current language and the given harmony to standard output.
          CHORD_SPEC can be:
          'ROOT_DIATONIC_PITCH HARMONY_NAME INVERSION'
          or
          "ROOT_DIATONIC_PITCH = HARMONY_NAME INVERSION"
          Using double quotes allows for shell variables substitutions, as in:
          HARMONY="maj7"
          INVERSION=2
          xml2ly -show-chord-analysis "bes ${HARMONY} ${INVERSION}"
\end{lstlisting}

For example, one can obtain the structure of the B\Flat\ dominant minor ninth chord's second inversion this way:
\begin{lstlisting}[language=MusicXML]
menu@macbookprojm > xml2ly -show-chord-analysis 'bes dommin9 2'
The analysis of chord 'bes dommin9' inversion 2 is:

  Chord 'bes dommin9' inversion 2 contents, 5 intervals:
    d     : majorThird
    bes   : perfectUnison
    ces   : minorNinth
    aes   : minorSeventh
    f     : perfectFifth

  Chord 'bes dommin9' inversion 2 inner intervals:
      f     -> aes   : minorThird          (perfectFifth         -> minorSeventh)
      f     -> ces   : diminishedFifth     (perfectFifth         -> minorNinth)
      f     -> bes   : perfectFourth       (perfectFifth         -> perfectUnison)
      f     -> d     : majorSixth          (perfectFifth         -> majorThird)

      aes   -> ces   : minorThird          (minorSeventh         -> minorNinth)
      aes   -> bes   : majorSecond         (minorSeventh         -> perfectUnison)
      aes   -> d     : augmentedFourth     (minorSeventh         -> majorThird)

      ces   -> bes   : majorSeventh        (minorNinth           -> perfectUnison)
      ces   -> d     : augmentedSecond     (minorNinth           -> majorThird)

      bes   -> d     : majorThird          (perfectUnison        -> majorThird)
  This chord contains 2 tritons
\end{lstlisting}


