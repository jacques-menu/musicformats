% !TEX root = MusicFormatsUserGuide.tex

% -------------------------------------------------------------------------
\chapter{Shell basics}
% -------------------------------------------------------------------------

Since this document is about using \mf\ from the \CLI\ by musicians, let's start by a short presentation of shell usage. This chapter can be skipped of course by shell-savvy users.

A \shell\ is an application that reads commands and executes them. In the early ages of physical \MainIt{terminal}s, they were typically typed on a keyboard. With \GUI\ nowadays, they can be typed in a so-called terminal window.
A 

The syntax of shell commands is meant to be simple, without complex structuring features such as those found in programming languages.

A number of shells have been used over the years. Most of the ones used at the time of this writing belong to the \code{sh} family, among then \bash\ and \zsh\ (\code{zsh}). \\
The commands we use in this document can be run on any shell in this family.

A shell maintains a so-called {\it \currentWorkingDirectory}, which a directory in the files system.

A so-called \MainIt{prompt} is displayed by a shell when is it ready to read a command and execute it. This document uses two kinds of prompts:
\begin{itemize}
\item one contains only the user name and machine name, such as:
\begin{lstlisting}[language=Terminal]
jacquesmenu@macmini: ~ > 
\end{lstlisting}

\item the other one displays the \currentWorkingDirectory: it is used when the latter has to be set at a specific value for the command:
\begin{lstlisting}[language=Terminal]
jacquesmenu@macmini: ~/musicformats-git-dev/musicxmlfiles > 
\end{lstlisting}

\end{itemize}


% -------------------------------------------------------------------------
\section{Basic shell builtins}
% -------------------------------------------------------------------------

Many builtins have very short names for ease of interactive use. Many vowels were left out to minimize typing. For example, there are:
\begin{itemize}
\item \code{pwd} to show the \currentWorkingDirectory;
\item \code{cd} to change directory;
\item \code{echo} to produce output in the terminal window.
\end{itemize}

When a shell is launched, a directory is chosen as the \currentWorkingDirectory, usually the user's \MainIt{home directory}.
\begin{lstlisting}[language=Terminal]
jacquesmenu@macmini: ~ > pwd
/Users/jacquesmenu

jacquesmenu@macmini: ~ > cd musicformats-git-dev

jacquesmenu@macmini: ~/musicformats-git-dev > pwd
/Users/jacquesmenu/musicformats-git-dev
jacquesmenu@macmini: ~/musicformats-git-dev >
\end{lstlisting}


% -------------------------------------------------------------------------
\section{Commands}
% -------------------------------------------------------------------------

A command name is either provided by the shell itself, a so-called \MainIt{builtin}, or the name of a piece of software that can be executed.

In this example, the command name is xml2lyy:
\begin{lstlisting}[language=Terminal]
jacquesmenu@macmini > xml2lyy +sdf 45
-bash: xml2lyy: command not found
\end{lstlisting}

The shell can be queried about a command name:
\begin{lstlisting}[language=Terminal]
jacquesmenu@macmini > type cd
cd is a shell builtin

jacquesmenu@macmini > type xml2lyy
-bash: type: xml2lyy: not found

jacquesmenu@macmini > type xml2ly
xml2ly is hashed (/Users/jacquesmenu/musicformats-git-dev/build/bin/xml2ly)
\end{lstlisting}


% -------------------------------------------------------------------------
\section{Obtaining help about the shell commands}
% -------------------------------------------------------------------------

The \code{man} command (Manual page) can be used for that:
\begin{lstlisting}[language=Terminal]
jacquesmenu@macmini > man
What manual page do you want?
\end{lstlisting}

It can be used to obtain help about itself, too:
\begin{lstlisting}[language=Terminal]
jacquesmenu@macmini > man man
\end{lstlisting}

produces:
\begin{lstlisting}[language=Terminal]
man(1)                                          General Commands Manual                                         man(1)

NAME
       man - format and display the on-line manual pages

SYNOPSIS
       man [-acdfFhkKtwW] [--path] [-m system] [-p std::string] [-C config_file] [-M pathlist] [-P pager] [-B browser] [-H
       htmlpager] [-S section_list] [section] name ...

DESCRIPTION
       man formats and displays the on-line manual pages.  If you specify section, man only looks in that section of
       the manual.  name is normally the name of the manual page, which is typically the name of a command, function,
       or file.  However, if name contains a slash (/) then man interprets it as a file specification, so that you can
       do man ./foo.5 or even man /cd/foo/bar.1.gz.

       See below for a description of where man looks for the manual page files.
     
... ...
\end{lstlisting}


% -------------------------------------------------------------------------
\section{Multiple commands in a line}
% -------------------------------------------------------------------------

One can submit several commands in a single line,  separating them with a \code{;}:
\begin{lstlisting}[language=Terminal]
jacquesmenu@macmini: ~/musicformats-git-dev/musicxmlfiles > cd Clefs; ls -sal
total 88
 0 drwxr-xr-x    8 jacquesmenu  staff    256 Feb 26 09:03 .
 0 drwxr-xr-x  108 jacquesmenu  staff   3456 Feb 25 07:46 ..
16 -rw-r--r--@   1 jacquesmenu  staff   6148 Feb 26 00:23 .DS_Store
 8 -rw-r--r--    1 jacquesmenu  staff   3168 Feb 26 00:19 ClefChange.ly
24 -rw-r--r--@   1 jacquesmenu  staff  12286 Jan  2 17:01 ClefChange.xml
 8 -rw-r--r--    1 jacquesmenu  staff   1662 Apr 22  2021 ClefChanges.xml
 8 -rw-r--r--    1 jacquesmenu  staff   3045 Feb 26 00:19 Clefs.ly
24 -rw-r--r--@   1 jacquesmenu  staff   9120 Jan  2 17:02 Clefs.xml
jacquesmenu@macmini: ~/musicformats-git-dev/musicxmlfiles/Clefs > 
\end{lstlisting}

This is handy for interactive use, which avoids the use of multi-line commands, presented at \sectionRef{Multi-line commands}.


% -------------------------------------------------------------------------
\section{Paths}
% -------------------------------------------------------------------------

The files on a computer are organized as file-systems. A \fspath\ is a way to access a file on a file system:
\begin{itemize}
\item on \Unix\-like system, there is a single tree of so-called \MainIt{directories}, the root being named \code{/}. A sub-directory is preceded by \code{/} in the paths;

\item on \Windows\ systems, there is a set of trees, their roots being the physical or virtual drives, such as \code{C:}. A \code{\textbackslash} is used to indicate a sub-directory.

\item \code{.} is the \currentWorkingDirectory;

\item \code{..} is the directory containing the \currentWorkingDirectory, i.e. one level up in the directories hierarchy;
\end{itemize}

This document uses \Unix\-like \path s.


% -------------------------------------------------------------------------
\section{Quoting, variables and aliases}\label{Quoting, variables and aliases}
% -------------------------------------------------------------------------

Shell commands are submitted as a sequence of words separated by spaces. If a word, such a file name, contains \MainIt{space}s, it has to be surrounded by \quotes\ or \doubleQuotes\ in order to be seen by the shell as a single word:
\begin{lstlisting}[language=Terminal]
jacquesmenu@macmini > xml2ly Nice file.xml
Several input file names supplied, only one can be used
The arguments std::vector contains 2 elements:
   0: "Nice"
   1: "file.xml"

jacquesmenu@macmini > xml2ly 'Nice file.xml'
can't open file Nice file.xml
### Conversion from MusicXML to LilyPond failed ###

jacquesmenu@macmini > xml2ly "Nice file.xml"
can't open file Nice file.xml
### Conversion from MusicXML to LilyPond failed ###
\end{lstlisting}

Note that if a quote or double quote is part of word, the word should be inclosed by the other such:
\begin{lstlisting}[language=Terminal]
jacquesmenu@macmini: ~/musicformats-git-dev/musicxmlfiles > xml2ly -find "tuplet's"
0 occurrence of std::string "tuplet's" has been found

jacquesmenu@macmini: ~/musicformats-git-dev/musicxmlfiles > 
jacquesmenu@macmini: ~/musicformats-git-dev/musicxmlfiles > xml2ly -find 'tuplet"s'
0 occurrence of std::string "tuplet"s" has been found
\end{lstlisting}

A shell \MainIt{variable} is a name for a piece of text, called its \MainIt{value}, that can be used instead of that text in commands. The value of the variable can be seen in the terminal with the \code{echo} command:
\begin{lstlisting}[language=Terminal]
jacquesmenu@macmini: ~/musicformats-git-dev > DOC_DIR=documentation

jacquesmenu@macmini: ~/musicformats-git-dev > echo $DOC_DIR
documentation
\end{lstlisting}

Variables can be used surrounded by curly brackets, too:
\begin{lstlisting}[language=Terminal]
jacquesmenu@macmini: ~/musicformats-git-dev/documentation > echo ${DOC_DIR}
documentation
\end{lstlisting}
This notation provides further possibilities such as std::string replacement, which are out of the scope of this document.

Using variables is interesting when there are several uses of its value: changing the value at one place causing the new value to be used at every such use:
\begin{lstlisting}[language=Terminal]
jacquesmenu@macmini: ~/musicformats-git-dev > ls $DOC_DIR
CommonLaTeXFiles		MusicFormatsUserGuide	presentation
IntroductionToMusicXML		MusicFormatsMaintainanceGuide
MusicFormatsAPIGuide	graphics

jacquesmenu@macmini: ~/musicformats-git-dev > cd $DOC_DIR

jacquesmenu@macmini: ~/musicformats-git-dev/documentation > pwd
/Users/jacquesmenu/musicformats-git-dev/documentation
\end{lstlisting}

The difference between \quotes\ and \doubleQuotes\ is how variables are handled:
\begin{itemize}
\item the characters between quotes are used literally;
\item variables occurring between \doubleQuotes\ are replaced by their value.
\end{itemize}

\begin{lstlisting}[language=Terminal]
jacquesmenu@macmini: ~/musicformats-git-dev > DOC_DIR=documentation

jacquesmenu@macmini: ~/musicformats-git-dev > cd '$DOC_DIR'
-bash: cd: $DOC_DIR: No such file or directory

jacquesmenu@macmini: ~/musicformats-git-dev > cd "$DOC_DIR"

jacquesmenu@macmini: ~/musicformats-git-dev/documentation > pwd
/Users/jacquesmenu/musicformats-git-dev/documentation
\end{lstlisting}

Here is an example combining quotes and \doubleQuotes:
\begin{lstlisting}[language=Terminal]
jacquesmenu@macmini: ~/musicformats-git-dev/musicxmlfiles > DOC_DIR=documentation
jacquesmenu@macmini: ~/musicformats-git-dev/musicxmlfiles > echo "DOC_DIR's value is: ${DOC_DIR}"
DOC_DIR's value is: documentation
\end{lstlisting}


% -------------------------------------------------------------------------
\section{Input/Output redirection and pipes}
% -------------------------------------------------------------------------

Shells can read commands from text files named \MainIt{script}s through \inputRedirection, \outputRedirection\ and \pipe s.
%%%JMI


% -------------------------------------------------------------------------
\section{Functions}
% -------------------------------------------------------------------------

The shells allow the creation of \MainIt{functions}, that contain several commands under a single name. An example is \function{checkVersions}, which displays the versions of the main \mf\ \service s:
\begin{lstlisting}[language=Terminal]
function checkVersions ()
{
#  set -x

  xml2ly -v
  xml2brl -v
  xml2xml -v
  xml2gmn -v

  Mikrokosmos3Wandering -v

  msdlconverter -v
#  set +x
}
\end{lstlisting}


% -------------------------------------------------------------------------
\section{MusicFormatsBashDefinitions.bash}
% -------------------------------------------------------------------------

\fileName{MusicFormatsBashDefinitions.bash} contains a set of variables, aliases and function definitions used by this author. One of them is \function{checkVersions} above.

Feel free to use them, adapt them or ignore them depending on your taste.

Some settings we use in this document are:
\begin{lstlisting}[language=Terminal]
jacquesmenu@macmini > type ll
ll is a function
ll () 
{ 
    ls -salGTF $*
}
\end{lstlisting}
The options to \code{ls} may vary depending the on the \OS.

% -------------------------------------------------------------------------
\section{Scripts}
% -------------------------------------------------------------------------

A \shell\ script is a text file that can be executed using its name as a command. The first line tells which shell should be used to execute the commands in the remainder of the file, \code{sh} by default.

This author uses scripts as the one below as handy interactive short-cuts. It groups commands to copy a \mxml\ file exported after scanning a PDF file to another file, convert the latter to \lily\ with \xmlToLy\ and open the result with \fresco\ to produce the PDF score:
\begin{lstlisting}[language=Terminal]
jacquesmenu@macmini > cat doBethena_SaxTenor.bash 
#/bin/bash

cp -p Bethena_SaxTenor_original.xml Bethena_SaxTenor.xml

xml2ly -include  Bethena_SaxTenor_OptionsAndArguments.txt

open Bethena_SaxTenor.ly
\end{lstlisting}

After creating the script file, make it executable:
\begin{lstlisting}[language=Terminal]
jacquesmenu@macmini > chmod +x doBethena_SaxTenor.bash

jacquesmenu@macmini > ls -sal doBethena_SaxTenor.bash 
8 -rwxr-xr-x@ 1 jacquesmenu  staff  154 Feb 20 07:46 doBethena_SaxTenor.bash
\end{lstlisting}

The script can then be executed with:
\begin{lstlisting}[language=Terminal]
jacquesmenu@macmini > ./doBethena_SaxTenor.bash 
\end{lstlisting}

There is another script example at \sectionRef{Using the MFSL interpreter in shell scripts}.


% -------------------------------------------------------------------------
\section{Multi-line commands}\label{Multi-line commands}
% -------------------------------------------------------------------------

A command can be split into several lines if it is long or to enhance readability. A \code{\textbackslash} character is used for that:
\begin{lstlisting}[language=Terminal]
jacquesmenu@macmini > xml2brl \
> -trace-passes \
> foo.xml

%--------------------------------------------------------------
  Handle the options and arguments from argc/argv
%--------------------------------------------------------------
This is xml2brl v0.9.51 (October 12, 2021) from MusicFormats v0.9.61 (February 18, 2022)
Launching the conversion of "foo.xml" to Braille
Time is Saturday 2022-02-26 @ 09:53:43 CET
The command line is:
  xml2brl -trace-passes foo.xml
or with options long names:
  xml2brl -trace-passes foo.xml
or with options short names:
   -tpasses foo.xml
Braille will be written to standard output
The command line options and arguments have been analyzed

%--------------------------------------------------------------
  Pass 1: Create an MXSR reading a MusicXML file
%--------------------------------------------------------------
can't open file foo.xml
### Conversion from MusicXML to Braille failed ###
\end{lstlisting}

This feature is most useful in scripts.


% -------------------------------------------------------------------------
\section{Some useful shell commands}
% -------------------------------------------------------------------------

% -------------------------------------------------------------------------
\subsection{{\tt for}}
% -------------------------------------------------------------------------

The \code{for} command is analoguous to the same in programming languages, with features specific to shell usage.

For example, \mf\ comes with some \mxml\ files whose name contains '\code{Stanza}':
\begin{lstlisting}[language=Terminal]
jacquesmenu@macmini: ~/musicformats-git-dev/musicxmlfiles/lyrics > for FILE in $(ls *Stanza*.xml); do echo ${FILE}; done
MultipleStanzas.xml
StanzaMaster.xml
\end{lstlisting}

They can be converted to \lily\ with this \code{for} command. Note the use of \code{\$(ls *Stanza*.xml)} to execute the \code{ls} command to obtain the files names to be handled:
\begin{lstlisting}[language=Terminal]
jacquesmenu@macmini: ~/musicformats-git-dev/musicxmlfiles/lyrics > for FILE in $(ls *Stanza*.xml); do echo "---> converting ${FILE} to LilyPond"; xml2ly -auto-output-file-name ${FILE}; done
---> converting MultipleStanzas.xml to LilyPond
---> converting StanzaMaster.xml to LilyPond
\end{lstlisting}

The resulting files are:
\begin{lstlisting}[language=Terminal]
jacquesmenu@macmini: ~/musicformats-git-dev/musicxmlfiles/lyrics > ls -sal *Stanza*.ly
8 -rw-r--r--  1 jacquesmenu  staff  3659 Feb 26 10:38 MultipleStanzas.ly
8 -rw-r--r--  1 jacquesmenu  staff  2236 Feb 26 10:38 StanzaMaster.ly
\end{lstlisting}

Such long \code{for} commands are not easy to type as a single line interactively.


% -------------------------------------------------------------------------
\subsection{{\tt find}}
% -------------------------------------------------------------------------

The \code{find} command looks for files in directory satisfying some criterion. One of them is a name pattern:
\begin{lstlisting}[language=Terminal]
jacquesmenu@macmini: ~/musicformats-git-dev/musicxmlfiles > find . -name *Mozart*.xml
./xmlsamples3.1/MozartTrio.xml
./challenging/MozartPianoSonata.xml
./multistaff/MozartAnChloeK524Sample.xml
./multistaff/MozartAdagioKV410.xml
./multistaff/MozartSecondTrio.xml
\end{lstlisting}

\code{find} applies actions to the files that mach the criterion, \code{-print} by default. There are other actions, such as \code{-exec}. 

Each file found in turn is designated by '\code{\{\}}', and the command must be terminated by '\code{\textbackslash};':
\begin{lstlisting}[language=Terminal]
jacquesmenu@macmini: ~/musicformats-git-dev/musicxmlfiles > find . -name *Mozart*.xml -exec ls -sal {} \;
184 -rw-r--r--  1 jacquesmenu  staff  92127 Apr 22  2021 ./xmlsamples3.1/MozartTrio.xml
56 -rw-r--r--@ 1 jacquesmenu  staff  25664 Apr 22  2021 ./challenging/MozartPianoSonata.xml
152 -rw-r--r--  1 jacquesmenu  staff  76231 Apr 22  2021 ./multistaff/MozartAnChloeK524Sample.xml
320 -rw-r--r--  1 jacquesmenu  staff  163503 Apr 22  2021 ./multistaff/MozartAdagioKV410.xml
120 -rw-r--r--  1 jacquesmenu  staff  60833 Apr 22  2021 ./multistaff/MozartSecondTrio.xml
\end{lstlisting}

Note that, as can be seen above, \code{find} searches the whose set of files contained in its first argument, including in sub-directories.

More informations is available with \code{man find}.
