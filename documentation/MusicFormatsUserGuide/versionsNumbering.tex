% !TEX root = MusicFormatsUserGuide.tex

% -------------------------------------------------------------------------
\chapter{Versions numbering}\label{Versions numbering}
% -------------------------------------------------------------------------

\mf\ uses \MainIt{semantic} version numbering, such as \code{v0.9.61}:%%%JMI
\begin{itemize}
\item the library itself gets a new number right after a new branch as been created for it. Branching to "v0.9.60" causes the library to be numbered "v0.9.61" with \code{SetMusicFormatsVersionNumber.bash};

\item each representation, converter or pass that is modified this new branch has been created gets a new history element with the same number as the library.
\end{itemize}

Thus all the executable tools have the same version number as the library, but their components may have older version numbers. This can be seen with options \optionNameBoth{mf-version}{mfv}, \optionNameBoth{version}{v} and \optionNameBoth{version-full}{vf} :
\begin{lstlisting}[language=Terminal]%%%JMI do this again
jacquesmenu@macmini > xml2ly -mf-version
MusicFormats library v0.9.61 (February 28, 2022)

jacquesmenu@macmini > xml2ly -version
Command line version of musicxml2lilypond converter v0.9.61 (February 28, 2022)

jacquesmenu@macmini > xml2ly -version-full
Command line version of musicxml2lilypond converter v0.9.61 (February 28, 2022)

Representations versions:
  MXSR                
    v0.9.50 (October 6, 2021)
  MSR                 
    v0.9.52 (November 27, 2021)
  LPSR                
    v0.9.50 (October 6, 2021)

Passes versions:
  mxsr2msr            
    v0.9.60 (February 21, 2022)
  msr2msr             
    v0.9.51 (November 15, 2021)
  msr2lpsr            
    v0.9.50 (October 6, 2021)
  lpsr2lilypond       
    v0.9.60 (February 21, 2022)
\end{lstlisting}
