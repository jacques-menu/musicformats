% -------------------------------------------------------------------------
%  MusicFormats Library
%  Copyright (C) Jacques Menu 2016-2023

%  This Source Code Form is subject to the terms of the Mozilla Public
%  License, v. 2.0. If a copy of the MPL was not distributed with this
%  file, you can obtain one at http://mozilla.org/MPL/2.0/.

%  https://github.com/jacques-menu/musicformats
% -------------------------------------------------------------------------

% !TEX root = mfmaintainanceguide.tex

% -------------------------------------------------------------------------
\chapter{Repeats handling}\label{Repeats handling}
% -------------------------------------------------------------------------

Repeats are presented at \sectionRef{Repeats}.


% -------------------------------------------------------------------------
\section{Translating repeats from MXSR to MSR}
% -------------------------------------------------------------------------

This is done in \mxsrToMsr{}.

The tough part is to handle \mxml\ \musicXmlMarkup{barLine} markups, since they are meant for \drawing, and do not structure repeats as such.

Recognizing the structure of repeat relies on the attributes of the barLines. The following \enumType s are defined in \msr{msrBarLines.h} local to \class{msrBarLine}:
\begin{lstlisting}[language=CPlusPlus]
    // location
    enum class msrBarLineLocationKind {
      kBarLineLocationNone,

      kBarLineLocationLeft,
      kBarLineLocationMiddle,
      kBarLineLocationRight // by default
    };
\end{lstlisting}

\begin{lstlisting}[language=CPlusPlus]
    // style
    enum class msrBarLineStyleKind {
      kBarLineStyleNone,

      kBarLineStyleRegular,  // by default

      kBarLineStyleDotted, kBarLineStyleDashed, kBarLineStyleHeavy,
      kBarLineStyleLightLight, kBarLineStyleLightHeavy,
      kBarLineStyleHeavyLight, kBarLineStyleHeavyHeavy,
      kBarLineStyleTick, kBarLineStyleShort
    };
\end{lstlisting}

\begin{lstlisting}[language=CPlusPlus]
    // repeat direction
    enum class msrBarLineRepeatDirectionKind {
      kBarLineRepeatDirectionNone,
      kBarLineRepeatDirectionForward, kBarLineRepeatDirectionBackward
    };
\end{lstlisting}

\begin{lstlisting}[language=CPlusPlus]
    // ending type
    enum class msrBarLineEndingTypeKind {
      kBarLineEndingTypeNone,

      msrBarLineEndingTypeKind::kBarLineEndingTypeStart,
      msrBarLineEndingTypeKind::kBarLineEndingTypeStop,
      msrBarLineEndingTypeKind::kBarLineEndingTypeDiscontinue
    };
\end{lstlisting}

\begin{lstlisting}[language=CPlusPlus]
    // category
    enum class msrBarLineCategoryKind {
      kBarLineCategory_UNKNOWN_,

      kBarLineCategoryStandalone,

      kBarLineCategoryRepeatStart, kBarLineCategoryRepeatEnd,

      kBarLineCategoryHookedEndingStart, kBarLineCategoryHookedEndingEnd,
      kBarLineCategoryHooklessEndingStart, kBarLineCategoryHooklessEndingEnd
    };
\end{lstlisting}

\begin{lstlisting}[language=CPlusPlus]
    // segno
    enum class msrBarLineHasSegnoKind {
      kBarLineHasSegnoYes, kBarLineHasSegnoNo
    };
\end{lstlisting}

\begin{lstlisting}[language=CPlusPlus]
    // coda
    enum class msrBarLineHasCodaKind {
      kBarLineHasCodaYes, kBarLineHasCodaNo
    };

\end{lstlisting}

\begin{lstlisting}[language=CPlusPlus]
    // repeat winged
    enum class msrBarLineRepeatWingedKind {
      kBarLineRepeatWingedNone,

      kBarLineRepeatWingedStraight, kBarLineRepeatWingedCurved,
      kBarLineRepeatWingedDoubleStraight, kBarLineRepeatWingedDoubleCurved
    };
\end{lstlisting}

The attributes of \musicXmlMarkup{barline} are deciphered upon the first visit of {\tt S_barline} in\\
\mxsrToMsr{mxsr2msrTranslator.cpp}:%%%JMI
\begin{lstlisting}[language=CPlusPlus]
void mxsr2msrTranslator::visitStart ( S_barline& elt )
{
	// ... ... ...

  // location

  {
    std::string
      location =
        elt->getAttributeValue ("location");

    fCurrentBarLineLocationKind =
      msrBarLineLocationKind::kBarLineLocationRight; // by default

    if       (location == "left") {
      fCurrentBarLineLocationKind = msrBarLineLocationKind::kBarLineLocationLeft;
    }
    else  if (location == "middle") {
      fCurrentBarLineLocationKind = msrBarLineLocationKind::kBarLineLocationMiddle;
    }
    else if  (location == "right") {
      fCurrentBarLineLocationKind = msrBarLineLocationKind::kBarLineLocationRight;
    }
    else {
      std::stringstream ss;

      ss <<
        "barLine location \"" << location <<
        "\" is unknown, using 'right' by default";

   // JMI   musicxmlError (
      musicxmlWarning (
        gServiceRunData->getInputSourceName (),
        inputLineNumber,
   //     __FILE__, __LINE__,
        ss.str ());
    }
  }

  fOnGoingBarLine = true;
}
\end{lstlisting}

Then the \class{msrBarLine} instance is created upon the second visit of {\tt S_barline}:%%%JMI
\begin{lstlisting}[language=CPlusPlus]
void mxsr2msrTranslator::visitEnd ( S_barline& elt )
{
	// ... ... ...

  // create the barLine
  S_msrBarLine
    barLine =
      msrBarLine::create (
        inputLineNumber,
        fCurrentBarLineLocationKind,
        fCurrentBarLineStyleKind,
        fCurrentBarLineRepeatDirectionKind,
        fCurrentBarLineEndingTypeKind,
        fCurrentBarLineEndingNumber,
        fCurrentBarLineTimes,
        msrBarLineCategoryKind::kBarLineCategory_UNKNOWN_, // will be set afterwards
        fCurrentBarLineHasSegnoKind,
        fCurrentBarLineHasCodaKind,
        fCurrentBarLineRepeatWingedKind);

	// ... ... ...

  // wait until its category is defined
  // to append the barLine to the current segment

  // handle the barLine according to: JMI
  // http://www.musicxml.com/tutorial/the-midi-compatible-part/repeats/

  Bool barLineHasBeenHandled = false;

  switch (fCurrentBarLineLocationKind) {
    case msrBarLineLocationKind::kBarLineLocationNone:
      // should not occur
      break;

    case msrBarLineLocationKind::kBarLineLocationLeft:
      if (
        fCurrentBarLineEndingTypeKind
          ==
        msrBarLineEndingTypeKind::kBarLineEndingTypeStart
      ) {
        // ending start, don't know yet whether it's hooked or hookless
        // ------------------------------------------------------
        if (! fCurrentBarLineEndingNumber.size ()) {
          musicxmlWarning (
            gServiceRunData->getInputSourceName (),
            inputLineNumber,
            "mandatory ending number is missing, assuming \"1\"");

          fCurrentBarLineEndingNumber = "1";
        }

        // don't know yet whether repeat ending start barLine is hooked or hookless
        // remember it in fCurrentRepeatEndingStartBarLine,
        fCurrentRepeatEndingStartBarLine = barLine;

        // handle the repeat ending start
        handleRepeatEndingStart (barLine);

        barLineHasBeenHandled = true;
      }

      else if (
        fCurrentBarLineRepeatDirectionKind
          ==
        msrBarLineRepeatDirectionKind::kBarLineRepeatDirectionForward
      ) {
        // repeat start
        // ------------------------------------------------------
        // set the barLine category
        barLine->
          setBarLineCategory (
            msrBarLineCategoryKind::kBarLineCategoryRepeatStart);

        // handle the repeat start
        handleRepeatStart (barLine);

        barLineHasBeenHandled = true;
      }
      break;

    case msrBarLineLocationKind::kBarLineLocationMiddle:
      // JMI ???
      break;

    case msrBarLineLocationKind::kBarLineLocationRight:
      {
        if (
          fCurrentBarLineEndingTypeKind == msrBarLineEndingTypeKind::kBarLineEndingTypeStop
            &&
          fCurrentBarLineEndingNumber.size () != 0
        ) {
          // hooked ending end
          // ------------------------------------------------------
          // set current barLine ending start category
          fCurrentRepeatEndingStartBarLine->
            setBarLineCategory (
              msrBarLineCategoryKind::kBarLineCategoryHookedEndingStart);

          // set this barLine's category
          barLine->
            setBarLineCategory (
              msrBarLineCategoryKind::kBarLineCategoryHookedEndingEnd);

          // handle the repeat hooked ending end
          handleRepeatHookedEndingEnd (barLine);

          barLineHasBeenHandled = true;
        }

        else if (
          fCurrentBarLineRepeatDirectionKind
            ==
          msrBarLineRepeatDirectionKind::kBarLineRepeatDirectionBackward
        ) {
          // repeat end
          // ------------------------------------------------------

          // set this barLine's category
          barLine->
            setBarLineCategory (
              msrBarLineCategoryKind::kBarLineCategoryRepeatEnd);

          // handle the repeat end
          handleRepeatEnd (barLine);

          barLineHasBeenHandled = true;
        }

        else if (
          fCurrentBarLineEndingTypeKind == msrBarLineEndingTypeKind::kBarLineEndingTypeDiscontinue
            &&
          fCurrentBarLineEndingNumber.size () != 0
        ) {
          // hookless ending end
          // ------------------------------------------------------
          // set current barLine ending start category
          fCurrentRepeatEndingStartBarLine->
            setBarLineCategory (
              msrBarLineCategoryKind::kBarLineCategoryHooklessEndingStart);

          // set this barLine's category
          barLine->
            setBarLineCategory (
              msrBarLineCategoryKind::kBarLineCategoryHooklessEndingEnd);

          // handle the repeat hookless ending end
          handleRepeatHooklessEndingEnd (barLine);

          barLineHasBeenHandled = true;
        }

        // forget about current repeat ending start barLine
        fCurrentRepeatEndingStartBarLine = nullptr;
      }
      break;
  } // switch

  // set the barLine category to stand alone if not yet handled
  if (! barLineHasBeenHandled) {
    switch (fCurrentBarLineStyleKind) {
      case msrBarLineStyleKind::kBarLineStyleRegular:
      case msrBarLineStyleKind::kBarLineStyleDotted:
      case msrBarLineStyleKind::kBarLineStyleDashed:
      case msrBarLineStyleKind::kBarLineStyleHeavy:
      case msrBarLineStyleKind::kBarLineStyleLightLight:
      case msrBarLineStyleKind::kBarLineStyleLightHeavy:
      case msrBarLineStyleKind::kBarLineStyleHeavyLight:
      case msrBarLineStyleKind::kBarLineStyleHeavyHeavy:
      case msrBarLineStyleKind::kBarLineStyleTick:
      case msrBarLineStyleKind::kBarLineStyleShort:
        barLine->
          setBarLineCategory (
            msrBarLineCategoryKind::kBarLineCategoryStandalone);

        // append the bar line to the current part
				// ... ... ...

        fCurrentPart->
          appendBarLineToPart (barLine);

        barLineHasBeenHandled = true;
        break;

      case msrBarLineStyleKind::kBarLineStyleNone:
        std::stringstream ss;

        ss <<
          "barLine " <<
          barLine->asString () <<
          " has no barLine style";

        musicxmlWarning (
          gServiceRunData->getInputSourceName (),
          inputLineNumber,
     //     __FILE__, __LINE__,
          ss.str ());
        break;
    } // switch
  }

  // has this barLine been handled?
  if (! barLineHasBeenHandled) {
    std::stringstream ss;

    ss << std::left <<
      "cannot handle a barLine containing: " <<
      barLine->asString ();

    msrInternalWarning (
      gServiceRunData->getInputSourceName (),
      inputLineNumber,
      ss.str ());
  }

  fOnGoingBarLine = false;
}
\end{lstlisting}


% -------------------------------------------------------------------------
\section{Translating repeats from MXSR to MSR}
% -------------------------------------------------------------------------

This is done in \mxsrToMsr{}.


% -------------------------------------------------------------------------
\section{Translating repeats from MSR to MSR}
% -------------------------------------------------------------------------

This is done in \msrToMsr{}.


% -------------------------------------------------------------------------
\section{Translating repeats from MSR to LPSR}
% -------------------------------------------------------------------------

This is done in \msrToLpsr{}.


% -------------------------------------------------------------------------
\section{Translating repeats from LPSR to LilyPond}
% -------------------------------------------------------------------------

This is done in \lpsrToLilypond{}.


