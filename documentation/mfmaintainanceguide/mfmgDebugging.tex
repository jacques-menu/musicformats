% !TEX root = mfmaintainanceguide.tex


% -------------------------------------------------------------------------
\chapter{Debugging}
% -------------------------------------------------------------------------

Debugging \mf\ can be quite time-consuming. The trace options available have been designed to provide fine-grained trace information to help locate issues.

\Function{catchSignals} at the beginning of the \functionName {main} function can be commented out in order to run a service under a debugger.

File \mflibrary{mfPreprocessorSettings.h} contains:
\begin{lstlisting}[language=CPlusPlus]
/*
  MusicFormats Library
  Copyright (C) Jacques Menu 2016-2023

  This Source Code Form is subject to the terms of the Mozilla Public
  License, v. 2.0. If a copy of the MPL was not distributed with this
  file, you can obtain one at http://mozilla.org/MPL/2.0/.

  https://github.com/jacques-menu/musicformats
*/

#ifndef ___mfPreprocessorSettings___
#define ___mfPreprocessorSettings___

/*
  This file groups several build-time setting that influence
  code security and speed, as well as a couple of facilities
*/

//______________________________________________________________________________
// uncomment the following definition if no exceptions display is desired
#define MF_CAUGHT_EXCEPTIONS_DISPLAY_IS_ENABLED

//______________________________________________________________________________
// comment the following definition if abort on internal errors is desired
// CAUTION: DON'T USE THIS IN PRODUCTION CODE,
// since that could kill a session on a web server, for example
#define MF_ABORT_TO_DEBUG_ERRORS_IS_ENABLED

//______________________________________________________________________________
// uncomment the following definition if abort on internal errors is desired
#define MF_SANITY_CHECKS_ARE_ENABLED

//______________________________________________________________________________
// uncomment the following definition if harmonies extra options are desired
#define MF_HARMONIES_EXTRA_IS_ENABLED

  #include "oahHarmoniesExtraOah.h"

//______________________________________________________________________________
// comment the following definition if no trace is desired
#define MF_TRACE_IS_ENABLED

#ifdef MF_TRACE_IS_ENABLED
  #include "mfTraceOah.h"
#endif // MF_TRACE_IS_ENABLED


#endif // ___mfPreprocessorSettings___
\end{lstlisting}


% -------------------------------------------------------------------------
\section{Useful options}
% -------------------------------------------------------------------------

Here are the most basing options used when debugging:
\begin{itemize}
\item \optionBoth{trace-passes}{tpasses}
	this is the first option to use, to locate in which pass the problem arises

\item \optionBoth{input-line-numbers}{iln}
	this option produces the music elements input-line numbers in the output files

\item the \optionName{display*} options

\end{itemize}


% -------------------------------------------------------------------------
\section{Removing the results of a build}
% -------------------------------------------------------------------------

The contents of \distrib\ after a build is:
\begin{lstlisting}[language=Terminal]
jacquesmenu@macmini: ~/musicformats-git-dev > ll build/
total 80
 0 drwxr-xr-x  10 jacquesmenu  staff    320 May  2 18:18:50 2022 ./
 0 drwxr-xr-x  39 jacquesmenu  staff   1248 May  2 18:18:51 2022 ../
16 -rw-r--r--@  1 jacquesmenu  staff   6148 May  2 15:45:10 2022 .DS_Store
 8 -rw-r--r--@  1 jacquesmenu  staff   1815 Jun 29 09:10:50 2021 Building.md
32 -rw-r--r--   1 jacquesmenu  staff  14849 May  2 18:18:50 2022 CMakeLists.txt
 8 -rw-r--r--   1 jacquesmenu  staff    291 Jun 29 09:10:50 2021 MakePkg.bat
16 -rw-r--r--   1 jacquesmenu  staff   7463 May  2 18:18:50 2022 Makefile
 0 drwxr-xr-x@ 27 jacquesmenu  staff    864 May  2 18:21:17 2022 bin/
 0 drwxr-xr-x@  7 jacquesmenu  staff    224 May  2 18:21:11 2022 lib/
 0 drwxr-xr-x  12 jacquesmenu  staff    384 May  2 18:19:25 2022 libdir/
\end{lstlisting}

The built files are in \distrib{bin}, \distrib{lib} and \distrib{libdir}. There is no \code{clean} target in \Makefile. They can be removed in a single step with this alias:
\begin{lstlisting}[language=Terminal]
jacquesmenu@macmini: ~/musicformats-git-dev > type rmbuild
rmbuild is aliased to `cd ${MUSIC_FORMATS_DEV}/build ; rm -r bin lib libdir; ls -sal'
\end{lstlisting}


% -------------------------------------------------------------------------
\section{Reverting to a previous MusicFormats version}
% -------------------------------------------------------------------------

The \github\ \mf\ \repo\ keeps a number of recents releases, such as:
\begin{lstlisting}[language=Terminal]
jacquesmenu@macmini: ~/musicformats-git-dev > git branch
* master
  v0.9.60
  v0.9.61
  v0.9.62
\end{lstlisting}

Then \code{master} branch contains the development version. To switch back to another version, one should check it out:
\begin{lstlisting}[language=Terminal]
jacquesmenu@macmini: ~/musicformats-git-dev > git checkout v0.9.61
Switched to branch 'v0.9.61'
Your branch is up to date with 'master/v0.9.61'.
jacquesmenu@macmini: ~/musicformats-git-dev > git branch
  master
  v0.9.60
* v0.9.61
  v0.9.62
\end{lstlisting}

Now building this version with \code{make}, we get:
\begin{lstlisting}[language=Terminal]
jacquesmenu@macmini: ~/musicformats-git-dev/build > xml2ly -version
Command line version of musicxml2lilypond converter v0.9.61 (March 3, 2022)
\end{lstlisting}


% -------------------------------------------------------------------------
\chapter{Locating a bug with Git's bisection}
% -------------------------------------------------------------------------

A bug appeared in v0.9.63, in which the \optionNameBoth{display-lpsr-short}{dlpsrshort} option causes \xmlToLy\ to crash.
The symptom varies with the \OS, pointing to a probable memory corruption.

Reverting to \mf\ previous versions show that:
\begin{itemize}
\item v0.9.60 behaves alright;
\item v0.9.61 exhibits the bug:

\begin{lstlisting}[language=Terminal]
jacquesmenu@macmini: ~/Desktop > lldb -- xml2ly  fullbarrests/MeasureRestWithoutBarLine.xml -display-lpsr-short -aofn
(lldb) target create "xml2ly"
Current executable set to 'xml2ly' (x86_64).
(lldb) settings set -- target.run-args  "fullbarrests/MeasureRestWithoutBarLine.xml" "-display-lpsr" "-aofn"
(lldb) r
Process 28676 launched: '/Users/jacquesmenu/musicformats-git-dev/build/bin/xml2ly' (x86_64)
        The measure with ordinal number 3 is now registered with a duration of 1/1 in part Part_POne (partID "P1", partName "Soprano"), fPartMeasuresWholeNotessVector.size () = 2
        The measure with ordinal number 4 is now registered with a duration of 1/1 in part Part_POne (partID "P1", partName "Soprano"), fPartMeasuresWholeNotessVector.size () = 2
%--------------------------------------------------------------
  Pass (ptional): displaying the LPSR as text
%--------------------------------------------------------------

... ... ...

    [PartGroup "PartGroup_1 ('0', fPartGroupName "Implicit")" (1 part), line 0
      fPartGroupName            : "Implicit"

Process 28676 stopped
* thread #1, queue = 'com.apple.main-thread', stop reason = EXC_BAD_ACCESS (code=1, address=0x1)
    frame #0: 0x00000001007b879a xml2ly`MusicFormats::msrPartGroup::print (this=0x0000600003e08100, os=0x000000010ca04650) const at msrPartGroups.cpp:1185:13
   1182	      i      = iBegin;
   1183
   1184	    for ( ; ; ) {
-> 1185	      os << (*i);
   1186	      if (++i == iEnd) break;
   1187	      os << std::endl;
   1188	    } // for
Target 0: (xml2ly) stopped.
(lldb)
\end{lstlisting}

\end{itemize}

This bug has thus been introduced between v0.9.60 and v0.9.61.
Three have been several \code{git push} occurrences leading from v0.9.60 to v0.9.61.


% -------------------------------------------------------------------------
\section{Locating a bug at random in the Git log}
% -------------------------------------------------------------------------

\git\ provides various ways to display the commits history of the \repo\ through \code{git log} options, for example:
\begin{lstlisting}[language=Terminal]
jacquesmenu@macmini: ~/musicformats-git-dev > git log --pretty=format:"%h - %ad : %s"
ea338fd - Tue May 3 10:09:04 2022 +0200 : Complement to the Makefile
083db8a - Tue May 3 07:43:09 2022 +0200 : Before reverting to v0.9.60
12b6d93 - Mon May 2 09:41:37 2022 +0200 : Prior to bisecting
03d98be - Tue Apr 26 11:15:23 2022 +0200 : Finalized sone trace options
3dd7b72 - Tue Apr 26 10:10:38 2022 +0200 : Finalized sone trace options
7f7507c - Thu Apr 14 17:01:14 2022 +0200 : Finalized MFSL symtax and semantics, fixed a couple of issues
06109d3 - Thu Apr 14 17:00:50 2022 +0200 : Finalized MFSL symtax and semantics, fixed a couple of issues
62aa64c - Thu Apr 7 07:18:44 2022 +0200 : v0.9.62
671ffa4 - Thu Apr 7 06:26:34 2022 +0200 : Pre v0.9.62
bf9eb63 - Wed Apr 6 23:42:44 2022 +0200 : Pre v0.9.62
db4397c - Wed Apr 6 22:14:43 2022 +0200 : Pre v0.9.62
9a80b24 - Mon Apr 4 13:06:12 2022 +0200 : Added MFSL (MusicFormats Script Language)
2ef1150 - Mon Apr 4 12:07:07 2022 +0200 : Added MFSL (MusicFormats Script Language)
3f56d52 - Tue Mar 29 16:34:23 2022 +0200 : Added MFSL (MusicFormats Script Language)
fc1ea21 - Tue Mar 29 08:58:34 2022 +0200 : Added MFSL (MusicFormats Script Language)
737b996 - Mon Mar 28 23:42:03 2022 +0200 : Added MFSL (MusicFormats Script Language)
8c91155 - Sat Mar 26 08:35:55 2022 +0100 : Added MFSL (MusicFormats Script Language)
fc68a93 - Fri Mar 18 15:11:19 2022 +0100 : Added MFSL (MusicFormats Script Language)
01430a9 - Fri Mar 18 15:11:12 2022 +0100 : Added MFSL (MusicFormats Script Language)
4082813 - Thu Mar 17 18:50:02 2022 +0100 : Added MFSL (MusicFormats Script Language)
2696628 - Sun Mar 13 00:48:05 2022 +0100 : Added MFSL (MusicFormats Script Language)
a828231 - Thu Mar 10 14:28:11 2022 +0100 : Added MFSL (MusicFormats Script Language)
bf04937 - Wed Mar 9 12:53:17 2022 +0100 : Added MFSL (MusicFormats Script Language)
a855ee4 - Tue Mar 8 16:39:28 2022 +0100 : Added MFSL (MusicFormats Script Language)
b636816 - Mon Mar 7 14:49:54 2022 +0100 : Added MFSL (MusicFormats Script Language)
ecd5eaa - Sun Mar 6 00:11:05 2022 +0100 : Added 'keep-msr-voice, kmv' option
ec1c8ef - Sat Mar 5 08:48:39 2022 +0100 : Added 'ignore-msr-voice, imv' option
8246467 - Thu Mar 3 16:11:37 2022 +0100 : v0.9.61
603e19c - Thu Mar 3 13:43:48 2022 +0100 : Switched from C++11 to C++17 for <filesystem>
77d3d29 - Thu Mar 3 07:56:00 2022 +0100 : Switched from C++11 to C++17 for <filesystem>
a880063 - Thu Mar 3 07:44:08 2022 +0100 : Switched from C++11 to C++17 for <filesystem>
38b584f - Wed Mar 2 12:44:22 2022 +0100 : Switched from C++11 to C++17 for <filesystem>
662454a - Tue Mar 1 17:14:47 2022 +0100 : Pre-v0.9.61
2cb4d5f - Mon Feb 28 11:53:04 2022 +0100 : Renamed some documentation folders and files
c7839a8 - Mon Feb 28 09:56:46 2022 +0100 : Renamed some documentation folders and files
0e85f99 - Mon Feb 28 09:06:16 2022 +0100 : Renames some documentation folders and files
c5a43d9 - Fri Feb 25 17:48:29 2022 +0100 : Finalized files/musicxmlfiles/Makefile
21e3898 - Thu Feb 24 21:54:28 2022 +0100 : Added '-replicate-msr-measure'  option
9738598 - Thu Feb 24 21:53:00 2022 +0100 : Added '-replicate-msr-measure'  option
ae751c3 - Mon Feb 21 09:58:35 2022 +0100 : Added various options
f2d2f57 - Sat Feb 19 08:09:02 2022 +0100 : Workflow to publish Mac OS release
ac5ad6b - Sat Feb 19 08:00:57 2022 +0100 : Initializa npm package
29de34d - Fri Feb 18 11:00:42 2022 +0100 : v0.9.60
7c067d6 - Fri Feb 18 10:56:17 2022 +0100 : v0.9.60
5e3ba90 - Fri Feb 18 09:57:50 2022 +0100 : Pre v0.9.60
dfeb7be - Fri Feb 18 09:56:07 2022 +0100 : Pre v0.9.60
c31dde3 - Wed Feb 16 11:50:42 2022 +0100 : Updates to the make and cmake configuration
fd6fef0 - Wed Feb 16 09:45:44 2022 +0100 : Complements to the installation doc
b7ad2af - Tue Feb 15 17:40:53 2022 +0100 : Distrib test 17
50a904c - Tue Feb 15 17:37:53 2022 +0100 : Distrib test 16
cf65bd3 - Tue Feb 15 08:41:14 2022 +0100 : Distrib test 15
9cda15e - Tue Feb 15 08:38:57 2022 +0100 : Distrib test 14
74a2b7f - Tue Feb 15 08:30:21 2022 +0100 : Distrib test 13
ee011e9 - Mon Feb 14 15:06:59 2022 +0100 : Distrib test 13
4d6f9cb - Mon Feb 14 08:54:23 2022 +0100 : Distrib test 12
7692b7b - Mon Feb 14 08:52:50 2022 +0100 : Distrib test 12
d9e943d - Sat Feb 12 09:30:45 2022 +0100 : Distrib test 11
3dde810 - Sat Feb 12 09:29:51 2022 +0100 : Distrib test 11
f07b02a - Fri Feb 11 17:55:34 2022 +0100 : Distrib test 10
5113824 - Fri Feb 11 16:47:17 2022 +0100 : Distrib test 9
7f0fa8e - Fri Feb 11 16:45:21 2022 +0100 : Distrib test 8
93f72f4 - Fri Feb 11 16:32:33 2022 +0100 : Distrib test 6
121aa64 - Fri Feb 11 14:35:38 2022 +0100 : Distrib test 5
547556f - Fri Feb 11 11:30:17 2022 +0100 : Creation of MusicFormats repository
\end{lstlisting}

One can pick one of the commits, revert to it and check wether the bug is present in it.


% -------------------------------------------------------------------------
\section{Locating a bug in the commits with Git's bisection}
% -------------------------------------------------------------------------

Locating the particular push that introduced the bug can be facilitated by \git 's \MainIt{bisect} facility.
Here is how it works:
\begin{lstlisting}[language=Terminal]
jacquesmenu@macmini: ~/musicformats-git-dev > git bisect start

jacquesmenu@macmini: ~/musicformats-git-dev > git bisect good v0.9.60

jacquesmenu@macmini: ~/musicformats-git-dev > git bisect bad v0.9.61
Bisecting: 7 revisions left to test after this (roughly 3 steps)
[0e85f994ab00ea2dd94ddcb1895cbae5a32f072a] Renames some documentation folders and files

jacquesmenu@macmini: ~/musicformats-git-dev > git branch
* (no branch, bisect started on v0.9.61)
  master
  v0.9.60
  v0.9.61
  v0.9.62
\end{lstlisting}

The bisection proposes commit \code{0e85f994ab00ea2dd94ddcb1895cbae5a32f072a} as a middle point between v0.9.60 and v0.9.61.
So let us check it out:
\begin{lstlisting}[language=Terminal]
jacquesmenu@macmini: ~/musicformats-git-dev > git checkout 0e85f994ab00ea2dd94ddcb1895cbae5a32f072a
HEAD is now at 0e85f99 Renames some documentation folders and files

jacquesmenu@macmini: ~/musicformats-git-dev > git branch
* (no branch, bisect started on v0.9.61)
  master
  v0.9.60
  v0.9.61
  v0.9.62
\end{lstlisting}

Then, building this intermediate development version leads to:
\begin{lstlisting}[language=Terminal]
jacquesmenu@macmini: ~/musicformats-git-dev/build > make
make macos
cd libdir && cmake .. -G Xcode   -Wno-dev
-- VERSION: v0.9.61

-- Configuring version v0
.v9
.v61

... ... ...

/Users/jacquesmenu/musicformats-git-dev/src/components/mfcLibraryComponent.cpp:12:10: fatal error:
      '../../../MusicFormatsVersionNumber.h' file not found
#include "../../../MusicFormatsVersionNumber.h"
         ^~~~~~~~~~~~~~~~~~~~~~~~~~~~~~~~~~~~~~
1 error generated.

... ... ...

** BUILD FAILED **
\end{lstlisting}

Well, this dev version had been pushed to have new files and/or contents saved on the \mf\ \repo\ \dots and we should try other commits around it.

A first possibility is to use \code{git bisect skip}, that moves to :
\begin{lstlisting}[language=Terminal]
jacquesmenu@macmini: ~/musicformats-git-dev > git bisect skip
Bisecting: 7 revisions left to test after this (roughly 3 steps)
[c7839a87549660963a8b1ef0898d5cbcce8305aa] Renamed some documentation folders and files
\end{lstlisting}

Checking commit c7839a87549660963a8b1ef0898d5cbcce8305aa out, we get:
\begin{lstlisting}[language=Terminal]
jacquesmenu@macmini: ~/musicformats-git-dev > git checkout c7839a87549660963a8b1ef0898d5cbcce8305aa
HEAD is now at c7839a8 Renamed some documentation folders and files
\end{lstlisting}

Building that leads to the same error as above. Let us skip one again:
\begin{lstlisting}[language=Terminal]
jacquesmenu@macmini: ~/musicformats-git-dev > git bisect skip
Bisecting: 7 revisions left to test after this (roughly 3 steps)
[a880063c134a7ba49b31f5fb52b47f682058f64a] Switched from C++11 to C++17 for <filesystem>
\end{lstlisting}

It turns out the commit \code{a880063c134a7ba49b31f5fb52b47f682058f64a} does not build either. Let us skip to the next commit:
\begin{lstlisting}[language=Terminal]
jacquesmenu@macmini: ~/musicformats-git-dev > git bisect skip
Bisecting: 7 revisions left to test after this (roughly 3 steps)
[2cb4d5f2133d34a19fea8f47d9ed2ccfb24d0fad] Renamed some documentation folders and files

jacquesmenu@macmini: ~/musicformats-git-dev > git checkout 2cb4d5f2133d34a19fea8f47d9ed2ccfb24d0fad
HEAD is now at 2cb4d5f Renamed some documentation folders and files
\end{lstlisting}

Here the code base builds alright, and bug does not show up, so we should continue skipping.


% -------------------------------------------------------------------------
\section{Locating the bug in the code base}
% -------------------------------------------------------------------------

The bug we're after is found to have been introduced at this point:
\begin{itemize}
\item commit \code{2cb4d5f}, does not exhibit the bug;
\item the next one, commit \code{662454a}, does.
\end{itemize}

The changes brought by commit \code{} can be shown with: %%% git show 662454a   git show --stat 662454a
\begin{lstlisting}[language=Terminal]
jacquesmenu@macmini: ~/musicformats-git-dev > git log --patch -1 662454a > patch_662454a.txt

jacquesmenu@macmini: ~/musicformats-git-dev > ls -sal patch_662454a.txt
256 -rw-r--r--@ 1 jacquesmenu  staff  80782 May  4 10:59 patch_662454a.txt
\end{lstlisting}

The bug shows up with \mfFiles{musicxmlfiles/fullbarrests/MeasureRestWithoutBarLine.xml}, but not with others such as \mfFiles{musicxmlfiles/multistaff/SATBExample.xml}.

Analysing the patch description in \fileName{patch_662454a.txt}, we find that nothing in the differences between those two successive patches can explain  the crash. The problem thus lies elsewhere\dots

%%%JMI


