% -------------------------------------------------------------------------
%  MusicFormats Library
%  Copyright (C) Jacques Menu 2016-2023

%  This Source Code Form is subject to the terms of the Mozilla Public
%  License, v. 2.0. If a copy of the MPL was not distributed with this
%  file, you can obtain one at http://mozilla.org/MPL/2.0/.

%  https://github.com/jacques-menu/musicformats
% -------------------------------------------------------------------------

% !TEX root = mfmaintainanceguide.tex

% -------------------------------------------------------------------------
\chapter{Documentation}
% -------------------------------------------------------------------------

The \mf\ documentation is written in \LaTeX, the pictures being created with the \tikzpgf\ package, see \url{https://www.bu.edu/math/2013/08/tikzpgfmanual.pdf}.

All the \LaTeX\ files have an initial '\code{!~TEX root}' comment. This is a \texshop\ specific feature, allowing a documentation to be composed from any of the files that it imports, such as:
\begin{lstlisting}[language=Latex]
% !TEX root = mfmaintainanceguide.tex
\end{lstlisting}

The \directoryName{documentation} directory contains:
\begin{lstlisting}[language=Terminal]
jacquesmenu@macmini: ~/musicformats-git-dev/documentation > ls -sal
total 48
 0 drwxr-xr-x@  10 jacquesmenu  staff    320 Feb 28 07:42 .
 0 drwxr-xr-x   38 jacquesmenu  staff   1216 Feb 27 12:14 ..
48 -rw-r--r--@   1 jacquesmenu  staff  22532 Feb 28 07:49 .DS_Store
 0 drwxr-xr-x   18 jacquesmenu  staff    576 Feb 28 08:27 mflatexlib
 0 drwxr-xr-x   38 jacquesmenu  staff   1216 Feb 18 08:39 IntroductionToMusicXML
 0 drwxr-xr-x   57 jacquesmenu  staff   1824 Feb 28 10:08 mfapiguide
 0 drwxr-xr-x  116 jacquesmenu  staff   3712 Feb 28 10:22 mfmaintainanceguide
 0 drwxr-xr-x   53 jacquesmenu  staff   1696 Feb 28 10:07 mfuserguide
 0 drwxr-xr-x   27 jacquesmenu  staff    864 Feb 14 08:54 graphics
 0 drwxr-xr-x    5 jacquesmenu  staff    160 Jan 23 16:33 presentation
\end{lstlisting}

The \directoryName{mflatexlib} directory contains \LaTeX\ settings used by the various documentation files and the code for pictures:
\begin{lstlisting}[language=Terminal]
jacquesmenu@macmini:~/musicformats-git-dev/documentation/mflatexlib > ls -sal *.tex
 8 -rwxr--r--@  1 jacquesmenu  staff    241 Aug 17 14:21 CreateMSRScoreRepresentation.zsh
 8 -rw-r--r--   1 jacquesmenu  staff    507 Jun 28 20:52 LaTeXBoxes.tex
16 -rw-r--r--   1 jacquesmenu  staff   6494 Jun 28 20:52 LaTeXCommonSettings.tex
 8 -rw-r--r--   1 jacquesmenu  staff   1907 Jun 28 20:52 LaTeXDivisionsCommands.tex
 8 -rw-r--r--   1 jacquesmenu  staff    957 Jun 28 20:52 LaTeXFontsAndColors.tex
 8 -rw-r--r--   1 jacquesmenu  staff    604 Jun 28 20:52 LaTeXGraphicsAndPictures.tex
 8 -rw-r--r--   1 jacquesmenu  staff   1128 Jun 28 20:52 LaTeXIndexing.tex
24 -rw-r--r--   1 jacquesmenu  staff  10728 Jun 28 20:52 LaTeXListings.tex
 8 -rw-r--r--   1 jacquesmenu  staff   1527 Jun 28 20:52 LaTeXMusicFormatsCommands.tex
24 -rw-r--r--   1 jacquesmenu  staff  11735 Aug 14 16:50 LaTeXMusicFormatsFilesAndFolders.tex
 8 -rw-r--r--@  1 jacquesmenu  staff   2151 Jun 28 20:52 LaTeXMusicFormatsNames.tex
 8 -rw-r--r--   1 jacquesmenu  staff    441 Jun 28 20:52 LaTeXMusicNotation.tex
 8 -rw-r--r--   1 jacquesmenu  staff   1535 Jun 28 20:52 LaTeXReferencing.tex
32 -rw-r--r--   1 jacquesmenu  staff  15145 Aug 14 16:50 LaTeXShortcuts.tex
 8 -rw-r--r--   1 jacquesmenu  staff   2171 Jun 28 20:52 LaTeXTablesAndLists.tex
40 -rw-r--r--@  1 jacquesmenu  staff  17746 Jun 28 20:52 MSRClassesHierarchyPicture.tex
40 -rw-r--r--@  1 jacquesmenu  staff  16711 Aug 17 14:21 MSRScoreRepresentation.tex
 8 -rw-r--r--   1 jacquesmenu  staff    321 Jun 28 20:52 MusicFormats.ist
48 -rw-r--r--   1 jacquesmenu  staff  21749 Jun 28 20:52 MusicFormatsArchitecturePicture.tex
 8 -rw-r--r--   1 jacquesmenu  staff   1558 Aug 15 23:36 OAHClassesHierarchyPicture.log
16 -rw-r--r--@  1 jacquesmenu  staff   5906 Aug 16 08:02 OAHClassesHierarchyPicture.tex
\end{lstlisting}

It is to be noted that \doc{mflatexlib/MSRScoreRepresentation.tex} is generated from \doc{mflatexlib/MSRClassesHierarchyPicture.tex} with this script, which should be run only if there is any change in the \msrRepr\ classes hierarchy:
\begin{lstlisting}[language=Terminal]
jacquesmenu@macmini-1:~/musicformats-git-dev/documentation/mflatexlib > cat CreateMSRScoreRepresentation.zsh
#!/bin/zsh

# create a LaTeX file for the user guide from the maintainance one

sed 's/msr//g' MSRClassesHierarchyPicture.tex \
  | \
sed 's/The MSR classes hierarchy/The MSR music score representation/g' \
  > \
MSRScoreRepresentation.tex
\end{lstlisting}

\Directory{graphics} contains \Main{PNG} files showing screenshots of the results of using the \mf\ tools. % JMI ??? v0.9.67

\Directory{libmusicxml2Presentation} contains a presentation of \libmusicxml\ written by \fober.

\Directory{IntroductionToMusicxml} contains a presentation done by this author at the \\
'MUSIC ENGRAVING IN THE 21ST CENTURY -- DEVELOPMENTS AND PERSPECTIVES' conference at Mozarteum in Salzburg in January 2020 (https://www.uni-mozarteum.at/en/kunst/music-engraving-conference.php).

%\begin{itemize} JMI v0.9.66
%%\item \doc{LaTeXCommonSettings.tex}: a set of commands and settings used by all the documentation \LaTeX\ files;
%\item ;
%\item \doc{libmusicxml2Presentation/}: Dom Fober's original documentation describing \libmusicxml;
%\item \doc{MusicFormatsArchitecture.tex}: the image describing the architecture of \mf, used in other documents;
%\item \doc{mfuserguide/}: the user guide;
%\item \doc{maintainersGuideToMusicFormats/}: this maintiainer's guide.
%\end{itemize}

\LaTeX\ needs to be run {\it three} times when the chapter/section/subsection hierarchy is modified. Check that the last page number, at the bottom of any page, is not less than the one before.

The following files contain the current \mf\ version number and date:
\begin{itemize}
\item the \src{MusicFormatsVersionNumber.h} and \src{MusicFormatsVersionDate.h} files are used by the C++ code base;
\item \file{MusicFormatsVersionNumber.txt} and \file{MusicFormatsVersionDate.txt} are used by the \LaTeX\ source files
\end{itemize}
Those files should be re-generated when a new version of \mf\ is created, for example:
\begin{lstlisting}[language=Terminal]
jacquesmenu@macmini: ~/musicformats-git-dev > devtools/SetMusicFormatsVersionNumber.bash "0.9.61"
==> PWD is:
/Users/jacquesmenu/musicformats-git-dev

==> Writing MusicFormats version number 0.9.61 to MusicFormatsVersionNumber.txt

8 -rw-r--r--@ 1 jacquesmenu  staff  6 Mar  2 13:43:04 2022 MusicFormatsVersionNumber.txt
0.9.61
==> PWD is:
/Users/jacquesmenu/musicformats-git-dev/src

==> Writing MusicFormats version number 0.9.61 to MusicFormatsVersionNumber.h

8 -rw-r--r--@ 1 jacquesmenu  staff  45 Mar  2 13:43:04 2022 MusicFormatsVersionNumber.h
#define MUSICFORMATS_VERSION_NUMBER "0.9.61"
\end{lstlisting}

and:

\begin{lstlisting}[language=Terminal]
jacquesmenu@macmini: ~/musicformats-git-dev > devtools/SetMusicFormatsVersionDate.bash "March 2, 2022"
==> PWD is:
/Users/jacquesmenu/musicformats-git-dev

==> Writing MusicFormats version date March 2, 2022 to MusicFormatsVersionDate.txt

8 -rw-r--r--@ 1 jacquesmenu  staff  14 Mar  2 13:43:32 2022 MusicFormatsVersionDate.txt
March 2, 2022

==> PWD is:
/Users/jacquesmenu/musicformats-git-dev/src

==> Writing MusicFormats version date March 2, 2022 to MusicFormatsVersionDate.h

8 -rw-r--r--@ 1 jacquesmenu  staff  50 Mar  2 13:43:32 2022 MusicFormatsVersionDate.h
#define MUSICFORMATS_VERSION_DATE "March 2, 2022"
\end{lstlisting}

Avoid editing these files manually. In particular, \fileName{MusicFormatsVersionNumber.txt} should {\bf NOT} be terminated by an end of line, since its contents is used in the name of library files generated in \build{lib}.


% -------------------------------------------------------------------------
\section{\LaTeX\ macros}
% -------------------------------------------------------------------------

The \mf\ documentation uses a number of macros both to simplify formatting of frequent texts and to feed the many indexes at the end. All of them are grouped in \doc{mflatexlib}:
\begin{lstlisting}[language=Terminal]
jacquesmenu@macstudio:~/musicformats-git-dev/documentation/mflatexlib > ls -sal LaTeX*.tex
 8 -rw-r--r--  1 jacquesmenu  staff    507 Jun 28 20:52 LaTeXBoxes.tex
16 -rw-r--r--  1 jacquesmenu  staff   6494 Jun 28 20:52 LaTeXCommonSettings.tex
 8 -rw-r--r--  1 jacquesmenu  staff   1907 Jun 28 20:52 LaTeXDivisionsCommands.tex
 8 -rw-r--r--  1 jacquesmenu  staff    957 Jun 28 20:52 LaTeXFontsAndColors.tex
 8 -rw-r--r--  1 jacquesmenu  staff    604 Jun 28 20:52 LaTeXGraphicsAndPictures.tex
 8 -rw-r--r--  1 jacquesmenu  staff   1128 Jun 28 20:52 LaTeXIndexing.tex
24 -rw-r--r--  1 jacquesmenu  staff  10728 Jun 28 20:52 LaTeXListings.tex
 8 -rw-r--r--  1 jacquesmenu  staff   1527 Jun 28 20:52 LaTeXMusicFormatsCommands.tex
24 -rw-r--r--@ 1 jacquesmenu  staff  11735 Jul 29 09:02 LaTeXMusicFormatsFilesAndFolders.tex
 8 -rw-r--r--@ 1 jacquesmenu  staff   2151 Jun 28 20:52 LaTeXMusicFormatsNames.tex
 8 -rw-r--r--  1 jacquesmenu  staff    441 Jun 28 20:52 LaTeXMusicNotation.tex
 8 -rw-r--r--  1 jacquesmenu  staff   1535 Jun 28 20:52 LaTeXReferencing.tex
32 -rw-r--r--@ 1 jacquesmenu  staff  14665 Jun 28 20:52 LaTeXShortcuts.tex
 8 -rw-r--r--  1 jacquesmenu  staff   2171 Jun 28 20:52 LaTeXTablesAndLists.tex
\end{lstlisting}

For example:
\begin{lstlisting}[language=Latex]
\newcommand{\CLI}{command line\index[Main]{command line}}
\end{lstlisting}

\begin{lstlisting}[language=Latex]
\newcommand{\musicXmlMarkup}[1]{%
{\tt <#1/>}\index[Main]{{\tt $<$#1 /$>$}}\index[MusicXML]{{\tt #1 $<$/$>$}}%
}
\newcommand{\musicXmlAttribute}[1]{%
{\tt "#1"}\index[Main]{{\tt $<$#1 /$>$}}\index[MusicXML]{{\tt #1 ""}}%
}
\end{lstlisting}

\begin{lstlisting}[language=Latex]
\newcommand{\Main}[1]{%
#1\index[Main]{#1}%
}
\newcommand{\MainName}[1]{%
\index[Main]{#1}%
}

\newcommand{\code}[1]{%
{\tt #1}\index[Main]{{\tt #1}}%
}
\end{lstlisting}

Some command exist in two forms, differing in the capitalization of the first character:
\begin{lstlisting}[language=Latex]
\newcommand{\enumType}{enumeration type\index[Main]{enumeration type}}
\newcommand{\EnumType}{Enumeration type\index[Main]{enumeration type}}
\end{lstlisting}

Some command names are of the form \code{*Both*}:
\begin{lstlisting}[language=Latex]
\newcommand{\fileName}[1]{%
{\tt #1}\index[Main]{{\tt #1}}\index[Files]{{\tt #1}}%
}
\newcommand{\fileNameBoth}[1]{%
{\textcolor{brown}{\tt *#1.h/.cpp}}\index[Main]{#1.h/.cpp@{{tt *#1.h/.cpp}}}\index[Files]{#1.h/.cpp@{{tt *#1.h/.cpp}}}%
}
\end{lstlisting}

\begin{lstlisting}[language=Latex]
\newcommand{\msrToMsr}[1]{%
{\textcolor{brown}{\tt src/passes/msr2msr/#1}}%
}
\newcommand{\msrToMsrBoth}[1]{%
{\textcolor{brown}{\tt src/passes/msr2msr/#1.h/.cpp}}%
}
\end{lstlisting}

Some command names are of the form \code{star*}:
\begin{lstlisting}[language=Latex]
\newcommand{\methodName}[1]{%
{\tt #1~()}\index[Main]{{\tt #1}~()}\index[MethodsAndFields]{{\tt #1}~()}%
}
\newcommand{\starMethodName}[1]{%
{\tt *#1~()}\index[Main]{#1~()@{\tt *#1}}\index[MethodsAndFields]{*#1~()@{\tt *#1~()}}%
}
\end{lstlisting}

Some commands have a variant of the form \code{*Name*} to produce only their arguments, with no additional text:
\begin{lstlisting}[language=Latex]
\newcommand{\file}[1]{%
file {\tt #1}\index[Main]{{\tt #1}}\index[Files]{{\tt #1}}%
}
\newcommand{\File}[1]{%
File {\tt #1}\index[Main]{{\tt #1}}\index[Files]{{\tt #1}}%
}

\newcommand{\fileName}[1]{%
{\tt #1}\index[Main]{{\tt #1}}\index[Files]{{\tt #1}}%
}
\newcommand{\fileNameBoth}[1]{%
{\textcolor{brown}{\tt *#1.h/.cpp}}\index[Main]{#1.h/.cpp@{{tt *#1.h/.cpp}}}\index[Files]{#1.h/.cpp@{{tt *#1.h/.cpp}}}%
}
\end{lstlisting}

Some commands are in the form \code{*Repr} : the designate the name of a representation, such as:
\begin{lstlisting}[language=Latex]
\newcommand{\msrRepr}{MSR\index[Main]{MSR}}
\end{lstlisting}


% -------------------------------------------------------------------------
\section{About this document}
% -------------------------------------------------------------------------

This document provides cross views of the information needed for \mf\ maintainance. It is organized in a number of parts:
\begin{itemize}
\item the first part provides an overview of the library, together with the concepts is uses;
\item then the two-phase visitors pattern, which is central to \mf, is presented;
\item the third part is dedicated to the programming style and conventions used throughout the code base;
\item the OAH (Options and help), a pervasive feature in \mf, is detailed;
\item the fifth part details the formats provided by the library;
\item the following parts are dedicated to passes, generators and converters, respectively;
\item the ninth part presents the interfaces to the formats, passes and converters;
\item the tenth part provides a longitudinal view of the handling of selected music score contents elements, grouped by such elements such as staves, tuplets and harmonies;
\item and finally, the last part is dedicated to the implementation of the MSDL language.
\end{itemize}


% -------------------------------------------------------------------------
\section{The MusicFormats architecture}
% -------------------------------------------------------------------------


% -------------------------------------------------------------------------
\section{User guide}
% -------------------------------------------------------------------------

\doc{mfuserguide/mfuserguide.pdf} is the usual user guide. It presents the use of \mf\ with the \CLI\ for the time being.


% -------------------------------------------------------------------------
\section{API guide}
% -------------------------------------------------------------------------

\doc{mfapiguide/mfapiguide.pdf} presents the use of \mf\ through the \API s. The latter are used internally by the \CLI\ services, and can be used from applications at will, such as in a \Web\ site.


% -------------------------------------------------------------------------
\section{Maintainance guide}
% -------------------------------------------------------------------------

\doc{mfmaintainanceguide/mfmaintainanceguide.pdf} describes the internals of \mf\ from a maintainer's point of view. It contains a detailed presentation of the various types used, and a part dedicated to selected topics: this is to have a longitudinal view of how various music elements are handled in the various passes.

