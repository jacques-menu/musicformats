% -------------------------------------------------------------------------
%  MusicFormats Library
%  Copyright (C) Jacques Menu 2016-2024

%  This Source Code Form is subject to the terms of the Mozilla Public
%  License, v. 2.0. If a copy of the MPL was not distributed with this
%  file, you can obtain one at http://mozilla.org/MPL/2.0/.

%  https://github.com/jacques-menu/musicformats
% -------------------------------------------------------------------------

% !TEX root = mfmaintainanceguide.tex

% -------------------------------------------------------------------------
\chapter{The two-phase visitors pattern}\label{The two-phase visitors pattern}
% -------------------------------------------------------------------------

\mf\ uses a two-phase visitors pattern designed by \fober\ to traverse data structures such an {\tt xmlElement tree} or an MSR description, handling each node in the structure in a systematic way. This is in contrast to a programmed top-down traversal.

Such data structures traversals is actually \MainIt{data driven}: a visitor can decide to 'see' only selected node types.

There are case where visiting is not the way to go, see the sections below.


% -------------------------------------------------------------------------
\section{Basic mechanism}
% -------------------------------------------------------------------------

Visiting a node in a data structure is done in this order:
\begin{itemize}
\item first phase: visit the node for the fist time, top-down;
\item visit the node contents, using the same two-phase visitors pattern;
\item second phase: visit the node for the second time, bottom-up.
\end{itemize}

The first can be used to prepare data needed for the node contents visit, for example.
Then the second phase can used such data, if relevant, as well as data created by the node contents visit, do consolidate the whole.

A visitor class   should:
\begin{itemize}
\item inherit from {\tt basevisitor};
\item inherit from the smart pointer classes it visits;
\item define methods \methodName{visitStart} and/or \methodName{visitEnd} depending on which phases it wants to handle. The parameter of all such \methodName{visit*} methods is always a reference to a smart pointer.
\end{itemize}

{\tt basevisitor} is defined in \visitors{basevisitor.h}, and contains nothing:
\begin{lstlisting}[language=CPlusPlus]
class   basevisitor
{
	public:
		virtual ~basevisitor() {}
};
\end{lstlisting}

It is used as the base class   of all visitors in \methodName{browsedata} methods:
\begin{lstlisting}[language=CPlusPlus]
void msrWords::acceptIn (basevisitor* v)
{
#ifdef MF_TRACE_IS_ENABLED
  if (gMsrOahGroup->getTraceMsrVisitors ()) {
    std::stringstream ss;

    ss <<
      "% ==> msrWords::acceptIn ()";

    gWaeHandler->waeTraceWithoutInputLocation (
      __FILE__, __LINE__,
      ss.str ());
  }
#endif // MF_TRACE_IS_ENABLED

  if (visitor<S_msrWords>*
    p =
      dynamic_cast<visitor<S_msrWords>*> (v)) {
        S_msrWords elem = this;

#ifdef MF_TRACE_IS_ENABLED
        if (gMsrOahGroup->getTraceMsrVisitors ()) {
          std::stringstream ss;

          ss <<
            "% ==> Launching msrWords::visitStart ()";

          gWaeHandler->waeTraceWithoutInputLocation (
            __FILE__, __LINE__,
            ss.str ());
        }
#endif // MF_TRACE_IS_ENABLED
        p->visitStart (elem);
  }
}
\end{lstlisting}


% -------------------------------------------------------------------------
\section{Browser template classes}
% -------------------------------------------------------------------------

There are several such classes, all with the same specification as the one in\\
\lib{tree_browser.h}, named to allow easy seach for them in the code base.\\
For example, in \msr{msrElements.h}, there is:
\begin{lstlisting}[language=CPlusPlus]
//______________________________________________________________________________
template <typename T> class   msrBrowser : public browser <T>
{
  public:

    msrBrowser (basevisitor* v) : fVisitor (v) {}

    virtual               ~msrBrowser () {}

  public:

    virtual void set (basevisitor* v) { fVisitor = v; }

    virtual void browse (T& t) {
      enter (t);

      t.browseData (fVisitor);

      leave (t);
    }

  protected:

    basevisitor*  fVisitor;

    virtual void enter (T& t) { t.acceptIn  (fVisitor); }
    virtual void leave (T& t) { t.acceptOut (fVisitor); }
};
\end{lstlisting}


% -------------------------------------------------------------------------
\section{A first example: counting notes in MusicMXL data}
% -------------------------------------------------------------------------

In \libmusicxmlSamples{countnotes.cpp}, counting the notes in \mxml\ data needs only see {\tt S_note} nodes. class   {\tt countnotes} thus inherits only from a visitor for this type of node, and all the other node types are simply ignored.

\VisitorMethod{countnotes}{visitStart} only has to increment the notes count:
\begin{lstlisting}[language=CPlusPlus,caption={{\tt countnotes.cpp}}]
class   countnotes :
	public visitor<S_note>
{
	public:
		int	fCount;

		countnotes() : fCount (0)	{}

		virtual ~countnotes () {}

		void visitStart ( S_note& elt )		{ fCount++; }
};
\end{lstlisting}

% -------------------------------------------------------------------------
\section{A more complex example}
% -------------------------------------------------------------------------

Let's look at the \musicXmlMarkup{scaling} \mxml\ element:
\begin{lstlisting}[language=MusicXML]
		<scaling>
			<millimeters>7</millimeters>
			<tenths>40</tenths>
		</scaling>
\end{lstlisting}

It contains a \musicXmlMarkup{millimeter} and a \musicXmlMarkup{tenth} element. The latter two don't contain any other elements, so \methodName{visitStart} is enough for them.

There is nothing to do on the visit start upon \musicXmlMarkup{scaling}, so there is no such method.
On the visit end upon \musicXmlMarkup{scaling}, though, the values grabbed from the \musicXmlMarkup{millimeter} and \musicXmlMarkup{tenth} elements are used to create the \class{msrScaling} description.

Should a visit start method have been written, the execution order would have been:
\begin{lstlisting}[language=CPlusPlus]
  mxsr2msrTranslator::visitStart ( S_scaling& elt)
  	mxsr2msrTranslator::visitStart ( S_millimeters& elt )
		mxsr2msrTranslator::visitStart ( S_tenths& elt )
  mxsr2msrTranslator::visitEnd ( S_scaling& elt)
\end{lstlisting}

or, depending on the order in which the subelements of \musicXmlMarkup{scaling} are visited:
\begin{lstlisting}[language=CPlusPlus]
  mxsr2msrTranslator::visitStart ( S_scaling& elt)
		mxsr2msrTranslator::visitStart ( S_tenths& elt )
  	mxsr2msrTranslator::visitStart ( S_millimeters& elt )
  mxsr2msrTranslator::visitEnd ( S_scaling& elt)
\end{lstlisting}

In \mxsrToMsr{mxsr2msrTranslator.cpp}, visiting a \musicXmlMarkup{scaling} element is handled this way:
\begin{lstlisting}[language=CPlusPlus,caption={Visiting {\tt $<$scaling /$>$}}]
void mxsr2msrTranslator::visitStart ( S_millimeters& elt )
{
#ifdef MF_TRACE_IS_ENABLED
  if (gGlobalMxsr2msrOahGroup->getTraceMxsrVisitors ()) {
		std::stringstream ss;

		ss <<
      "--> Start visiting S_millimeters" <<
      ", line " << elt->getInputStartLineNumber ();

    gWaeHandler->waeTraceWithoutInputLocation (
      __FILE__, __LINE__,
      ss.str ());
  }
#endif // MF_TRACE_IS_ENABLED

  fCurrentMillimeters = (float)(*elt);
}

void mxsr2msrTranslator::visitStart ( S_tenths& elt )
{
#ifdef MF_TRACE_IS_ENABLED
  if (gGlobalMxsr2msrOahGroup->getTraceMxsrVisitors ()) {
		std::stringstream ss;

		ss <<
      "--> Start visiting S_tenths" <<
      ", line " << elt->getInputStartLineNumber ();

    gWaeHandler->waeTraceWithoutInputLocation (
      __FILE__, __LINE__,
      ss.str ());
  }
#endif // MF_TRACE_IS_ENABLED

  fCurrentTenths = (float)(*elt);
}

void mxsr2msrTranslator::visitEnd ( S_scaling& elt)
{
  int inputLineNumber =
    elt->getInputStartLineNumber ();

#ifdef MF_TRACE_IS_ENABLED
  if (gGlobalMxsr2msrOahGroup->getTraceMxsrVisitors ()) {
		std::stringstream ss;

		ss <<
      "--> End visiting S_scaling" <<
      ", line " << inputLineNumber;

    gWaeHandler->waeTraceWithoutInputLocation (
      __FILE__, __LINE__,
      ss.str ());
  }
#endif // MF_TRACE_IS_ENABLED

  // create a scaling
  S_msrScaling
    scaling =
      msrScaling::create (
        inputLineNumber,
        fCurrentMillimeters,
        fCurrentTenths);

#ifdef MF_TRACE_IS_ENABLED
  if (gTraceOahGroup->getTraceGeometry ()) {
    gLog <<
      "There are " << fCurrentTenths <<
      " tenths for " <<  fCurrentMillimeters;

    gWaeHandler->waeTraceWithoutInputLocation (
      __FILE__, __LINE__,
      ss.str ());
  }
#endif // MF_TRACE_IS_ENABLED

  // set the MSR score's scaling
  fMsrScore->
    setScaling (scaling);
}
\end{lstlisting}


% -------------------------------------------------------------------------
\section{Data browsing order}
% -------------------------------------------------------------------------

The order of the visit of a node's subnodes is programmed in \methodName{browseData} methods, such as:
\begin{lstlisting}[language=CPlusPlus,caption={{\tt msrDoubleTremolo::browseData (basevisitor* v)}}]
void msrDoubleTremolo::browseData (basevisitor* v)
{
  if (fDoubleTremoloFirstElement) {
    // browse the first element
    msrBrowser<msrElement> browser (v);
    browser.browse (*fDoubleTremoloFirstElement);
  }

  if (fDoubleTremoloSecondElement) {
    // browse the second element
    msrBrowser<msrElement> browser (v);
    browser.browse (*fDoubleTremoloSecondElement);
  }
}
\end{lstlisting}

Since this order is set in the \methodName{browsedata} methods, it cannot be influenced by the visitors of the corresponding class   instances.

There are cases where the data should be sorted prior to being browsed, such as the staves in parts: this ensures that they are browsed in this order: harmonies staff, other staves, figured bass staff.


% -------------------------------------------------------------------------
\section{Selectively inhibiting data browsing}
% -------------------------------------------------------------------------


% -------------------------------------------------------------------------
\subsection{Inhibiting data browsing in the code}
% -------------------------------------------------------------------------

In some cases, it is desirable not to browse part of the data. This is the case when a given class   contains non-normalized data, i.e. data that occurs elsewhere and will be browsed in another class   instance.

For example, \class{msrMultiMeasureRest} contains \class{msrMeasure} instances. \class{msrScore} contains:
\begin{lstlisting}[language=CPlusPlus]
    // in <multiple-rest/>, the multi-measure rests are explicit,
    // whereas LilyPond only needs the number of multi-measure rests
    Bool                  fInhibitMultiMeasureRestsBrowsing;

    void                  setInhibitMultiMeasureRestsBrowsing ()
                              {
                                fInhibitMultiMeasureRestsBrowsing = true;
                              }

    Bool                  getInhibitMultiMeasureRestsBrowsing () const
                              {
                                return
                                  fInhibitMultiMeasureRestsBrowsing;
                              };
\end{lstlisting}

\Class{lpsr2lilypondTranslator} checks this setting:
\begin{lstlisting}[language=CPlusPlus]
void lpsr2lilypondTranslator::visitEnd (S_msrNote& elt)
{
  // ... ... ...

  if (fOnGoingMultiMeasureRests) {
    switch (elt->getNoteKind ()) {
      case msrNoteKind::kNoteRestInMeasure:
        // don't handle multi-measure restss, that's done in visitEnd (S_msrMultiMeasureRest&)
          if (elt->getNoteOccupiesAFullMeasure ()) {
            Bool inhibitMultiMeasureRestsBrowsing =
              fVisitedLpsrScore->
                getMsrScore ()->
                  getInhibitMultiMeasureRestsBrowsing ();

            if (inhibitMultiMeasureRestsBrowsing) {
#ifdef MF_TRACE_IS_ENABLED
              if (
                gTraceOahGroup->getTraceNotes ()
                  ||
                gTraceOahGroup->getTraceMultiMeasureRests ()
              ) {
                gLog <<
                  "% ==> end visiting multi-measure rests is ignored" <<
                  std::endl;
              }
#endif // MF_TRACE_IS_ENABLED

#ifdef MF_TRACE_IS_ENABLED
              if (gTraceOahGroup->getTraceNotesDetails ()) {
                gLog <<
                  "% ==> returning from visitEnd (S_msrNote&)" <<
                  std::endl;
              }
#endif // MF_TRACE_IS_ENABLED

              noteIsToBeIgnored = true;
            }
          }
        break;
  // ... ... ...
}
\end{lstlisting}

Another example is in the \class{lpsr2lilypondTranslator} constructor:
\begin{lstlisting}[language=CPlusPlus]
lpsr2lilypondTranslator::lpsr2lilypondTranslator (
  S_lpsrScore     lpsrScore,
  const S_msrOahGroup&  msrOpts,
  const S_lpsrOahGroup& lpsrOpts,
  std::ostream&        lilypondCodeStream)
  : fLilypondCodeStream (
      lilypondCodeStream)
{
  fMsrOahGroup  = msrOpts;
  fLpsrOahGroup = lpsrOpts;

  // the LPSR score we're visiting
  fVisitedLpsrScore = lpsrScore;

  // inhibit the browsing of grace notes groups before,
  // since they are handled at the note level
  fVisitedLpsrScore->
    getMsrScore ()->
      setInhibitGraceNotesGroupsBeforeBrowsing ();

  // inhibit the browsing of grace notes groups after,
  // since they are handled at the note level
  fVisitedLpsrScore->
    getMsrScore ()->
      setInhibitGraceNotesGroupsAfterBrowsing ();
\end{lstlisting}

The test for browsing inhibition is done in \msr{msrNotes.cpp}:
\begin{lstlisting}[language=CPlusPlus]
void msrNote::browseData (basevisitor* v)
{
  // browse the grace notes group before note if any
  if (fGraceNotesGroupBeforeNote) {
    // fetch the score
    S_msrScore
      score =
        fetchNoteUpLinkToScore ();

    if (score) {
      Bool
        inhibitGraceNotesGroupsBeforeBrowsing =
          score->
            getInhibitGraceNotesGroupsBeforeBrowsing ();

      if (inhibitGraceNotesGroupsBeforeBrowsing) {
#ifdef MF_TRACE_IS_ENABLED
        if (
          gMsrOahGroup->getTraceMsrVisitors ()
            ||
          gTraceOahGroup->getTraceNotes ()
            ||
          gTraceOahGroup->getTraceGraceNotes ()
        ) {
          std::stringstream ss;

          ss <<
            "% ==> visiting grace notes groups before notes is inhibited";

          gWaeHandler->waeTraceWithoutInputLocation (
            __FILE__, __LINE__,
            ss.str ());
        }
#endif // MF_TRACE_IS_ENABLED
      }
      else {
        // browse the grace notes group before note
        msrBrowser<msrGraceNotesGroup> browser (v);
        browser.browse (*fGraceNotesGroupBeforeNote);
      }
    }
  }

  // ... ... ...
}
\end{lstlisting}


% -------------------------------------------------------------------------
\subsection{Inhibiting data browsing by options}
% -------------------------------------------------------------------------

Choosing which elements to browse can be more selective:
\begin{lstlisting}[language=Terminal]
void msrStaff::browseData (basevisitor* v)
{
  // ... ... ...

  if (fStaffAllVoicesList.size ()) {
    for (const S_msrVoice& voice : fStaffAllVoicesList) {
      // is this voice name in the ignore voices set?
      Bool ignoreVoice (false);

      std::string voiceName =
        voice->
          getVoiceName ();

      const std::set <std::string>&
        ignoreMsrVoicesSet =
          gGlobalMsr2msrOahGroup->
            getIgnoreMsrVoicesSet ();

  		// ... ... ...

      if (ignoreMsrVoicesSet.size ()) {
        ignoreVoice =
          mfStringIsInStringSet (
            voiceName,
            ignoreMsrVoicesSet);
      }

      if (ignoreVoice) {
#ifdef MF_TRACE_IS_ENABLED // JMI
        if (gTraceOahGroup->getTraceVoices ()) {
          std::stringstream ss;

          ss <<
            "Ignoring voice \"" <<
            voiceName <<
            "\"";

          gWaeHandler->waeTraceWithoutInputLocation (
            __FILE__, __LINE__,
            ss.str ());
        }
#endif // MF_TRACE_IS_ENABLED
      }

      else {
        msrBrowser<msrVoice> browser (v);
        browser.browse (*voice);
      }
    } // for
  }

  // ... ... ...
}
\end{lstlisting}

% -------------------------------------------------------------------------
\section{Adapting visitors to data browsing order with booleans}
% -------------------------------------------------------------------------

A frequent situation is when the visitor's actions should be tuned depending upon which elements are being visited. In simple case, this can be handled with boolean variables.

For example, \musicXmlMarkup{system-layout} may occur both in the \musicXmlMarkup{defaults} and \musicXmlMarkup{print} \mxml\ markups:
\begin{lstlisting}[language=MusicXML]
	<defaults>
		<scaling>
			<millimeters>7.3</millimeters>
			<tenths>40</tenths>
		</scaling>
		<page-layout>
			<page-height>1534</page-height>
			<page-width>1151</page-width>
			<page-margins type="both">
				<left-margin>54.7945</left-margin>
				<right-margin>54.7945</right-margin>
				<top-margin>27.3973</top-margin>
				<bottom-margin>27.3973</bottom-margin>
			</page-margins>
		</page-layout>
		<system-layout>
			<system-margins>
				<left-margin>15</left-margin>
				<right-margin>0</right-margin>
			</system-margins>
			<system-distance>92.5</system-distance>
			<top-system-distance>27.5</top-system-distance>
		</system-layout>

  // ... ... ...

	<part id="P1">
		<measure number="1">
			<print>
				<system-layout>
					<system-margins>
						<left-margin>75.625</left-margin>
						<right-margin>0</right-margin>
					</system-margins>
					<top-system-distance>410.9375</top-system-distance>
				</system-layout>
				<staff-layout>
					<?DoletSibelius JustifyAllStaves=false?>
					<?DoletSibelius ExtraSpacesAbove=3?>
				</staff-layout>
				<measure-layout>
					<measure-distance>20</measure-distance>
				</measure-layout>
			</print>
\end{lstlisting}

To know which element is being visited, we use boolean {\tt fOnGoing*} variables, such as {\tt fOnGoingPrintLayout} in \class{msr2mxsrTranslator}.

It is assigned in:
\begin{lstlisting}[language=CPlusPlus]
void msr2mxsrTranslator::visitStart (S_msrPrintLayout& elt)
{
 // ... ... ...

 fOnGoingPrintLayout = true;
}

void msr2mxsrTranslator::visitEnd (S_msrPrintLayout& elt)
{
  // ... ... ...

  fOnGoingPrintLayout = false;
}
\end{lstlisting}

and checked for example in:
\begin{lstlisting}[language=CPlusPlus]
void msr2mxsrTranslator::visitStart (S_msrSystemLayout& elt)
{
  // ... ... ...

  // create a system layout element
  Sxmlelement
    systemLayoutElement =
      createMxmlElement (k_system_layout, "");

  if (fOnGoingPrintLayout) {
    // append it to the current print element
    fCurrentPrintElement->push (
      systemLayoutElement);
  }
  else {
    // don't append it at once to the score defaults element
    fScoreDefaultsSystemLayoutElement = systemLayoutElement;
  }
\end{lstlisting}


When the data browsing order does not fit the needs of a visitor, the latter has to store the values gathered until they can be processed. This occurs for exemple in {\tt mxsr2msrTranslator}, which uses {\tt fCurrentPrintLayout} for this purpose:
\begin{lstlisting}[language=CPlusPlus]
void mxsr2msrTranslator::visitStart ( S_system_layout& elt )
{
  // ... ... ...

  // create the system layout
  fCurrentSystemLayout =
    msrSystemLayout::create (
      inputLineNumber);

  fOnGoingSystemLayout = true;
}

void mxsr2msrTranslator::visitEnd ( S_system_layout& elt )
{
  // ... ... ...

  if (fOnGoingPrint) {
    // set the current print layout's system layout
    fCurrentPrintLayout->
      setSystemLayout (
        fCurrentSystemLayout);
  }
  else {
    // set the MSR score system layout
    fMsrScore->
      setSystemLayout (
        fCurrentSystemLayout);
  }

  // forget about the current system layout
  fCurrentSystemLayout = nullptr;

  fOnGoingSystemLayout = false;
}
\end{lstlisting}


% -------------------------------------------------------------------------
\section{Adapting visitors to data browsing order with stacks}
% -------------------------------------------------------------------------

In more complex cases, the visiting order leads to have several on-going elements simultaneously. This is the case with \class{msrTuplet}, which can be nested.

They are handled in \mxsrToMsr{mxsr2msrTranslator} and \lpsrToLilypondBoth{lpsr2lilypondTranslator}, for example, using a stack to keep track of them.

\mf\ never uses C++ STL stacks, because they cannot be iterated over:
\begin{lstlisting}[language=CPlusPlus]
    std::list <S_msrTuplet>    fOnGoingTupletsStack;
\end{lstlisting}

\begin{lstlisting}[language=CPlusPlus]
void lpsr2lilypondTranslator::visitStart (S_msrTuplet& elt)
{
	// ... ... ...

  if (fOnGoingTupletsStack.size ()) {
    // elt is a nested tuplet

    const S_msrTuplet&
      containingTuplet =
        fOnGoingTupletsStack.front ();

    // unapply containing tuplet factor,
    // i.e 3/2 inside 5/4 becomes 15/8 in MusicXML...
    elt->
      unapplySoundingFactorToTupletMembers (
        containingTuplet->
          getTupletFactor ());
  }

	// ... ... ...

  // push the tuplet on the tuplets stack
  fOnGoingTupletsStack (elt);

	// ... ... ...
}

void lpsr2lilypondTranslator::visitEnd (S_msrTuplet& elt)
{
	// ... ... ...

  // pop the tuplet from the tuplets stack
  fOnGoingTupletsStack ();

	// ... ... ...
}
\end{lstlisting}


% -------------------------------------------------------------------------
\section{Avoiding the visiting pattern by cascading}\index[Main]{cascading}
% -------------------------------------------------------------------------

There are cases where we need a deterministic traversal of some data handled by \mf. For example, appending a {\tt msrStaffDetails} instance to a part should be \cascaded\ to its staves. It would be an overkill to create a specific browser for this purpose.

This is what \method{msrPart}{appendStaffDetailsToPart} does:
\begin{lstlisting}[language=CPlusPlus]
void msrPart::appendStaffDetailsToPart (
  const S_msrStaffDetails& staffDetails)
{
	// ... ... ...

  // register staff details in part
  fCurrentPartStaffDetails = staffDetails;

  // append staff details to registered staves
  for (
    std::map <int, S_msrStaff>::const_iterator i =
      getPartStavesMap.begin ();
    i != getPartStavesMap.end ();
    ++i
  ) {
    S_msrStaff
      staff = (*i).second;

    staff->
      appendStaffDetailsToStaff (
        staffDetails);
  } // for
}
\end{lstlisting}

Another case is the handling the various elements attached to an \class{msrNote} instance, among them chords, grace notes groups and tuplet, all of which contain notes too. \\
Doing things in the right order can be tricky, see \lpsrToLilypondBoth{lpsr2lilypondTranslator}.

The time-oriented representation of scores in \msrRepr\ is also printed by \cascading through \methodName{printSlices} methods, see \chapterRef{MSR time-oriented represention}.
