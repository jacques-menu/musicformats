% !TEX root = MusicFormatsMaintainanceGuide.tex

% -------------------------------------------------------------------------
\chapter{MSR time-oriented represention}\label{MSR time-oriented represention}
% -------------------------------------------------------------------------

In order to represent the music according to simultaneous sounding time, MSR builds:
\begin{itemize}
\item a flat list of measures at the voice and staff levels;
\item from this, a vector of measures slices at the voice, staff, part, part group and score levels.
\end{itemize}

The source files are in \msrBoth{msrMeasuresSlices}.


% -------------------------------------------------------------------------
\section{Note events}
% -------------------------------------------------------------------------

Notes start and stop are represented by \enumType{msrNoteEventKind}:
\begin{lstlisting}[language=CPlusPlus]
enum class msrNoteEventKind {
  kNoteEventStart,
  kNoteEventStop
};
\end{lstlisting}

A note event is described in class   {\tt }:
\begin{lstlisting}[language=CPlusPlus]
class   msrNoteEvent : public smartable
{
	// ... ... ...

  private:

    // private fields
    // ------------------------------------------------------

    Rational              fNoteEventMeasurePosition;
    S_msrNote             fNoteEventNote;
    msrNoteEventKind      fNoteEventKind;
};
\end{lstlisting}


% -------------------------------------------------------------------------
\section{Simultaneous notes chunks}
% -------------------------------------------------------------------------

Such a chunk is a set of notes or rests played simultaneously, i.e. that start and stop at the same time. The set is stored as a list actually:
\begin{lstlisting}[language=CPlusPlus]
class   msrSimultaneousNotesChunk : public smartable
{
	// ... ... ...

  private:

    // private fields
    // ------------------------------------------------------

    Rational              fChunkMeasurePosition;
    std::list<S_msrNote>  fChunkNotesList;
    Rational              fChunkDurationWholeNotes;
};
\end{lstlisting}


% -------------------------------------------------------------------------
\section{Measures slices}
% -------------------------------------------------------------------------

A measures slice, described by \class{msrMeasuresSlice}, is a 'vertical'\index{vertical!vertical cut} cut in the score across voices: is contains all the measures starting at the same time, one per voice:
\begin{lstlisting}[language=CPlusPlus]
class EXP msrMeasuresSlice : public smartable
{
	// ... ... ...

  protected:

    // protected fields
    // ------------------------------------------------------

    int                   fSlicePuristMeasureNumber;
    std::string           fSliceMeasureNumber;

    // the measures in the slice
    std::vector<S_msrMeasure>  fSliceMeasuresVector;

    // notes flat list
    std::list<S_msrNote>  fSliceNotesFlatList;

    // note events list
    std::list<S_msrNoteEvent>
                          fSliceNoteEventsList;

    // simultaneous notes chunks list
    std::list<S_msrSimultaneousNotesChunk>
                          fSliceSimultaneousNotesChunksList;
};

\end{lstlisting}


% -------------------------------------------------------------------------
\section{Measures slices sequences}
% -------------------------------------------------------------------------

A \class{msrMeasuresSlicesSequence} contains a vector of {\tt S_msrMeasuresSlice} instances:
\begin{lstlisting}[language=CPlusPlus]
class EXP msrMeasuresSlicesSequence : public smartable
{
	// ... ... ...

  private:

    // private fields
    // ------------------------------------------------------

    std::string           fMeasuresOrigin;

    std::vector<S_msrMeasuresSlice>
                          fMeasuresSlicesVector;
};
\end{lstlisting}

A \smart\ to am {\tt msrMeasuresSlicesSequence} instance is stored in {\tt msrVoice}, {\tt msrStaff}, {\tt msrPart}, {\tt msrPartGroup} and {\tt msrScore}.


% -------------------------------------------------------------------------
\section{Building the measures slices}
% -------------------------------------------------------------------------

% -------------------------------------------------------------------------
\subsection{Part measures slices}
% -------------------------------------------------------------------------

At the part level, this is done in \method{msrPart}{collectPartMeasuresSlices}:
\begin{lstlisting}[language=CPlusPlus]
void msrPart::collectPartMeasuresSlices (
  int inputLineNumber)
{
	// ... ... ...

  // create the part measures slices sequence
  fPartMeasuresSlicesSequence =
    msrMeasuresSlicesSequence::create (
      fPartName); // origin

  // populate it
  for (S_msrStaff staff : fPartAllStavesList) {
		// ... ... ...

    ++gIndenter;

    S_msrMeasuresSlicesSequence
      staffMeasuresSlicesSequence =
        staff->
          getStaffMeasuresSlicesSequence ();

    if (! staffMeasuresSlicesSequence) {
      std::stringstream ss;

      ss <<
        "The staffMeasuresSlicesSequence of staff \"" <<
        staff->getStaffName () <<
        "\" is null";

      musicxmlWarning (
        gGlobalCurrentServiceRunData->getInputSourceName (),
        inputLineNumber,
        ss.str ());
    }
    else {
      fPartMeasuresSlicesSequence->
        mergeWithMeasuresSlicesSequence (
          inputLineNumber,
          getPartCombinedName (),
          staffMeasuresSlicesSequence);
    }

    --gIndenter;
  } // for

	// ... ... ...
}
\end{lstlisting}


% -------------------------------------------------------------------------
\subsection{Staff measures slices}
% -------------------------------------------------------------------------

\Method{msrStaff}{collectStaffMeasuresSlices} builds them:
\begin{lstlisting}[language=CPlusPlus]
void msrStaff::collectStaffMeasuresSlices (
  int inputLineNumber)
{
	// ... ... ...

  // create the staff measures slices sequence
  fStaffMeasuresSlicesSequence =
    msrMeasuresSlicesSequence::create (
      fStaffName); // origin

  // populate it
  for (const S_msrVoice& voice : fStaffAllVoicesList) {
		// ... ... ...

    // get the voice measures slices sequence
    S_msrMeasuresSlicesSequence
      voiceMeasuresSlicesSequence =
        voice->
          getVoiceMeasuresSlicesSequence ();

    // merge it with the voice measures slices sequence
    if (voiceMeasuresSlicesSequence) { // JMI
      fStaffMeasuresSlicesSequence =
        fStaffMeasuresSlicesSequence->
          mergeWithMeasuresSlicesSequence (
            inputLineNumber,
            fStaffName,
            voiceMeasuresSlicesSequence);
    }

    // identify the solo notes and rests in the staff
    fStaffMeasuresSlicesSequence->
      identifySoloNotesAndRests ();

    --gIndenter;
  } // for

	// ... ... ...
}
\end{lstlisting}


% -------------------------------------------------------------------------
\section{Solo notes and rests}
% -------------------------------------------------------------------------

A solo note or rest is one that occurs alone at some point in time for its whole duration, without any other note being played at the same time.

Identifying such solo notes or rests is done in \method{msrMeasuresSlicesSequence}{identifySoloNotesAndRests} using the measures slices of the staff they occur in, called \method{msrStaff}{collectStaffMeasuresSlices} as shown above:
\begin{lstlisting}[language=CPlusPlus]
void msrMeasuresSlicesSequence::identifySoloNotesAndRests ()
{
	// ... ... ...

  // collect the notes from the sequence's measures slices
  for (
    std::vector<S_msrMeasuresSlice>::const_iterator i =
      fMeasuresSlicesVector.begin ();
    i != fMeasuresSlicesVector.end ();
    ++i
  ) {
    S_msrMeasuresSlice measuresSlice = (*i);

    measuresSlice->
      collectNonSkipNotesFromMeasuresSliceMeasures ();
  } // for
}
\end{lstlisting}


% -------------------------------------------------------------------------
\section{A measures slices example}
% -------------------------------------------------------------------------


