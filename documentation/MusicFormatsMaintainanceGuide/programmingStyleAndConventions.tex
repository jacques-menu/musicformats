% -------------------------------------------------------------------------
%  MusicFormats Library
%  Copyright (C) Jacques Menu 2016-2023

%  This Source Code Form is subject to the terms of the Mozilla Public
%  License, v. 2.0. If a copy of the MPL was not distributed with this
%  file, you can obtain one at http://mozilla.org/MPL/2.0/.

%  https://github.com/jacques-menu/musicformats
% -------------------------------------------------------------------------

% !TEX root = MusicFormatsMaintainanceGuide.tex

% -------------------------------------------------------------------------
\chapter{Programming style and conventions}
% -------------------------------------------------------------------------


% -------------------------------------------------------------------------
\section{Files naming conventions}
% -------------------------------------------------------------------------

Most file names start with an identification of the component they belong to, such as '\fileName{\tt oah}', '\fileName{mxsr}', '\fileName{msr}', '\fileName{lpsr}', '\fileName{lilypond}', '\fileName{bsr}', '\fileName{braille}', '\fileName{xml2ly}', '\fileName{xml2brl}' and \fileName{msdl}.\\
The ancillary files such as \mfutilitiesBoth{mfIndentedTextOutput} follow this rule too, with an \code{mf} prefix.

The '{\tt *Oah.*}' files handle the options and help for the corresponding component, such as\\ '\msrToMsrBoth{msr2msrOah}'.

The '\oahBoth{\tt traceOah}', \oahBoth{musicxmlOah}\, '{\tt extra}' and '{\tt general}' prefixes are about the corresponding help groups.

There are a couple of 'globlal' files not related to any particular component, placed in \mfutilities{} with an \code{mf} name prefix:
\begin{lstlisting}[language=Terminal]
jacquesmenu@macmini: ~/musicformats-git-dev/src > ll mfutilities/
total 200
 0 drwxr-xr-x  15 jacquesmenu  staff    480 Oct 22 06:25:57 2021 ./
 0 drwxr-xr-x  19 jacquesmenu  staff    608 Oct 22 05:29:29 2021 ../
 8 -rw-r--r--@  1 jacquesmenu  staff   3255 Oct 18 20:22:16 2021 mfBool.cpp
16 -rw-r--r--@  1 jacquesmenu  staff   4917 Oct 18 19:56:51 2021 mfBool.h
 8 -rw-r--r--@  1 jacquesmenu  staff   1336 Oct 15 18:48:10 2021 mfEnumAll.h
16 -rw-r--r--@  1 jacquesmenu  staff   7182 Nov  8 13:08:51 2021 mfIndentedTextOutput.cpp
16 -rw-r--r--@  1 jacquesmenu  staff   7715 Nov  8 13:08:40 2021 mfIndentedTextOutput.h
 8 -rw-r--r--@  1 jacquesmenu  staff    889 Oct 15 20:34:47 2021 mfMusicformatsErrorKind.cpp
 8 -rw-r--r--@  1 jacquesmenu  staff    629 Oct 15 20:34:47 2021 mfMusicformatsErrors.h
 8 -rw-r--r--@  1 jacquesmenu  staff   2541 Nov  5 11:29:25 2021 oahOptionsVector.cpp
 8 -rw-r--r--@  1 jacquesmenu  staff    972 Oct 15 20:16:51 2021 oahBasicTypes.h
64 -rw-r--r--@  1 jacquesmenu  staff  29773 Oct 15 18:48:10 2021 mfStringsHandling.cpp
16 -rw-r--r--@  1 jacquesmenu  staff   6269 Oct 15 18:55:46 2021 mfStringsHandling.h
16 -rw-r--r--@  1 jacquesmenu  staff   5028 Oct  7 20:03:27 2021 mfTiming.cpp
 8 -rw-r--r--@  1 jacquesmenu  staff   3726 Oct  8 08:21:09 2021 mfTiming.h
\end{lstlisting}

The files {\tt *Elements.h/.cpp} contain base classes to variants, such as \lpsrBoth{lpsrElements}, whose {\tt lpsrElement} class   is used in a number of other files:
\begin{lstlisting}[language=Terminal]
jacquesmenu@macmini: ~/musicformats-git-dev/src > grep -r 'public lpsrElement' *
formats/lpsr/lpsrStaves.h:29:class EXP lpsrNewStaffgroupBlock : public lpsrElement
formats/lpsr/lpsrStaves.h:87:class EXP lpsrNewStaffTuningBlock : public lpsrElement
formats/lpsr/lpsrStaves.h:150:class EXP lpsrNewStaffBlock : public lpsrElement
formats/lpsr/lpsrStaves.h:208:class EXP lpsrStaffBlock : public lpsrElement
formats/lpsr/lpsrVariables.h:27:class EXP lpsrVariableUseCommand : public lpsrElement
formats/lpsr/lpsrScores.h:35:class EXP lpsrScore : public lpsrElement
formats/lpsr/lpsrBarNumbers.h:26:class EXP lpsrBarNumberCheck : public lpsrElement
formats/lpsr/lpsrBarNumbers.h:85:class EXP lpsrBarCommand : public lpsrElement
formats/lpsr/lpsrLyrics.h:31:class EXP lpsrNewLyricsBlock : public lpsrElement
formats/lpsr/lpsrComments.h:25:class EXP lpsrComment : public lpsrElement
formats/lpsr/lpsrVoices.h:29:class EXP lpsrUseVoiceCommand : public lpsrElement
formats/lpsr/lpsrParts.h:27:class EXP lpsrPartBlock : public lpsrElement
formats/lpsr/lpsrPapers.h:32:class EXP lpsrPaper : public lpsrElement
formats/lpsr/lpsrPartGroups.h:29:class EXP lpsrPartGroupBlock : public lpsrElement
formats/lpsr/lpsrParallelMusic.h:28:class EXP lpsrParallelMusicBLock : public lpsrElement
formats/lpsr/lpsrLayouts.h:23:class EXP lpsrLayout : public lpsrElement
formats/lpsr/lpsrHeaders.h:27:class EXP lpsrHeader : public lpsrElement
formats/lpsr/lpsrScheme.h:29:class EXP lpsrSchemeVariable : public lpsrElement
formats/lpsr/lpsrScheme.h:140:class EXP lpsrSchemeFunction : public lpsrElement
formats/lpsr/lpsrBookBlockElements.h:35:class EXP lpsrBookBlockElement : public lpsrElement
formats/lpsr/lpsrBookBlockElements.h:237:class EXP lpsrBookBlock : public lpsrElement
formats/lpsr/lpsrContexts.h:30:class EXP lpsrContext : public lpsrElement
\end{lstlisting}

\newpage

There are a number of self-explaning {\tt *BasicTypes.h/.cpp} file names:%%%JMI
\begin{lstlisting}[language=Terminal]
./formats/msdl/msdlEnumTypes.h
./formats/msdl/msdlEnumTypes.cpp
./formats/msr/msrBasicTypes.cpp
./formats/msr/msrBasicTypes.h
./formats/lpsr/lpsrEnumTypes.cpp
./formats/lpsr/lpsrEnumTypes.h
./formats/bsr/bsrEnumTypes.h
./formats/bsr/bsrEnumTypes.cpp
./oah/oahBasicTypes.h
./oah/oahBasicTypes.cpp
./formatsgeneration/msrGeneration/msrGenerationBasicTypes.cpp
./formatsgeneration/msrGeneration/msrGenerationBasicTypes.h
\end{lstlisting}

The files are grouped in the {\tt src} folder according to the component they belong to:
\begin{itemize}
\item {\tt converters}
\item {\tt generators}
\item {\tt interfaces}
\item {\tt oah}
\item {\tt formatsgeneration}
\item {\tt passes}
\item {\tt formats}
\item {\tt utilities}
\item {\tt wae}
\end{itemize}


% -------------------------------------------------------------------------
\section{Adding C++ files}
% -------------------------------------------------------------------------

Building \mf\ relies on {\tt build/CMakeLists.txt} to find the C++ files that should be compiled.

When building \mf\ with:
\begin{lstlisting}[language=Terminal]
cd build
make
\end{lstlisting}
a \cmake\ \cache\ is created in file {\tt build/libdir/CMakeCache.txt}, containing the list of all the C++ files in the library, including those of the embedded \libmusicxml.

Adding individual files in fine, but adding new folders in the {\tt src} hierarchy implies to update variable {\tt SRC_FOLDERS} in {\tt build/CMakeLists.txt} accordingly and to remove the {\tt build/libdir/CMakeCache.txt} \cache.

Care must be taken when adding a new file on a case insensitive file system: the type case in its name should be what is needed in the first place.

For example, renaming \msr{msrBarlines.h} to \msr{msrBarLines.h} (this author's experience) leads \mf\ not to build successfully on Linux if you develop on Windows or MacOS:
\begin{itemize}
\item the latter two usually use case insensitive file names (even though one may choose to format as disk to be case-sensitive), but Linux does not;
\item the renaming above is not pushed to the repository by {\tt git push} on case insensitive file systems.
\end{itemize}

The best solution here, both for files and folders names, is to use '{\tt git mv}' to do the renaming instead of the \OS\ tools:
\begin{itemize}
\item {\tt git mv oldName newName}
\item {\tt git commit "..." -a}
\item {\tt git push}
\end{itemize}


% -------------------------------------------------------------------------
\section{Renaming C++ files}
% -------------------------------------------------------------------------

Renaming a C++ file causes {\tt build/libdir/CMakeCache.txt} to be obsolete: it then has to be removed, and the library should be built anew.

For example, this author uses the \code{rmcache} bash alias to remove the \cache:
\begin{lstlisting}[language=Terminal]
jacquesmenu@macmini: ~/musicformats-git-dev > type rmcache
rmcache is aliased to `rm /Users/jacquesmenu/musicformats-git-dev/build/libdir/CMakeCache.txt'
\end{lstlisting}

Running \make\ will re-create this \cache\ with the new file name.

Caution has to be taken when a file name case is changed in a case-insensitive development environment such as Windows or MacOS. Cloning \mf\ in Linux will then fail to find the file under its new name.

In such a case, the following \Main{Git} command has to be used to actually change the file name in \mf\ repository:
\begin{lstlisting}[language=Terminal]
git mv -f <old name> <new name>
\end{lstlisting}

Changing the name of a directory in \src\ should be propagated to \build{CMakeLists.txt}, since this is where the set of files to be compiled is determined, as in:
\begin{lstlisting}[language=Terminal]
if (MSR)
	set (SRC_FOLDERS ${SRC_FOLDERS} passes/musicxml2mxsr formats/mxsr passes/mxsr2msr formats/msr passes/msr2msr formatsgeneration/multiGeneration formatsgeneration/msrGeneration generators/Mikrokosmos3Wandering generators/LilyPondIssue34)
endif()
\end{lstlisting}
In such a case, the \cache\ should be removed before building, see \code{rmcache} above.


% -------------------------------------------------------------------------
\section{Source code layout}
% -------------------------------------------------------------------------

The following text-editing conventions are used:
\begin{itemize}
\item tabs are not used before the first non-space character in a line, two spaces are used instead;

\item the code is not tightly packed: declarations in classes have the members' names aligned vertically, with many spaces before them if needed, and empty lines are used to separate successive activities in methods.
\end{itemize}


% -------------------------------------------------------------------------
\section{Defensive programming}
% -------------------------------------------------------------------------

The code base of \xmlToLy\ is \MainIt{defensive programming} oriented, which means that:
\begin{itemize}
\item identifiers are explicit and long if needed -- only very local ones are short, such as iteration loops indexes;

\item the code is organized in sections, with an initial comment documenting what the code does;

\item the \CPlusplus 's {\tt auto} declaration feature is used only for \enumType s, see below. Writing the explicit types in a large code base helps the maintainer mastering the code;

\item \function{mfAssert} is used to perform sanity checks, such as detect a null pointer prior to using it.
\end{itemize}

The few uses of {\tt auto} declarations are in range-based {\tt for} loops over \enumType s. There the type of the index is explicit from the {\tt Enum*} being enumerated over. For example, in \msr{msrBasicTypes.cpp}:
\begin{lstlisting}[language=CPlusPlus]
    for (auto e : mfEnumAll<msrHarmonyKind> ()) {
      // create the harmony structure
      S_msrHarmonyStructure
        harmonyStructure =
          msrHarmonyStructure::create (
            e);

      // register it in the map
      gGlobalHarmonyStructuresMap [e] =
        harmonyStructure;
    } // for
\end{lstlisting}

\newpage

\Class{mfEnumAll} is defined in \mfutilities{mfEnumAll.h} as:
\begin{lstlisting}[language=CPlusPlus]
template< typename T >
class   mfEnumAll
{
  public:

    class   Iterator
    {
      public:

        Iterator (int value)
          : fIterationIndex (value)
            {}

        T operator* (void) const
            { return (T) fIterationIndex; }

        void operator++ (void)
            {  ++fIterationIndex; }

        Bool operator != (Iterator rhs)
            {  return fIterationIndex != rhs.fIterationIndex; }

      private:

        int fIterationIndex;
     };
};
\end{lstlisting}


% -------------------------------------------------------------------------
\section{Sanity checks}
% -------------------------------------------------------------------------

They are performed to ensure that the formats in \mf\ are consistent, to avoid ugly crashes. An example is:
\begin{lstlisting}[language=CPlusPlus]
    // get voice to insert harmonies into
    S_msrVoice
      voiceToInsertHarmoniesInto =
        fCurrentPart->
          getPartHarmoniesVoice ();

#ifdef MF_SANITY_CHECKS_ARE_ENABLED
  	// sanity check
		mfAssert (
			__FILE__, __LINE__,
			voiceToInsertHarmoniesInto != nullptr,
			"voiceToInsertHarmoniesInto is null");
#endif // MF_SANITY_CHECKS_ARE_ENABLED
\end{lstlisting}


% -------------------------------------------------------------------------
\section{{\tt JMI} comments}
% -------------------------------------------------------------------------

Comments containg {\tt JMI} indicates that the code may have to be reconsidered in the future, should a problem arise. They are removed when it becomes obvious that the code is fine. JMI was the acronym for the author's activity as a software contractor long time ago.


% -------------------------------------------------------------------------
\section{Exported symbols}
% -------------------------------------------------------------------------

The classes and functions\index{functions} that need to be exported from the \mf\ library in the Windows meaning are marked as such with an {\tt EXP} specification:
\begin{lstlisting}[language=CPlusPlus]
class EXP smartable {
	// ... ... ...
};
\end{lstlisting}

\begin{lstlisting}[language=CPlusPlus]
EXP S_mxsrOahGroup createGlobalMxsrOahGroup ();
\end{lstlisting}

\code{EXP} is defined in \interface{exports.h} and is non-blank only when building on \Windows:
\begin{lstlisting}[language=CPlusPlus]
#if defined(WIN32) // && !defined (GCC)

# ifdef MSVC
#  pragma warning (disable : 4267)
#  pragma warning (disable : 4275)
#  pragma warning (disable : 4251)
#  pragma warning (disable : 4786)
#  pragma warning (disable : 4251)
#  pragma warning (disable : 4275)
# endif

# ifdef LIBMUSICXML_EXPORTS
#  define EXP __declspec(dllexport)

# elif defined(LIBMUSICXML_STATIC)
#  define EXP

# else
#  define EXP __declspec(dllimport)
# endif

#else

# ifdef LIBMUSICXML_EXPORTS
#  define EXP	__attribute__ ((visibility("default")))
# else
#  define EXP
# endif

#endif
\end{lstlisting}


% -------------------------------------------------------------------------
\section{Smart pointers}
% -------------------------------------------------------------------------

\libmusicxml\ provides what \fober\ named smart pointers, because:
\begin{itemize}
\item a smart pointer is an instance of a class   that contains the actual pointer in the usual C++ sense;
\item the actual pointer is guaranteed to be initialized to {\tt nullptr};
\item garbage collection is implicit, using reference counts.
\end{itemize}
The definitions are in \lib{smartpointer.h}.

The reference counting is done in class   {\tt smartable}:
\begin{lstlisting}[language=CPlusPlus]
class EXP smartable {
	private:
		unsigned 	refCount;
	public:
		//! gives the reference count of the object
		unsigned refs() const         { return refCount; }
		//! addReference increments the ref count and checks for refCount overflow
		void addReference()           { refCount++; assert(refCount != 0); }
		//! removeReference delete the object when refCount is zero
		void removeReference()		  { if (--refCount == 0) delete this; }

	protected:
		smartable() : refCount(0) {}
		smartable(const smartable&): refCount(0) {}
		//! destructor checks for non-zero refCount
		virtual ~smartable()    { assert (refCount == 0); }
		smartable& operator=(const smartable&) { return *this; }
};

A smart pointer type is created with template class   {\tt SMARTP}, for example:
template<class   T> class   SMARTP {
class EXP msrElement : public smartable
{
	// ... ... ...
};
typedef SMARTP<msrElement> S_msrElement;

\end{lstlisting}

Smart pointer type name belonging to \fober's work can be told from those of \mf\ by their prefix:
\begin{itemize}
\item  in \libmusicxml, smart pointer type names start with an '{\tt S}', such as {\tt \$xmlelement};
\item in \mf, they start with '{\tt S_}', such as {\tt S_oahHandler}.
\end{itemize}

Inheriting from class   {\tt smartable} is used to create smart pointer types, as in \wae{waeHandlers.h}:
\begin{lstlisting}[language=CPlusPlus]
class EXP waeHandler : public smartable
{
  public:

    // creation
    // ------------------------------------------------------

    static SMARTP<waeHandler> create ();

  public:

    // constructors/destructor
    // ------------------------------------------------------

                          waeHandler ();

    virtual               ~waeHandler ();

  public:

    // set and get
    // ------------------------------------------------------

  public:

    // public services
    // ------------------------------------------------------

  public:

    // print
    // ------------------------------------------------------

    std::string           asString () const;

    void                  print (std::ostream& os) const;

  private:

    // private fields
    // ------------------------------------------------------
};
typedef SMARTP<waeHandler> S_waeHandler;
EXP std::ostream& operator << (std::ostream& os, const S_waeHandler& elt);
\end{lstlisting}

The creation of the instances in \wae{waeHandlers.cpp} is done with:
\begin{lstlisting}[language=CPlusPlus]
S_waeHandler waeHandler::create ()
{
  waeHandler* obj =
    new waeHandler ();
  assert (obj != nullptr);
  return obj;
}
\end{lstlisting}


% -------------------------------------------------------------------------
\section{Files contents layout}
% -------------------------------------------------------------------------

Indentation is done by two spaces, avoiding TAB characters.

In {\tt *.h} files, the classes declarations contain all of part of the following:
\begin{itemize}
\item public data types, usually \enumType s, if any;
\item public static class   \methodName{create*} methods, except for pure virtual classes, in which case they are commented out;
\item constructors ad destructor;
\item public \methodName{set*} and \methodName{get*} methods;
\item public services if any;
\item public visiting methods, i.e. \methodName{acceptIn}, \methodName{acceptOut} and \methodName{browseData}, if the class   contains browsable data such as STL lists, vectors, maps and sets;
\item public print methods, such as \methodName{asString} and \methodName{print};
\item private methods if any;
\item private fields.
\item private work methods if any;%%%JMI
\item private work fields.
\end{itemize}

A work method is used internally by the class, while a work field is one that evolves as the class   contents is populated.

Most class   declarations are followed by a smart pointer type and a {\tt operator <<}, such as:
\begin{lstlisting}[language=CPlusPlus]
typedef SMARTP<msrHarmonyDegree> S_msrHarmonyDegree;
EXP std::ostream& operator << (std::ostream& os, const S_msrHarmonyDegree& elt);
\end{lstlisting}

The same order for constructors, destructor and methods is followed in most {\tt .cpp} files.


% -------------------------------------------------------------------------
\section{{\tt \#define DEBUG*} code sections}
% -------------------------------------------------------------------------

Some sections of code in \starFileName{.cpp} are controlled by such definitions:
\begin{itemize}
\item \constantName{//\#define DEBUG_EARLY_OPTIONS} in \oah{oahEarlyOptions.cpp};
\item \constantName{//\#define DEBUG_INDENTER} in \oah{mfIndentedTextOutput.cpp};
\item \constantName{//\#define DEBUG_SPLITTING} in \mfutilities{mfStringsHandling.cpp}
\end{itemize}

These can be uncommented to obtain development-time trace information, without there being a need for such in \mf\ library regular use.


% -------------------------------------------------------------------------
\section{Identifiers choice conventions}
% -------------------------------------------------------------------------

The following rules apply:
\begin{itemize}

\item all enumeration type names describing variants in classes end in '{\tt Kind}';%%%JMI
\item all enumeration constants start with '{\tt k}'. common prefixes are used to help locate all occurences of constants of the given type in a text editor, as:
\begin{lstlisting}[language=CPlusPlus]
enum class msrBassFigurePrefixKind {
  kBassFigurePrefix_UNKNOWN_,
  kBassFigurePrefixDoubleFlat, kBassFigurePrefixFlat, kBassFigurePrefixFlatFlat,
  kBassFigurePrefixNatural,
  kBassFigurePrefixSharpSharp, kBassFigurePrefixSharp, kBassFigurePrefixDoubleSharp
};
\end{lstlisting}

\item all classes names have a prefix indicating which part of \mf\ there belong to, such as \class{msrTimeSignature}, {\tt oahAtomStoringAValue} and {\tt msdlKeywordsLanguageAtom};

\item all classes member fields start with '{\tt f}';

\item all class static fields start with '{\tt s}';

\item all global variables start with '{\tt pGlobal}', in which case there are {\code set\dots} and {\code get\dots} macros to access them, starting with '{\tt g}' -- this is to facilitate typing and names completion in code editors;
\item some (all? JMI) start with '{\tt g}';

\item all variables private to methods start with '{\tt pPrivate}';

\item some {\tt K_*} constants are declared as static class constant members, such as in \msr{msrParts.h}:
\begin{lstlisting}[language=CPlusPlus]
//______________________________________________________________________________
class EXP msrPart : public msrPartGroupElement
{
  public:

    // constants
    // ------------------------------------------------------

    static const int K_PART_HARMONIES_STAFF_NUMBER;
    static const int K_PART_HARMONIES_VOICE_NUMBER;

    static const int K_PART_FIGURED_BASS_STAFF_NUMBER;
    static const int K_PART_FIGURED_BASS_VOICE_NUMBER;

   // ... ... ...
};
\end{lstlisting}

with the definition in \msr{msrParts.cpp}:
\begin{lstlisting}[language=CPlusPlus]
//______________________________________________________________________________
// constants
const int msrPart::K_PART_HARMONIES_STAFF_NUMBER = 10;
const int msrPart::K_PART_HARMONIES_VOICE_NUMBER = 11;

const int msrPart::K_PART_FIGURED_BASS_STAFF_NUMBER = 20;
const int msrPart::K_PART_FIGURED_BASS_VOICE_NUMBER = 21;
\end{lstlisting}

\end{itemize}

When a field is an STL container, such a {\tt vector}, {\tt list}, {\tt map} or {\tt set}, this in indicated as part of the identifier, such as:
\begin{lstlisting}[language=CPlusPlus]
    std::map<std::string, std::string>
                          fPartsRenamingMap;
\end{lstlisting}
or
\begin{lstlisting}[language=CPlusPlus]
                          fStringToDalSegnoKindMapVariable;
\end{lstlisting}
or
\begin{lstlisting}[language=CPlusPlus]
    std::map<std::string, Sxmlelement>  fPartMeasureNumbersToElementsMap;
\end{lstlisting}

All \methodName{create*} methods create class   instances, and are paired with an explicit constructor with the same parameters:
\begin{lstlisting}[language=CPlusPlus]
    // creation from MusicXML
    // ------------------------------------------------------

    static SMARTP<msrHarmonyDegree> create (
                            int                      inputLineNumber,
                            int                      harmonyDegreeValue,
                            msrAlterationKind        harmonyDegreeAlterationKind,
                            msrHarmonyDegreeTypeKind harmonyDegreeTypeKind);

  protected:

    // constructors/destructor
    // ------------------------------------------------------

                          msrHarmonyDegree (
                            int                      inputLineNumber,
                            int                      harmonyDegreeValue,
                            msrAlterationKind        harmonyDegreeAlterationKind,
                            msrHarmonyDegreeTypeKind harmonyDegreeTypeKind);

    virtual               ~msrHarmonyDegree ();
\end{lstlisting}

Some classes use private fields and methods for their internal working. A field in the form {\tt fCurrent*} denotes something whose value is not permanent once set. Fields named {\tt fPending*} contain values gathered to be used later, such as {\tt fPendingHarmoniesList} in \mxsrToMsrBoth{mxsr2msrTranslator}.


% -------------------------------------------------------------------------
\section{Exceptions and warnings/errors reporting}
% -------------------------------------------------------------------------

\mf\ defines exceptions for its needs in \waeBoth{waeExceptions}. These exceptions can be related to a format, a pass or a converter. Exceptions named {\tt *Internal*} are raised when something that should not happen occurs: this to avoid ugly crashes later in the execution.

All the std::exception classes are derived from {\tt }, that contains:
\begin{lstlisting}[language=CPlusPlus]
class EXP mfException: public std::exception
{
  public:

    // constructors/destructor
    // ------------------------------------------------------

                          mfException (
                            std::string const& exceptionDescription = "",
                            int           exceptionNumber = 0,
                            int           exceptionLevel = 0
                          ) throw ()
                            : fExceptionDescription (
                                "mfException: " + exceptionDescription),
                              fExceptionNumber (exceptionNumber),
                              fExceptionLevel (exceptionLevel)
                          {}

  // ... ... ...

  private:

    std::string           fExceptionDescription;

    int                   fExceptionNumber;
    int                   fExceptionLevel;
};
\end{lstlisting}

An example of std::exception is:
\begin{lstlisting}[language=CPlusPlus]
class EXP mxsr2msrException: public mfException
{
  public:
    mxsr2msrException (std::string const& exceptionDescription) throw ()
      : mfException (exceptionDescription)
    {}
};
typedef SMARTP<musicxmlException> S_musicxmlException;
\end{lstlisting}

There are warning and error reporting functions\index{functions} in \waeBoth{waeInterface.h}.
Examples are:
\begin{lstlisting}[language=CPlusPlus]
void oahValueFittedAtom::applyValueLessAtom (std::ostream& os)
{
  std::stringstream ss;

  ss <<
    "Applying value-fitted atom '" <<
    fetchNames () <<
    "' without a value";

  oahInternalError (ss.str ());
}

	fSelected = true;
\end{lstlisting}

and:
\begin{lstlisting}[language=CPlusPlus]
    case msrPedalTypeKind::kPedalType_UNKNOWN_:
      {
        // should not occur

        std::stringstream ss;

        ss <<
          "msrPedal '" <<
          elt->asShortString () <<
          "' has no pedal type";

        msrInternalError (
          gServiceRunData->getInputSourceName (),
          inputLineNumber,
          __FILE__, __LINE__,
          ss.str ());
      }
      break;
\end{lstlisting}

Another one is:
\begin{lstlisting}[language=CPlusPlus]
void mxsr2msrTranslator::visitEnd ( S_accordion_registration& elt )
{
  int inputLineNumber =
    elt->getInputLineNumber ();

#ifdef MF_TRACE_IS_ENABLED
  if (gGlobalMxsrOahGroup->getTraceMxsrVisitors ()) {
		std::stringstream ss;

		ss <<
      "--> End visiting S_accordion_registration" <<
      ", line " << inputLineNumber;

    gWaeHandler->waeTraceWithoutInputLocation (
      __FILE__, __LINE__,
      ss.str ());
  }
#endif // MF_TRACE_IS_ENABLED

  // An accordion-registration element needs to have
  // at least one of the child elements present

  if (fCurrentAccordionNumbersCounter == 0) {
    musicxmlWarning (
      gServiceRunData->getInputSourceName (),
      inputLineNumber,
      "accordion-registration has 0 child element, ignoring it");
  }

  else {
    // create the accordion registration
    S_msrAccordionRegistration
      accordionRegistration =
        msrAccordionRegistration::create (
          inputLineNumber,
          fCurrentAccordionHigh,
          fCurrentAccordionMiddle,
          fCurrentAccordionLow);

    // append it to the current part
    fCurrentPart->
      appendAccordionRegistrationToPart (
        accordionRegistration);
  }
}
\end{lstlisting}


% -------------------------------------------------------------------------
\section{Exporting symbols for Windows DLLs}
% -------------------------------------------------------------------------

\Main{Windows} needs export specifications for the symbols used by clients of a \Main{DLL}.

\url{https://docs.microsoft.com/en-us/cpp/build/exporting-from-a-dll-using-declspec-dllexport?view=msvc-160} is titled "Exporting from a DLL Using \code{__declspec(dllexport)}". It states that:
\begin{itemize}
\item to export functions, the \code{__declspec(dllexport)} keyword must appear to the left of the calling-convention keyword, if a keyword is specified. For example:
\begin{lstlisting}[language=CPlusPlus]
__declspec(dllexport) void __cdecl Function1(void);
\end{lstlisting}

\item to export all of the public data members and member functions in a class, the keyword must appear to the left of the class   name as follows:
\begin{lstlisting}[language=CPlusPlus]
class   __declspec(dllexport) CExampleExport : public CObject
{ ... class   definition ... };
\end{lstlisting}

\end{itemize}

\mf\ uses symbol \code{EXP}, supplied by \libmusicxmlSamples. \\
It is defined in \elements{exports.h} to be empty except on Windows, where it is a default visibility attribute:
\begin{lstlisting}[language=CPlusPlus]
#ifndef __exports__
#define __exports__

#if defined(WIN32) // && !defined (GCC)

# ifdef MSVC
#  pragma warning (disable : 4267)
#  pragma warning (disable : 4275)
#  pragma warning (disable : 4251)
#  pragma warning (disable : 4786)
#  pragma warning (disable : 4251)
#  pragma warning (disable : 4275)
# endif

# ifdef LIBMUSICXML_EXPORTS
#  define EXP __declspec(dllexport)

# elif defined(LIBMUSICXML_STATIC)
#  define EXP

# else
#  define EXP __declspec(dllimport)
# endif

#else

# ifdef LIBMUSICXML_EXPORTS
#  define EXP	__attribute__ ((visibility("default")))
# else
#  define EXP
# endif

#endif

#endif
\end{lstlisting}


% -------------------------------------------------------------------------
\section{Dynamic type checking}
% -------------------------------------------------------------------------

\EnumType s are not ideal to distinguish variants when inheritance is used, mainly because adding new derived types imposes the addition of new constants, thus impacting other areas in the code base.

\code{dynamic_cast} is used in those cases, such as:
\begin{lstlisting}[language=CPlusPlus]
  // handle the option
  if (
    // options group?
    S_oahGroup
      group =
        dynamic_cast<oahGroup*>(&(*element))
  ) {
    registerOahElementUse (
      group, optionNameUsed, ""); // "===group==="); // JMI to debug
  }

  else if (
    // options subgroup?
    S_oahSubGroup
      subGroup =
        dynamic_cast<oahSubGroup*>(&(*element))
  ) {
    registerOahElementUse (
      subGroup, optionNameUsed, ""); // "===subGroup==="); // JMI to debug
  }
\end{lstlisting}


% -------------------------------------------------------------------------
\section{Input line numbers}
% -------------------------------------------------------------------------

The passes and converters in \mf\ convert formats of scores from one format to another. In order to produce helpful warning and error messages, several descriptions contain a field:
\begin{lstlisting}[language=CPlusPlus]
    int                   fInputLineNumber;
\end{lstlisting}

An input line number in the {\tt xmlelement} class   is the only thing that has had to be added to \libmusicxml\ for the needs of \mf.

Also, many methods contain an {\tt int inputLineNumber} parameter, which is always the first one:
\begin{lstlisting}[language=CPlusPlus]
msrElement::msrElement (
  int inputLineNumber)
{
  fInputLineNumber = inputLineNumber;
}
\end{lstlisting}

Such input line numbers can be present in the output of the converters, such as:
\begin{lstlisting}[language=Terminal]
jacquesmenu@macmini: ~/musicformats-git-dev/files > xml2ly -query input-line-numbers
--- Help for atom "input-line-numbers" in subgroup "Output"
    -iln, -input-line-numbers
          Generate after each note and barLine a comment containing
          its MusicXML input line number.
          This is useful when debugging xml2ly.
\end{lstlisting}

Generators such as {\tt Mikrokosmos3Wandering} don't read any input, and the input line numbers the use are the ones in the source code, which is helpful for debugging. For example:
\begin{lstlisting}[language=CPlusPlus]
S_msrStaff Mikrokosmos3WanderingGenerator::createStaffInPart (
  int              staffNumber,
  const S_msrPart& part)
{
  // create the staff
  S_msrStaff
    staff =
      msrStaff::create (
        __LINE__, msrStaffKind::kStaffKindRegular, staffNumber, part);

  // append it to the part
  part ->
    addStaffToPartCloneByItsNumber ( // JMI NOT clone???
      staff);

  return staff;
}
\end{lstlisting}

In all output produced by \mf\ services, including trace informations and comments in the generated output refering to the input data, line numbers appear as:
\begin{lstlisting}[language=Terminal]
	line <number>
\end{lstlisting}
with a single space before the number, as in:
\begin{lstlisting}[language=Lilypond]
	e16 %{ line 153 %} ] %{ line 163 kBeamEnd %}
\end{lstlisting}

This helps locating such occurrences in the debug process.


% -------------------------------------------------------------------------
\section{Static declarations}
% -------------------------------------------------------------------------

They are used for:
\begin{itemize}
\item classes methods such as \method{msrTie}{create}, \method{msrTempo}{createTempoPerMinute} and \method{msrTemp}{msrTempoBeatUnitsKindAsString};

\item classes constant fields such as \constant{msrStanza}{K_STANZA_NUMBER_UNKNOWN_}, to be prefered to C-style {\tt \#define} preprocessor specifications for type safety;

\item functions of methods remanent variables such as \functionPrivateVariable{pPrivateThisMethodHasBeenRun} in \function{initializeMsrGenerationAPI};

\item library-wide variables such as \globalVariable{gGlobalOStreamIndenter} and \globalVariable{sGlobalTimingItemsList}, that would be too cumbersome to pass to each and every method or function that uses them.
\end{itemize}


% -------------------------------------------------------------------------
\section{Avoiding MusicFormats multiple initializations}
% -------------------------------------------------------------------------

Such behaviour would create data structures several times, the result being unnecessary activities being performed. Avoiding it is done with \functionPrivateVariable{pPrivateThisMethodHasBeenRun}, here in \\
\bsr{bsr.cpp}:
\begin{lstlisting}[language=CPlusPlus]
void initializeBSR ()
{
  // protect library against multiple initializations
  static Bool pPrivateThisMethodHasBeenRun (false);

  if (! pPrivateThisMethodHasBeenRun) {
#ifdef MF_TRACE_IS_ENABLED
    if (gEarlyOptions.getTraceEarlyOptions () && ! gEarlyOptions.getEarlyQuietOption ()) {
      gLog <<
        "Initializing BSR basic types handling";

      gWaeHandler->waeTraceWithoutInputLocation (
        __FILE__, __LINE__,
        ss.str ());
    }
#endif // MF_TRACE_IS_ENABLED

    // BSR braille output kinds handling
    // ------------------------------------------------------

    initializeBsrBrailleOutputKindsMap ();

    // BSR texts languages handling
    // ------------------------------------------------------

    initializeBsrTextsLanguageKindsMap ();

    pPrivateThisMethodHasBeenRun = true;
  }
}
\end{lstlisting}


% -------------------------------------------------------------------------
\section{Enumeration types}
% -------------------------------------------------------------------------

All \enumType s use the \CPlusplus\ '{\tt enum class}' feature, such as:

\begin{lstlisting}[language=CPlusPlus]
enum class msrSlurTypeKind {
  kSlurType_UNKNOWN_,

  kSlurTypeRegularStart, kSlurTypePhrasingStart,
  kSlurTypeContinue,
  kSlurTypeRegularStop, kSlurTypePhrasingStop
};
\end{lstlisting}

This prevents enumeration constants name conflicts across \enumType s, and qualified names such as \constant{msrSlurTypeKind}{kSlurTypeRegularStart} are quite explicit.

Many enumerations names end in '{\tt Kind}', which is a way distinguish them from rather similar classes names in some cases.

Enumeration constants in the form {\tt k*_UNKNOWN_} are used to indicate a value that \MainIt{has not been set yet}. There are always the first on in the corresponding enumeration, to benefit from the the \CPlusplus\ implicit \initialization\  to the the equivalent of 0.

An enumeration constant may end in '{\tt None}', meaning that {\tt "none"} is actually a possible value for the corresponding type:
\begin{lstlisting}[language=CPlusPlus]
    // repeat winged
    enum class msrBarLineRepeatWingedKind {
      kBarLineRepeatWingedNone,

      kBarLineRepeatWingedStraight, kBarLineRepeatWingedCurved,
      kBarLineRepeatWingedDoubleStraight, kBarLineRepeatWingedDoubleCurved
    };
\end{lstlisting}

Here how the {\tt "winged"} \mxml\ attribute of \musicXmlMarkup{repeat} is analysed in {\tt }:%%%JMI
\begin{lstlisting}[language=CPlusPlus]
void mxsr2msrTranslator::visitStart ( S_repeat& elt )
{
    // ... ... ...

    std::string winged = elt->getAttributeValue ("winged");

    fCurrentBarLineRepeatWingedKind =
      msrBarLineRepeatWingedKind::kBarLineRepeatWingedNone; // default value

    if (winged.size ()) {
      if       (winged == "none") {
        fCurrentBarLineRepeatWingedKind =
          msrBarLineRepeatWingedKind::kBarLineRepeatWingedNone;
      }
      else if (winged == "straight") {
        fCurrentBarLineRepeatWingedKind =
          msrBarLineRepeatWingedKind::kBarLineRepeatWingedStraight;
      }
      else  if (winged == "curved") {
        fCurrentBarLineRepeatWingedKind =
          msrBarLineRepeatWingedKind::kBarLineRepeatWingedCurved;
      }
      else  if (winged == "double-straight") {
        fCurrentBarLineRepeatWingedKind =
          msrBarLineRepeatWingedKind::kBarLineRepeatWingedDoubleStraight;
      }
      else  if (winged == "double-curved") {
        fCurrentBarLineRepeatWingedKind =
          msrBarLineRepeatWingedKind::kBarLineRepeatWingedDoubleCurved;
      }
      else {
        std::stringstream ss;

        ss <<
          "repeat winged \"" << winged <<
          "\" is unknown";

        musicxmlError (
          gServiceRunData->getInputSourceName (),
          inputLineNumber,
          __FILE__, __LINE__,
          ss.str ());
      }
    }

    // ... ... ...
\end{lstlisting}

The \mf\ \enumType s come with a \code{$<$$<$} operator to display their values in a user friendly way, such as \type{msrChordInKind} in \msrBoth{msrBasicTypes}:
\begin{lstlisting}[language=CPlusPlus]
enum class msrChordInKind {
  kChordIn_UNKNOWN_,

  kChordInMeasure,
  kChordInTuplet,
  kChordInGraceNotesGroup
};

EXP std::string msrChordInKindAsString (
  msrChordInKind chordInKind);

std::ostream& operator << (std::ostream& os, const msrChordInKind& elt);
\end{lstlisting}

The implementation is:
\begin{lstlisting}[language=CPlusPlus]
std::string msrChordInKindAsString (
  msrChordInKind chordInKind)
{
  std::string result;

  switch (chordInKind) {
    case msrChordInKind::kChordIn_UNKNOWN_:
      result = "kChordIn_UNKNOWN_";
      break;
    case msrChordInKind::kChordInMeasure:
      result = "kChordInMeasure";
      break;
    case msrChordInKind::kChordInTuplet:
      result = "kChordInTuplet";
      break;
    case msrChordInKind::kChordInGraceNotesGroup:
      result = "kChordInGraceNotesGroup";
      break;
  } // switch

  return result;
}

std::ostream& operator << (std::ostream& os, const msrChordInKind& elt)
{
  os << msrChordInKindAsString (elt);
  return os;
}
\end{lstlisting}


% -------------------------------------------------------------------------
\section{yes/no enumerations types}
% -------------------------------------------------------------------------

Boolean argument to methods calls are fine in simple cases such as:
\begin{lstlisting}[language=CPlusPlus]
    void                  setCombinedBooleanVariables (Bool value);
\end{lstlisting}

But when there are multiple arguments, the semantics of the {\tt true} or {\tt false} constants if far from obvious.

This is why we use enum classes such as:
\begin{lstlisting}[language=CPlusPlus]
enum class msrVoiceCreateInitialLastSegmentKind {
  kCreateInitialLastSegmentYes,
  kCreateInitialLastSegmentNo
};
\end{lstlisting}

in such cases, so that the arguments bare a clear semantics:
\begin{lstlisting}[language=CPlusPlus]
  fPartHarmoniesVoice =
    msrVoice::create (
      inputLineNumber,
      msrVoiceKind::kVoiceKindHarmonies,
      partHarmoniesVoiceNumber,
      msrVoiceCreateInitialLastSegmentKind::kCreateInitialLastSegmentYes,
      fPartHarmoniesStaff);
\end{lstlisting}


% -------------------------------------------------------------------------
\section{Boolean values anyway}
% -------------------------------------------------------------------------

Defining a yes/no enumeration type for 'true' boolean values such as the variables containing the OAH options would be cumbersome. The C++ \code{bool} type suffers from the C heritage, in which integers and even pointers can be mixed in and considered as boolean values.

Moreover, a \code{bool} variable not explicitly initialized in the developper's code can lead to hard to fix bugs, in particular when the \mf\ library is used on various hardware and \OS s.

For these reasons, \mf\ features a \class{Bool} defined in \mfutilitiesBoth{mfBool}. It \Main{encapsulates} the actual \code{bool} value, enforcing that its initial value is not random, but known to the developper, through constructors. This also avoids in particular long sequences of initializations in the passes constructors.


% -------------------------------------------------------------------------
\section{On/off values}
% -------------------------------------------------------------------------

Some elements in \mf\ are represented by a \MainIt{three-state} value.

\OnOff values are handled by \enumType{mfOnOffKind}, declared in \mflibrary{mfOnOff.h}:
\begin{lstlisting}[language=Terminal]
//______________________________________________________________________________
enum class mfOnOffKind {
  kMfOnOffUnknown,
  kMfOnOffOn, kMfOnOffOff
};

Bool mfOnOffKindAsBool (
  mfOnOffKind onOffKind);

std::string mfOnOffKindAsString (
  mfOnOffKind onOffKind);

std::ostream& operator << (std::ostream& os, const mfOnOffKind elt);
\end{lstlisting}

This is the case for example of page ragging information in \lpsr{lpsr/lpsrPapers.h}:
\begin{lstlisting}[language=Terminal]
    // on/off values
    mfOnOffKind           fRaggedBottom;
    S_oahOnOffAtom        fRaggedBottomAtom;

    mfOnOffKind           fRaggedLast;
    S_oahOnOffAtom        fRaggedLastAtom;

    mfOnOffKind           fRaggedLastBottom;
    S_oahOnOffAtom        fRaggedLastBottomAtom;

    mfOnOffKind           fRaggedRight;
    S_oahOnOffAtom        fRaggedRightAtom;
\end{lstlisting}

Care must be taken in \mflibrary{mfOnOff.cpp} when converting an \enumType{mfOnOffKind} value to \type{Bool}:
\begin{lstlisting}[language=Terminal]
Bool mfOnOffKindAsBool (
  mfOnOffKind onOffKind)
{
  Bool result;

  switch (onOffKind) {
    case mfOnOffKind::kMfOnOffUnknown:
      {
        mfError (
          __FILE__, __LINE__,
          "mfOnOffKind::kMfOnOffUnknown cannot be converted to Bool");
      }
      break;
    case mfOnOffKind::kMfOnOffOn:
      result = true;
      break;
    case mfOnOffKind::kMfOnOffOff:
      result = false;
      break;
  } // switch

  return result;
}
\end{lstlisting}


% -------------------------------------------------------------------------
\section{Iterating over numeration types}
% -------------------------------------------------------------------------

Such iterations rely on template classes. For this to work, the enumeration type should provide specific {\tt AllFirst} and {\tt AllLast} 'aliases' for the first and last constants in the type.

This is done for \enumType{msrHarmonyKind} in \msr{msrBasicTypes.h}:
\begin{lstlisting}[language=CPlusPlus]
// harmonies
//______________________________________________________________________________
enum class msrHarmonyKind {
  kHarmony_UNKNOWN_,

  // MusicXML harmonies

  kHarmonyMajor, kHarmonyMinor,
  kHarmonyAugmented, kHarmonyDiminished,

	// ... ... ...

  // other

  kHarmonyOther,

  kHarmonyNone,

  // aliases
  // ------------------------------------

  AllFirst = kHarmony_UNKNOWN_,
  AllLast  = kHarmonyNone,

	// ... ... ...
};
\end{lstlisting}

class   {\tt mfEnumAll} is defined in \mfutilities{mfEnumAll.h}:
\begin{lstlisting}[language=CPlusPlus]
//______________________________________________________________________________
/*
  https://stackoverflow.com/questions/8498300/allow-for-range-based-for-with-enum-classes
*/
template< typename T >
class   mfEnumAll
{
  public:

    class   Iterator
    {
      public:

        Iterator (int value)
          : fIterationIndex (value)
            {}

        T operator* (void) const
            { return (T) fIterationIndex; }

        void operator++ (void)
            {  ++fIterationIndex; }

        Bool operator != (Iterator rhs)
            {  return fIterationIndex != rhs.fIterationIndex; }

      private:

        int fIterationIndex;
     };
};

template< typename T >
typename mfEnumAll<T>::Iterator begin (mfEnumAll<T>)
{
  return typename mfEnumAll<T>::Iterator ((int)T::AllFirst);
}

template< typename T >
typename mfEnumAll<T>::Iterator end (mfEnumAll<T>)
{
  return typename mfEnumAll<T>::Iterator (((int)T::AllLast) + 1);
}
\end{lstlisting}

The {\tt mfEnumAll} template class, defined in \mfutilities{mfEnumAll.h} can then be used to iterate from \constant{msrHarmonyKind}{AllFirst} to \constant{msrHarmonyKind}{AllLast}, here in \msr{msrBasicTypes.cpp}:
\begin{lstlisting}[language=CPlusPlus]
void initializeHarmonyStructuresMap ()
{
  // protect library against multiple initializations
  static Bool pPrivateThisMethodHasBeenRun (false);

  if (! pPrivateThisMethodHasBeenRun) {
    for (auto e : mfEnumAll<msrHarmonyKind> ()) {
      // create the harmony structure
      S_msrHarmonyStructure
        harmonyStructure =
          msrHarmonyStructure::create (
            e);

      // register it in the map
      gGlobalHarmonyStructuresMap [e] =
        harmonyStructure;
    } // for

    pPrivateThisMethodHasBeenRun = true;
  }
}
\end{lstlisting}

The {\tt mfEnumAll} template class, defined in \mfutilities{mfEnumAll.h} can then be used to iterate from \constant{msdlTokenKind}{AllFirst} to \constant{msdlTokenKind}{AllLast}, here in \msdl{msdlTokens.cpp}:
\begin{lstlisting}[language=CPlusPlus]
    for (auto e : EnumNonSeparators<msdlTokenKind> ()) {
      std::string
        nonSeparatorTokenAsMsdlString  =
          msdlTokenKindAsMsdlString (
            e,
            keywordsLanguageKind);

			// ... ... ...
    } // for
\end{lstlisting}

All such \class{Enum*} classes in \mf\ are:
\begin{lstlisting}[language=Terminal]
jacquesmenu@macmini: ~/musicformats-git-dev/src > grep -r 'class   Enum' *
formats/msdl/msdlTokens.h:class   EnumNonSeparators
formats/msdl/msdlTokens.h:class   EnumLanguageIndependent
formats/msdl/msdlTokens.h:class   EnumLanguageDependent
formats/msr/msrBasicTypes.h:class   EnumTrueHarmonies
utilities/mfutilities.h:class   mfEnumAll
\end{lstlisting}

For example \class{EnumTrueHarmonies}, that relies on \constant{msrHarmonyKind}{TrueHarmoniesFirst} and\\
 \constant{msrHarmonyKind}{TrueHarmoniesLast}:
\begin{lstlisting}[language=CPlusPlus]
void msrHarmonyStructure::printAllHarmoniesStructures (std::ostream& os)
{
  os <<
    "All the known harmonies structures are:" <<
    std::endl << std::endl;

  ++gIndenter;

  for (auto e : EnumTrueHarmonies<msrHarmonyKind> ()) {
    // create the harmony intervals
    S_msrHarmonyStructure
      harmonyStructure =
        msrHarmonyStructure::create (
          e);

    // print it
    os <<
      harmonyStructure <<
      std::endl;
  } // for

  --gIndenter;
}
\end{lstlisting}


% -------------------------------------------------------------------------
\section{mfRational numbers}
% -------------------------------------------------------------------------

\mf\ uses rationals for notes sounding and display whole notes and measure positionss, among others.
\Class {mfRational} is defined by \libmusicxml\ in \libBoth{mfRational}:
\begin{lstlisting}[language=CPlusPlus]
class EXP mfRational {
	// ... ... ...

  private:

    // private fields
    // ------------------------------------------------------

    long int              fNumerator;
    long int              fDenominator;

    // used by rationalise()
//    long int              gcd(long int a, long int b);

	// ... ... ...
};
\end{lstlisting}

Rationals are not used, however, for tuplets factors, see .


% -------------------------------------------------------------------------
\section{Don't mix pure and typed scalar types}
% -------------------------------------------------------------------------

A whole notes value is described by \class{msrWholeNotes}, defined in \msrBoth{msrNotesDurations}. It contains a numerator and denominator, as \class{msrRational} does, but those types are distinct, since an \className{msrWholeNotes} instance is a fraction of a whole note, note a mere typeless fraction.

\mf\ uses \className{msrWholeNotes} instances for notes sounding and display whole notes values;

For the same reason, \class{msrTupletFactor}, defined in \msrBoth{msrTupletFactors}, is used with specifically named fields:
\begin{lstlisting}[language=CPlusPlus]
class EXP msrTupletFactor
{
	// ... ... ...

  private:

    // private fields
    // ------------------------------------------------------

    int                   fTupletActualNotes;
    int                   fTupletNormalNotes;
};
\end{lstlisting}


% -------------------------------------------------------------------------
\section{Default values}
% -------------------------------------------------------------------------

The guide lines for \mf\ in this matter are:
\begin{itemize}
\item smart pointers are initialized to {\tt nullptr} in the \class{SMARTP} constructor (they're smart after all), defined by \libmusicxml\ in \lib{smartpointer.h}:
\begin{lstlisting}[language=CPlusPlus]
template<class   T> class   SMARTP {
	private:
		//! the actual pointer to the class
		T* fSmartPtr;

	public:
		//! an empty constructor - points to null
		SMARTP()	: fSmartPtr(0) {}

	// ... ... ...
\end{lstlisting}

\item all variables and classes fields of non-class   types, such as {\tt int}, {\tt float} and \enumType s, are to be initialized explicitly;

\item \mf\ functions\index{functions} and methods parameters never have default values: overloading is used instead.
\end{itemize}


% -------------------------------------------------------------------------
\section{{\tt create*} methods}%%%JMI
% -------------------------------------------------------------------------

All concrete classes, i.e. those that are not pure virtual, have \methodName{create*} methods paired with a constructor with the exact same parameters.

In most cases, there are justs named \methodName{create*}, but a couple of them have more explicit names.

One case is that of \class{msrTempo} in \msrBoth{msrTempos}, because calls to them would be hard to distinguish at first glance otherwise:
\begin{lstlisting}[language=CPlusPlus]
 class EXP msrTempo : public msrMeasureElement
{
 		// ... ... ...

	  static SMARTP<msrTempo> createTempoWordsOnly (
      int               inputLineNumber,
      S_msrWords        tempoWords);

    static SMARTP<msrTempo> createTempoPerMinute (
      int               inputLineNumber,
      msrDottedNotesDuration tempoBeatUnit,
      std::string       tempoPerMinute,
      msrTempoParenthesizedKind
                        tempoParenthesizedKind,
      msrPlacementKind  tempoPlacementKind);

    static SMARTP<msrTempo> createTempoBeatUnitEquivalent (
      int               inputLineNumber,
      msrDottedNotesDuration tempoBeatUnit,
      msrDottedNotesDuration tempoEquivalentBeatUnit,
      msrTempoParenthesizedKind
                        tempoParenthesizedKind,
      msrPlacementKind  tempoPlacementKind);

    static SMARTP<msrTempo> createTempoNotesRelationship (
      int               inputLineNumber,
      S_msrTempoNotesRelationshipElements
                        tempoNotesRelationshipLeftElements,
      msrTempoNotesRelationshipKind
                        tempoNotesRelationshipKind,
      S_msrTempoNotesRelationshipElements
                        tempoNotesRelationshipRightElements,
      msrTempoParenthesizedKind
                        tempoParenthesizedKind,
      msrPlacementKind  tempoPlacementKind);

		// ... ... ...
};
\end{lstlisting}

Another case is that of \class{msrKey} in \msrBoth{msrKeys}, in which the variant selected is made explicit:
\begin{lstlisting}[language=CPlusPlus]
class EXP msrKey : public msrMeasureElement
{
		// ... ... ...

    static SMARTP<msrKey> createTraditional (
                            int                      inputLineNumber,
                            msrQuarterTonesPitchKind keyTonicPitchKind,
                            msrModeKind              modeKind,
                            int                      keyCancel);

    static SMARTP<msrKey> createHumdrumScot (
      int                  inputLineNumber);

		// ... ... ...
};
\end{lstlisting}


% -------------------------------------------------------------------------
\section{{\tt get*()}, {\tt set*()} and {\tt fetch*()} methods}
% -------------------------------------------------------------------------

As is usual, classes privates member variables are accessed through \methodName{set*()} and \methodName{get*()} methods. The name of these methods is obtained by replacing the '\code{f}' in the field name by '\code{set}' and '\code{get}', respectively. In \msr{msrTies.h}, one finds:
\begin{lstlisting}[language=CPlusPlus]
    // set and get
    // ------------------------------------------------------

    void                  setTieKind (msrTieKind tieKind)
                              { fTieKind = tieKind; }

    msrTieKind            getTieKind () const
                              { return fTieKind; }

    void                  setTiePlacementKind (msrPlacementKind placementKind)
                              { fTiePlacementKind = placementKind; }

    msrPlacementKind      getTiePlacementKind () const
                              { return fTiePlacementKind; }
\end{lstlisting}


\code{fetch} is used when the result is not store in a variable, but has to computed in some way. \msr{msrSegment.cpp} contains:
\begin{lstlisting}[language=CPlusPlus]
S_msrStaff msrSegment::fetchSegmentUpLinkToStaff () const
{
  S_msrStaff result;

  if (fSegmentUpLinkToVoice) {
    result =
      fSegmentUpLinkToVoice->
        getVoiceUpLinkToStaff ();
  }

  return result;
}
\end{lstlisting}

Some methods exist in two versions, the second one with a \code{NonConst} suffix in the name. This can be the case if a private class   field is to be modified outside the class, or if there are specific needs.

This happens for example in \msrBoth{mfslNotes}:
\begin{lstlisting}[language=CPlusPlus]
    // articulations
    const std::list<S_msrArticulation>&
                          getNoteArticulations () const
                              { return fNoteArticulations; }

    std::list<S_msrArticulation>&
                          getNoteArticulationsNonConst ()
                              { return fNoteArticulations; }
\end{lstlisting}

Another case is in \mfslBoth{mfslDriver}:
\begin{lstlisting}[language=CPlusPlus]
    const yy::location&   getScannerLocation () const
                              { return fScannerLocation; }

    yy::location&         getScannerLocationNonConst ()
                            // no const here
                            // due to constraints in the Flex-generated code
                              { return fScannerLocation; }
\end{lstlisting}


% -------------------------------------------------------------------------
\section{{\tt initialize*()} and {\tt finalize*()} methods}
% -------------------------------------------------------------------------

When a description contains many fields, the ones initialized by the values of the constructor's parameters are initialized in the latter, and the others are in an \methodName{initialize*()} method, such as:
\begin{lstlisting}[language=CPlusPlus]
msrPart::msrPart (
  int            inputLineNumber,
  std::string         partID,
  S_msrPartGroup partUpLinkToPartGroup)
    : msrPartGroupElement (inputLineNumber)
{
  // replace spaces in partID to set fPartID
  for_each (
    partID.begin (),
    partID.end (),
    mfStringSpaceReplacer (fPartID, '_'));

/* JMI
#ifdef MF_SANITY_CHECKS_ARE_ENABLED
  // sanity check
  mfAssert (
    __FILE__, __LINE__,
    partUpLinkToPartGroup != nullptr,
    "partUpLinkToPartGroup is null");
#endif // MF_SANITY_CHECKS_ARE_ENABLED

  // set part number
  fPartAbsoluteNumber = ++sPartsCounter;

  // set part's part group upLink
  fPartUpLinkToPartGroup = partUpLinkToPartGroup;

  // do other initializations
  initializePart ();
}
\end{lstlisting}

Some \methodName{finalize*()} methods exist. %%%JMI


% -------------------------------------------------------------------------
\section{{\tt *asString()} and {\tt *fromString()} functions\index{functions}}
% -------------------------------------------------------------------------

Each \enumType comes with an \functionName{*AsString()} function\index{functions}, to display the constant values as strings. Some also have a \functionName{fromString} function\index{functions} to convert strings to the corresponding constant. For example, one finds in \msrBoth{msrBasicTypes}:
\begin{lstlisting}[language=CPlusPlus]
// placement
//______________________________________________________________________________
enum class msrPlacementKind {
  kPlacement_UNKNOWN_,

  kPlacementAbove, kPlacementBelow
};
\end{lstlisting}

\begin{lstlisting}[language=CPlusPlus]
// placement
//______________________________________________________________________________
msrPlacementKind msrPlacementKindFromString (
  int    inputLineNumber,
  std::string placementString)
{
  msrPlacementKind result = msrPlacementKind::kPlacement_UNKNOWN_; // default value

  if      (placementString == "above")
    result = msrPlacementKind::kPlacementAbove;
  else if (placementString == "below")
    result = msrPlacementKind::kPlacementBelow;
  else {
    if (placementString.size ()) {
      std::stringstream ss;

      ss <<
        "placement \"" << placementString <<
        "\" should be 'above' or 'below'";

      musicxmlError (
        gServiceRunData->getInputSourceName (),
        inputLineNumber,
        __FILE__, __LINE__,
        ss.str ());
    }
  }

  return result;
}
\end{lstlisting}

\begin{lstlisting}[language=CPlusPlus]
std::string msrPlacementKindAsString (
  msrPlacementKind placementKind)
{
  std::string result;

  switch (placementKind) {
    case msrPlacementKind::kPlacement_UNKNOWN_:
      result = "noPlacement";
      break;
    case msrPlacementKind::kPlacementAbove:
      result = "placementAbove";
      break;
    case msrPlacementKind::kPlacementBelow:
      result = "placementBelow";
      break;
  } // switch

  return result;
}
\end{lstlisting}

Many classes have \methodName{asStringShort()} methods to provide more compact a description as the one provided by the corresponding \methodName{asString()} method.%%%JMI


% -------------------------------------------------------------------------
\section{{\tt translate*()} methods\index{methods} and {\tt convert*() functions\index{functions}}}
% -------------------------------------------------------------------------

To translate and to convert are aliass in the context of \mf.

For semantic clearness, \methodName{translate*()} methods are supplied by the individual translators, as in \msrToMsrBoth{msr2msrTranslator}:
\begin{lstlisting}[language=CPlusPlus]
//________________________________________________________________________
class EXP msr2msrTranslator :

  // MSR score

  public visitor<S_msrScore>,

	// ... ... ...

{
  public:

                          msr2msrTranslator ();

    virtual               ~msr2msrTranslator ();

    S_msrScore            translateMsrToMsr (
                            const S_msrScore& theMsrScore);
	// ... ... ...

};
\end{lstlisting}

\begin{lstlisting}[language=CPlusPlus]
S_msrScore msr2msrTranslator::translateMsrToMsr (
  const S_msrScore& theMsrScore)
{
#ifdef MF_SANITY_CHECKS_ARE_ENABLED
  // sanity check
  mfAssert (
    __FILE__, __LINE__,
    theMsrScore != nullptr,
    "theMsrScore is null");
#endif // MF_SANITY_CHECKS_ARE_ENABLED

  // the MSR score we're visiting
  fVisitedMsrScore = theMsrScore;

  // create the resulting MSR score
  fResultingNewMsrScore =
    msrScore::create (
      K_MF_INPUT_LINE_UNKNOWN_,
      "msrScore::create()");

  // create a msrScore browser
  msrBrowser<msrScore> browser (this);

  // browse the visited score with the browser
  browser.browse (*fVisitedMsrScore);

  // forget about the visited MSR score
  fVisitedMsrScore = nullptr;

  return fResultingNewMsrScore;
}
\end{lstlisting}

The \functionName{convert*()} functions are the interfaces to the translators, for example in \msrToMsrBoth{msr2msrInterface}:
\begin{lstlisting}[language=CPlusPlus]
S_msrScore translateMsrToMsr (
  S_msrScore        originalMsrScore,
  const S_msrOahGroup&     msrOpts,
  S_msr2msrOahGroup msr2msrOpts,
  mfPassIDKind          passIDKind,
  const std::string&     passDescription)
{
	// ... ... ...

  // the msr2msrTranslator
  msr2msrTranslator
    translator;

  // build the resulting MSR score
  S_msrScore
    resultingNewMsrScore =
      translator.translateMsrToMsr (
        originalMsrScore);

	// ... ... ...
}
\end{lstlisting}


% -------------------------------------------------------------------------
\section{{\tt context} arguments}
% -------------------------------------------------------------------------

Some methods have such an argument, a std::string, to provide helpful information to the maintainer of \mf. An exemple is \method{msrMeasureRepeat}{displayMeasureRepeat}, defined in \msrBoth{msrMeasureRepeats}:
\begin{lstlisting}[language=CPlusPlus]
void msrMeasureRepeat::displayMeasureRepeat (
  int    inputLineNumber,
  const std::string& context)
{
  gLog <<
    std::endl <<
    "*********>> MeasureRepeat " <<
    ", measureRepeatMeasuresNumber: '" <<
    fMeasureRepeatMeasuresNumber <<
    ", measureRepeatSlashesNumber: '" <<
    fMeasureRepeatSlashesNumber <<
    "', voice:" <<
    std::endl <<
    fUpLinkToMeasureRepeatToVoice->getVoiceName () <<
    " (" << context << ")" <<
    ", line " << inputLineNumber <<
    " contains:" <<
    std::endl;

  ++gIndenter;
  print (gLog);
  --gIndenter;

  gLog <<
    " <<*********" <<
    std::endl << std::endl;
\end{lstlisting}

An call example in \msr{msrVoices.h} is:
\begin{lstlisting}[language=CPlusPlus]
    displayVoiceMeasureRepeatAndVoice (
      inputLineNumber,
      "createMeasureRepeatFromItsFirstMeasures() 1");
\end{lstlisting}


% -------------------------------------------------------------------------
\section{Sorting and {\tt compare*()} methods}
% -------------------------------------------------------------------------

\mf\ sometimes needs to sort some data structures:
\begin{lstlisting}[language=Terminal]
jacquesmenu@macmini: ~/musicformats-git-dev/src > grep -r '\.sort (' *
oah/oahBasicTypes.cpp:  optionsMapElementsNamesList.sort ();
passes/lpsr2lilypond/lpsr2lilypondTranslator.cpp:    frameFrameNotesList.sort (
formats/msr/msrMeasuresSlices.cpp:  fSliceNotesFlatList.sort (
formats/msr/msrMeasuresSlices.cpp:  fSliceNoteEventsList.sort (
formats/msr/msrStaves.cpp:        fStaffAllVoicesList.sort (
formats/msr/msrStaves.cpp:        fStaffAllVoicesList.sort (
formats/msr/msrStaves.cpp:  fStaffRegularVoicesList.sort (
formats/msr/msrMeasures.cpp:    fMeasureElementsList.sort (
formats/msr/msrMeasures.cpp:    fMeasureElementsList.sort (
formats/msr/msrParts.cpp:    fPartAllStavesList.sort (
formats/msr/msrParts.cpp:    fPartAllStavesList.sort (
formats/lpsr/lpsrParts.cpp:    fPartBlockElementsList.sort (
\end{lstlisting}

There are thus a number of \methodName{compare*} methods according to the needs:
\begin{lstlisting}[language=Terminal]
jacquesmenu@macmini: ~/musicformats-git-dev/src > grep -r compare * | grep '\.h'
oah/oahBasicTypes.h:    const std::multiset<S_oahElement, compareOahElements>&
oah/oahBasicTypes.h:    std::multiset<S_oahElement, compareOahElements>
oah/oahElements.h:struct compareOahElements {
formats/msr/msrParts.h:    static bool           compareStavesToHaveFiguredBassesBelowCorrespondingPart (
formats/msr/msrNotes.h:    static bool           compareNotesByIncreasingMeasurePosition (
formats/msr/msrMeasureElementsWithoutUpLinkToMeasure.h:    static bool           compareMeasureElementsByIncreasingMeasurePosition (
formats/msr/msrStaves.h:    static bool           compareVoicesByIncreasingNumber (
formats/msr/msrStaves.h:    static bool           compareVoicesToHaveHarmoniesAboveCorrespondingVoice (
formats/msr/msrStaves.h:    static bool           compareVoicesToHaveFiguredBassesBelowCorrespondingVoice (
formats/msr/msrMeasuresSlices.h:    static bool           compareNotesEventsByIncreasingMeasurePosition (
formats/msr/msrMeasuresSlices.h:    static bool           compareSimultaneousNotesChunksByIncreasingMeasurePosition (
formats/lpsr/lpsrParts.h:    static bool           compareElementsToHaveHarmoniesAboveCorrespondingStaff (
formats/lpsr/lpsrParts.h:    static bool           compareStaffBlockWithOtherElement (
formats/lpsr/lpsrParts.h:    static bool           compareChordNamesContextWithOtherElement (
utilities/mfutilities.h:    // compare indentation value
\end{lstlisting}

An example is:
\begin{lstlisting}[language=Terminal]
bool msrPart::compareStavesToHaveFiguredBassesBelowCorrespondingPart (
  const S_msrStaff& first,
  const S_msrStaff& second)
{
  int
    firstStaffNumber =
      first->getStaffNumber (),
    secondStaffNumber =
      second->getStaffNumber ();

  if (firstStaffNumber > msrPart::K_PART_FIGURED_BASS_STAFF_NUMBER) {
    firstStaffNumber -= msrPart::K_PART_FIGURED_BASS_STAFF_NUMBER + 1;
  }
  if (secondStaffNumber > msrPart::K_PART_FIGURED_BASS_STAFF_NUMBER) {
    secondStaffNumber -= msrPart::K_PART_FIGURED_BASS_STAFF_NUMBER + 1;
  }

  bool result =
    firstStaffNumber > secondStaffNumber;

  return result;
}
\end{lstlisting}


% -------------------------------------------------------------------------
\section{Mutually dependent classes}
% -------------------------------------------------------------------------

\CPlusplus\ has some constraints on how such classes can be defined, see for example \url{https://cplusplus.com/forum/articles/10627/}. \mf\ sticks to having one header file per important class, with one exception.


%/Users/jacquesmenu/musicformats-git-dev/src/passes/msr2lpsr/msr2lpsrTranslator.cpp:3374:23: error: member access into incomplete type 'MusicFormats::msrTempo'
%      ", line " << elt->getInputLineNumber () <<
      std::endl;

%                      ^
%/Users/jacquesmenu/musicformats-git-dev/src/representations/msr/msrTypesForwardDeclarations.h:307:9: note: forward declaration of 'MusicFormats::msrTempo'
%class   msrTempo;
%        ^
%
%
%/Users/jacquesmenu/musicformats-git-dev/src/passes/msr2lpsr/msr2lpsrTranslator.cpp:5319:7: error: incomplete type 'MusicFormats::msrAfterGraceNotesGroup' named in nested name specifier
%      msrAfterGraceNotesGroup::create (
%      ^~~~~~~~~~~~~~~~~~~~~~~~~
%/Users/jacquesmenu/musicformats-git-dev/src/representations/msr/msrTypesForwardDeclarations.h:31:9: note: forward declaration of 'MusicFormats::msrAfterGraceNotesGroup'
%class   msrAfterGraceNotesGroup;
%        ^


%If, for example, class A uses class B, then class B is one of class A's dependencies. Whether it can be forward declared or needs to be included depends on how B is used within A:
%
%
%- do nothing if: A makes no references at all to B
%- do nothing if: The only reference to B is in a friend declaration
%- forward declare B if: A contains a B pointer or reference: B* myb;
%- forward declare B if: one or more functions has a B object/pointer/reference
%as a parementer, or as a return type: B MyFunction(B myb);
%- #include "b.h" if: B is a parent class of A
%- #include "b.h" if: A contains a B object: B myb;
%
%A circular dependency is when two (or more) classes depend on each other. For example, class A depends on class B, and class B depends on class A. If you stick to "the right way" and forward declare when you can instead of #including needlessly, this usually isn't a problem. As long as the circle is broken with a forward declaration at some point, you're fine.


%https://learn.microsoft.com/en-us/cpp/error-messages/compiler-errors-1/compiler-error-c2027?view=msvc-170
%
%
%It is possible to declare a pointer to a declared but undefined type. But C++ does not allow a reference to an undefined type.
%The following sample generates C2027.
%C++
%
%Copy
%// C2027_b.cpp
%class A;
%A& CreateA();
%
%class B;
%B* CreateB();
%
%int main() {
%   CreateA();   // C2027
%   CreateB();   // OK
%}



In some cases, two classes have to know about each other, such as \className{msrPartGroup} and \className{msrPartGroupElement}. This is because \Main{part group}s can be nested: a part group contains part group element, which can be staves as well as other part groups. This loop can be seen in red in \figureRef{The MSR classes hierarchy}.


% -------------------------------------------------------------------------
\subsection{Pre-declarations}
% -------------------------------------------------------------------------

Handling such a \Main{mutual dependency} in C++ is rather easy if both classes are declared in one and the same \code{.h} file, with a \Main{pre-declaration} of one of them before the other one is declared, such as:
\begin{lstlisting}[language=CPlusPlus]
class   msrPartGroup;
typedef SMARTP<msrPartGroup> S_msrPartGroup;
\end{lstlisting}

All one can do with a forward declared type is to declare a {\it pointer} or {\it reference} to said type. In particular, this precludes a forward declared type to be that of a class member or the return type of a class method.

%%%JMI I had a similar error message which i resolved by moving the #includes lower down the header file so that the class definition and method definition that the other #includes desire come earlier than the other header includes.

As a matter of taste, \mf\ follows \libmusicxml\ by declaring the \functionName{create} static class methods this way:
\begin{lstlisting}[language=CPlusPlus]
//______________________________________________________________________________
class EXP msrBarLine : public msrMeasureElement
{
  public:

    // creation from MusicXML
    // ------------------------------------------------------

    static SMARTP<msrBarLine> create (
\end{lstlisting}

As an alternative, the same could be written as follows, at the cost of an extra pre-declaration for each class:
\begin{lstlisting}[language=CPlusPlus]
class EXP msrBarLine;
typedef   SMARTP<msrBarLine> S_msrBarLine;

class EXP msrBarLine : public msrMeasureElement
{
  public:

    // creation from MusicXML
    // ------------------------------------------------------

    static S_msrBarLine create (...)

  // ... ... ...
};
\end{lstlisting}


There are many classes pre-declations in \mf. We thus use a couple of \fileName{*EnumTypes*} files to contain most of them, in order to avoid numerous header files includes:
\begin{lstlisting}[language=CPlusPlus]
jacquesmenu@macstudio-1:~/JMI_Developpement/musicformats-git-dev/src > look EnumTypes
./representations/msdl/msdlEnumTypes.h
./representations/msdl/msdlEnumTypes.cpp
./representations/msr/msrTupletsEnumTypes.h
./representations/msr/msrTemposEnumTypes.h
./representations/msr/msrTupletsEnumTypes.cpp
./representations/msr/msrMeasuresEnumTypes.cpp
./representations/msr/msrTemposEnumTypes.cpp
./representations/msr/msrStavesEnumTypes.h
./representations/msr/msrNotesEnumTypes.h
./representations/msr/msrNotesEnumTypes.cpp
./representations/msr/msrStavesEnumTypes.cpp
./representations/msr/msrMeasuresEnumTypes.h
./representations/msr/msrRepeatsEnumTypes.cpp
./representations/msr/msrRepeatsEnumTypes.h
./representations/lpsr/lpsrEnumTypes.cpp
./representations/lpsr/lpsrEnumTypes.h
./representations/bsr/bsrEnumTypes.cpp
./representations/bsr/bsrEnumTypes.h
\end{lstlisting}


% -------------------------------------------------------------------------
\subsection{Simple mutual dependency using separate header files}
% -------------------------------------------------------------------------

A \className{msrPartGroupElement} is either a \className{msrPart} or another \className{msrPartGroupElement}, since the latter can be nested.

A \className{msrPartGroupElement} cannot contain an uplink to an instance of \className{S_msrPartGroup}: pre-declaring \type{S_msrPartGroup}:
\begin{lstlisting}[language=CPlusPlus]
class   msrPartGroup;
typedef SMARTP<msrPartGroup> S_msrPartGroup;
\end{lstlisting}

and using it as the type of a class member common to all \className{msrPartGroupElement} \subClass es leads to the following error:
\begin{lstlisting}[language=Terminal]
error: member access into incomplete type 'MusicFormats::msrPartGroup'
\end{lstlisting}

So \msr{msrPartGroupsElements.h} contains only:
\begin{lstlisting}[language=CPlusPlus]
/*
  Parts and part groups can be found in part groups,
  hence class   msrPartGroupElement
*/

class EXP msrPartGroupElement : public msrElement
/*
  a purely virtual common ancestor to the msrPartGroup and msrPart classes,
  which can be inside an msrPartGroup
*/
{
	// ... ... ...

  private:

    // private fields
    // ------------------------------------------------------

    /*
      The part group uplink is declared in the sub-classes,
      i.e. msrPart and msrPartGroup,
      to allow for separate *.h files, C++ constraint
    */
};
\end{lstlisting}

Then a \className{msrPartGroup}, a \subClass\ of \className{msrPartGroupElement} declared in \msr{msrPartGroups.h}, can contain such an uplink to a \className{msrPartGroup} instance:
\begin{lstlisting}[language=CPlusPlus]
#include "msrPartGroupElements.h"

// ... ... ...

class   msrPartGroup;
typedef SMARTP<msrPartGroup> S_msrPartGroup;

// ... ... ...

class EXP msrPartGroup : public msrPartGroupElement
{
	// ... ... ...

  private:

    // private fields
    // ------------------------------------------------------

    // upLinks

    S_msrPartGroup        fPartGroupUpLinkToPartGroup;
                            // part groups can be nested

		// ... ... ...

    // allowing for both parts and (sub-)part groups as elements

    std::list<S_msrPartGroupElement>
                          fPartGroupElementsList;
};
\end{lstlisting}

\Type{S_msrPartGroup} is used in the declaration of field \fieldName{fPartGroupElementsList}, hence its pre-declaration.

\Class{msrPart}, another \subClass of \className{msrPartGroupElement}, in defined in \msr{msrParts.h} this way:
\begin{lstlisting}[language=CPlusPlus]
#include "msrPartGroupElements.h"

// ... ... ...

class EXP msrPart : public msrPartGroupElement
{
	// ... ... ...

  public:

    // set and get
    // ------------------------------------------------------

    // upLinks

    void                  setPartUpLinkToPartGroup (
                            const S_msrPartGroup& partGroup)
                              { fPartUpLinkToPartGroup = partGroup; }

    S_msrPartGroup        getPartUpLinkToPartGroup () const
                              { return fPartUpLinkToPartGroup; }
		// ... ... ...

  private:

    // private fields
    // ------------------------------------------------------

    // upLinks

    S_msrPartGroup        fPartUpLinkToPartGroup;

		// ... ... ...
};
\end{lstlisting}

%This works fine since \field{msrPart}{fPartUpLinkToPartGroup} is defined in \msr{msrParts.h} itself. %%% JMI
This rather complex situation is depicted at the top of \figureRef{The MSR classes hierarchy}.


% -------------------------------------------------------------------------
\subsection{More complex mutual dependencies}
% -------------------------------------------------------------------------

The case of notes, chords, tuplets and grace notes groups is more intricate:
\begin{itemize}
\item a note can be standalone in a measure;
\item a note can be part of:
  \begin{itemize}
  \item a chord;
  \item a tuplet;
  \item a grace notes group;
  \item a double tremolo;
  \end{itemize}

\item a chord can be standalone in a measure;
\item a chord can be part of:
  \begin{itemize}
  \item a tuplet;
  \item a grace notes group;
  \end{itemize}

\item a tuplet can be standalone in a measure;
\item a tuplet can be part of:
  \begin{itemize}
  \item another tuplet;
  \end{itemize}

\item a grace notes group is attached to:
  \begin{itemize}
  \item a note;
  \end{itemize}

\item a double tremolo is standalone in a measure.
\end{itemize}

\Class{msrDoubleTremolo} is a \subClass\ of \className{msrMeasureElement}.

Regarding classes \className{msrNote}, \className{msrChord} and \className{msrTuplet}:
\begin{itemize}
\item they have to be sub-classes of \class{msrMeasureElement} in some way, since they can be standalone in a \className{msrMeasure} instance;

\item they should be sub-classes of \className{msrTupletElement}, since they can be members of a \className{msrTuplet} instance.
\end{itemize}

This leads to the following hierarchy:
\begin{itemize}
\item \className{msrNote}, \className{msrChord} and \className{msrTuplet} are direct sub-classes of \class{msrTupletElement}

\item \class{msrTupletElement} is a direct \subClass\ of \className{msrMeasureElement}.
\end{itemize}

This rather complex situation is depicted at the bottom of \figureRef{The MSR classes hierarchy}.

But then, the mutual dependency of need a more complex representation.


% -------------------------------------------------------------------------
\section{Templates and functional programming usage}
% -------------------------------------------------------------------------

There are currently few templates in the \mf\ code base, namely:
\begin{itemize}
\item some are used by the two-phase visitors pattern, see \chapterRef{The two-phase visitors pattern};

\item some exist for enumeration types, such as:
\begin{lstlisting}[language=CPlusPlus]
template< typename T >
class   EnumNonSeparators
{
  public:

    class   Iterator
    {
      public:

        Iterator (int value)
          : fIterationIndex (value)
            {}

        T operator* (void) const
            { return (T) fIterationIndex; }

        void operator++ (void)
            {  ++fIterationIndex; }

        Bool operator != (Iterator rhs)
            {  return fIterationIndex != rhs.fIterationIndex; }

      private:

        int fIterationIndex;
     };
};

template< typename T >
typename EnumNonSeparators<T>::Iterator begin (EnumNonSeparators<T>)
{
  return typename EnumNonSeparators<T>::Iterator ((int)T::NonSeparatorsFirst);
}

template< typename T >
typename EnumNonSeparators<T>::Iterator end (EnumNonSeparators<T>)
{
  return typename EnumNonSeparators<T>::Iterator (((int)T::NonSeparatorsLast) + 1);
}
\end{lstlisting}


\item some are used by the code created by \bison, like:
\begin{lstlisting}[language=CPlusPlus]
    /// Construct and fill.
    template <typename T>
    value_type (YY_RVREF (T) t)
      : yytypeid_ (&typeid (T))
    {
      ISCM_ASSERT (sizeof (T) <= size);
      new (yyas_<T> ()) T (YY_MOVE (t));
    }
\end{lstlisting}

\end{itemize}

There could be more templates use once \mf\ reaches a rather stable code base and it is clear what parts of it can be restructured with generic code.

In the same vein, there is little use as of this writing of higher-level facilities such as \code{lambda} and \Main{functor}s.

