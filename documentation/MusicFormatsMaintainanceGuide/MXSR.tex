% -------------------------------------------------------------------------
%  MusicFormats Library
%  Copyright (C) Jacques Menu 2016-2023

%  This Source Code Form is subject to the terms of the Mozilla Public
%  License, v. 2.0. If a copy of the MPL was not distributed with this
%  file, you can obtain one at http://mozilla.org/MPL/2.0/.

%  https://github.com/jacques-menu/musicformats
% -------------------------------------------------------------------------

% !TEX root = MusicFormatsMaintainanceGuide.tex

% -------------------------------------------------------------------------
\chapter{MusicXML Scores Representation (MXSR)}
% -------------------------------------------------------------------------

This format is provided by \libmusicxml, even though \fober\ didn't give it that name. It is a tree of \class{mxmlelement} nodes, mapped one to one to the \mxml\ markups.

The files in \fileName{libmusicxml/src}.

A set of interface functions\index{functions} is contained in \mxsrBoth{mxsr}:
\begin{lstlisting}[language=Terminal]
jacquesmenu@macmini: ~/musicformats-git-dev/src > ll formats/mxsr/
total 48
 0 drwxr-xr-x   6 jacquesmenu  staff   192 May 26 08:20:55 2021 ./
 0 drwxr-xr-x  10 jacquesmenu  staff   320 Jun 25 05:39:49 2021 ../
 8 -rw-r--r--@  1 jacquesmenu  staff  3292 Jun  6 06:35:19 2021 mxsr.cpp
 8 -rw-r--r--@  1 jacquesmenu  staff  1555 Jun  6 06:35:19 2021 mxsrGeneration.h
16 -rw-r--r--@  1 jacquesmenu  staff  7781 Jun  6 06:35:19 2021 mxsrOah.cpp
16 -rw-r--r--@  1 jacquesmenu  staff  4829 Jun  6 06:35:19 2021 mxsrOah.h
\end{lstlisting}


% -------------------------------------------------------------------------
\section{MusicXML elements and attributes}
% -------------------------------------------------------------------------

\mxml\ data contains so-called elements, written as {\tt <...~/>} markups, that can be nested:
\begin{lstlisting}[language=MusicXML]
			<system-margins>
				<left-margin>15</left-margin>
				<right-margin>0</right-margin>
			</system-margins>
\end{lstlisting}
In the example above, the values of the two margins are {\tt 15} and {\tt 0}, respectively.

\mxml\ elements can have attributes, such as {\tt version} below:
\begin{lstlisting}[language=MusicXML]
<score-partwise version="3.1">
\end{lstlisting}

The values of the elements and attributes are strings.

There are two special elements at the beginning of \mxml\ data:
\begin{itemize}
\item a \musicXmlMarkup{?xml} element indicating the characters encoding used;
\item a \musicXmlMarkup{$"!"$DOCTYPE} element telling that the contents is in ’score-partwise’ mode and containing the URL of the DTD.
\end{itemize}

An exemple is:
\begin{lstlisting}[language=MusicXML]
<?xml version="1.0" encoding="UTF-8"?>
<!DOCTYPE score-partwise PUBLIC "-//Recordare//DTD MusicXML 2.0 Partwise//EN"
                                "http://www.musicxml.org/dtds/partwise.dtd">
\end{lstlisting}


% -------------------------------------------------------------------------
\section{The {\tt xmlelement} and {\tt xmlattribute} types}\className{xmlelement}\className{xmlattribute}
% -------------------------------------------------------------------------

These two classes are defined in \elementsBoth{xml}:
\begin{lstlisting}[language=CPlusPlus]
class   xmlelement;
class   xmlattribute;

typedef SMARTP<xmlattribute> 	Sxmlattribute;
typedef SMARTP<xmlelement> 		Sxmlelement;
\end{lstlisting}

class   {\tt xmlattribute} contains:
\begin{lstlisting}[language=CPlusPlus]
//______________________________________________________________________________
class EXP xmlattribute : public smartable {
	//! the attribute name
	std::string	fName;
	//! the attribute value
	std::string 	fValue;

	// ... ... ...

		//! returns the attribute value as a int
		operator int () const;
		//! returns the attribute value as a long
		operator long () const;
		//! returns the attribute value as a float
		operator float () const;
\end{lstlisting}

class   {\tt xmlelement} contains:
\begin{lstlisting}[language=CPlusPlus]
class EXP xmlelement : public ctree<xmlelement>, public visitable
{
	private:
		//! the element name
		std::string fName;
		//! the element value
		std::string fValue;
		//! list of the element attributes
		std::vector<Sxmlattribute> fAttributes;

	protected:
		// the element type
		int fType;
		// the input line number for messages to the user
		int fInputLineNumber;

	// ... ... ...

		//! returns the element value as a long
		operator long () const;
		//! returns the element value as a int
		operator int () const;
		//! returns the element value as a float
		operator float () const;
		//! elements comparison
		Bool operator ==(const xmlelement& elt) const;
		Bool operator !=(const xmlelement& elt) const { return !(*this == elt); }

		//! adds an attribute to the element
		long add (const Sxmlattribute& attr);

	// ... ... ...
};
\end{lstlisting}

Type {\tt Sxmlelement} is a \smart\ to an {\tt xmlelement}, so it is an {\tt xmlelement} tree, since {\tt xmlelement} is a recursive type.

{\tt fInputLineNumber} is used for example in warning and error messages, to help the user locate the problem.

{\tt fType} typically contains a value of some \enumType, more on this below.


% -------------------------------------------------------------------------
\section{Enumeration types for {\tt xmlelement}'s {\tt fType}}\className{xmlelement}\fieldIndex{xmlelement}{fType}
% -------------------------------------------------------------------------

\libmusicxml\ uses {\tt elements/templates/elements.bash}, a Bash script, to generate the \enumType\ constants and classes source code from the \mxml\ DTD. This is not done in the {\tt Makefile}, since it is to be run by hand only once.

The DTD files we use as reference are in {\tt libmusicxml/dtds/3.1/schema};
\begin{lstlisting}[language=Terminal]
jacquesmenu@macmini: ~/musicformats-git-dev/libmusicxml/dtds/3.1/schema > ls -sal *.mod
 40 -rwxr-xr-x@ 1 jacquesmenu  staff  20238 Apr 22 15:49 attributes.mod
 16 -rwxr-xr-x  1 jacquesmenu  staff   4943 Apr 22 15:49 barLine.mod
 80 -rwxr-xr-x@ 1 jacquesmenu  staff  37932 Apr 22 15:49 common.mod
 88 -rwxr-xr-x@ 1 jacquesmenu  staff  41960 Apr 22 15:49 direction.mod
 16 -rwxr-xr-x@ 1 jacquesmenu  staff   4097 Apr 22 15:49 identity.mod
 24 -rwxr-xr-x@ 1 jacquesmenu  staff  10266 Apr 22 15:49 layout.mod
  8 -rwxr-xr-x@ 1 jacquesmenu  staff   2833 Apr 22 15:49 link.mod
104 -rwxr-xr-x@ 1 jacquesmenu  staff  51384 Apr 22 15:49 note.mod
 32 -rwxr-xr-x@ 1 jacquesmenu  staff  15476 Apr 22 15:49 score.mod
\end{lstlisting}

The first result of running \templates{elements.bash} is an anonymous \enumType\ defined in \elements{elements.h}:
\begin{lstlisting}[language=CPlusPlus]
enum class {
	kNoElement,
	kComment,
	kProcessingInstruction,
	k_accent,
	k_accidental,
	k_accidental_mark,
	k_accidental_text,

	// ... ... ...

	k_work,
	k_work_number,
	k_work_title,
	kEndElement
};
\end{lstlisting}

The constants {\tt kNoElement}, {\tt kComment} and {\tt kProcessingInstruction} are added by {\tt elements.bash}.


% -------------------------------------------------------------------------
\section{Classes for the {\tt xmlelement}s}
% -------------------------------------------------------------------------

All the \mxml\ classes are instantiated from the {\tt musicxml} template class, defined in \elements{types.h}. This is where {\tt fType} gets its value:
\begin{lstlisting}[language=CPlusPlus]
template <int elt> class   musicxml : public xmlelement
{
  protected:
    musicxml (int inputLineNumber) : xmlelement (inputLineNumber)	{ fType = elt; }
};
\end{lstlisting}

The \smart\ s to the various elements are defined in \elements{typedef.h}, using an anonymous \enumType:
\begin{lstlisting}[language=CPlusPlus]
typedef SMARTP<musicxml<kComment> >					S_comment;
typedef SMARTP<musicxml<kProcessingInstruction> >	S_processing_instruction;

typedef SMARTP<musicxml<k_accent> >		S_accent;
typedef SMARTP<musicxml<k_accidental> >		S_accidental;
typedef SMARTP<musicxml<k_accidental_mark> >		S_accidental_mark;
typedef SMARTP<musicxml<k_accidental_text> >		S_accidental_text;

// ... ... ...

typedef SMARTP<musicxml<k_work> >		S_work;
typedef SMARTP<musicxml<k_work_number> >		S_work_number;
typedef SMARTP<musicxml<k_work_title> >		S_work_title;
\end{lstlisting}

The two-way correspondance of \mxml\ elements names to type {\tt Sxmlelement} is stored {\tt fMap} and {\tt fType2Name}, defined in \elements{factory.h}:
\begin{lstlisting}[language=CPlusPlus]
class EXP factory : public singleton<factory>{

	std::map<std::string, functor<Sxmlelement>*> fMap;
	std::map<int, const char*>	fType2Name;

	// ... ... ...
};
\end{lstlisting}

Those two maps are initialized in \libmusicxmlSamples{elements/factory.cpp}:
\begin{lstlisting}[language=CPlusPlus]
factory::factory()
{
	fMap["comment"] 		= new newElementFunctor<kComment>;
	fMap["pi"] 				= new newElementFunctor<kProcessingInstruction>;
	fType2Name[kComment] 	= "comment";
	fType2Name[kProcessingInstruction]  = "pi";

	fMap["accent"] 	= new newElementFunctor<k_accent>;
	fMap["accidental"] 	= new newElementFunctor<k_accidental>;
	fMap["accidental-mark"] 	= new newElementFunctor<k_accidental_mark>;
	fMap["accidental-text"] 	= new newElementFunctor<k_accidental_text>;

	// ... ... ...

	fMap["work"] 	= new newElementFunctor<k_work>;
	fMap["work-number"] 	= new newElementFunctor<k_work_number>;
	fMap["work-title"] 	= new newElementFunctor<k_work_title>;

	fType2Name[k_accent] 	= "accent";
	fType2Name[k_accidental] 	= "accidental";
	fType2Name[k_accidental_mark] 	= "accidental-mark";
	fType2Name[k_accidental_text] 	= "accidental-text";

	// ... ... ...

	fType2Name[k_work] 	= "work";
	fType2Name[k_work_number] 	= "work-number";
	fType2Name[k_work_title] 	= "work-title";
}
\end{lstlisting}

class   {\tt newElementFunctor} is defined in {\tt } to provide call operator as:
\begin{lstlisting}[language=CPlusPlus]
template<int elt>
class   newElementFunctor : public functor<Sxmlelement>
{
  public:

    Sxmlelement operator ()()
        { return musicxml<elt>::new_musicxml (libmxmllineno); }
};
\end{lstlisting}


% -------------------------------------------------------------------------
\section{{\tt xmlelement} trees}
% -------------------------------------------------------------------------

This section describes features supplied by \libmusicxml.

An {\tt xmlelement} is the basic brick to represent a \mxml\ element.

Smart pointer type {\tt SXMLFile} is defined in \libFiles{xmlfile.h}:
\begin{lstlisting}[language=CPlusPlus]
//______________________________________________________________________________
class EXP TXMLFile : public smartable
{
  private:
    TXMLDecl*             fXMLDecl;
    TDocType*             fDocType;
    Sxmlelement           fXMLTree;

  protected:
			 TXMLFile () : fXMLDecl(0), fDocType(0) {}
    virtual ~TXMLFile () { delete fXMLDecl; delete fDocType; }

  public:
    static SMARTP<TXMLFile> create();

  public:
    TXMLDecl* 		getXMLDecl ()			{ return fXMLDecl; }
    TDocType* 		getDocType ()			{ return fDocType; }
    Sxmlelement		elements () 			{ return fXMLTree; }

    void 			set (Sxmlelement root)	{ fXMLTree = root; }
    void 			set (TXMLDecl * dec)	{ fXMLDecl = dec; }
    void 			set (TDocType * dt)		{ fDocType = dt; }

    void 			print (std::ostream& s);
};
typedef SMARTP<TXMLFile> SXMLFile;
\end{lstlisting}


% -------------------------------------------------------------------------
	\subsection{Creating {\tt xmlelement} trees from textual data}
% -------------------------------------------------------------------------

Reading \mxml\ data creates instances of {\tt xmlelement}. This is done by and instance of {\tt xmlreader}, defined in \libFilesBoth{xmlreader}, which provides methods:
\begin{lstlisting}[language=CPlusPlus]
		SXMLFile readbuff(const char* file);
		SXMLFile read(const char* file);
		SXMLFile read(FILE* file);
\end{lstlisting}

These three functions\index{functions} are defined this way:
\begin{lstlisting}[language=CPlusPlus]
//_______________________________________________________________________________
SXMLFile xmlreader::readbuff(const char* buffer)
{
	fFile = TXMLFile::create();
	debug("read buffer", '-');
	return readbuffer (buffer, this) ? fFile : 0;
}

//_______________________________________________________________________________
SXMLFile xmlreader::read(const char* file)
{
	fFile = TXMLFile::create();
	debug("read", file);
	return readfile (file, this) ? fFile : 0;
}

//_______________________________________________________________________________
SXMLFile xmlreader::read(FILE* file)
{
	fFile = TXMLFile::create();
	return readstream (file, this) ? fFile : 0;
}
\end{lstlisting}


% -------------------------------------------------------------------------
	\subsection{Printing {\tt xmlelement} trees}
% -------------------------------------------------------------------------

An {\tt xmlelement} can be printed by \function{printMxsr}, defined in \mxsrBoth{mxsr}:
\begin{lstlisting}[language=CPlusPlus]
void printMxsr (const Sxmlelement theMxsr, std::ostream& os)
{
  xmlvisitor v (os);
  tree_browser<xmlelement> browser (&v);
  browser.browse (*theMxsr);
}
\end{lstlisting}

 This how \mxml\ and Guido output are generated.


% -------------------------------------------------------------------------
\section{The {\tt SXMLFile} type}\index[Alphabetica]{SXMLFile}\index[Types]{SXMLFile}
% -------------------------------------------------------------------------

\smartPointerType{SXMLFile} is defined in \factory{musicxmlfactory.h} as a \smart\ to \class{TXMLFile}:
\begin{lstlisting}[language=CPlusPlus]
//______________________________________________________________________________
class EXP TXMLFile : public smartable
{
  private:
    TXMLDecl*             fXMLDecl;
    TDocType*             fDocType;
    Sxmlelement           fXMLTree;

  protected:
			 TXMLFile () : fXMLDecl(0), fDocType(0) {}
    virtual ~TXMLFile () { delete fXMLDecl; delete fDocType; }

  public:
    static SMARTP<TXMLFile> create();

  public:
    TXMLDecl* 		getXMLDecl ()			{ return fXMLDecl; }
    TDocType* 		getDocType ()			{ return fDocType; }
    Sxmlelement		elements () 			{ return fXMLTree; }

    void 			set (Sxmlelement root)	{ fXMLTree = root; }
    void 			set (TXMLDecl * dec)	{ fXMLDecl = dec; }
    void 			set (TDocType * dt)		{ fDocType = dt; }

    void 			print (std::ostream& s);
};
typedef SMARTP<TXMLFile> SXMLFile;
\end{lstlisting}

{\tt fXMLDecl} describes the \musicXmlMarkup{?xml} element and {\tt fDocType} contains the \musicXmlMarkup{$"!"$DOCTYPE} element.


