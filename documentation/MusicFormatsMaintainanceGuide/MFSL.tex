% !TEX root = MusicFormatsMaintainanceGuide.tex

% -------------------------------------------------------------------------
\chapter{MFSL (MusicFormats Scripting Language}
% -------------------------------------------------------------------------

% -------------------------------------------------------------------------
\section{A script example}
% -------------------------------------------------------------------------

This \script\ illustrates the basic features of \mfslLang:
\begin{lstlisting}[language=MFSL]
#!//Users/jacquesmenu/musicformats-git-dev/build/bin/mfsl

# the MusicFormats tool to be used
tool : xml2ly

# the input file
input :test.mfsl

# parts
  -keep-musicxml-part-id P1

# the voices choice
choice VOICES_CHOICE : voice1Only | voice2Only ;
  # could be : choice VOICES_CHOICE : ... ... ... ;

set VOICES_CHOICE = voice1Only ;
  # change this to voice2Only to switch to another subset of options
  # could even be parameter to the script such a $1

# choose which options to use according to VOICES_CHOICE
case VOICES_CHOICE :
	voice1Only:
    -title "Joli morceau - voix 1"
    -ignore-msr-voice Part_POne_Staff_One_Voice_Two
  ;

	voice2Only:
    -title "Joli morceau - voix 2"
    --ignore-msr-voice Part_POne_Staff_One_Voice_One

    -display-options-values

    -global-staff-size 25.5
  ;
;
\end{lstlisting}

This first line of an \mfslLang\ \script\ is the so-called \MainIt{shebang} containing the \filePath\ to the interpreter, allow for running such \script s by their name provided they are made executable.


% -------------------------------------------------------------------------
\section{Implementation principles}
% -------------------------------------------------------------------------

\mfslLang\ is implemented with the \flex\ and \bison\ C++ code generators:
\begin{itemize}
\item \mfsl{mfslScanner.ll} contains the \flex\ lexical description of \mfslLang.

It is used to create \mfsl{mfslScanner.cpp};

\item \mfsl{mfslParser.yy} is the syntax and semantics description of \mfslLang.

From it, \bison\ creates \mfsl{mfslParser.h}, \mfsl{mfslParser.cpp} and \mfsl{mfslParser.output}.

The latter file can be used to check the grammar, in particular if \LR\ conflicts are detected;

\item communication between the code generated this way is done by a so-called \MainIt{driver}, along the lines of the \code{C++-calc} example provided by \bison\ v3.8.1;

\item the way the tokens description is shared by the scanner and parser is described at \sectionRef{Tokens description};

\item the whole power of \oahRepr\ is used to handle the contents of \mfslLang\ \script s as well as the options to the \mfslLangInterp\ itself.
\end{itemize}

Only the predefined \code{bool} type is used, since the generated C++ code relies on this. This is why \methodName{getValue} is used in \clisamples{mfsl.cpp}:
\begin{lstlisting}[language=CPlusPlus]
    string                 theMfTool;
    string                 theInputFile;
    oahOptionsAndArguments optionsAndArguments;

    err =
      launchMfslInterpreter (
        inputSourceName,
        traceScanning.getValue (),
        traceParsing.getValue (),
        displayTokens.getValue (),
        displayNonTerminals.getValue (),
        theMfTool,
        theInputFile,
        optionsAndArguments);
\end{lstlisting}


% -------------------------------------------------------------------------
\section{The contents of the MFSL folder}
% -------------------------------------------------------------------------

\mfsl{location.hh} defines the \className{yy::location} class, that contains the script file name and input line number:
\begin{lstlisting}[language=TerminalSmall]
jacquesmenu@macmini: ~/musicformats-git-dev/src/interpreters/mfsl > ls -sal
total 776
  0 drwxr-xr-x  28 jacquesmenu  staff    896 Mar 15 05:16 .
  0 drwxr-xr-x@  4 jacquesmenu  staff    128 Mar 13 00:47 ..
 16 -rw-r--r--@  1 jacquesmenu  staff   6148 Mar 14 10:18 .DS_Store
  8 -rw-r--r--@  1 jacquesmenu  staff   1266 Mar 15 05:15 Makefile
 16 -rw-r--r--@  1 jacquesmenu  staff   7864 Mar 15 05:16 location.hh
 24 -rw-r--r--@  1 jacquesmenu  staff  10106 Mar 14 18:26 mfslBasicTypes.cpp
 16 -rw-r--r--@  1 jacquesmenu  staff   4568 Mar 14 18:16 mfslBasicTypes.h
  8 -rw-r--r--@  1 jacquesmenu  staff   1585 Mar 15 05:12 mfslDriver.cpp
  8 -rw-r--r--@  1 jacquesmenu  staff   3413 Mar 15 05:12 mfslDriver.h
  8 -rw-r--r--@  1 jacquesmenu  staff   3041 Mar  9 07:35 mfslInterpreterComponent.cpp
  8 -rw-r--r--@  1 jacquesmenu  staff    661 Mar  9 07:02 mfslInterpreterInterface.h
 24 -rw-r--r--@  1 jacquesmenu  staff  11981 Mar 10 11:38 mfslInterpreterInsiderHandler.cpp
 16 -rw-r--r--@  1 jacquesmenu  staff   5270 Mar 10 07:11 mfslInterpreterInsiderHandler.h
  8 -rw-r--r--@  1 jacquesmenu  staff   1161 Mar 15 05:13 mfslInterpreterInterface.h
 16 -rw-r--r--@  1 jacquesmenu  staff   7116 Mar 14 15:53 mfslInterpreterOah.cpp
 16 -rw-r--r--@  1 jacquesmenu  staff   4692 Mar 14 15:51 mfslInterpreterOah.h
 24 -rw-r--r--@  1 jacquesmenu  staff  10070 Mar 14 15:53 mfslInterpreterRegularHandler.cpp
  8 -rw-r--r--@  1 jacquesmenu  staff   3533 Mar  9 08:22 mfslInterpreterRegularHandler.h
 88 -rw-r--r--   1 jacquesmenu  staff  43880 Mar 15 05:16 mfslParser.cpp
 96 -rw-r--r--   1 jacquesmenu  staff  45868 Mar 15 05:16 mfslParser.h
 24 -rw-r--r--@  1 jacquesmenu  staff  10722 Mar 13 16:57 mfslParser.output
 16 -rw-r--r--@  1 jacquesmenu  staff   5930 Mar 14 18:19 mfslParser.yy
136 -rw-r--r--   1 jacquesmenu  staff  68514 Mar 15 05:16 mfslScanner.cpp
 24 -rw-r--r--@  1 jacquesmenu  staff  11251 Mar 15 05:12 mfslScanner.ll
144 -rw-r--r--@  1 jacquesmenu  staff  71091 Mar 15 05:14 mfslScanner.log
  8 -rw-r--r--@  1 jacquesmenu  staff   2047 Mar  9 11:45 mfslWae.cpp
  8 -rw-r--r--@  1 jacquesmenu  staff   3681 Mar  9 11:44 mfslWae.h
  8 -rwxr-xr-x@  1 jacquesmenu  staff    817 Mar 14 18:20 test.mfsl
\end{lstlisting}


% -------------------------------------------------------------------------
\section{The MFSL basic types}
% -------------------------------------------------------------------------

%%%JMI


% -------------------------------------------------------------------------
\section{The MFSL Makefile}\label{The MFSL Makefile}
% -------------------------------------------------------------------------

This \Makefile\ is quite simple: the options to \flex\ and \bison\ are placed in \mfsl{mfslScanner.ll} and \mfsl{mfslParser.yy}, respectively:
\begin{lstlisting}[language=Terminal]
jacquesmenu@macmini: ~/musicformats-git-dev/src/interpreters/mfsl > cat Makefile
# ... ... ...

# variables
# ---------------------------------------------------------------------------

MAKEFILE = Makefile

GENERATED_FILES  = mfslParser.h mfslScanner.cpp mfslParser.cpp

BISON = bison
FLEX  = flex

CXXFLAGS = -I.. -DMAIN


# implicit target
# ---------------------------------------------------------------------------

all : $(GENERATED_FILES)


# generation rules
# ---------------------------------------------------------------------------

mfslScanner.cpp : $(MAKEFILE) mfslScanner.ll
	$(FLEX) -omfslScanner.cpp mfslScanner.ll


mfslParser.h mfslParser.cpp : $(MAKEFILE) mfslParser.yy
	$(BISON) --defines=mfslParser.h -o mfslParser.cpp mfslParser.yy


# clean
# ---------------------------------------------------------------------------

clean:
	rm -f $(GENERATED_FILES)
\end{lstlisting}


% -------------------------------------------------------------------------
\section{Locations handling}\label{Locations handling}
% -------------------------------------------------------------------------

%%%JMI


% -------------------------------------------------------------------------
\section{Tokens description}\label{Tokens description}
% -------------------------------------------------------------------------

The tokens are described in \mfsl{mfslParser.yy}, such as:
\begin{lstlisting}[language=Bison]
%token <string> OPTION "option"
\end{lstlisting}

Both \code{OPTION} and \code{"option"} can be used in the productions, but the grammar is more readable if the capitalized name is used:
\begin{lstlisting}[language=Bison]
Option
  : OPTION
    {
	    if (drv.getDisplayNonTerminals ()) {
        gLogStream <<
          "  ==> option " << $1 <<
          endl << endl;
			}

			$$ = oahOptionNameAndValue::create ($1, "");
    }

  | OPTION OptionValue
    {
      if (drv.getDisplayNonTerminals ()) {
        gLogStream <<
          "  ==> option " << $1 << ' ' << $2 <<
          endl << endl;
      }

      $$ = oahOptionNameAndValue::create ($1, $2);
    }
;
\end{lstlisting}

In case of error, \code{"option"} is used to display a message to the user.

The name \code{OPTION} is used in \mfsl{mfslScanner.ll} prefixed by \code{yy::parser::make_}:
\begin{lstlisting}[language=Flex]
"--"{name} |
"-"{name} {
  if (drv.getTraceTokens ()) {
    gLogStream << "--> " << drv.getScannerLocation () <<
    ": option [" << yytext << "]" <<
    endl;
  }
  return yy::parser::make_OPTION (yytext, loc);
}
\end{lstlisting}

The suffix after \code{make_} has to be defined in the \mfsl{mfslParser.yy} for this to do the link between the Flex-generated and Bison-generated code:
\begin{lstlisting}[language=Terminal]
%token <string> OPTION "option"
\end{lstlisting}

In \mfsl{mfslParser.cpp}, this becomes:
\begin{lstlisting}[language=CPlusPlus]
      case symbol_kind::S_OPTION: // "option"
\end{lstlisting}

We don't have to create \method{yy::parser}{make_OPTION} ourselves, though: it is taken care of by Bison itself, since it returns a \type{char*}.

The \code{calc++} example in the \bison\ documentation contains the case of numbers:
\begin{lstlisting}[language=Flex]
%{
  // A number symbol corresponding to the value in S.
  yy::parser::symbol_type
  make_NUMBER (const std::string &s, const yy::parser::location_type& loc);
%}

// ... ... ...

yy::parser::symbol_type
make_NUMBER (const std::string &s, const yy::parser::location_type& loc)
{
  errno = 0;
  long n = strtol (s.c_str(), NULL, 10);
  if (! (INT_MIN <= n && n <= INT_MAX && errno != ERANGE))
    throw yy::parser::syntax_error (loc, "integer is out of range: " + s);
  return yy::parser::make_NUMBER ((int) n, loc);
}
\end{lstlisting}


% -------------------------------------------------------------------------
\section{The driver}
% -------------------------------------------------------------------------

\mfsl{mfslDriver.h} contains a prototype of \function{yylex}:
\begin{lstlisting}[language=CPlusPlus]
//______________________________________________________________________________
// Give Flex the prototype of yylex we want ...
# define YY_DECL \
  yy::parser::symbol_type yylex (mfslDriver& drv)
// ... and declare it for the parser's sake.
YY_DECL;
\end{lstlisting}

Then it contains the declaration of \class{mfslDriver}:
\begin{lstlisting}[language=CPlusPlus]
//______________________________________________________________________________
// Conducting the whole scanning and parsing of MFSL
class mfslDriver
{
  public:

    // // constructor/destructor
    // ------------------------------------------------------

                          mfslDriver (
                              bool traceScanning,
                              bool tTraceParsing,
                              bool displayTokens,
                              bool displayNonTerminals);

	// ... ... ...

  public:

    // public services
    // ------------------------------------------------------

    // run the parser on file inputFileName,
    // return 0 on success

    int                   parseFile (const string& inputFileName);

    // handling the scanner

    void                  scanBegin ();
    void                  scanEnd ();

  protected:

    // public fields
    // ------------------------------------------------------

    // The name of the file being parsed.
    string                fInputFileName;

    // scanning
    bool                  fTraceScanning;
    yy::location          fScannerLocation;

    // parsing
    bool                  fTraceParsing;

    // display
    bool                  fTraceTokens;
    bool                  fDisplayNonTerminals;

    // variables handling
    map<string, int>      fScannerVariables;

		string                fToolName;

    int                   fDriverResult;
};
\end{lstlisting}

The definitions are placed in two files due to the specificity of the sharing of variables and function in the \flex\ and \bison-generated code:
\begin{itemize}

\item \mfsl{mfslDriver.cpp} contains the \constructorName{mfslDriver} constructor, that runs the parser:
\begin{lstlisting}[language=CPlusPlus]
mfslDriver::mfslDriver (
  bool traceScanning,
  bool tTraceParsing,
  bool displayTokens,
  bool displayNonTerminals)
{
  fTraceParsing  = traceScanning;
  fTraceScanning = tTraceParsing;

  fTraceTokens       = displayTokens;
  fDisplayNonTerminals = displayNonTerminals;

  fScannerVariables ["one"] = 1;
  fScannerVariables ["two"] = 2;
}

int mfslDriver::parseFile (const string &inputFileName)
{
  // input file name
  fInputFileName = inputFileName;

  // initialize scanner location
  fScannerLocation.initialize (
    &fInputFileName);

  // begin scan
  scanBegin ();

  // do the parsing
  yy::parser theParser (*this);

  theParser.set_debug_level (
    fTraceParsing);

  int result = theParser ();

  // end scan
  scanEnd ();

  return result;
}
\end{lstlisting}

\item the remaining code is placed in the third part of (service code) of \mfsl{mfslScanner.ll}, since it needs to access variables in the code generated by \flex:
\begin{lstlisting}[language=CPlusPlus]
void mfslDriver::scanBegin ()
{
  yy_flex_debug = fTraceScanning;

  if (fInputFileName.empty () || fInputFileName == "-") {
    yyin = stdin;
  }

  else if (!(yyin = fopen (fInputFileName.c_str (), "r")))
    {
      std::cerr <<
        "cannot open " << fInputFileName << ": " << strerror (errno) <<
        '\n';

      exit (EXIT_FAILURE);
    }
}

void mfslDriver::scanEnd ()
{
  fclose (yyin);
}
\end{lstlisting}

\end{itemize}


% -------------------------------------------------------------------------
\section{Lexical analysis}
% -------------------------------------------------------------------------

% -------------------------------------------------------------------------
\subsection{Flex options}
% -------------------------------------------------------------------------

\begin{lstlisting}[language=Flex]
%option interactive
%option debug

%option yylineno
%option noyywrap

%option nounput noinput
\end{lstlisting}


% -------------------------------------------------------------------------
\subsection{Flex regular expressions}
% -------------------------------------------------------------------------

\begin{lstlisting}[language=Flex]
blank                     [ \t\r]
endOfLine                 [\n]
character                 .

letter                    [A-Za-zéèêàâòôùûî]
digit                     [0-9]

name                      {letter}(_|-|\.|{letter}|{digit})*
integer                   {digit}+
exponent                  [eE][+-]?{integer}

singleQuote             [']
doubleQuote               ["]
tabulator                 [\t]
backSlash                 [\\]

... ... ...

%{
  // Code run each time a pattern is matched.
  # define YY_USER_ACTION  loc.columns (yyleng);
%}
\end{lstlisting}


\begin{lstlisting}[language=Flex]
%x                        SINGLE_QUOTED_STRING_MODE
%x                        DOUBLE_QUOTED_STRING_MODE

%x                        COMMENT_TO_END_OF_LINE_MODE
%x                        PARENTHESIZED_COMMENT_MODE
\end{lstlisting}

\begin{lstlisting}[language=Flex]
/* strings */

#define                   STRING_BUFFER_SIZE 1024
char                      pStringBuffer [STRING_BUFFER_SIZE];

  // A handy shortcut to the location held by the mfslDriver.
  yy::location& loc = drv.getScannerLocation ();
  // Code run each time yylex is called.
  loc.step ();
\end{lstlisting}

\begin{lstlisting}[language=Flex]
{blank}+   loc.step ();
\n+        loc.lines (yyleng); loc.step ();
\end{lstlisting}


\begin{lstlisting}[language=Flex]
{integer}"."{integer}({exponent})? |
{integer}{exponent} {
  if (drv.getTraceTokens ()) {
    gLogStream <<
    	"--> " << drv.getScannerLocation () <<
			" double: " << yytext <<
			endl;
  }
  return yy::parser::make_DOUBLE (yytext, loc);
}

{integer} {
  if (drv.getTraceTokens ()) {
    gLogStream <<
    	"--> " << drv.getScannerLocation () <<
			" integer: " << yytext <<
			endl;
  }
  return yy::parser::make_INTEGER (yytext, loc);
}
\end{lstlisting}


\begin{lstlisting}[language=Flex]
"tool" {
  if (drv.getTraceTokens ()) {
    gLogStream <<
    	"--> " << drv.getScannerLocation () << ": " << yytext <<
			endl;
  }
  return yy::parser::make_TOOL (loc);
}
\end{lstlisting}


\begin{lstlisting}[language=Flex]
{name} {
  if (drv.getTraceTokens ()) {
    gLogStream << "--> " << drv.getScannerLocation () <<
    ": name [" << yytext << "]" <<
    endl;
  }
  return yy::parser::make_NAME (yytext, loc);
}



"--"{name} |
"-"{name} {
  if (drv.getTraceTokens ()) {
    gLogStream << "--> " << drv.getScannerLocation () <<
    ": option [" << yytext << "]" <<
    endl;
  }
  return yy::parser::make_OPTION (yytext, loc);
}



"(" {
  if (drv.getTraceTokens ()) {
    gLogStream <<
    	"--> " << drv.getScannerLocation () << ": " << yytext <<
			endl;
  }
  return yy::parser::make_LEFT_PARENTHESIS (loc);
}
\end{lstlisting}


\begin{lstlisting}[language=Flex]
. {
   throw yy::parser::syntax_error (
     loc,
     "invalid character: " + std::string (yytext));
}


<<EOF>>    return yy::parser::make_YYEOF (loc);
\end{lstlisting}


% -------------------------------------------------------------------------
\section{Syntax and semantic analysis}
% -------------------------------------------------------------------------


% -------------------------------------------------------------------------
\subsection{{\tt Bison} options}
% -------------------------------------------------------------------------

\begin{lstlisting}[language=Bison]
%skeleton "lalr1.cc" // -*- C++ -*-
%require "3.8.1"
%defines

%printer { yyo << $$; } <*>;

%define api.token.raw

%define api.token.constructor
%define api.value.type variant
%define parse.assert

%code requires {
  #include <string>

  class mfslDriver;
}

// The parsing context
%param { mfslDriver& drv } // declaration, any parameter name is fine

%locations

%verbose // to produce mfslParser.output

%define parse.trace
%define parse.error detailed
%define parse.lac full

//%define parse.report
//%define parse.verbose

%code {
# include "mfslDriver.h"
}
\end{lstlisting}

\begin{lstlisting}[language=Bison]
%define api.token.prefix {TOK_}
%token
  BAR  "|"
  SEMICOLON ";"
  COLON ":"
  EQUAL  "="

  SLASH  "/"
  COMMA  ","

  STAR  "*"

  LEFT_PARENTHESIS  "("
  RIGHT_PARENTHESIS ")"

  TOOL    "tool"
  INPUT   "input"
  CHOICE  "choice"
  SET     "set"
  CASE    "case"
;

%token <std::string> INTEGER "integer"
%token <std::string> DOUBLE  "double"

%token <std::string> SINGLE_QUOTED_STRING "single quoted_string"
%token <std::string> DOUBLE_QUOTED_STRING "double quoted_string"

%token <std::string> NAME "name"

%token <std::string> OPTION "option"
\end{lstlisting}


% -------------------------------------------------------------------------
\subsection{{\tt Bison} regular expressions}
% -------------------------------------------------------------------------

\begin{lstlisting}[language=Bison]
/* the MFSL non-terminals */
/* ---------------------- */

%nterm <std::string> Number
%nterm <std::string> String
%nterm <std::string> OptionValue



/* the MFSL axiom */
/* -------------- */

%start script
\end{lstlisting}


% -------------------------------------------------------------------------
\section{Interface to the MFSL parser}
% -------------------------------------------------------------------------

This is provided by \mfsl{mfslInterpreterInterface.h}:
\begin{lstlisting}[language=Terminal]
EXP extern mfMusicformatsError launchMfslInterpreter (
  const string&           inputSourceName,
  bool                    traceScanning,
  bool                    traceParsing,
  bool                    displayTokens,
  bool                    displayNonTerminals,
  string&                 theMfTool,
  string&                 theInputFile,
  oahOptionsAndArguments& optionsAndArguments);
\end{lstlisting}

The definition of this function is placed in \mfsl{mfslScanner.ll}:
\begin{lstlisting}[language=Terminal]
mfMusicformatsError launchMfslInterpreter (
  const string&           inputSourceName,
  bool                    traceScanning,
  bool                    traceParsing,
  bool                    displayTokens,
  bool                    displayNonTerminals,
  string&                 theMfTool,
  string&                 theInputFile,
  oahOptionsAndArguments& optionsAndArguments)
{
  mfMusicformatsError
    result =
      mfMusicformatsError::k_NoError;

  mfslDriver
    theDriver (
      traceScanning,
      traceParsing,
      displayTokens,
      displayNonTerminals);

  int parseResult =
  	theDriver.parseFile (inputSourceName);

  gLogStream <<
    "--> parseResult: " << parseResult <<
    endl;

  if (! parseResult) {
    result =
      mfMusicformatsError::kErrorInvalidFile;
  }

	gLogStream <<
    "inputFileName: " << theDriver.getInputFileName () <<
  	endl <<
		"toolName: " << theDriver.getToolName () <<
  	endl;

  theMfTool    = theDriver.getToolName ();
  theInputFile = theDriver.getInputFileName ();

	return result;
}
\end{lstlisting}


% -------------------------------------------------------------------------
\section{Running the example MFSL script}
% -------------------------------------------------------------------------

Let's show show the \mfslLangInterp\ uses the options above:
\begin{lstlisting}[language=Terminal]
jacquesmenu@macmini: ~/musicformats-git-dev/src/interpreters/mfsl > ./test.mfsl -display-tokens -display-non-terminals -display-options-values
  The options values for //Users/jacquesmenu/musicformats-git-dev/build/bin/mfsl are:
    MFSL group (-help-mfsl-group, -hmfsl-group), 2 atoms chosen:
    --------------------------
      MFSL (-help-mfsl, -hmfsl), 2 atoms chosen:
        fTraceTokens                     : true, set by user
        fDisplayNonTerminals               : true, set by user

    Options and help group (-help-oah-group, -hoah-group), 1 atom chosen:
    --------------------------
      Options and help (-help-oah, -hoah), 1 atom chosen:
        fDisplayOptionsValues              : true, set by user


--> ./test.mfsl:2.1-39: tool
--> ./test.mfsl:2.41: :
--> ./test.mfsl:2.43-48: name [xml2ly]
  ==> tool: xml2ly

--> ./test.mfsl:4.1-22: input
--> ./test.mfsl:4.24: :
--> ./test.mfsl:4.25-33: name [test.mfsl]
  ==> input: test.mfsl

--> test.mfsl:6.11-32: option [-keep-musicxml-part-id]
--> test.mfsl:6.34-35: name [P1]
  ==> option -keep-musicxml-part-id P1

--> test.mfsl:8.1-26: choice
--> test.mfsl:8.28-40: name [VOICES_CHOICE]
--> test.mfsl:8.42: :
--> test.mfsl:8.44-53: name [voice1Only]
--> test.mfsl:8.55: |
--> test.mfsl:8.57-66: name [voice2Only]
--> test.mfsl:8.68: ;
  ==> ChoiceDeclaration VOICES_CHOICE : ...

--> test.mfsl:10.1-3: set
--> test.mfsl:10.5-17: name [VOICES_CHOICE]
--> test.mfsl:10.19: =
--> test.mfsl:10.21-30: name [voice1Only]
--> test.mfsl:10.32: ;
  ==> ChoiceSetting, set VOICES_CHOICE = voice1Only

--> test.mfsl:12.1-61: case
--> test.mfsl:12.63-75: name [VOICES_CHOICE]
--> test.mfsl:12.77: :
--> test.mfsl:13.2-11: name [voice1Only]
--> test.mfsl:13.12: :
--> test.mfsl:14.5-10: option [-title]
--> test.mfsl:14.12-34: double quoted string ["]
  ==> option -title "

--> test.mfsl:15.5-21: option [-ignore-msr-voice]
--> test.mfsl:15.23-51: name [Part_POne_Staff_One_Voice_Two]
  ==> option -ignore-msr-voice Part_POne_Staff_One_Voice_Two

--> test.mfsl:16.3: ;
  ==> Case voice1Only : ...

--> test.mfsl:18.2-11: name [voice2Only]
--> test.mfsl:18.12: :
--> test.mfsl:19.5-10: option [-title]
--> test.mfsl:19.12-34: double quoted string ["]
  ==> option -title "

--> test.mfsl:20.5-22: option [--ignore-msr-voice]
--> test.mfsl:20.24-52: name [Part_POne_Staff_One_Voice_One]
  ==> option --ignore-msr-voice Part_POne_Staff_One_Voice_One

--> test.mfsl:22.5-27: option [-display-options-values]
--> test.mfsl:24.5-22: option [-global-staff-size]
  ==> option -display-options-values

--> test.mfsl:24.24-27 double: 25.5
  ==> option -global-staff-size 25.5

--> test.mfsl:25.3: ;
  ==> Case voice2Only : ...

--> test.mfsl:26.1: ;
  ==> CaseStatement, VOICES_CHOICE : ...

--> parseResult: 0
==> inputFileName: test.mfsl
==> toolName:      xml2ly
jacquesmenu@macmini: ~/musicformats-git-dev/src/interpreters/mfsl >
\end{lstlisting}


% -------------------------------------------------------------------------
\subsection{Error recovery}
% -------------------------------------------------------------------------

The \mfslLangInterp\ uses a variant of the \MainIt{stopper sets} method that was present in the early Pascal and Pascal-S converters. The latter passed a set of tokens not to be overtaken to the procedures in charge of accepting the various statements in the language. Strangely enough, this was not done for declarations.

We use a stack of tokens sets that grows and shrinks in parallel with the accepting functions\index{functions}, to know more contextual informations when deciding wether to consume a token or not. The corresponding term is {it shift}
when building the analysis tables in LR technology.


