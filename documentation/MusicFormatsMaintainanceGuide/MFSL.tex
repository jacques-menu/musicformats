% !TEX root = MusicFormatsMaintainanceGuide.tex

% -------------------------------------------------------------------------
\chapter{MFSL (MusicFormats Scripting Language}
% -------------------------------------------------------------------------

% -------------------------------------------------------------------------
\section{A script example}
% -------------------------------------------------------------------------

This \script\ illustrates the basic features of \mfslLang:
\begin{lstlisting}[language=MFSL]
#!//Users/jacquesmenu/musicformats-git-dev/build/bin/mfsl

# the MusicFormats tool to be used
tool : xml2ly

# the input file
input :test.mfsl

# parts
  -keep-musicxml-part-id P1

# the voices choice
choice VOICES_CHOICE : voice1Only | voice2Only ;
  # could be : choice VOICES_CHOICE : ... ... ... ;

set VOICES_CHOICE = voice1Only ;
  # change this to voice2Only to switch to another subset of options
  # could even be parameter to the script such a $1

# choose which options to use according to VOICES_CHOICE
case VOICES_CHOICE :
	voice1Only:
    -title "Joli morceau - voix 1"
    -ignore-msr-voice Part_POne_Staff_One_Voice_Two
  ;

	voice2Only:
    -title "Joli morceau - voix 2"
    --ignore-msr-voice Part_POne_Staff_One_Voice_One

    -display-options-values

    -global-staff-size 25.5
  ;
;
\end{lstlisting}

This first line of an \mfslLang\ \script\ is the so-called \MainIt{shebang} containing the \filePath\ to the interpreter, allow for running such \script s by their name provided they are made executable.


% -------------------------------------------------------------------------
\section{Implementation principles}
% -------------------------------------------------------------------------

\mfslLang\ is implemented with the \flex\ and \bison\ C++ code generators:
\begin{itemize}
\item \mfsl{mfslScanner.ll} contains the \flex\ lexical description of \mfslLang.

It is used to create \mfsl{mfslScanner.cpp};

\item \mfsl{mfslParser.yy} is the syntax and semantics description of \mfslLang.

From it, \bison\ creates \mfsl{mfslParser.h}, \mfsl{mfslParser.cpp} and \mfsl{mfslParser.output}.

The latter file can be used to check the grammar, in particular if \LR\ conflicts are detected;

\item communication between the code generated this way is done by a so-called \MainIt{driver}, along the lines of the \code{C++-calc} example provided by \bison\ v3.8.1;

\item the way the tokens description is shared by the scanner and parser is described at \sectionRef{Tokens description};

\item the whole power of \oahRepr\ is used to handle the contents of \mfslLang\ \script s as well as the options to the \mfslInterp\ itself.
\end{itemize}

Only the predefined \code{bool} type is used, since the generated C++ code relies on this. This is why \methodName{getValue} is used in \clisamples{mfsl.cpp}:
\begin{lstlisting}[language=CPlusPlus]
    std::string       theMfTool;
    std::string       theInputFile;
    oahOptionsAndArguments optionsAndArguments;

    err =
      launchMfslInterpreter (
        inputSourceName,
        traceScanning.getValue (),
        traceParsing.getValue (),
        displayTokens.getValue (),
        displayNonTerminals.getValue (),
        theMfTool,
        theInputFile,
        optionsAndArguments);
\end{lstlisting}


% -------------------------------------------------------------------------
\section{The contents of the MFSL folder}
% -------------------------------------------------------------------------

\mfsl{location.hh} defines the \className{yy::location} class, that contains the script file name and input line number:
\begin{lstlisting}[language=TerminalSmall]
jacquesmenu@macmini: ~/musicformats-git-dev/src/interpreters/mfsl > ls -sal
total 776
  0 drwxr-xr-x  28 jacquesmenu  staff    896 Mar 15 05:16 .
  0 drwxr-xr-x@  4 jacquesmenu  staff    128 Mar 13 00:47 ..
 16 -rw-r--r--@  1 jacquesmenu  staff   6148 Mar 14 10:18 .DS_Store
  8 -rw-r--r--@  1 jacquesmenu  staff   1266 Mar 15 05:15 Makefile
 16 -rw-r--r--@  1 jacquesmenu  staff   7864 Mar 15 05:16 location.hh
 24 -rw-r--r--@  1 jacquesmenu  staff  10106 Mar 14 18:26 mfslBasicTypes.cpp
 16 -rw-r--r--@  1 jacquesmenu  staff   4568 Mar 14 18:16 mfslBasicTypes.h
  8 -rw-r--r--@  1 jacquesmenu  staff   1585 Mar 15 05:12 mfslDriver.cpp
  8 -rw-r--r--@  1 jacquesmenu  staff   3413 Mar 15 05:12 mfslDriver.h
  8 -rw-r--r--@  1 jacquesmenu  staff   3041 Mar  9 07:35 mfslInterpreterComponent.cpp
  8 -rw-r--r--@  1 jacquesmenu  staff    661 Mar  9 07:02 mfslInterpreterInterface.h
 24 -rw-r--r--@  1 jacquesmenu  staff  11981 Mar 10 11:38 mfslInterpreterInsiderHandler.cpp
 16 -rw-r--r--@  1 jacquesmenu  staff   5270 Mar 10 07:11 mfslInterpreterInsiderHandler.h
  8 -rw-r--r--@  1 jacquesmenu  staff   1161 Mar 15 05:13 mfslInterpreterInterface.h
 16 -rw-r--r--@  1 jacquesmenu  staff   7116 Mar 14 15:53 mfslInterpreterOah.cpp
 16 -rw-r--r--@  1 jacquesmenu  staff   4692 Mar 14 15:51 mfslInterpreterOah.h
 24 -rw-r--r--@  1 jacquesmenu  staff  10070 Mar 14 15:53 mfslInterpreterRegularHandler.cpp
  8 -rw-r--r--@  1 jacquesmenu  staff   3533 Mar  9 08:22 mfslInterpreterRegularHandler.h
 88 -rw-r--r--   1 jacquesmenu  staff  43880 Mar 15 05:16 mfslParser.cpp
 96 -rw-r--r--   1 jacquesmenu  staff  45868 Mar 15 05:16 mfslParser.h
 24 -rw-r--r--@  1 jacquesmenu  staff  10722 Mar 13 16:57 mfslParser.output
 16 -rw-r--r--@  1 jacquesmenu  staff   5930 Mar 14 18:19 mfslParser.yy
136 -rw-r--r--   1 jacquesmenu  staff  68514 Mar 15 05:16 mfslScanner.cpp
 24 -rw-r--r--@  1 jacquesmenu  staff  11251 Mar 15 05:12 mfslScanner.ll
144 -rw-r--r--@  1 jacquesmenu  staff  71091 Mar 15 05:14 mfslScanner.log
  8 -rw-r--r--@  1 jacquesmenu  staff   2047 Mar  9 11:45 mfslWae.cpp
  8 -rw-r--r--@  1 jacquesmenu  staff   3681 Mar  9 11:44 mfslWae.h
  8 -rwxr-xr-x@  1 jacquesmenu  staff    817 Mar 14 18:20 test.mfsl
\end{lstlisting}


% -------------------------------------------------------------------------
\section{The MFSL basic types}
% -------------------------------------------------------------------------

%%%JMI


% -------------------------------------------------------------------------
\section{The MFSL Makefile}\label{The MFSL Makefile}
% -------------------------------------------------------------------------

This \Makefile\ is quite simple: the options to \flex\ and \bison\ are placed in \mfsl{mfslScanner.ll} and \mfsl{mfslParser.yy}, respectively:
\begin{lstlisting}[language=Terminal]
jacquesmenu@macmini: ~/musicformats-git-dev/src/interpreters/mfsl > cat Makefile
# ... ... ...

# variables
# ---------------------------------------------------------------------------

MAKEFILE = Makefile

GENERATED_FILES  = mfslParser.h mfslScanner.cpp mfslParser.cpp

BISON = bison
FLEX  = flex

CXXFLAGS = -I.. -DMAIN


# implicit target
# ---------------------------------------------------------------------------

all : $(GENERATED_FILES)


# generation rules
# ---------------------------------------------------------------------------

mfslScanner.cpp : $(MAKEFILE) mfslScanner.ll
	$(FLEX) -omfslScanner.cpp mfslScanner.ll


mfslParser.h mfslParser.cpp : $(MAKEFILE) mfslParser.yy
	$(BISON) --defines=mfslParser.h -o mfslParser.cpp mfslParser.yy


# clean
# ---------------------------------------------------------------------------

clean:
	rm -f $(GENERATED_FILES)
\end{lstlisting}


% -------------------------------------------------------------------------
\section{Locations handling}\label{Locations handling}
% -------------------------------------------------------------------------

%%%JMI


% -------------------------------------------------------------------------
\section{Tokens description}\label{Tokens description}
% -------------------------------------------------------------------------

The tokens are described in \mfsl{mfslParser.yy}, such as:
\begin{lstlisting}[language=Bison]
%token <std::string> OPTION "option"
\end{lstlisting}

Both \code{OPTION} and \code{"option"} can be used in the productions, but the grammar is more readable if the capitalized name is used:
\begin{lstlisting}[language=Bison]
Option
  : OPTION
    {
	    if (drv.getDisplayNonTerminals ()) {
        gLog <<
          "  ==> option " << $1 <<
          std::endl << std::endl;
			}

			$$ = oahOptionNameAndValue::create ($1, "");
    }

  | OPTION OptionValue
    {
      if (drv.getDisplayNonTerminals ()) {
        gLog <<
          "  ==> option " << $1 << ' ' << $2 <<
          std::endl << std::endl;
      }

      $$ = oahOptionNameAndValue::create ($1, $2);
    }
;
\end{lstlisting}

In case of error, \code{"option"} is used to display a message to the user.

The name \code{OPTION} is used in \mfsl{mfslScanner.ll} prefixed by \code{yy::parser::make_}:
\begin{lstlisting}[language=Flex]
"--"{name} |
"-"{name} {
  if (drv.getTraceTokens ()) {
    gLog << "--> " << drv.getScannerLocation () <<
    ": option [" << yytext << ']' <<
    std::endl;
  }
  return yy::parser::make_OPTION (yytext, loc);
}
\end{lstlisting}

The suffix after \code{make_} has to be defined in the \mfsl{mfslParser.yy} for this to do the link between the Flex-generated and Bison-generated code:
\begin{lstlisting}[language=Terminal]
%token <std::string> OPTION "option"
\end{lstlisting}

In \mfsl{mfslParser.cpp}, this becomes:
\begin{lstlisting}[language=CPlusPlus]
      case symbol_kind::S_OPTION: // "option"
\end{lstlisting}

We don't have to create \method{yy::parser}{make_OPTION} ourselves, though: it is taken care of by Bison itself, since it returns a \type{char*}.

The \code{calc++} example in the \bison\ documentation contains the case of numbers:
\begin{lstlisting}[language=Flex]
%{
  // A number symbol corresponding to the value in S.
  yy::parser::symbol_type
  make_NUMBER (const std::string &s, const yy::parser::location_type& loc);
%}

// ... ... ...

yy::parser::symbol_type
make_NUMBER (const std::string &s, const yy::parser::location_type& loc)
{
  errno = 0;
  long n = strtol (s.c_str(), NULL, 10);
  if (! (INT_MIN <= n && n <= INT_MAX && errno != ERANGE))
    throw yy::parser::syntax_error (loc, "integer is out of range: " + s);
  return yy::parser::make_NUMBER ((int) n, loc);
}
\end{lstlisting}


% -------------------------------------------------------------------------
\section{The driver}
% -------------------------------------------------------------------------

\Class{mfslDriver} contains everything needed to let the code generated by \flex\ and \bison\ communicate with each other, as well as any work variables needed during the analysis of \mfslLang\ input.
This latter point allows for multiple analyzers to coexist.

\mfsl{mfslDriver.h} contains a prototype of \function{yylex}:
\begin{lstlisting}[language=CPlusPlus]
//______________________________________________________________________________
// Give Flex the prototype of yylex we want ...
# define YY_DECL \
  yy::parser::symbol_type yylex (mfslDriver& drv)
// ... and declare it for the parser's sake.
YY_DECL;
\end{lstlisting}

Then it contains the declaration of \class{mfslDriver}:
\begin{lstlisting}[language=CPlusPlus]
// Conducting the whole scanning and parsing of MFSL
class   mfslDriver
{
  public:

    // constants
    // ------------------------------------------------------

    static const std::string   K_ALL_PSEUDO_LABEL_NAME;

    // // constructor/destructor
    // ------------------------------------------------------

                          mfslDriver ();

    virtual               ~mfslDriver ();

	// ... ... ...

  public:

    // public services
    // ------------------------------------------------------

    // run the parser, return 0 on success
    int                   parseInput_Pass1 ();

    // handling the scanner
    void                  scanBegin ();
    void                  scanEnd ();

	// ... ... ...

  private:

    // private fields
    // ------------------------------------------------------

    // the name of the MusicFormats tool
		std::string           fTool;

    // the name of the MusicFormats script
		std::string           fScriptName;

    // the names of the input sources
    std::list<std::string>
                          fInputSoucesList;


    // scanning
    bool                  fTraceScanning;
    mfsl::location        fScannerLocation;

	// ... ... ...
};
\end{lstlisting}

The definitions are placed in two files due to the specificity of the sharing of variables and function in the \flex\ and \bison-generated code: %%%JMI
\begin{itemize}

\item \mfsl{mfslDriver.cpp} contains \method{mfslDriver}{parseInput_Pass1}, that runs the parser:
\begin{lstlisting}[language=CPlusPlus]
int mfslDriver::parseInput_Pass1 ()
{
  // initialize scanner location
  fScannerLocation.initialize (
    &fScriptName);

  // begin scan
  scanBegin ();

  if (fScriptName.empty () || fScriptName == "-") {
    fScriptName = "stdin"; // nicer for warning and error messages
  }

  // do the parsing
  mfsl::parser theParser (*this);

  theParser.set_debug_level (
    fTraceParsing);

  int parseResult = theParser ();

  // end scan
  scanEnd ();

	// ... ... ...

  // do the final semantics check
  finalSemanticsCheck ();

  return parseResult;
}
\end{lstlisting}

\item the remaining code is placed in the third part (service code) of \mfsl{mfslScanner.ll}, since it needs to access variables in the code generated by \flex:
\begin{lstlisting}[language=CPlusPlus]
void mfslDriver::scanBegin ()
{
  yy_flex_debug = fTraceScanning;

  if (fScriptName.empty () || fScriptName == "-") {
    yyin = stdin;
  }

  else if (!(yyin = fopen (fScriptName.c_str (), "r")))
    {
      std::stringstream ss;

      char*
        errorCString =
          mfStrErrorCString ();

      if (errorString != nullptr) {
        ss <<
          gLanguage->cannotOpenScriptForWriting (fScriptName) <<
          ": " <<
          errorString <<
          std::endl;

        mfslFileError (
          fScriptName,
          ss.str ());
      }
    }
}

void mfslDriver::scanEnd ()
{
  fclose (yyin);
}
\end{lstlisting}

\end{itemize}


% -------------------------------------------------------------------------
\section{Lexical analysis}
% -------------------------------------------------------------------------

The lexical definition of \mfslLang\ in \mfsl{mfslScanner.ll} is described below.


% -------------------------------------------------------------------------
\subsection{Flex options}
% -------------------------------------------------------------------------
The \code{prefix} is used to allow for multiple \flex -generated analyzers to coexist:
\begin{lstlisting}[language=Flex]
%option prefix="mfsl"

%option yylineno

%option noyywrap

%option nounput noinput debug interactive
\end{lstlisting}


% -------------------------------------------------------------------------
\subsection{Flex regular expressions}
% -------------------------------------------------------------------------

The basic ones are:
\begin{lstlisting}[language=Flex]
blank                     [ \t\r]
endOfLine                 [\n]
character                 .

letter                    [A-Za-zéèêàâòôùûî]
digit                     [0-9]

name                      {letter}(_|-|\.|{letter}|{digit})*
integer                   {digit}+
exponent                  [eE][+-]?{integer}

singleleQuote             [']
doubleQuote               ["]
tabulator                 [\t]
backSlash                 [\\]

... ... ...

%{
  // Code run each time a pattern is matched.
  # define YY_USER_ACTION  loc.columns (yyleng);
%}
\end{lstlisting}

Some exclusive modes are used for strings and comments:
\begin{lstlisting}[language=Flex]
%x                        SINGLE_QUOTED_STRING_MODE
%x                        DOUBLE_QUOTED_STRING_MODE

%x                        COMMENT_TO_END_OF_LINE_MODE
%x                        PARENTHESIZED_COMMENT_MODE
\end{lstlisting}

Strings must be stored in a private buffer:
\begin{lstlisting}[language=Flex]
/* strings */

#define                   STRING_BUFFER_SIZE 1024
char                      pStringBuffer [STRING_BUFFER_SIZE];

// A handy shortcut to the location held by the mfslDriver
mfsl::location& loc = drv.getScannerLocationNonConst ();
\end{lstlisting}

Locating the tokens in the the MFSL input text is done with:
\begin{lstlisting}[language=Flex]
// Code run each time yylex() is called
loc.step ();
\end{lstlisting}

This lead for example to:
\begin{lstlisting}[language=Flex]
{blank} {
  loc.step ();
}

{endOfLine} {
  loc.lines (yyleng); loc.step ();
}
\end{lstlisting}

The numbers are handled by:
\begin{lstlisting}[language=Flex]
{integer}"."{integer}({exponent})? |
{integer}{exponent} {
  if (drv.getTraceTokens ()) {
    gLog <<
    	"--> " << drv.getScannerLocation () <<
			" double: " << yytext <<
			std::endl;
  }
  return yy::parser::make_DOUBLE (yytext, loc);
}

{integer} {
  if (drv.getTraceTokens ()) {
    gLog <<
    	"--> " << drv.getScannerLocation () <<
			" integer: " << yytext <<
			std::endl;
  }
  return yy::parser::make_INTEGER (yytext, loc);
}
\end{lstlisting}

The \mfslLang\ keywords are handled with the \code{make_...} facility:
\begin{lstlisting}[language=Flex]
"tool" {
  if (drv.getTraceTokens ()) {
    gLog <<
    	"--> " << drv.getScannerLocation () << ": " << yytext <<
			std::endl;
  }
  return yy::parser::make_TOOL (loc);
}
\end{lstlisting}

The names and the options are handled by:
\begin{lstlisting}[language=Flex]
{name} {
  if (drv.getDisplayTokens ()) {
    gLog << "--> " << drv.getScannerLocation () <<
    ": name [" << yytext << ']' <<
    std::endl;
  }

  loc.begin.column += yyleng;
  loc.step ();

  return
    mfsl::parser::make_NAME (yytext, loc);
}



"--"{name} |
"-"{name} {
  if (drv.getTraceTokens ()) {
    gLog << "--> " << drv.getScannerLocation () <<
    ": option [" << yytext << ']' <<
    std::endl;
  }
  return yy::parser::make_OPTION (yytext, loc);
}



"(" {
  if (drv.getTraceTokens ()) {
    gLog <<
    	"--> " << drv.getScannerLocation () << ": " << yytext <<
			std::endl;
  }
  return yy::parser::make_LEFT_PARENTHESIS (loc);
}
\end{lstlisting}

The catchall rule issues an error message:
\begin{lstlisting}[language=Flex]
. {
   throw mfsl::parser::syntax_error (
     loc,
     "### invalid character: " + std::string (yytext));
}
\end{lstlisting}

And the end of the \mfslLang\ input is handled this way:
\begin{lstlisting}[language=Flex]
<<EOF>> {
  return
    mfsl::parser::make_YYEOF (loc);
}
\end{lstlisting}


% -------------------------------------------------------------------------
\section{Syntax and semantic analysis}
% -------------------------------------------------------------------------


% -------------------------------------------------------------------------
\subsection{{\tt Bison} options for MFSL}
% -------------------------------------------------------------------------

Setting \code{api.prefix} allows for multiple analyzers to coexist:
\begin{lstlisting}[language=Bison]
%skeleton "lalr1.cc" // -*- C++ -*-
%require "3.8.1"
%defines

%define api.prefix {mfsl}

%define api.token.raw

%define api.token.constructor
%define api.value.type variant
%define parse.assert

%code requires {
  #include <string>

  class   mfslDriver;
}

// the parsing context
%param { mfslDriver& drv } // declaration, any parameter name is fine

%verbose // to produce mfslParser.output

%locations

// other Bison options
%define parse.trace
%define parse.error detailed
%define parse.lac full
// %define api.pure full

%printer { yyo << $$; } <*>;


%code {
  #include "mfslBasicTypes.h"
}
\end{lstlisting}


% -------------------------------------------------------------------------
\subsection{The MFSL tokens}
% -------------------------------------------------------------------------

The \mfslLang\ tokens are:
\begin{lstlisting}[language=Bison]
%define api.token.prefix {MFSL_TOK_}

%token
  BAR         "|"
  AMPERSAND   "&"
  EQUAL        "="
  SEMICOLON   ";"
  COLON       ":"
  COMMA       ","

  TOOL        "tool"
  INPUT       "input"

  CHOICE      "choice"
  DEFAULT     "default"

  CASE        "case"

  SELECT      "select"
  ALL         "all"
;

%code {
  #include "mfslDriver.h"
}

%token <std::string> INTEGER "integer number"
%token <std::string> DOUBLE  "double number"

%token <std::string> SINGLE_QUOTED_STRING "single quoted_string"
%token <std::string> DOUBLE_QUOTED_STRING "double quoted_string"

%token <std::string> NAME "name"

%token <std::string> OPTION "option"
\end{lstlisting}


% -------------------------------------------------------------------------
\subsection{The MFSL non-terminals and axiom}
% -------------------------------------------------------------------------

They are:
\begin{lstlisting}[language=Bison]
// the MFSL non-terminals
//_______________________________________________________________________________

%nterm <std::string> Number

%nterm <std::string> SingleString
%nterm <std::string> std::string

%nterm <std::string> OptionValue

%nterm <std::string> LabelName


// the MFSL axiom
//_______________________________________________________________________________

%start Script
\end{lstlisting}


% -------------------------------------------------------------------------
\section{Interface to the MFSL parser}
% -------------------------------------------------------------------------

This is provided by \mfsl{mfslInterpreterInterface.h}:
\begin{lstlisting}[language=Terminal]
EXP extern mfMusicformatsErrorKind launchMfslInterpreter (
  const std::string&      inputSourceName,
  bool                    traceScanning,
  bool                    traceParsing,
  bool                    displayTokens,
  bool                    displayNonTerminals,
  std::string&                 theMfTool,
  std::string&                 theInputFile,
  oahOptionsAndArguments& optionsAndArguments);
\end{lstlisting}

The definition of this function is placed in \mfsl{mfslScanner.ll}:
\begin{lstlisting}[language=Terminal]
EXP mfMusicformatsErrorKind launchMfslInterpreter (
  const std::string&      inputSourceName,
  bool                    traceScanning,
  bool                    traceParsing,
  bool                    displayTokens,
  bool                    displayNonTerminals,
  std::string&                 theMfTool,
  std::string&                 theInputFile,
  oahOptionsAndArguments& optionsAndArguments)
{
  mfMusicformatsErrorKind
    result =
      mfMusicformatsErrorKind::kMusicformatsError_NONE;

  mfslDriver
    theDriver (
      traceScanning,
      traceParsing,
      displayTokens,
      displayNonTerminals);

  int parseResult =
  	theDriver.parseFile (inputSourceName);

  gLog <<
    "--> parseResult: " << parseResult <<
    std::endl;

  if (! parseResult) {
    result =
      mfMusicformatsErrorKind::kMusicformatsErrorInvalidFile;
  }

	gLog <<
    "inputFileName: " << theDriver.getInputFileName () <<
  	std::endl <<
		"toolName: " << theDriver.getToolName () <<
  	std::endl;

  theMfTool    = theDriver.getToolName ();
  theInputFile = theDriver.getInputFileName ();

	return result;
}
\end{lstlisting}


% -------------------------------------------------------------------------
\section{Running the example MFSL script}
% -------------------------------------------------------------------------

Let's show show the \mfslInterp\ uses the options above:
\begin{lstlisting}[language=Terminal]
jacquesmenu@macmini: ~/musicformats-git-dev/src/interpreters/mfsl > ./test.mfsl -display-tokens -display-non-terminals -display-options-values
  The options values for //Users/jacquesmenu/musicformats-git-dev/build/bin/mfsl are:
    MFSL group (-help-mfsl-group, -hmfsl-group), 2 atoms selected:
    --------------------------
      MFSL (-help-mfsl, -hmfsl), 2 atoms selected:
        fTraceTokens                     : true, selected option
        fDisplayNonTerminals               : true, selected option

    Options and help group (-help-oah-group, -hoah-group), 1 atom selected:
    --------------------------
      Options and help (-help-oah, -hoah), 1 atom selected:
        fDisplayOptionsValues              : true, selected option


--> ./test.mfsl:2.1-39: tool
--> ./test.mfsl:2.41: :
--> ./test.mfsl:2.43-48: name [xml2ly]
  ==> tool: xml2ly

--> ./test.mfsl:4.1-22: input
--> ./test.mfsl:4.24: :
--> ./test.mfsl:4.25-33: name [test.mfsl]
  ==> input: test.mfsl

--> test.mfsl:6.11-32: option [-keep-musicxml-part-id]
--> test.mfsl:6.34-35: name [P1]
  ==> option -keep-musicxml-part-id P1

--> test.mfsl:8.1-26: choice
--> test.mfsl:8.28-40: name [VOICES_CHOICE]
--> test.mfsl:8.42: :
--> test.mfsl:8.44-53: name [voice1Only]
--> test.mfsl:8.55: |
--> test.mfsl:8.57-66: name [voice2Only]
--> test.mfsl:8.68: ;
  ==> ChoiceDeclaration VOICES_CHOICE : ...

--> test.mfsl:10.1-3: set
--> test.mfsl:10.5-17: name [VOICES_CHOICE]
--> test.mfsl:10.19: =
--> test.mfsl:10.21-30: name [voice1Only]
--> test.mfsl:10.32: ;
  ==> ChoiceSetting, set VOICES_CHOICE = voice1Only

--> test.mfsl:12.1-61: case
--> test.mfsl:12.63-75: name [VOICES_CHOICE]
--> test.mfsl:12.77: :
--> test.mfsl:13.2-11: name [voice1Only]
--> test.mfsl:13.12: :
--> test.mfsl:14.5-10: option [-title]
--> test.mfsl:14.12-34: double quoted std::string ["]
  ==> option -title "

--> test.mfsl:15.5-21: option [-ignore-msr-voice]
--> test.mfsl:15.23-51: name [Part_POne_Staff_One_Voice_Two]
  ==> option -ignore-msr-voice Part_POne_Staff_One_Voice_Two

--> test.mfsl:16.3: ;
  ==> Case voice1Only : ...

--> test.mfsl:18.2-11: name [voice2Only]
--> test.mfsl:18.12: :
--> test.mfsl:19.5-10: option [-title]
--> test.mfsl:19.12-34: double quoted std::string ["]
  ==> option -title "

--> test.mfsl:20.5-22: option [--ignore-msr-voice]
--> test.mfsl:20.24-52: name [Part_POne_Staff_One_Voice_One]
  ==> option --ignore-msr-voice Part_POne_Staff_One_Voice_One

--> test.mfsl:22.5-27: option [-display-options-values]
--> test.mfsl:24.5-22: option [-global-staff-size]
  ==> option -display-options-values

--> test.mfsl:24.24-27 double: 25.5
  ==> option -global-staff-size 25.5

--> test.mfsl:25.3: ;
  ==> Case voice2Only : ...

--> test.mfsl:26.1: ;
  ==> CaseStatement, VOICES_CHOICE : ...

--> parseResult: 0
==> inputFileName: test.mfsl
==> toolName:      xml2ly
jacquesmenu@macmini: ~/musicformats-git-dev/src/interpreters/mfsl >
\end{lstlisting}


% -------------------------------------------------------------------------
\subsection{Error recovery}
% -------------------------------------------------------------------------

The \mfslInterp\ uses a variant of the \MainIt{stopper sets} method that was present in the early Pascal and Pascal-S converters. The latter passed a set of tokens not to be overtaken to the procedures in charge of accepting the various statements in the language. Strangely enough, this was not done for declarations.

We use a stack of tokens sets that grows and shrinks in parallel with the accepting functions\index{functions}, to know more contextual informations when deciding wether to consume a token or not. The corresponding term is {it shift}
when building the analysis tables in LR technology.


