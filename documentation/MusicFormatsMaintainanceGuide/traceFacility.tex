% -------------------------------------------------------------------------
%  MusicFormats Library
%  Copyright (C) Jacques Menu 2016-2023

%  This Source Code Form is subject to the terms of the Mozilla Public
%  License, v. 2.0. If a copy of the MPL was not distributed with this
%  file, you can obtain one at http://mozilla.org/MPL/2.0/.

%  https://github.com/jacques-menu/musicformats
% -------------------------------------------------------------------------

% !TEX root = MusicFormatsMaintainanceGuide.tex

% -------------------------------------------------------------------------
\chapter{The trace facility}
% -------------------------------------------------------------------------


\mf\ is instrumented with an optionnal, full-fledged trace facility, with numerous options to display what is going on when using the library.
One can build the library with or without trace, which applies to the whole code base.

% -------------------------------------------------------------------------
\section{Activating the trace}
% -------------------------------------------------------------------------

Trace is controlled by {\tt MF_TRACE_IS_ENABLED}, defined or nor in \oah{mfStaticSettings.h}:

\begin{lstlisting}[language=CPlusPlus]
#ifndef ___enableTraceIfDesired___
#define ___enableTraceIfDesired___

#ifndef MF_TRACE_IS_ENABLED
  // comment the following definition if no trace is desired
  #define MF_TRACE_IS_ENABLED
#endif // MF_TRACE_IS_ENABLED

#endif // ___enableTraceIfDesired___
\end{lstlisting}

This file should be included when the trace facility is used:
\begin{lstlisting}[language=CPlusPlus]
#include "mfStaticSettings.h"
\end{lstlisting}

The files \oahBoth{TraceOah} contain the options to the trace facility itself.

Be sure to build \mf\ with \code{MF_TRACE_IS_ENABLED} both active and commented out before creating a new \code{v*} \version\ \branch, to check that variables scopes are fine.

For example, {\tt xml2ly -insider -help-trace}\fileName{xml2ly}\option{insider} produces:
\begin{lstlisting}[language=Terminal]
menu@macbookprojm > xml2ly -insider -help-trace
--- Help for group "OAH Trace" ---
  OAH Trace (-ht, -help-trace) (use this option to show this group)
    There are trace options transversal to the successive passes,
      showing what's going on in the various translation activities.
      They're provided as a help for the maintainance of MusicFormats,
      as well as for the curious.
      The options in this group can be quite verbose, use them with small input data!
      All of them imply '-trace-passes, -tpasses'.
  --------------------------
    Options handling trace    (-htoh, -help-trace-options-handling):
      -toah, -trace-oah
            Write a trace of options and help handling to standard error.
      -toahd, -trace-oah-details
            Write a trace of options and help handling with more details to standard error.
    Score to voices           (-htstv, -help-trace-score-to-voices):
      -t<SHORT_NAME>, -trace-<LONG_NAME>
            Trace SHORT_NAME/LONG_NAME in books to voices.
      The 10 known SHORT_NAMEs are:
        book, scores, pgroups, pgroupsd, parts, staves, st, schanges,
        .
      The 10 known LONG_NAMEs are:
        -books, -scores, -part-groups, -part-groups-details,
        -parts, -staves, -staff-details, -staff-changes, -voices and
        -voices-details.
... ... ...
\end{lstlisting}


% -------------------------------------------------------------------------
\section{Trace categories}
% -------------------------------------------------------------------------

% -------------------------------------------------------------------------
\section{Using traces in practise}
% -------------------------------------------------------------------------

In \lpsrToLilypond{lpsr2lilypondTranslator.cpp}, the trace for the generation of LilyPond code for a regular note in a measure is produced by:
\begin{lstlisting}[language=CPlusPlus]
void lpsr2lilypondTranslator::generateCodeForNoteRegularInMeasure (
  const S_msrNote& note)
{
  int inputLineNumber =
    note->getInputLineNumber ();

#ifdef MF_TRACE_IS_ENABLED
  if (gTraceOahGroup->getTraceNotes ()) {
    std::stringstream ss;

    ss <<
      std::endl <<
      "% --> generating code for noteRegularInMeasure " <<
      note->asString () <<
      ", line " << inputLineNumber <<
      std::endl;

    gLog          << ss.str ();
    fLilypondCodeStream << ss.str ();
  }
#endif // MF_TRACE_IS_ENABLED

\end{lstlisting}


% -------------------------------------------------------------------------
\section{Debugging traces handling}
% -------------------------------------------------------------------------

If case there is a null pointer in a case such as:
\begin{lstlisting}[language=CPlusPlus]
    gMsrOahGroup->getUseFilenameAsWorkCreditTypeTitle ()
\end{lstlisting}

the way to go is to:%%%JMI
\begin{itemize}
\item locate {\tt gGlobalMxsr2msrOahGroup} in the {\tt *.h} it is declared in;
\item check that the creation method in the same, such as \methodName{createGlobalMxsr2msrOahGroup}, is called in the {\tt *InsiderHandler.cpp} file for the service that crashed, which may require including that \starFileName{.h} header in \starFileName{InsiderHandler.cpp}.
\end{itemize}

