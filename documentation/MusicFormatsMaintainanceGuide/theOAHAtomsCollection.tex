% !TEX root = MusicFormatsMaintainanceGuide.tex

% -------------------------------------------------------------------------
\chapter{The OAH atoms collection}\label{The OAH atoms collection}
% -------------------------------------------------------------------------

These handly general-purpose OAH atoms are used in \mf\ itself. They are defined in \oahBoth{oahAtomsCollection}:
\begin{lstlisting}[language=Terminal]
jacquesmenu@macmini: ~/musicformats-git-dev/src/oah > grep class   oahAtomsCollection.h
class EXP oahAtomAlias : public oahAtom
class EXP oahMacroAtom : public oahAtom
class EXP oahOptionsUsageAtom : public oahPureHelpAtomWithoutAValue
class EXP oahHelpAtom : public oahPureHelpAtomWithoutAValue
class EXP oahHelpSummaryAtom : public oahPureHelpAtomWithoutAValue
class EXP oahAboutAtom : public oahPureHelpAtomWithoutAValue
class EXP oahVersionAtom : public oahPureHelpAtomWithoutAValue
class EXP oahContactAtom : public oahPureHelpAtomWithoutAValue
class EXP oahBooleanAtom : public oahAtom
class EXP oahTwoBooleansAtom : public oahBooleanAtom
class EXP oahThreeBooleansAtom : public oahBooleanAtom
class EXP oahCombinedBooleansAtom : public oahAtom
class EXP oahCommonPrefixBooleansAtom : public oahAtom
class EXP oahIntegerAtom : public oahAtomStoringAValue
class EXP oahTwoIntegersAtom : public oahIntegerAtom
class EXP oahFloatAtom : public oahAtomStoringAValue
class EXP oahStringAtom : public oahAtomStoringAValue
class EXP oahFactorizedStringAtom : public oahAtom
class EXP oahStringWithDefaultValueAtom : public oahStringAtom
class EXP oahRationalAtom : public oahAtomStoringAValue
class EXP oahNaturalNumbersSetElementAtom : public oahAtomStoringAValue
class EXP oahColorRGBAtom : public oahAtomStoringAValue
class EXP oahIntSetElementAtom : public oahAtomStoringAValue
class EXP oahStringSetElementAtom : public oahAtomStoringAValue
class EXP oahStringToIntMapElementAtom : public oahAtomStoringAValue
class EXP oahStringAndIntegerAtom : public oahAtomStoringAValue
class EXP oahStringAndTwoIntegersAtom : public oahAtomStoringAValue
class EXP oahLengthUnitKindAtom : public oahAtomStoringAValue
class EXP oahLengthAtom : public oahAtomStoringAValue
class EXP oahMidiTempoAtom : public oahAtomStoringAValue
class EXP oahOptionNameHelpAtom : public oahStringWithDefaultValueAtom
class EXP oahQueryOptionNameAtom : public oahPureHelpAtomExpectingAValue
class EXP oahFindStringAtom : public oahPureHelpAtomExpectingAValue
\end{lstlisting}


% -------------------------------------------------------------------------
\section{OAH macro atoms}
% -------------------------------------------------------------------------

A OAH macro atom is a combination, a list of several options under a single name. The \className{oahMacroAtom} class   is defined in \oahBoth{oahAtomsCollection}:
\begin{lstlisting}[language=CPlusPlus]
class EXP oahMacroAtom : public oahAtom
{
/*
  a list of atoms
*/

	// ... ... ...

  public:

    // public services
    // ------------------------------------------------------

    void                  appendAtomToMacro (S_oahAtom atom);

    void                  applyElement (ostream& os) override;



  private:

    // private fields
    // ------------------------------------------------------

     list<S_oahAtom>      fMacroAtomsList;
};
\end{lstlisting}

Populating \field{oahMacroAtom}{fMacroAtomsList} is straightfoward:
\begin{lstlisting}[language=CPlusPlus]
void oahMacroAtom::appendAtomToMacro (S_oahAtom atom)
{
  // sanity check
  mfAssert (
    __FILE__, __LINE__,
    atom != nullptr,
    "atom is null");

  fMacroAtomsList.push_back (atom);
}
\end{lstlisting}

Applying the macro atom is done in \method{oahMacroAtom}{applyElement}:
\begin{lstlisting}[language=CPlusPlus]
void oahMacroAtom::applyElement (ostream& os)
{
#ifdef TRACING_IS_ENABLED
  if (gGlobalOahEarlyOptions.getEarlyTracingOah ()) {
    gLogStream <<
      "==> option '" << fetchNames () << "' is a oahMacroAtom" <<
      endl;
  }
#endif

  for (
    list<S_oahAtom>::const_iterator i =
      fMacroAtomsList.begin ();
    i != fMacroAtomsList.end ();
    ++i
  ) {
    S_oahAtom atom = (*i);

    if (
      // oahAtomStoringAValue?
      S_oahAtomStoringAValue
        atomWithVariable =
          dynamic_cast<oahAtomStoringAValue*>(&(*atom))
    ) {
//       atomWithVariable-> JMI ???
//         applyAtomWithValue (theString, os);
    }
    else {
      // valueless atom
      atom->
        applyElement (os);
    }
  } // for
}
\end{lstlisting}


% -------------------------------------------------------------------------
\section{A OAH macro atom example}
% -------------------------------------------------------------------------

\xmlToBrl\ has the \optionNameBoth{auto-utf8}{au8d} option:
\begin{lstlisting}[language=Terminal]
jacquesmenu@macmini > xml2brl -query auto-utf8d
--- Help for atom "auto-utf8d" in subgroup "Files"
    -auto-utf8d, -au8d
          Combines -auto-output-file-name, -utf8d and -use-encoding-in-file-name
\end{lstlisting}

This macro options is defined in \brailleGeneration{brailleGenerationOah.cpp} the following way:
\begin{lstlisting}[language=CPlusPlus]
void brailleGenerationOahGroup::initializeMacroOptions ()
{
  S_oahSubGroup
    subGroup =
      oahSubGroup::create (
        "Macros",
        "help-braille-generation-macros", "hbgm",
R"()",
      oahElementVisibilityKind::kElementVisibilityWhole,
      this);

  appendSubGroupToGroup (subGroup);

  // create the auto utfd8 macro

  S_oahMacroAtom
    autoUTFd8MacroAtom =
      oahMacroAtom::create (
        "auto-utf8d", "au8d",
        "Combines -auto-output-file-name, -utf8d and -use-encoding-in-file-name");

  subGroup->
    appendAtomToSubGroup (
      autoUTFd8MacroAtom);

  // populate it
  autoUTFd8MacroAtom->
    appendAtomToMacro (
      gGlobalOutputFileOahGroup->getAutoOutputFileNameAtom ());

  fBrailleOutputKindAtom->
    applyAtomWithValue (
      "utf8d",
      gLogStream);
  autoUTFd8MacroAtom->
    appendAtomToMacro (
      fBrailleOutputKindAtom);

  autoUTFd8MacroAtom->
    appendAtomToMacro (
      fUseEncodingInFileNameAtom);
}
\end{lstlisting}


% -------------------------------------------------------------------------
\section{LilyPond octave entry}
% -------------------------------------------------------------------------

Pass {\tt lpsr2lilypond} has three options to choose this, all controlling one and the same variable:
\begin{lstlisting}[language=Terminal]
jacquesmenu@macmini > xml2ly -query absolute
--- Help for atom "absolute" in subgroup "Notes"
    -abs, -absolute
          Use absolute octave entry in the generated LilyPond code.
\end{lstlisting}

\begin{lstlisting}[language=Terminal]
jacquesmenu@macmini > xml2ly -query relative
--- Help for atom "relative" in subgroup "Notes"
    -rel, -relative
          Use relative octave entry reference PITCH_AND_OCTAVE in the generated LilyPond code.
          PITCH_AND_OCTAVE is made of a diatonic pitch and
          an optional sequence of commas or single quotes.
          It should be placed between double quotes if it contains single quotes, such as:
            -rel "c''".
          The default is to use LilyPond's implicit reference 'f'.
\end{lstlisting}

\begin{lstlisting}[language=Terminal]
jacquesmenu@macmini > xml2ly -query fixed
--- Help for atom "fixed" in subgroup "Notes"
    -fixed
          Use fixed octave entry reference PITCH_AND_OCTAVE in the generated LilyPond code.
          PITCH_AND_OCTAVE is made of a diatonic pitch and
          an optional sequence of commas or single quotes.
          It should be placed between double quotes if it contains single quotes, such as:
            -fixed "c''"
\end{lstlisting}

This is done in \lilypondGenerationBoth{lpsr2lilypondOah} using a single instance of \class{msrOctaveEntryVariable}:
\begin{lstlisting}[language=CPlusPlus]
class EXP msrOctaveEntryVariable : public smartable
{
	// ... ... ...

  private:

    // private fields
    // ------------------------------------------------------

    string                fVariableName;
    msrOctaveEntryKind    fOctaveEntryKind;

    Bool                  fSetByAnOption;
};

\end{lstlisting}

The three classes:
\begin{itemize}
\item  {\tt lilypondAbsoluteOctaveEntryAtom}
\item  {\tt lilypondRelativeOctaveEntryAtom}
\item  {\tt lilypondFixedOctaveEntryAtom}
\end{itemize}
all contain an alias for an \class{msrOctaveEntryVariable} variable:
\begin{lstlisting}[language=CPlusPlus]
    // private fields
    // ------------------------------------------------------

    msrOctaveEntryVariable&
                          fOctaveEntryKindVariable;
\end{lstlisting}

The {\tt fOctaveEntryVariable} filed of class   {\tt lpsr2lilypondOahGroup} shared be all three options atoms is:
\begin{lstlisting}[language=CPlusPlus]
    // notes
    // --------------------------------------

    msrOctaveEntryVariable
                          fOctaveEntryVariable;
\end{lstlisting}

