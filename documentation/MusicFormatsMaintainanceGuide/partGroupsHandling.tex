% !TEX root = MusicFormatsMaintainanceGuide.tex

% -------------------------------------------------------------------------
\chapter{Part groups handling}\label{Part groups handling}
% -------------------------------------------------------------------------

\mf\ part groups are presented at \chapterRef{Part groups}.

In \mxml, part groups can overlap, even though no one seems ever to have needed that. That seems to be more a feature in the Finale handling of \mxml export that a true musical need.

\msrRepr\ does not support overlapping part group. Handling part groups is done in \mxsrToMsrBoth{mxsr2msrTranslator.h}, where overlapping groups are identified and rejected:
\begin{lstlisting}[language=Terminal]
jacquesmenu@macmini > xml2ly partgroups/OverlappingPartGroups.xml
### MusicXML ERROR ### partgroups/OverlappingPartGroups.xml:169:
There are overlapping part groups, namely:
  '2' -=> PartGroup_6 ('2', partGroupName "1
2"), lines 164..169
and
  '1' -=> PartGroup_2 ('1', partGroupName ""), lines 76..170

Please contact the maintainers of MusicFormats (see option '-c, -contact'):
  either you found a bug in the xml2ly converter,
  or this MusicXML data is the first-ever real-world case
  of a score exhibiting overlapping part groups.
  Exception caught: mfException:
There are overlapping part groups, namely:
  '2' -=> PartGroup_6 ('2', partGroupName "1
2"), lines 164..169
and
  '1' -=> PartGroup_2 ('1', partGroupName ""), lines 76..170

Please contact the maintainers of MusicFormats (see option '-c, -contact'):
  either you found a bug in the xml2ly converter,
  or this MusicXML data is the first-ever real-world case
  of a score exhibiting overlapping part groups.

  Error message(s) were issued for input line 169
  ### xml2ly gIndenter final value: 1 ###
### Conversion from MusicXML to LilyPond failed ###
\end{lstlisting}

class   {\tt mxmlPartGroupDescr} contains:
\begin{lstlisting}[language=CPlusPlus]
struct mxmlPartGroupDescr : public smartable
{
/*
  positions represent the order in which the parts appear in <part-list />
*/

	// ... ... ...

  private:

    // private fields
    // ------------------------------------------------------

    int                   fStartInputLineNumber;
    int                   fStopInputLineNumber;

    int                   fPartGroupNumber; // may be reused later

    S_msrPartGroup        fPartGroup;

    int                   fStartPosition;
    int                   fStopPosition;
};
\end{lstlisting}

Part groups numbers number re-used and they can be nested, so there is an implicit part group at the top of their hierachy, attached to the \class{msrScore}:
\begin{lstlisting}[language=CPlusPlus]
class EXP mxsr2msrSkeletonBuilder :
	// ... ... ...

    // an implicit part group has to be created to contain everything,
    // since there can be parts out of any explicit part group
    S_mxmlPartGroupDescr      fImplicitPartGroupDescr;
    S_msrPartGroup            fImplicitPartGroup;

    void                      createImplicitPartGroup ();

    // part groups numbers can be re-used, they're no identifier
    // we use a map to access them by part group number
    int                       fPartGroupsCounter;
    vector<S_mxmlPartGroupDescr>
                              fPartGroupDescsVector;
    map<int, S_mxmlPartGroupDescr>
                              fAllPartGroupDescrsMap;
    map<int, S_mxmlPartGroupDescr>
                              fStartedPartGroupDescrsMap;

	// ... ... ...
\end{lstlisting}

