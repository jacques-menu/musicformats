% -------------------------------------------------------------------------
%  MusicFormats Library
%  Copyright (C) Jacques Menu 2016-2023

%  This Source Code Form is subject to the terms of the Mozilla Public
%  License, v. 2.0. If a copy of the MPL was not distributed with this
%  file, you can obtain one at http://mozilla.org/MPL/2.0/.

%  https://github.com/jacques-menu/musicformats
% -------------------------------------------------------------------------

% !TEX root = MusicFormatsMaintainanceGuide.tex

% -------------------------------------------------------------------------
\chapter{MusicFormats components (MFC)}\label{MusicFormats components (MFC)}
% -------------------------------------------------------------------------

\mf\ supports keeping the history of its \component s versions using a dedicated representation, as an alternative to separate \Main{release notes}. The source files are in \mfc.


% -------------------------------------------------------------------------
\section{Versions numbers}\label{Versions numbers}
% -------------------------------------------------------------------------

The basic data structure is \class{mfcVersionNumber}:
\begin{lstlisting}[language=CPlusPlus]
class   mfcVersionNumber: public smartable
{
	// ... ... ...

  public:

    // public services
    // ------------------------------------------------------

    Bool                  operator== (const mfcVersionNumber& other) const;

    Bool                  operator!= (const mfcVersionNumber& other) const;

    Bool                  operator<  (const mfcVersionNumber& other) const;

    Bool                  operator>= (const mfcVersionNumber& other) const;

    Bool                  operator>  (const mfcVersionNumber& other) const;

    Bool                  operator<= (const mfcVersionNumber& other) const;

  public:

    // print
    // ------------------------------------------------------

    std::string           asString () const;

    void                  print (std::ostream& os) const;

  private:

    // fields
    // ------------------------------------------------------

    int                   fMajorNumber;
    int                   fMinorNumber;
    int                   fPatchNumber;
    std::string           fPreRelease;
};
\end{lstlisting}


% -------------------------------------------------------------------------
\section{Versions descriptions}\label{Versions descriptions}
% -------------------------------------------------------------------------

Each component version is described by a \class{mfcVersionDescr} instance:
\begin{lstlisting}[language=CPlusPlus]
class   mfcVersionDescr : public smartable
{
	// ... ... ...

  private:

    // fields
    // ------------------------------------------------------

    S_mfcVersionNumber    fVersionNumber;
    std::string           fVersionDate;
    std::list<std::string>
                          fVersionDescriptionItems;
};
\end{lstlisting}


% -------------------------------------------------------------------------
\section{Versions histories}\label{Versions histories}
% -------------------------------------------------------------------------

An instance of \className{mfcVersionsHistory} is essentially a list of \className{mfcVersionsHistory} instances:
\begin{lstlisting}[language=CPlusPlus]
class   mfcVersionsHistory : public smartable
{
	// ... ... ...

  public:

    // public services
    // ------------------------------------------------------

    void                  appendVersionDescrToHistory (
                            const S_mfcVersionDescr& versionDescr);

    S_mfcVersionDescr     fetchMostRecentVersion () const;

    S_mfcVersionNumber    fetchMostRecentVersionNumber () const;

	// ... ... ...

  protected:

    // protected fields
    // ------------------------------------------------------

    std::list<S_mfcVersionDescr>
                          fVersionsList;
};
\end{lstlisting}

The current version of a component is the last one appended to \fieldName{fVersionsList}:
\begin{lstlisting}[language=CPlusPlus]
S_mfcVersionDescr mfcVersionsHistory::fetchMostRecentVersion () const
{
#ifdef MF_SANITY_CHECKS_ARE_ENABLED
  // sanity check
  mfAssert (
    __FILE__, __LINE__,
    fVersionsList.size () > 0,
    "fVersionsList is empty");
#endif // MF_SANITY_CHECKS_ARE_ENABLED

  return fVersionsList.back ();
}
\end{lstlisting}


% -------------------------------------------------------------------------
\section{Components descriptions}\label{Components descriptions}
% -------------------------------------------------------------------------

The \component s of \mf\ are described by \enumType{mfcComponentKind}:
\begin{lstlisting}[language=CPlusPlus]
enum class mfcComponentKind {
  kComponentRepresentation,
  kComponentPass,
  kComponentGenerator,
  kComponentConverter,
  kComponentLibrary
};
\end{lstlisting}

The \purelyVirtualClass{mfcComponent} is a superclass  to the ones describing formats, passes, generators, converters and the \mf\ library itself:
\begin{lstlisting}[language=CPlusPlus]
class   mfcComponent : public smartable
{
	// ... ... ...

  public:

    // public services
    // ------------------------------------------------------

    S_mfcVersionDescr     fetchComponentMostRecentVersion () const
                              {
                                return
                                  fVersionsHistory->
                                     fetchMostRecentVersion ();
                              }

	// ... ... ...

  public:

    // print
    // ------------------------------------------------------

    std::string           asString () const;

    std::string           mostRecentVersionNumberAndDateAsString () const;

    virtual void          print (std::ostream& os) const;

    virtual void          printVersion (std::ostream& os) const;
    virtual void          printHistory (std::ostream& os) const;

  protected:

    // protected services
    // ------------------------------------------------------

    virtual void          printOwnHistory (std::ostream& os) const;

  protected:

    // protected fields
    // ------------------------------------------------------

    std::string           fComponentName;

    mfcComponentKind       fComponentKind;

    S_mfcVersionsHistory  fVersionsHistory;
};
\end{lstlisting}

The virtual \methodName{printVersion} and \methodName{printHistory} methods are called by the \optionNameBoth{-v}{-version} and \optionNameBoth{-hist}{-history} options to the various generators and converters.


Representations and passes have a single, linear history, whereas the generators, the converters and \mf\ itself use several of them, each with its own history. This leads to a hierarchy of classes:
\begin{itemize}
\item \class{mfcRepresentationComponent} for formats;
\item \class{mfcPassComponent} for passes;
\item \purelyVirtualClass{mfcMultiComponent} for the generators, converters and \mf\ library, itself the superclass  of:


\begin{itemize}
	\item \class{mfcGeneratorComponent};
	\item \class{mfcConverterComponent};
	\item \class{mfcLibraryComponent}.
	\end{itemize}

\end{itemize}

Multi-components have their own history, hence field \method{mfcComponent}{printOwnHistory}.
\Class{mfcMultiComponent} is described below.


% -------------------------------------------------------------------------
\section{Multi-components}\label{Multi-components}
% -------------------------------------------------------------------------

\Class{mfcMultiComponent} contains lists of the formats and passes used:
\begin{lstlisting}[language=CPlusPlus]
class   mfcMultiComponent : public mfcComponent
{
	// ... ... ...

  protected:

    // protected fields
    // ------------------------------------------------------

    std::list<S_mfcRepresentationComponent>
                          fRepresentationComponentsList;
    std::list<S_mfcPassComponent>
                          fPassComponentsList;

    // should the version number be at least equal to
    // the ones of the components?
    mfcMultiComponentEntropicityKind
                          fComponentEntropicityKind;

    mfcMultiComponentUsedFromTheCLIKind
                          fComponentUsedFromTheCLIKind;
};
\end{lstlisting}

\EnumType{mfcMultiComponentEntropicityKind} is used to check that the version number of a \className{mfcMultiComponent} instance is at least equal to the version numbers of the formats and passes it uses:
\begin{lstlisting}[language=CPlusPlus]
enum class mfcMultiComponentEntropicityKind {
  kComponentEntropicityYes,
  kComponentEntropicityNo
};
\end{lstlisting}

\EnumType{mfcMultiComponentUsedFromTheCLIKind} is used to display context sensitive output with the \optionNameBoth{version}{v} and \optionNameBoth{history}{hist} options when the library is used from \CLI\ services or through the functional \API:
\begin{lstlisting}[language=CPlusPlus]
enum class mfcMultiComponentUsedFromTheCLIKind {
  kComponentUsedFromTheCLIYes,
  kComponentUsedFromTheCLINo
};
\end{lstlisting}

This allows for the maintainers of little used tools not to worry about using components with version numbers greater than their own. \\
Only \constant{mfcMultiComponentUsedFromTheCLIKind}{kComponentUsedFromTheCLIYes} is used at the time of this writing.

\Method{mfcMultiComponent}{print} displays the regular version numbers:
\begin{lstlisting}[language=Terminal]
jacquesmenu@macmini: ~ > xml2ly -version
Command line version of musicxml2lilypond converter v0.9.51 (October 12 2021)

Representations versions:
  MXSR
    v0.9.5 (October 6 2021)
  MSR
    v0.9.51 (October 14 2021)
  LPSR
    v0.9.5 (October 6 2021)

Passes versions:
  mxsr2msr
    v0.9.5 (October 6 2021)
  msr2msr
    v0.9.5 (October 6 2021)
  msr2lpsr
    v0.9.5 (October 6 2021)
  lpsr2lilypond
    v0.9.5 (October 6 2021)
\end{lstlisting}

\Method{mfcMultiComponent}{printHistory} displays information analogous to \Main{release notes}:
\begin{lstlisting}[language=Terminal]
jacquesmenu@macmini: ~ > xml2brl -history
Command line version of musicxml2braille converter v0.9.51 (October 12 2021)

Own history:
  v0.9.5 (October 6 2021):
    Start of sequential versions numbering

  v0.9.51 (October 12 2021):
    Fixed trace OAH issue

Representations history:
  MXSR
    v0.9.5 (October 6 2021):
      Start of sequential versions numbering

  MSR
    v0.9.5 (October 6 2021):
      Start of sequential versions numbering

    v0.9.51 (October 14 2021):
      Refined MSR names and summary display options

  BSR
    v0.9.5 (October 6 2021):
      Start of sequential versions numbering

Passes history:
  mxsr2msr
    v0.9.5 (October 6 2021):
      Start of sequential versions numbering

  msr2msr
    v0.9.5 (October 6 2021):
      Start of sequential versions numbering

  msr2bsr
    v0.9.5 (October 6 2021):
      Start of sequential versions numbering

  bsr2bsr
    v0.9.5 (October 6 2021):
      Start of sequential versions numbering

  bsr2braille
    v0.9.5 (October 6 2021):
      Start of sequential versions numbering
\end{lstlisting}

% -------------------------------------------------------------------------
\section{Versions history creation}\label{Versions history creation}
% -------------------------------------------------------------------------

\mf\ uses \MainIt{semantic} version numbering, such as \code{v0.9.61}:%%%JMI
\begin{itemize}
\item the library itself gets a new number right after a new branch as been created for it. Branching to "v0.9.61" causes the library to be numbered "v0.9.61" with \code{SetMusicFormatsVersionNumber.bash};

\item each representation, converter or pass that is modified this new branch has been created gets a new history element with the same number as the library.
\end{itemize}

The versions history must exist before the \optionNameBoth{version}{v} and \optionNameBoth{history}{hist} options are handled. They are thus created early by specific functions, placed in \starFileNameBoth{Component} files.

The functions that create them ensure than that is done at most once.


% -------------------------------------------------------------------------
\subsection{Representations and passes components creation}\label{Representations and passes components creation}
% -------------------------------------------------------------------------

This is done in \functionName{create*RepresentationComponent} and \functionName{create*PassComponent} functions, respectively.

For example, \msrRepr\ versions are handled by \function{createMsrRepresentationComponent} \\
in \msrBoth{msrHistory}:
\begin{lstlisting}[language=CPlusPlus]
S_mfcRepresentationComponent EXP createMsrRepresentationComponent ()
{
  static S_mfcRepresentationComponent pRepresentationComponent;

  // protect library against multiple initializations
  if (! pRepresentationComponent) {

#ifdef MF_TRACE_IS_ENABLED
    if (gEarlyOptions.getEarlyTraceComponents ()) {
	  	std::stringstream ss;

      ss <<
        "Initializing MSR format component" <<
        std::endl;

      gWaeHandler->waeTrace (
        __FILE__, __LINE__,
        ss.str ());
    }
#endif // MF_TRACE_IS_ENABLED

    // create the format component
    pRepresentationComponent =
      mfcRepresentationComponent::create (
        "MSR");

    // populate it
    pRepresentationComponent->
      appendVersionDescrToComponent (
        mfcVersionDescr::create (
          mfcVersionNumber::createFromString ("0.9.50"),
          "October 6, 2021",
          std::list<std::string> {
            "Start of sequential versions numbering"
          }
      ));

    pRepresentationComponent->
      appendVersionDescrToComponent (
        mfcVersionDescr::create (
          mfcVersionNumber::createFromString ("0.9.51"), // JMI
          "October 14, 2021",
          std::list<std::string> {
            "Refined MSR names and summary display options"
          }
      ));
  }

  return pRepresentationComponent;
}
\end{lstlisting}

The conversion of \mxml\ to \mxsrRepr does not belong to \mf\, since it is provided by \libmusicxml.


% -------------------------------------------------------------------------
\subsection{Generators and converters components creation}\label{Generators and converters components creation}
% -------------------------------------------------------------------------

In that case, the formats and passes components used by the multi-component should be created as well.

For example, the formats and passes used by the \fileName{musicxml2braille} converter are appended to the atoms versions list in its history in \function{createMusicxml2brailleConverterComponent} in \\
\musicxmlToBraille{musicxml2brailleConverterComponent.cpp}:
\begin{lstlisting}[language=CPlusPlus]
S_mfcConverterComponent EXP createMusicxml2brailleConverterComponent ()
{
  static S_mfcConverterComponent pConverterComponent;

  // protect library against multiple initializations
  if (! pConverterComponent) {

#ifdef MF_TRACE_IS_ENABLED
    if (gEarlyOptions.getEarlyTraceComponents ()) {
	  	std::stringstream ss;

      ss <<
        "Creating the musicxml2braille component" <<
        std::endl;

      gWaeHandler->waeTrace (
        __FILE__, __LINE__,
        ss.str ());
    }
#endif // MF_TRACE_IS_ENABLED

    // create the converter component
    pConverterComponent =
      mfcConverterComponent::create (
        "musicxml2braille",
        mfcMultiComponentEntropicityKind::kComponentEntropicityNo,
        mfcMultiComponentUsedFromTheCLIKind::kComponentUsedFromTheCLIYes); // JMI ???

    // populate the converter's own history
    pConverterComponent->
      appendVersionDescrToComponent (
        mfcVersionDescr::create (
          mfcVersionNumber::createFromString ("0.9.50"),
          "October 6, 2021",
          std::list<std::string> {
            "Start of sequential versions numbering"
          }
      ));

    pConverterComponent->
      appendVersionDescrToComponent (
        mfcVersionDescr::create (
          mfcVersionNumber::createFromString ("0.9.51"),
          "October 12, 2021",
          std::list<std::string> {
            "Fixed trace OAH issue"
          }
      ));

    // populate the converter's formats list
    pConverterComponent->
      appendRepresentationToMultiComponent (
        createMxsrRepresentationComponent ());
    pConverterComponent->
      appendRepresentationToMultiComponent (
        createMsrRepresentationComponent ());
    pConverterComponent->
      appendRepresentationToMultiComponent (
        createBsrRepresentationComponent ());

    pConverterComponent->
      appendPassToMultiComponent (
        createMxsr2msrComponent ());

    pConverterComponent->
      appendPassToMultiComponent (
        createMsr2msrComponent ());

    pConverterComponent->
      appendPassToMultiComponent (
        createMsr2bsrComponent ());

    pConverterComponent->
      appendPassToMultiComponent (
        createBsr2bsrComponent ());

    pConverterComponent->
      appendPassToMultiComponent (
        createBsr2brailleComponent ());
  }

  return pConverterComponent;
}
\end{lstlisting}


% -------------------------------------------------------------------------
\subsection{MusicFormats library component creation}\label{MusicFormats library component creation}
% -------------------------------------------------------------------------

This is done in \function{createLibraryComponent} in \mfcBoth{mfcLibraryComponent}:
\begin{lstlisting}[language=CPlusPlus]
S_mfcLibraryComponent EXP createLibraryComponent ()
{
  static S_mfcLibraryComponent pLibraryComponent;

  // protect library against multiple initializations
  if (! pLibraryComponent) {

#ifdef MF_TRACE_IS_ENABLED
    if (gEarlyOptions.getTraceEarlyOptions ()) {
	  	std::stringstream ss;

      ss <<
        "Creating the  MFC library component" <<
        std::endl;

      gWaeHandler->waeTrace (
        __FILE__, __LINE__,
        ss.str ());
    }
#endif // MF_TRACE_IS_ENABLED

    // create the library's history
    pLibraryComponent =
      mfcLibraryComponent::create (
        "musicformats",
        mfcMultiComponentEntropicityKind::kComponentEntropicityNo,
        mfcMultiComponentUsedFromTheCLIKind::kComponentUsedFromTheCLIYes); // JMI ???

    // populate the library's own history
    pLibraryComponent->
      appendVersionDescrToComponent (
        mfcVersionDescr::create (
          mfcVersionNumber::createFromString ("0.9.50"),
          "October 6, 2021",
          std::list<std::string> {
            "Start of sequential versions numbering"
          }
      ));

    pLibraryComponent->
      appendVersionDescrToComponent (
        mfcVersionDescr::create (
          mfcVersionNumber::createFromString ("0.9.51"),
          "October 12, 2021",
          std::list<std::string> {
            "Adding a version number to the MusicFormats library",
            "Fixed trace OAH issue in the musicxml2* converters)"
          }
      ));

    pLibraryComponent->
      appendVersionDescrToComponent (
        mfcVersionDescr::create (
          mfcVersionNumber::createFromString ("0.9.52"),
          "October 12, 2021",
          std::list<std::string> {
            "Added MusicFormats library versions history to '-hist, -history'"
          }
      ));

    pLibraryComponent->
      appendVersionDescrToComponent (
        mfcVersionDescr::create (
          mfcVersionNumber::createFromString ("0.9.53"),
          "October 22, 2021",
          std::list<std::string> {
            "Replaced bool by class   Bool in variables and fields",
            "Created MFC (MusicFormats components)"
          }
      ));

    pLibraryComponent->
      appendVersionDescrToComponent (
        mfcVersionDescr::create (
          mfcVersionNumber::createFromString ("0.9.54"),
          "Novermber 6, 2021",
          std::list<std::string> {
            "Replaced std::cout and std::cerr by gOutput and gLog respectively in the CLI samples",
            "Finalized components numbering (MFC)"
          }
      ));

    // populate the library's components history
    pLibraryComponent->
      appendRepresentationToMultiComponent (
        createMsrRepresentationComponent ());
    pLibraryComponent->
      appendRepresentationToMultiComponent (
        createLpsrRepresentationComponent ());
    pLibraryComponent->
      appendRepresentationToMultiComponent (
        createBsrRepresentationComponent ());
    pLibraryComponent->
      appendRepresentationToMultiComponent (
        createMxsrRepresentationComponent ());

    pLibraryComponent->
      appendPassToMultiComponent (
        createMsr2msrComponent ());

    pLibraryComponent->
      appendPassToMultiComponent (
        createMsr2lpsrComponent ());
    pLibraryComponent->
      appendPassToMultiComponent (
        createLpsr2lilypondComponent ());

    pLibraryComponent->
      appendPassToMultiComponent (
        createMsr2bsrComponent ());
    pLibraryComponent->
      appendPassToMultiComponent (
        createBsr2bsrComponent ());
    pLibraryComponent->
      appendPassToMultiComponent (
        createBsr2brailleComponent ());

    pLibraryComponent->
      appendPassToMultiComponent (
        createMsr2mxsrComponent ());

    pLibraryComponent->
      appendPassToMultiComponent (
        createMxsr2musicxmlComponent ());

    pLibraryComponent->
      appendPassToMultiComponent (
        createMxsr2guidoComponent ());
  }

  return pLibraryComponent;
}
\end{lstlisting}

\Function{createLibraryComponent} is called in \clisamples{displayMusicformatsVersion.cpp} and \clisamples{displayMusicformatsHistory.cpp}.


% -------------------------------------------------------------------------
\subsection{Version and history options handling}\label{Version and history options handling}
% -------------------------------------------------------------------------

In order to be able to execute the \optionNameBoth{version}{v} and \optionNameBoth{history}{hist} options of a generator or converter, a \className{oahHandler} instance must be supplied with a \className{mfcMultiComponent} instance.

\Field{oahHandler}{fHandlerMultiComponent} is used for this purpose:
\begin{lstlisting}[language=CPlusPlus]
//_______________________________________________________________________________
class EXP oahHandler : public smartable
{
	// ... ... ...

  protected:

    // protected initialization
    // ------------------------------------------------------

    virtual void          initializeHandlerMultiComponent () = 0;

  public:

    // set and get
    // ------------------------------------------------------

		// ... ... ...

    S_mfcMultiComponent   getHandlerMultiComponent () const
                              { return fHandlerMultiComponent; }

	// ... ... ...

  protected:

    // protected fields
    // ------------------------------------------------------

		// ... ... ...

    // compound versions
    S_mfcMultiComponent   fHandlerMultiComponent;
};
\end{lstlisting}

\Field{oahHandler}{fHandlerMultiComponent} is set in the \className{oahHandler} sub-classes constructors by a call to the overriden \methodName{initializeHandlerMultiComponent}.

For example in \constructor{xml2xmlInsiderHandler}:
\begin{lstlisting}[language=CPlusPlus]
xml2xmlInsiderHandler::xml2xmlInsiderHandler (
  const std::string& serviceName,
  std::string handlerHeader)
  : oahInsiderHandler (
      serviceName,
      handlerHeader,
R"(
          Welcome to the MusicXML to MusicXML converter
          delivered as part of the MusicFormats library.

      --- https://github.com/jacques-menu/musicformats ---
)",
R"(
Usage: xml2xml [[option]* [MusicXMLFile|-] [[option]*
)")
{
	// ... ... ...

  // initialize the multi-component
  initializeHandlerMultiComponent ();

	// ... ... ...
}
\end{lstlisting}

The overriden \methodName{initializeHandlerMultiComponent} methods merely get the atom or compound versions to assign it to \field{oahHandler}{fHandlerMultiComponent}.

For example, for \fileName{Mikrokosmos3Wandering}, the compound versions is simply set in the corresponding \insider\ class   \className{Mikrokosmos3WanderingInsiderHandler}:
\begin{lstlisting}[language=CPlusPlus]
void Mikrokosmos3WanderingInsiderHandler::initializeHandlerMultiComponent ()
{
  fHandlerMultiComponent =
    createMikrokosmos3WanderingGeneratorComponent ();
}
\end{lstlisting}


% -------------------------------------------------------------------------
\section{Accessing versions in regular handlers}\label{Accessing versions in regular handlers}
% -------------------------------------------------------------------------

A \regular\ handler merely gets the compound versions of the \insider\ handler it relies upon in its overriden \methodName{initializeHandlerMultiComponent} method:
\begin{lstlisting}[language=CPlusPlus]
class EXP oahRegularHandler : public oahHandler
/*
  A regular OAH handler relies on the existence of so-called 'insider' handler,
  that contains all the options values gathered from the user,
  grouped according to the internal representations and passes used.

  The variables containing the values of the options selected by the user
  are actually held by the insider handler.
*/
{
	// ... ... ... ...

  protected:

    // protected initialization
    // ------------------------------------------------------

	// ... ... ... ...

    void                  initializeHandlerMultiComponent () override
                            {
                              fHandlerMultiComponent =
                                fInsiderHandler->
                                  getHandlerMultiComponent ();
                            }

	// ... ... ... ...
};
\end{lstlisting}


% -------------------------------------------------------------------------
\section{Getting current version numbers}\label{Getting current version numbers}
% -------------------------------------------------------------------------

Apart from the version and history options, such current version numbers may be used in the output from generators and converters, depending on the options. A component description is the way to achieve that in the latter two cases.


% -------------------------------------------------------------------------
\subsection{Current version numbers in options}\label{Current version numbers in options}
% -------------------------------------------------------------------------

\OptionBoth{version}{v} displays the versions of generators and converters:
\begin{lstlisting}[language=Terminal]
jacquesmenu@macmini: ~/musicformats-git-dev/musicxml > xml2xml -version
Command line version of musicxml2musicxml converter v0.9.51 (October 12 2021)

Representations versions:
  MXSR
    v0.9.5 (October 6 2021)
  MSR
    v0.9.51 (October 14 2021)

Passes versions:
  mxsr2msr
    v0.9.5 (October 6 2021)
  msr2msr
    v0.9.5 (October 6 2021)
  msr2mxsr
    v0.9.5 (October 6 2021)
  mxsr2musicxml
    v0.9.5 (October 6 2021)
\end{lstlisting}

\OptionBoth{history}{hist} display the versions history of generators and converters:
\begin{lstlisting}[language=Terminal]
jacquesmenu@macmini: ~/musicformats-git-dev/musicxml > xml2gmn -history
Command line version of musicxml2guido converter v0.9.51 (October 12 2021)

Own history:
  v0.9.5 (October 6 2021):
    Start of sequential versions numbering

  v0.9.51 (October 12 2021):
    Fixed trace OAH issue

Representations history:
  MXSR
    v0.9.5 (October 6 2021):
      Start of sequential versions numbering

  MSR
    v0.9.5 (October 6 2021):
      Start of sequential versions numbering

    v0.9.51 (October 14 2021):
      Refined MSR names and summary display options

Passes history:
  mxsr2msr
    v0.9.5 (October 6 2021):
      Start of sequential versions numbering

  msr2msr
    v0.9.5 (October 6 2021):
      Start of sequential versions numbering

  msr2mxsr
    v0.9.5 (October 6 2021):
      Start of sequential versions numbering

  mxsr2guido
    v0.9.5 (October 6 2021):
      Start of sequential versions numbering
\end{lstlisting}


In \oahBoth{oahAtomsCollection}, \class{oahVersionAtom} contains method \methodName{printVersion}:
\begin{lstlisting}[language=CPlusPlus]
class EXP oahVersionAtom : public oahPureHelpValueLessAtom
{
	// ... ... ...

  public:

    // public services
    // ------------------------------------------------------

    void                  applyValueLessAtom (std::ostream& os) override;

	// ... ... ...

  public:

    // print
    // ------------------------------------------------------

		// ... ... ...

    void                  printVersion (std::ostream& os) const;
};
\end{lstlisting}

The option is applied by \method{oahVersionAtom}{applyElement}:
\begin{lstlisting}[language=CPlusPlus]
void oahVersionAtom::applyValueLessAtom (std::ostream& os)
{
#ifdef MF_TRACE_IS_ENABLED
  if (gEarlyOptions.getTraceEarlyOptions ()) {
		std::stringstream ss;

    ss <<
      "==> option '" << fetchNames () << "' is a oahVersionAtom" <<
      std::endl;

    gWaeHandler->waeTrace (
      __FILE__, __LINE__,
      ss.str ());
  }
#endif // MF_TRACE_IS_ENABLED

  int saveIndent = gIndenter.getIndent ();

  gIndenter.resetToZero ();

  printVersion (os);

  gIndenter.setIndent (saveIndent);

	fSelected = true;
}
\end{lstlisting}

The work is done by \method{oahVersionAtom}{printVersion}:
\begin{lstlisting}[language=CPlusPlus]
void oahVersionAtom::printVersion (std::ostream& os) const
{
  // get the handler version
  S_mfcMultiComponent
    handlerMultiComponent =
      fetchAtomUpLinkToHandler ()->
        getHandlerMultiComponent ();

#ifdef MF_SANITY_CHECKS_ARE_ENABLED
  // sanity check
  mfAssert (
    __FILE__, __LINE__,
    handlerMultiComponent != nullptr,
    "handlerMultiComponent is null");
#endif // MF_SANITY_CHECKS_ARE_ENABLED

  handlerMultiComponent->
    printVersion (os);
}
\end{lstlisting}

The situation is analog for histories with \methodName{printVersion} replaced by \methodName{printHistory}.


% -------------------------------------------------------------------------
\subsection{Current version numbers in formats}\label{Current version numbers in formats}
% -------------------------------------------------------------------------

When creating \lily\ output, the current version number of the converter used is indicated as a comment when the \optionBoth{lilypond-generation-infos}{lpgi} option is used:
\begin{lstlisting}[language=Terminal]
jacquesmenu@macmini: ~/musicformats-git-dev/musicxml > xml2ly --lilypond-generation-infos basic/HelloWorld.xml
\version "2.24.0"

% Pick your choice from the next two lines as needed
%myBreak = { \break }
myBreak = {}

% Pick your choice from the next two lines as needed
%myPageBreak = { \pageBreak }
myPageBreak = {}

% Generated by xml2ly v0.9.51 (October 12 2021)
% on Thursday 2021-11-11 @ 11:15:56 CET
% from "basic/HelloWorld.xml"

% ... ... ...
\end{lstlisting}

\Class{lpsrScore} contains an \mfcRepr\ component field:
\begin{lstlisting}[language=CPlusPlus]
class EXP lpsrScore : public lpsrElement
{
 	// ... ... ... ...

  private:

    // private fields
    // ------------------------------------------------------

	 	// ... ... ... ...

    // the multi-component
    // ------------------------------------------------------
    S_mfcMultiComponent   fMultiComponent;

 	// ... ... ... ...
};
\end{lstlisting}

In \lpsr{lpsrScores.cpp}, \constructor{lpsrScore} stores the multi-component value and uses it to create an \className{lpsrComment} instance:
\begin{lstlisting}[language=CPlusPlus]
lpsrScore::lpsrScore (
  int                 inputLineNumber,
  const S_msrScore&   theMsrScore,
  const S_mfcMultiComponent& multiComponent)
    : lpsrElement (inputLineNumber)
{
	// ... ... ... ...

  fMsrScore = theMsrScore;

  fMultiComponent = multiComponent;

  // should the initial comments about the service and the options used
  // be generated?
  if (gGlobalLpsr2lilypondOahGroup->getXml2lyInfos ()) {
    // create the 'input source name and translation date' comment
    {
      std::stringstream ss;

      ss <<
        "Generated by " <<
        gOahOahGroup->getOahOahGroupServiceName () <<
        ' ' <<
        fMultiComponent->
          mostRecentVersionNumberAndDateAsString () <<
        std::endl <<

        "% on " <<
        gServiceRunData->getTranslationDateFull () <<
        std::endl <<

        "% from ";

      if (gServiceRunData->getInputSourceName () == "-") {
        ss << "standard input";
      }
      else {
        ss << "\"" << gServiceRunData->getInputSourceName () << "\"";
      }

      fInputSourceNameComment =
        lpsrComment::create (
          inputLineNumber,
          ss.str (),
          lpsrCommentGapAfterwardsKind::kCommentGapAfterwardsNo);
    }

	 	// ... ... ... ...
 	}

 	// ... ... ... ...
}
\end{lstlisting}


% -------------------------------------------------------------------------
\subsection{Current version numbers in passes}\label{Current version numbers in passes}
% -------------------------------------------------------------------------

Another case is that of the generation of \mxml\ output:
\begin{lstlisting}[language=Terminal]
jacquesmenu@macmini: ~/musicformats-git-dev/musicxml > xml2xml -musicxml-generation-infos basic/HelloWorld.xml
<?xml version="1.0" encoding="UTF-8" standalone="no"?>
<!DOCTYPE score-partwise PUBLIC "-//Recordare//DTD MusicXML 3.1 Partwise//EN"
			"http://www.musicxml.org/dtds/partwise.dtd">
<score-partwise version="3.1">
    <!--
==================================================
Created by xml2xml v0.9.5 (October 6 2021)
on Thursday 2021-11-11 @ 11:04:06 CET
from basic/HelloWorld.xml
==================================================
-->
    <work>
        <work-number/>
        <work-title>Hello World!</work-title>
    </work>
    <movement-number/>
    <movement-title/>
    <identification>
        <encoding>
            <software>xml2xml v0.9.5 (October 6 2021), https://github.com/jacques-menu/musicformats</software>
            <encoding-date>2021-11-10</encoding-date>
        </encoding>
        <miscellaneous>
            <miscellaneous-field name="description"/>
        </miscellaneous>
    </identification>

 <!-- ... ... ... -->
\end{lstlisting}

In \msrToMxsr{msr2mxsrTranslator.cpp}, the start visitor of \className{msrScore} instances does that this way:
\begin{lstlisting}[language=CPlusPlus]
void msr2mxsrTranslator::visitStart (S_msrScore& elt)
{
 	// ... ... ... ...

  // get the pass component
  S_mfcPassComponent
    passComponent =
      createMsr2mxsrComponent ();

  // get the pass component current version number and date
  std::string
    passComponentMostRecentVersionNumberAndDateAsString =
      passComponent->
        mostRecentVersionNumberAndDateAsString ();

  // create the initial creation comment
  std::stringstream ss;
  ss <<
    std::endl <<
    "==================================================" <<
    std::endl <<
    "Created by " <<
    gOahOahGroup->getOahOahGroupServiceName () <<
    ' ' <<
    passComponentMostRecentVersionNumberAndDateAsString <<
    std::endl <<

    "on " <<
    gServiceRunData->getTranslationDateFull () <<
    std::endl <<

    "from " <<
    gServiceRunData->getInputSourceName () <<
    std::endl <<

    "==================================================" <<
    std::endl;

  // append the initial creation comment to the score part wise element
  fResultingMusicxmlelement->push (createMxmlelement (kComment, ss.str ()));

  // create a software element
  Sxmlelement
    softwareElement =
      createMxmlelement (
        k_software,
        gOahOahGroup->getOahOahGroupServiceName ()
          + ' '
          + passComponentMostRecentVersionNumberAndDateAsString +
          ", https://github.com/jacques-menu/musicformats");

  // append it to the identification encoding
  appendToScoreIdentificationEncoding (softwareElement);

 	// ... ... ... ...
}
\end{lstlisting}

