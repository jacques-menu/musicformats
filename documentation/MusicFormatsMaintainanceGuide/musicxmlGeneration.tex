% !TEX root = MusicFormatsMaintainanceGuide.tex

% -------------------------------------------------------------------------
\chapter{MusicXML generation}
% -------------------------------------------------------------------------

\mxml\ text is produced on the standard output stream, unless options '{\tt -output-file-name}' or '{\tt -auto-output-file-name}' are used.


% -------------------------------------------------------------------------
\section{Basic principle} \label{musicxmlGeneration}
% -------------------------------------------------------------------------

\mxml\ generation is done in two passes:
\begin{itemize}
\item first create and \mxsrRepr\ containing the data;
\item then simply write this tree.
\end{itemize}


% -------------------------------------------------------------------------
\section{Creating an {\tt xmlelement}}
% -------------------------------------------------------------------------

An simple example is:
\begin{lstlisting}[language=CPlusPlus]
  // create a direction element
  Sxmlelement directionElement = createMxmlelement (k_direction, "");

  // set it's "placement" attribute if relevant
  std::string
    placementString =
      msrPlacementKindAsMusicXMLString (placementKind);

  if (placementString.size ()) {
    directionElement->add (createMxmlAttribute ("placement",  placementString));
  }
\end{lstlisting}

This one supplies a value to the {\tt xmlelement} it creates:
\begin{lstlisting}[language=CPlusPlus]
void msr2mxsrTranslator::visitStart (S_msrIdentification& elt)
{
  // composers
  const std::list<std::string>&
    composersList =
      elt->getComposersList ();

  for (
    std::list<std::string>::const_iterator i=composersList.begin ();
    i!=composersList.end ();
    ++i
  ) {
    std::string variableValue = (*i);

    // create a creator element
    Sxmlelement creatorElement = createMxmlelement (k_creator, variableValue);

    // set its "type" attribute
    creatorElement->add (createMxmlAttribute ("type", "composer"));

    // append it to the composers elements list
    fComposersElementsList.push_back (creatorElement);
  } // for

	// ... ... ...
}
\end{lstlisting}


% -------------------------------------------------------------------------
\section{Creating an {\tt xmlelement} tree}
% -------------------------------------------------------------------------

In {\tt }, this code:
\begin{lstlisting}[language=CPlusPlus]
void msr2mxsrTranslator::visitStart (S_msrClef& elt)
{
    // ... ... ...

    Sxmlelement clefElement = createMxmlelement (k_clef, "");

    // set clefElement's "number" attribute if relevant
    /*
      0 by default in MSR,
      1 by default in MusicXML:
        The optional	number attribute refers to staff numbers within the part,
        from top to bottom on the system.
        A value of 1 is assumed if not present.
	  */

    int clefStaffNumber =
      elt->getClefStaffNumber ();

    if (clefStaffNumber > 1) {
      clefElement->add (
        createMxmlIntegerAttribute ("number", clefStaffNumber));
    }

    // populate clefElement
    switch (elt->getClefKind ()) {
      // ... ... ...

      case msrClefKind::kClefTrebleMinus8:
        {
          clefElement->push (
            createMxmlelement (
              k_sign,
              "G"));
          clefElement->push (
            createMxmlIntegerElement (
              k_line,
              2));
          clefElement->push (
            createMxmlIntegerElement (
              k_clef_octave_change,
              -1));
        }
        break;

      // ... ... ...
}
\end{lstlisting}

creates this \mxml\ element depending on the value returned by \method{msrClef}{getClefStaffNumber}:
\begin{lstlisting}[language=MusicXML]
 <clef number="2">
      <sign>G</sign>
      <line>2</line>
      <clef-octave-change>-1</clef-octave-change>
  </clef>

\end{lstlisting}

% -------------------------------------------------------------------------
\section{Browsing the visited MSR score}
% -------------------------------------------------------------------------

The creation of the tree is done in \msrToMxsrBoth{msr2mxsrTranslator}.

\Class{msr2mxsrTranslator} is defined in those files, it contains:
\begin{lstlisting}[language=CPlusPlus]
  public:

                          msr2mxsrTranslator (
                            const S_msrScore& visitedMsrScore);

    virtual               ~msr2mxsrTranslator ();

    Sxmlelement           translateMsrToMxsr ();

  // ... ... ...

  private:

    // the MSR score we're visiting
    // ------------------------------------------------------
    S_msrScore                fVisitedMsrScore;


    // the MXSR we're building
    // ------------------------------------------------------
    Sxmlelement               fResultingMusicxmlelement;
\end{lstlisting}

The \method{msr2mxsrTranslator}{translateMsrToMxsr} method does the following:
\begin{lstlisting}[language=CPlusPlus]
//________________________________________________________________________
Sxmlelement msr2mxsrTranslator::translateMsrToMxsr ()
{
#ifdef MF_SANITY_CHECKS_ARE_ENABLED
  // sanity check
  mfAssert (
    __FILE__, __LINE__,
    fVisitedMsrScore != nullptr,
    "fVisitedMsrScore is null");

  // create the current score part-wise element
  fResultingMusicxmlelement =
    createMxmlScorePartWiseElement ();

  // create a msrScore browser
  msrBrowser<msrScore> browser (this);

  // browse the visited score with the browser
  browser.browse (*fVisitedMsrScore);

  return fResultingMusicxmlelement;
}
\end{lstlisting}


% -------------------------------------------------------------------------
\section{Ancillary functions to create MXSR data}
% -------------------------------------------------------------------------

The \function{createMxmlScorePartWiseElement} is defined in \mxsrBoth{mxsr}:
\begin{lstlisting}[language=CPlusPlus]
//------------------------------------------------------------------------
Sxmlelement createMxmlScorePartWiseElement ()
{
  Sxmlelement result = factory::instance ().create (k_score_partwise);

  Sxmlattribute versionAttribute = createMxmlAttribute("version", "3.1");
  result->add (versionAttribute);

  return result;
}
\end{lstlisting}


