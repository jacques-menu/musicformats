% !TEX root = MusicFormatsMaintainanceGuide.tex

% -------------------------------------------------------------------------
\chapter{Braille generation}
% -------------------------------------------------------------------------


Braille is written to standard output or to files as binary data. Our reference is \url{http://www.brailleauthority.org/music/Music_Braille_Code_2015.pdf}.


% -------------------------------------------------------------------------
\section{Basic principle}
% -------------------------------------------------------------------------

Lilypond generation is done in \bsrToBrailleBoth{bsr2brailleTranslator}.

\Class{bsr2brailleTranslator} provides:
\begin{lstlisting}[language=CPlusPlus]
  public:

                          bsr2brailleTranslator (
                            const S_bsrScore&     bsrScore,
                            const S_bsrOahGroup& bsrOpts,
                            std::ostream&       brailleOutputStream);

    virtual               ~bsr2brailleTranslator ();

    void                  translateBsrToBraille ();
\end{lstlisting}

It contains these fields among others:
\begin{lstlisting}[language=CPlusPlus]
  private:

    // options
    // ------------------------------------------------------

    S_bsrOahGroup         fBsrOahGroup;

    // the BSR score we're visiting
    // ------------------------------------------------------

    S_bsrScore            fVisitedBsrScore;

    // the braille generator used
    // ------------------------------------------------------

    S_bsrBrailleGenerator fBrailleGenerator;

    // the output stream
    // ------------------------------------------------------

    std::ostream&              fBrailleOutputStream;
\end{lstlisting}


% -------------------------------------------------------------------------
\section{Output files name and contents options}
% -------------------------------------------------------------------------

he contents options use the following enumeration types:
\begin{lstlisting}[language=CPlusPlus]
enum class bsrUTFKind {
  kUTF8, kUTF16
};

enum class bsrByteOrderingKind {
  kByteOrderingNone,
  kByteOrderingBigEndian, kByteOrderingSmallEndian
};
\end{lstlisting}

\xmlToBrl\ supplies a \option{files} options subgroup:
\begin{lstlisting}[language=Terminal]
jacquesmenu@macmini > xml2brl -query files
--- Help for subgroup "files" in group "Files group" ---
  Files group (-files-group):
  --------------------------
    Files (-files):
      -o, -output-file-name FILENAME
            Write Braille to file FILENAME instead of standard output.
      -aofn, -auto-output-file-name
            This option can only be used when reading from a file.
            Write MusicXML code to a file in the current working directory.
            The file name is derived from that of the input file,
            replacing any suffix after the '.' by 'xml'
            or adding '.xml' if none is present.
      -bok, -braille-output-kind OUTPUT_KIND
            Use OUTPUT_KIND to write the generated Braille to the output.
            The 4 output kinds available are:
            ascii, utf16, utf8 and utf8d.
            'utf8d' leads to every line in the braille score to be generated
            as a line of cells followed by a line of text showing the contents
            for debug purposes.
            The default is 'ascii'.
      -ueifn, -use-encoding-in-file-name
            Append a description of the encoding used
            and the presence of a BOM if any to the file name before the '.'.
      -bom, -byte-ordering-mark BOM_ENDIAN
            Generate an initial BOM_ENDIAN byte ordering mark (BOM)
            ahead of the Braille nusic code,
            which can be one of 'big' or 'small'.
            By default, a big endian BOM is generated.
\end{lstlisting}


% -------------------------------------------------------------------------
\section{Braille generators}
% -------------------------------------------------------------------------

The following classes are defined in \brailleGenerationBoth{brailleGeneration}
\begin{lstlisting}[language=Terminal]
jacquesmenu@macmini: ~/musicformats-git-dev/src/formatsgeneration/brailleGeneration > grep class   brailleGeneration.h
enum class bsrUTFKind {
enum class bsrByteOrderingKind {
class EXP bsrBrailleGenerator : public smartable
/* this class is purely virtual
class EXP bsrAsciiBrailleGenerator : public bsrBrailleGenerator
class EXP bsrUTF8BrailleGenerator : public bsrBrailleGenerator
class EXP bsrUTF8DebugBrailleGenerator : public bsrUTF8BrailleGenerator
class EXP bsrUTF16BigEndianBrailleGenerator : public bsrBrailleGenerator
class EXP bsrUTF16SmallEndianBrailleGenerator : public bsrBrailleGenerator
\end{lstlisting}

The base \class{bsrBrailleGenerator} contains:
\begin{lstlisting}[language=CPlusPlus]
  public:

    // public services
    // ------------------------------------------------------

    virtual void          generateCodeForBrailleCell (
                            bsrCellKind cellKind) = 0;

    void                  generateCodeForCellsList (
                            const S_bsrCellsList& cellsList);

    virtual void          generateCodeForMusicHeading (
                            const S_bsrMusicHeading& musicHeading);

    virtual void          generateCodeForLineContents (
                            const S_bsrLineContents& lineContents);

		// ... ... ...

  protected:

    // protected fields
    // ------------------------------------------------------

    std::ostream&              fBrailleOutputStream;
\end{lstlisting}


% -------------------------------------------------------------------------
\section{Writing braille cells}
% -------------------------------------------------------------------------

Braille cells are output to an \fileName{std::ostream} as hexadecimal strings by \virtualMethod{generateCodeForBrailleCell} methods in \bsrToBraille{brailleGeneration.h}, depending on the kind of output selected.

For example, ASCII braille generation is done by:
\begin{lstlisting}[language=CPlusPlus]
void bsrAsciiBrailleGenerator::generateCodeForBrailleCell (
  bsrCellKind cellKind)
{
  std::string stringForCell;

  switch (cellKind) {
    case bsrCellKind::kCellUnknown:
      {
        std::stringstream ss;

        ss <<
          "cannot generate code for braille cell '" <<
          bsrCellKindAsString (cellKind) <<
          "'";
        msrInternalError (
          gServiceRunData->getInputSourceName (),
          -98, // inputLineNumber, TICINO JMI
          __FILE__, __LINE__,
          ss.str ());
      }
      break;

    case bsrCellKind::kCellEOL:    stringForCell = "\x0a"; break;
    case bsrCellKind::kCellEOP:    stringForCell = "\x0c"; break;

    case bsrCellKind::kDotsNone:   stringForCell = "\x20"; break;

    case bsrCellKind::kDots1:      stringForCell = "\x41"; break;
    case bsrCellKind::kDots2:      stringForCell = "\x31"; break;

		// ... ... ...

    case bsrCellKind::kDots23456:  stringForCell = "\x29"; break;
    case bsrCellKind::kDots123456: stringForCell = "\x3d"; break;
  } // switch

  fBrailleOutputStream <<
    stringForCell;
}
\end{lstlisting}

