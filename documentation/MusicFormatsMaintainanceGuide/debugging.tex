% !TEX root = MusicFormatsMaintainanceGuide.tex


% -------------------------------------------------------------------------
\chapter{Debugging}
% -------------------------------------------------------------------------

Debugging \mf\ can be quite time-consuming. The trace options available have designed to provide fine-grained tracing information to help locate issues.

\function{catchSignals} in \functionName {main} functions

File \wae{enableAbortToDebugErrors.h} contains:
\begin{lstlisting}[language=CPlusPlus]
/*
  MusicFormats Library
  Copyright (C) Jacques Menu 2016-2022

  This Source Code Form is subject to the terms of the Mozilla Public
  License, v. 2.0. If a copy of the MPL was not distributed with this
  file, You can obtain one at http://mozilla.org/MPL/2.0/.

  https://github.com/jacques-menu/musicformats
*/

#ifndef ___enableAbortToDebugErrorsIfDesired___
#define ___enableAbortToDebugErrorsIfDesired___


// comment the following definition if abort on internal errors is desired
// CAUTION: DON'T USE THIS IN PRODUCTION CODE,
// since that could kill a session on a \Web\ server, for example

#ifndef ABORT_TO_DEBUG_ERRORS
  #define ABORT_TO_DEBUG_ERRORS
#endif


#endif
\end{lstlisting}


% -------------------------------------------------------------------------
\section{Useful options}
% -------------------------------------------------------------------------

Here are the most basing options used when debugging:
\begin{itemize}
\item \optionBoth{trace-passes}{tpasses}
	this is the first option to use, to locate in which pass the problem arises

\item \optionBoth{input-line-numbers}{iln}
	this option produces the music elements input-line numbers in the output files

\item the \optionName{display*} options

\end{itemize}
