% -------------------------------------------------------------------------
%  MusicFormats Library
%  Copyright (C) Jacques Menu 2016-2023

%  This Source Code Form is subject to the terms of the Mozilla Public
%  License, v. 2.0. If a copy of the MPL was not distributed with this
%  file, you can obtain one at http://mozilla.org/MPL/2.0/.

%  https://github.com/jacques-menu/musicformats
% -------------------------------------------------------------------------

% !TEX root = MusicFormatsMaintainanceGuide.tex

% -------------------------------------------------------------------------
\chapter{Segments handling}\label{Segments handling}
% -------------------------------------------------------------------------

Segments are presented at \sectionRef{Segments}.

The segments concept used by \mf\ to describe music scores is not apparent to the users of GUI applications, in which music elements are \drawn\ on the page.
Their need is inherent to the representation of repeats, which contain music elements sequences (the segments) and even other repeats.

\MakeUppercase{All segments handling} in \mf\ \MakeUppercase{is done internally}: the \class{msrSegment} instances are created in voices and repeats \MakeUppercase{behind the curtains}.


% -------------------------------------------------------------------------
\section{Segments creation}\label{Segments creation}
% -------------------------------------------------------------------------

Instances of \class{msrSegment} are created at four places in \msr{msrVoices.cpp}:
\begin{lstlisting}[language=Terminal]
jacquesmenu@macmini: ~/musicformats-git-dev/src/formats/msr > grep 'msrSegment::create (' *.cpp
msrSegments.cpp:S_msrSegment msrSegment::create (
msrSegments.cpp:      msrSegment::create (
msrSegments.cpp:      msrSegment::create (
msrVoices.cpp:        msrSegment::create (
msrVoices.cpp:    msrSegment::create (
msrVoices.cpp:    msrSegment::create (
msrVoices.cpp:      msrSegment::create (
\end{lstlisting}

Calls to \method{msrSegment}{createSegmentNewbornClone} occurs only when visiting \class{msrSegment} instances in passes:
\begin{lstlisting}[language=Terminal]
jacquesmenu@macmini: ~/musicformats-git-dev/src/passes > grep createSegmentNewbornClone */*
msr2bsr/msr2bsrTranslator.cpp:      elt->createSegmentNewbornClone (
msr2lpsr/msr2lpsrTranslator.cpp:    elt->createSegmentNewbornClone (
msr2msr/msr2msrTranslator.cpp:    elt->createSegmentNewbornClone (
\end{lstlisting}

\Method{msrSegment}{createSegmentDeepClone} is not used at the time of this writing.

Explicit segments creation is thus entirely done in methods inside \msr{msrVoices.cpp}: the passes are not aware of this happening.

The first occurrence of \method{msrSegment}{create} is in \method{msrVoice}{initializeVoice}: when a voice is created, a segment is created and stored in its {\tt fVoiceLastSegment} if requested :
\begin{lstlisting}[language=CPlusPlus]
void msrVoice::initializeVoice (
  msrVoiceCreateInitialLastSegmentKind
    voiceCreateInitialLastSegmentKind)
{
	// ... ... ...

	  // create the initial last segment if requested
  switch (voiceCreateInitialLastSegmentKind) {
    case msrVoiceCreateInitialLastSegmentKind::kCreateInitialLastSegmentYes:
      // sanity check // JMI LAST
      mfAssert (
        __FILE__, __LINE__,
        fVoiceLastSegment == nullptr,
        "fVoiceLastSegment is null");

      // create the last segment
      fVoiceLastSegment =
        msrSegment::create (
          fInputLineNumber,
          this);

      if (! fVoiceFirstSegment) {
        fVoiceFirstSegment = fVoiceLastSegment;
      }
      break;
    case msrVoiceCreateInitialLastSegmentKind::kCreateInitialLastSegmentNo:
      break;
  } // switch

	// ... ... ...
};
\end{lstlisting}

\Method{msrVoice}{createMeasureRepeatFromItsFirstMeasures} is presented in \sectionRef{Measure repeats handling}, and the remaining two are presented in the next sections.


% -------------------------------------------------------------------------
\subsection{Creating a new last segment for a voice}
% -------------------------------------------------------------------------

There is \method{msrVoice}{createNewLastSegmentForVoice}, called at many places in \\
\msr{msrVoices.cpp}:
\begin{lstlisting}[language=CPlusPlus]
void msrVoice::createNewLastSegmentForVoice (
  int    inputLineNumber,
  const std::string& context)
{
	// ... ... ...

  // create the last segment
  fVoiceLastSegment =
    msrSegment::create (
      inputLineNumber,
      this);

  if (! fVoiceFirstSegment) {
    fVoiceFirstSegment = fVoiceLastSegment;
  }

	// ... ... ...
}
\end{lstlisting}

The calls to \method{msrVoice}{createNewLastSegmentForVoice} are in:
\begin{itemize}
\item \method{msrVoice}{createAMeasureAndAppendItToVoice}\\[-0.5ex]

\item \method{msrVoice}{appendStaffDetailsToVoice}\\[-0.5ex]

\item \method{msrVoice}{addGraceNotesGroupBeforeAheadOfVoiceIfNeeded}\\[-0.5ex]

\item \method{msrVoice}{handleVoiceLevelRepeatStart}
\item \method{msrVoice}{handleVoiceLevelRepeatEndWithoutStart}
\item \method{msrVoice}{handleVoiceLevelContainingRepeatEndWithoutStart}
\item \method{msrVoice}{handleVoiceLevelRepeatEndWithStart}%%%JMI commented
\item \method{msrVoice}{handleVoiceLevelRepeatEndingStartWithoutExplicitStart}
\item \method{msrVoice}{handleVoiceLevelRepeatEndingStartWithExplicitStart}\\[-0.5ex]

\item \method{msrVoice}{handleMultipleFullBarRestsStartInVoiceClone}\\[-0.5ex]

\item \method{msrVoice}{handleHooklessRepeatEndingEndInVoice}\\[-0.5ex]

\item \method{msrVoice}{appendBarLineToVoice}
\item \method{msrVoice}{appendSegnoToVoice}
\item \method{msrVoice}{appendCodaToVoice}
\item \method{msrVoice}{appendEyeGlassesToVoice}
\item \method{msrVoice}{appendPedalToVoice}
\item \method{msrVoice}{appendDampToVoice}
\item \method{msrVoice}{appendDampAllToVoice}
\end{itemize}


% -------------------------------------------------------------------------
\subsection{Creating a new last segment for a voice from its first measure}
% -------------------------------------------------------------------------

\Method{msrVoice}{createNewLastSegmentFromItsFirstMeasureForVoice} is used at several places in \msr{msrVoices.cpp}:
\begin{lstlisting}[language=CPlusPlus]
void msrVoice::createNewLastSegmentFromItsFirstMeasureForVoice (
  int          inputLineNumber,
  S_msrMeasure firstMeasure,
  std::string       context)
{
  // create the last segment
  fVoiceLastSegment =
    msrSegment::create (
      inputLineNumber,
      this);

  if (! fVoiceFirstSegment) {
    fVoiceFirstSegment = fVoiceLastSegment;
  }

	// ... ... ...

  // append firstMeasure to fVoiceLastSegment
  fVoiceLastSegment->
    appendMeasureToSegment (firstMeasure);

  // firstMeasure is the new voice last appended measure
  fVoiceLastAppendedMeasure = firstMeasure;

  // is firstMeasure the first one it the voice?
  if (! fVoiceFirstMeasure) {
    // yes, register it as such
    setVoiceFirstMeasure (
      firstMeasure);

    firstMeasure->
      setMeasureFirstInVoice ();
  }

	// ... ... ...
}
\end{lstlisting}

All the uses of this method concern repeats (\sectionRef{Repeats handling}), measure repeats (\sectionRef{Measure repeats handling}) and multiple full-bar rests(\sectionRef{Full-bar rests handling}).


% -------------------------------------------------------------------------
\section{Appending measures to a segment}
% -------------------------------------------------------------------------

\Method{msrSegment}{assertSegmentElementsListIsNotEmpty} is called as a sanity check by many methods in \msr{msrSegments.cpp}:
\begin{lstlisting}[language=CPlusPlus]
void msrSegment::assertSegmentElementsListIsNotEmpty (
  int inputLineNumber) const
{
  if (! fSegmentElementsList.size ()) {
#ifdef MF_TRACE_IS_ENABLED
  if (
    gTraceOahGroup->getTraceMeasuresDetails ()
      ||
    gTraceOahGroup->getTraceSegmentsDetails ()
      ||
    gTraceOahGroup->getTraceRepeatsDetails ()
  ) {
    fSegmentUpLinkToVoice->
      displayVoiceRepeatsStackMultipleFullBarRestsMeasureRepeatAndVoice (
        inputLineNumber,
        "assertSegmentElementsListIsNotEmpty()");
  }
#endif // MF_TRACE_IS_ENABLED

    gLog <<
      "assertSegmentElementsListIsNotEmpty()" <<
      ", fSegmentElementsList is empty" <<
      ", segment: " <<
      this->asString () <<
      ", in voice \"" <<
      fSegmentUpLinkToVoice->getVoiceName () <<
      "\"" <<
      "', line " << inputLineNumber <<
      std::endl;

    mfAssert (
      __FILE__, __LINE__,
      false,
      ", fSegmentElementsList is empty");
  }
}
\end{lstlisting}

One such call is:
\begin{lstlisting}[language=CPlusPlus]
void msrSegment::appendKeyToSegment (
  const S_msrKey& key)
{
#ifdef MF_TRACE_IS_ENABLED
  if (gTraceOahGroup->getTraceKeys ()) {
		std::stringstream ss;

    ss <<
      "Appending key " << key->asString () <<
      " to segment " << asString () <<
    ", in voice \"" <<
    fSegmentUpLinkToVoice->getVoiceName () <<
    "\"";

    gWaeHandler->waeTraceWithoutInputLocation (
      __FILE__, __LINE__,
      ss.str ());
  }
#endif // MF_TRACE_IS_ENABLED

#ifdef MF_SANITY_CHECKS_ARE_ENABLED
  // sanity check
  assertSegmentElementsListIsNotEmpty (
    key->getInputLineNumber ());
#endif // MF_SANITY_CHECKS_ARE_ENABLED

  ++gIndenter;

  // register key in segments's current measure
  fSegmentElementsList.back ()->
    appendKeyToMeasure (key);

  --gIndenter;
}
\end{lstlisting}


% -------------------------------------------------------------------------
\section{Translating from MXSR to MSR}
% -------------------------------------------------------------------------


% -------------------------------------------------------------------------
\section{Translating from MXSR to MSR}
% -------------------------------------------------------------------------

This is done in \mxsrToMsr{}.


% -------------------------------------------------------------------------
\section{Translating from MSR to MSR}
% -------------------------------------------------------------------------

This is done in \msrToMsr{}.


% -------------------------------------------------------------------------
\section{Translating from MSR to LPSR}
% -------------------------------------------------------------------------

This is done in \msrToLpsr{}.


% -------------------------------------------------------------------------
\section{Translating from LPSR to LilyPond}
% -------------------------------------------------------------------------

This is done in \lpsrToLilypond{}.


