% !TEX root = MusicFormatsMaintainanceGuide.tex

% -------------------------------------------------------------------------
\chapter{CPU measurements}
% -------------------------------------------------------------------------

\Option{cpu} displays the time spent in the successive passes, such as:
\begin{lstlisting}[language=Terminal]
Activity  Description                                             Kind       CPU (sec)
--------  ------------------------------------------------------  ---------  ---------

          Handle the options and arguments from argc/argv         mandatory    0.01187
Pass 1    Create an MXSR reading a MusicXML file              mandatory    0.00471
Pass 2a   Create an MSR skeleton from the MXSR                mandatory    0.00222
Pass 2b   Populate the MSR skeleton from MusicXML data            mandatory    0.00405
Pass 4    Convert the MSR into an LPSR                            mandatory    0.00137
Pass 5    Convert the LPSR score to LilyPond code                 mandatory    0.00136

Total (sec)  Mandatory  Optional
-----------  ---------  ---------
0.02558      0.02558    0.00000
\end{lstlisting}

These numbers are for the CPU only, not including input and output tasks.
The time spent in options handling is roughly always the same on a given machine.

\Class{mfTimingItemsList}, defined in \mfutilitiesBoth{mfTiming}, provides:
\begin{lstlisting}[language=CPlusPlus]
class EXP mfTimingItemsList {
	// ... ... ...

  public:

    // global variable for general use
    // ------------------------------------------------------

    static mfTimingItemsList         gGlobalTimingItemsList;

  public:

    // public services
    // ------------------------------------------------------

    // add an item
    void                  appendTimingItem (
                            string                     activity,
                            string                     description,
                            mfTimingItem::timingItemKind kind,
                            clock_t                    startClock,
                            clock_t                    endClock);

	// ... ... ...
}
\end{lstlisting}

\newpage

\Function {translateMsrToLpsrScore} in \msrToLpsr{msr2lpsrInterface.cpp} measures time to perform the conversion this way:
\begin{lstlisting}[language=CPlusPlus]
S_lpsrScore translateMsrToLpsr (
  S_msrScore          originalMsrScore,
  S_msrOahGroup       msrOpts,
  S_lpsrOahGroup      lpsrOpts,
  string              passNumber,
  string              passDescription,
  S_mfcMultiComponent multiComponent)
{
  if (gGlobalLpsr2lilypondOahGroup->getNoLilypondCode ()) {
    gLogStream <<
      "Option '-nolpc, -no-lilypond-code' is set, no LPSR is created" <<
      endl;

    return nullptr;
  }

  // sanity check
  mfAssert (
    __FILE__, __LINE__,
    originalMsrScore != nullptr,
    "originalMsrScore is null");

  // start the clock
  clock_t startClock = clock ();

#ifdef TRACING_IS_ENABLED
  if (gGlobalOahEarlyOptions.getEarlyTracePasses ()) {
    string separator =
      "%--------------------------------------------------------------";

    gLogStream <<
      endl <<
      separator <<
      endl <<
      gTab <<
      passNumber << ": " << passDescription <<
      endl <<
      separator <<
      endl;
  }
#endif

  // create an msr2lpsrTranslator
  msr2lpsrTranslator
    translator (
      originalMsrScore);

  // build the LPSR score
  S_lpsrScore
    resultingLpsr =
      translator.translateMsrToLpsr (
        originalMsrScore,
        multiComponent);

  clock_t endClock = clock ();

  // register time spent
  mfTimingItemsList::gGlobalTimingItemsList.appendTimingItem (
    passNumber,
    passDescription,
    mfTimingItem::kMandatory,
    startClock,
    endClock);
\end{lstlisting}
