% !TEX root = MusicFormatsMaintainanceGuide.tex

% -------------------------------------------------------------------------
\chapter{Components}
% -------------------------------------------------------------------------


% -------------------------------------------------------------------------
\section{Components terminology}
% -------------------------------------------------------------------------

In compiler writing terminology:
\begin{itemize}
\item an external format %%%JMI
\item an internal representation is a data structure representing the program being compiled;
\item there are often several internal representations, to simplify the compiler internal workings or for optimisation purposes;
\item the output of the compiler, such as binary code for some physical or emulated processor, is a last 'representation' of the program;
\item a pass converts an internal representation into another one, in a single step;
\item a multi-pass converter is a chain of passes, reading the input, converting it into a first internal representation, then a pass to convert it into another internal representation, and so on until the compiler output is produced.
\end{itemize}

\mf\ maps exactly to this model, providing the following components:
\begin{itemize}
\item internal representations (formats for short) of the music score: \msrRepr, \lpsrRepr, \bsrRepr\ and \mxsrRepr;
\item several passes are available to convert such formats into others;
\item a set of multi-pass converters are supplied, such as \xmlToLy\, \xmlToXml\ and \msdLangConv.%%%JMI
\end{itemize}

In the \mf\ user documentation, the term '\converter' is used because it is more meaningfull for musicians.

\mf\ provides high-level interfaces to its components as functions\index{functions} in \fileName{Interface} files:
\begin{lstlisting}[language=Terminal]
jacquesmenu@macmini: ~/musicformats-git-dev/src > look Interface
./formats/msr/msrInterface.cpp
./formats/msr/msrInterface.h
./formats/lpsr/lpsrInterface.cpp
./formats/lpsr/lpsrInterface.h
./formats/bsr/bsrInterface.h
./formats/bsr/bsrInterface.cpp
./passes/mxsr2musicxml/mxsr2musicxmlTranlatorInterface.h
./passes/mxsr2musicxml/mxsr2musicxmlTranlatorInterface.cpp
./passes/bsr2bsr/bsr2bsrFinalizerInterface.h
./passes/bsr2bsr/bsr2bsrFinalizerInterface.cpp
./passes/msr2mxsr/msr2mxsrInterface.cpp
./passes/msr2mxsr/msr2mxsrInterface.h
./passes/mxsr2msr/mxsr2msrSkeletonBuilderInterface.h
./passes/mxsr2msr/mxsr2msrTranslatorInterface.cpp
./passes/mxsr2msr/mxsr2msrTranslatorInterface.h
./passes/mxsr2msr/mxsr2msrSkeletonBuilderInterface.cpp
./passes/msr2msr/msr2msrInterface.h
./passes/msr2msr/msr2msrInterface.cpp
./passes/lpsr2lilypond/lpsr2lilypondInterface.h
./passes/lpsr2lilypond/lpsr2lilypondInterface.cpp
./passes/msr2lpsr/msr2lpsrInterface.cpp
./passes/msr2lpsr/msr2lpsrInterface.h
./passes/bsr2braille/bsr2brailleTranslatorInterface.h
./passes/bsr2braille/bsr2brailleTranslatorInterface.cpp
./passes/msr2bsr/msr2bsrInterface.h
./passes/msr2bsr/msr2bsrInterface.cpp
./passes/musicxml2mxsr/musicxml2mxsrInterface.h
./passes/musicxml2mxsr/musicxml2mxsrInterface.cpp
./passes/mxsr2guido/mxsr2guidoTranlatorInterface.h
./passes/mxsr2guido/mxsr2guidoTranlatorInterface.cpp
./converters/msr2guido/msr2guidoInterface.h
./converters/msr2guido/msr2guidoInterface.cpp
./converters/msr2braille/msr2brailleInterface.h
./converters/msr2braille/msr2brailleInterface.cpp
./converters/msdl2braille/msdl2brailleInterface.h
./converters/msdl2braille/msdl2brailleInterface.cpp
./converters/msdl2guido/msdl2guidoInterface.cpp
./converters/msdl2guido/msdl2guidoInterface.h
./converters/msdl2musicxml/msdl2musicxmlInterface.h
./converters/msdl2musicxml/msdl2musicxmlInterface.cpp
./converters/msdl2lilypond/msdl2lilypondInterface.h
./converters/msdl2lilypond/msdl2lilypondInterface.cpp
./converters/musicxml2braille/musicxml2brailleInterface.cpp
./converters/musicxml2braille/musicxml2brailleInterface.h
./converters/msr2lilypond/msr2lilypondInterface.cpp
./converters/msr2lilypond/msr2lilypondInterface.h
./converters/msr2musicxml/msr2musicxmlInterface.cpp
./converters/msr2musicxml/msr2musicxmlInterface.h
./converters/musicxml2musicxml/musicxml2musicxmlInterface.h
./converters/musicxml2musicxml/musicxml2musicxmlInterface.cpp
./converters/musicxml2lilypond/musicxml2lilypondInterface.h
./converters/musicxml2lilypond/musicxml2lilypondInterface.cpp
./converters/musicxml2guido/musicxml2guidoInterface.cpp
./converters/musicxml2guido/musicxml2guidoInterface.h
\end{lstlisting}

The converters are implemented as functions\index{functions} as well as CLI tools that use the latter.

\mf\ includes support for components versions numbering and history, see \chapterRef{MusicFormats components}.

\mfc{mfcComponents.h} includes all the components's header files.


% -------------------------------------------------------------------------
\section{Formats}
% -------------------------------------------------------------------------

The \format s are in \src{formats}:
\begin{lstlisting}[language=Terminal]
jacquesmenu@macmini: ~/musicformats-git-dev/src > ll formats/
total 32
 0 drwxr-xr-x   10 jacquesmenu  staff    320 Jun 25 05:39:49 2021 ./
 0 drwxr-xr-x   13 jacquesmenu  staff    416 Jun 17 17:16:37 2021 ../
24 -rw-r--r--    1 jacquesmenu  staff  10244 Jun 19 07:58:55 2021 .DS_Store
 0 drwxr-xr-x   60 jacquesmenu  staff   1920 Jun 18 07:32:14 2021 bsr/
 0 drwxr-xr-x   42 jacquesmenu  staff   1344 May 26 08:20:55 2021 lpsr/
 0 drwxr-xr-x   12 jacquesmenu  staff    384 Apr 22 15:49:23 2021 msdl/
 0 drwxr-xr-x   10 jacquesmenu  staff    320 May 26 08:20:55 2021 msdr/
 0 drwxr-xr-x  151 jacquesmenu  staff   4832 Jun 20 09:58:00 2021 msr/
 0 drwxr-xr-x    6 jacquesmenu  staff    192 May 26 08:20:55 2021 mxsr/
\end{lstlisting}

The formats interfaces are in files with the format's name:
\begin{lstlisting}[language=Terminal]
jacquesmenu@macmini: ~/musicformats-git-dev/src > ll formats/bsr/bsr.*
8 -rw-r--r--@ 1 jacquesmenu  staff   700 Jun  6 06:35:19 2021 formats/bsr/bsr.cpp
8 -rw-r--r--@ 1 jacquesmenu  staff  1206 Jun 18 10:04:45 2021 formats/bsr/bsr.h

jacquesmenu@macmini: ~/musicformats-git-dev/src > ll formats/lpsr/lpsr.*
8 -rw-r--r--@ 1 jacquesmenu  staff   703 Jun  6 06:35:19 2021 formats/lpsr/lpsr.cpp
8 -rw-r--r--@ 1 jacquesmenu  staff  1004 Jun  6 06:35:19 2021 formats/lpsr/lpsr.h

jacquesmenu@macmini: ~/musicformats-git-dev/src > ll formats/msdl/msdl.*
8 -rw-r--r--@ 1 jacquesmenu  staff  736 Jun  6 06:35:19 2021 formats/msdl/msdl.cpp
8 -rw-r--r--@ 1 jacquesmenu  staff  643 Jun  6 06:35:19 2021 formats/msdl/msdl.h

jacquesmenu@macmini: ~/musicformats-git-dev/src > ll formats/msdr/msdr.*
8 -rw-r--r--@ 1 jacquesmenu  staff  709 Jun  6 06:35:19 2021 formats/msdr/msdr.cpp
8 -rw-r--r--@ 1 jacquesmenu  staff  531 Jun  6 06:35:19 2021 formats/msdr/msdr.h

jacquesmenu@macmini: ~/musicformats-git-dev/src > ll formats/msr/msr.*
8 -rw-r--r--@ 1 jacquesmenu  staff   700 Jun  6 06:35:19 2021 formats/msr/msr.cpp
8 -rw-r--r--@ 1 jacquesmenu  staff  2410 Jun 20 09:58:38 2021 formats/msr/msr.h

jacquesmenu@macmini: ~/musicformats-git-dev/src > ll formats/mxsr/mxsr.*
8 -rw-r--r--@ 1 jacquesmenu  staff  3292 Jun  6 06:35:19 2021 formats/mxsr/mxsr.cpp
8 -rw-r--r--@ 1 jacquesmenu  staff  1555 Jun  6 06:35:19 2021 formats/mxsr/mxsrGeneration.h
\end{lstlisting}


% -------------------------------------------------------------------------
\section{Representations}
% -------------------------------------------------------------------------

The \representation s are in \src{representations}:
\begin{lstlisting}[language=Terminal]
jacquesmenu@macmini: ~/musicformats-git-dev/src > ll representations/
total 24
 0 drwxr-xr-x   11 jacquesmenu  staff    352 Dec 30 17:25:10 2021 ./
 0 drwxr-xr-x   18 jacquesmenu  staff    576 Jan 16 16:50:25 2022 ../
24 -rw-r--r--@   1 jacquesmenu  staff  10244 Jan  6 17:40:44 2022 .DS_Store
 0 drwxr-xr-x    8 jacquesmenu  staff    256 Dec 30 10:26:26 2021 braille/
 0 drwxr-xr-x   69 jacquesmenu  staff   2208 Jan  4 07:52:14 2022 bsr/
 0 drwxr-xr-x    4 jacquesmenu  staff    128 Dec 30 10:27:01 2021 guido/
 0 drwxr-xr-x   51 jacquesmenu  staff   1632 Jan  4 07:52:36 2022 lpsr/
 0 drwxr-xr-x   16 jacquesmenu  staff    512 Jan  4 07:52:55 2022 msdl/
 0 drwxr-xr-x   12 jacquesmenu  staff    384 Jan  4 07:53:13 2022 msdr/
 0 drwxr-xr-x  165 jacquesmenu  staff   5280 Jan  4 07:53:34 2022 msr/
 0 drwxr-xr-x   10 jacquesmenu  staff    320 Jan  4 07:53:54 2022 mxsr/
\end{lstlisting}


% -------------------------------------------------------------------------
\section{Passes}
% -------------------------------------------------------------------------

The \pass s are in \src{passes}:
\begin{lstlisting}[language=Terminal]
jacquesmenu@macmini: ~/musicformats-git-dev/src > ll passes
total 24
 0 drwxr-xr-x  14 jacquesmenu  staff    448 Nov 24 16:29:20 2021 ./
 0 drwxr-xr-x  20 jacquesmenu  staff    640 Nov 16 08:12:03 2021 ../
24 -rw-r--r--@  1 jacquesmenu  staff  10244 Nov 24 10:38:11 2021 .DS_Store
 0 drwxr-xr-x   8 jacquesmenu  staff    256 Oct 22 07:19:11 2021 bsr2braille/
 0 drwxr-xr-x   6 jacquesmenu  staff    192 Oct 22 07:20:34 2021 bsr2bsr/
 0 drwxr-xr-x  10 jacquesmenu  staff    320 Nov 16 10:09:27 2021 lpsr2lilypond/
 0 drwxr-xr-x  14 jacquesmenu  staff    448 Oct 22 07:22:09 2021 msdl2msr/
 0 drwxr-xr-x   8 jacquesmenu  staff    256 Oct 22 07:24:35 2021 msr2bsr/
 0 drwxr-xr-x   8 jacquesmenu  staff    256 Nov  1 16:31:34 2021 msr2lpsr/
 0 drwxr-xr-x   8 jacquesmenu  staff    256 Nov  1 16:31:34 2021 msr2msr/
 0 drwxr-xr-x   6 jacquesmenu  staff    192 Oct 22 07:27:46 2021 msr2mxsr/
 0 drwxr-xr-x   4 jacquesmenu  staff    128 Oct 22 07:28:37 2021 mxsr2guido/
 0 drwxr-xr-x  10 jacquesmenu  staff    320 Nov  1 16:31:34 2021 mxsr2msr/
 0 drwxr-xr-x   4 jacquesmenu  staff    128 Oct 22 07:29:50 2021 mxsr2musicxml/
\end{lstlisting}

Some passes are named translators (converters could have been used), and others are not. In \mxsrToMsr{}, \class{mxsr2msrSkeletonBuilder} does not translate \mxml\ data to another full representation: it merely creates a skeleton containing voices, are are empty:
\begin{lstlisting}[language=Terminal]
jacquesmenu@macmini: ~/musicformats-git-dev/src > ll passes/mxsr2msr/
total 1808
   0 drwxr-xr-x   8 jacquesmenu  staff     256 Jun 25 05:47:41 2021 ./
   0 drwxr-xr-x  16 jacquesmenu  staff     512 May 26 08:20:55 2021 ../
  96 -rw-r--r--@  1 jacquesmenu  staff   48389 Jun 21 07:43:20 2021 mxsr2msrOah.cpp
  40 -rw-r--r--@  1 jacquesmenu  staff   20327 Jun 16 10:41:37 2021 mxsr2msrOah.h
 192 -rw-r--r--@  1 jacquesmenu  staff   97896 Jun 25 08:58:38 2021 mxsr2msrSkeletonBuilder.cpp
  48 -rw-r--r--@  1 jacquesmenu  staff   20942 Jun 25 07:36:29 2021 mxsr2msrSkeletonBuilder.h
1280 -rw-r--r--@  1 jacquesmenu  staff  651474 Jun 25 07:49:52 2021 mxsr2msrTranslator.cpp
 152 -rw-r--r--@  1 jacquesmenu  staff   77039 Jun 21 07:43:20 2021 mxsr2msrTranslator.h
\end{lstlisting}

The passes functionality is available as functions\index{functions} in \fileName{*Interface.*}:
\begin{lstlisting}[language=Terminal]
jacquesmenu@macmini: ~/musicformats-git-dev/src > look Interface
./representations/msr/msrInterface.cpp
./representations/msr/msrInterface.h
./representations/lpsr/lpsrInterface.cpp
./representations/lpsr/lpsrInterface.h
./representations/bsr/bsrInterface.h
./representations/bsr/bsrInterface.cpp
./passes/mxsr2musicxml/mxsr2musicxmlTranlatorInterface.h
./passes/mxsr2musicxml/mxsr2musicxmlTranlatorInterface.cpp
./passes/bsr2bsr/bsr2bsrFinalizerInterface.h
./passes/bsr2bsr/bsr2bsrFinalizerInterface.cpp
./passes/msr2mxsr/msr2mxsrInterface.cpp
./passes/msr2mxsr/msr2mxsrInterface.h
./passes/mxsr2msr/mxsr2msrSkeletonBuilderInterface.h
./passes/mxsr2msr/mxsr2msrTranslatorInterface.cpp
./passes/mxsr2msr/mxsr2msrTranslatorInterface.h
./passes/mxsr2msr/mxsr2msrSkeletonBuilderInterface.cpp
./passes/msr2msr/msr2msrInterface.h
./passes/msr2msr/msr2msrInterface.cpp
./passes/lpsr2lilypond/lpsr2lilypondInterface.h
./passes/lpsr2lilypond/lpsr2lilypondInterface.cpp
./passes/msr2lpsr/msr2lpsrInterface.cpp
./passes/msr2lpsr/msr2lpsrInterface.h
./passes/bsr2braille/bsr2brailleTranslatorInterface.h
./passes/bsr2braille/bsr2brailleTranslatorInterface.cpp
./passes/msr2bsr/msr2bsrInterface.h
./passes/msr2bsr/msr2bsrInterface.cpp
./passes/musicxml2mxsr/musicxml2mxsrInterface.h
./passes/musicxml2mxsr/musicxml2mxsrInterface.cpp
./passes/mxsr2guido/mxsr2guidoTranlatorInterface.h
./passes/mxsr2guido/mxsr2guidoTranlatorInterface.cpp
./converters/msr2guido/msr2guidoInterface.h
./converters/msr2guido/msr2guidoInterface.cpp
./converters/msr2braille/msr2brailleInterface.h
./converters/msr2braille/msr2brailleInterface.cpp
./converters/msdl2braille/msdl2brailleInterface.h
./converters/msdl2braille/msdl2brailleInterface.cpp
./converters/msdl2guido/msdl2guidoInterface.cpp
./converters/msdl2guido/msdl2guidoInterface.h
./converters/msdl2musicxml/msdl2musicxmlInterface.h
./converters/msdl2musicxml/msdl2musicxmlInterface.cpp
./converters/msdl2lilypond/msdl2lilypondInterface.h
./converters/msdl2lilypond/msdl2lilypondInterface.cpp
./converters/musicxml2braille/musicxml2brailleInterface.cpp
./converters/musicxml2braille/musicxml2brailleInterface.h
./converters/msr2lilypond/msr2lilypondInterface.cpp
./converters/msr2lilypond/msr2lilypondInterface.h
./converters/msr2musicxml/msr2musicxmlInterface.cpp
./converters/msr2musicxml/msr2musicxmlInterface.h
./converters/musicxml2musicxml/musicxml2musicxmlInterface.h
./converters/musicxml2musicxml/musicxml2musicxmlInterface.cpp
./converters/musicxml2lilypond/musicxml2lilypondInterface.h
./converters/musicxml2lilypond/musicxml2lilypondInterface.cpp
./converters/musicxml2guido/musicxml2guidoInterface.cpp
./converters/musicxml2guido/musicxml2guidoInterface.h
\end{lstlisting}


% -------------------------------------------------------------------------
\section{Generators}
% -------------------------------------------------------------------------

A \generator\ is a multi-pass \CLI\ tool that creates an ouput from scratch, without reading anything. All of them use \multiGenerationBoth{multiGeneration} to offer a set of output formats:
\begin{itemize}
\item \clisamples{Mikrokosmos3Wandering.cpp}\\
			creates a score for this Bartok piece in various forms, depending on the options. It has been used to check th e MSR \API's:

\begin{lstlisting}[language=Terminal]
jacquesmenu@macmini: ~/musicformats-git-dev/src > ll formatsgeneration/multiGeneration/
total 56
 0 drwxr-xr-x   4 jacquesmenu  staff    128 Apr 22 15:49:16 2021 ./
 0 drwxr-xr-x  10 jacquesmenu  staff    320 May 26 08:20:55 2021 ../
40 -rw-r--r--@  1 jacquesmenu  staff  16774 Jun  6 06:38:55 2021 multiGenerationOah.cpp
16 -rw-r--r--@  1 jacquesmenu  staff   6750 Jun  6 06:38:55 2021 mfMultiGenerationOah.h
\end{lstlisting}

For example:
\begin{lstlisting}[language=Terminal]
jacquesmenu@macmini: ~ > Mikrokosmos3Wandering -lilypond -a
What LilyPondIssue34 does:

    This multi-pass generator creates a textual representation
    of the LilyPondIssue34 score.
    It basically performs 4 passes when generating LilyPond output output:

        Pass 1:  generate a first MSR for the LilyPondIssue34 score
        Pass 2:  converts the first MSR a second MSR;
        Pass 3:  converts the second MSR into a
                 LilyPond Score Representation (LPSR);
        Pass 4:  converts the LPSR to LilyPond code
                 and writes it to standard output.

    Other passes are performed according to the options, such as
    displaying views of the internal data or printing a summary of the score.

    The activity log and warning/error messages go to standard error.
\end{lstlisting}

\item \clisamples{LilyPondIssue34.cpp}
			creates a score for the \lily\ issue \#34 issue, also in various forms:;
\begin{lstlisting}[language=Terminal]
jacquesmenu@macmini: ~ > LilyPondIssue34 -musicxml -a
What LilyPondIssue34 does:

    This multi-pass generator creates a textual representation
    of the LilyPondIssue34 score.
    It basically performs 4 passes when generating MusicXML output output:

        Pass 1:  generate a first MSR for the LilyPondIssue34 score
        Pass 2:  converts the first MSR a second MSR, to apply options;
        Pass 3:  converts the second MSR into an MusicXML tree;
        Pass 4:  converts the MusicXML tree to MusicXML code
                 and writes it to standard output.

    Other passes are performed according to the options, such as
    displaying views of the internal data or printing a summary of the score.

    The activity log and warning/error messages go to standard error.
\end{lstlisting}
\end{itemize}


% -------------------------------------------------------------------------
\section{Converters}
% -------------------------------------------------------------------------

The \mf\ \converter s chain passes into a sequence, each pass reading the input or the format produced by the preceeding one:
\begin{lstlisting}[language=Terminal]
jacquesmenu@macmini: ~/musicformats-git-dev/src > ll converters/
total 32
 0 drwxr-xr-x  17 jacquesmenu  staff    544 May 26 08:20:55 2021 ./
 0 drwxr-xr-x  13 jacquesmenu  staff    416 Jun 17 17:16:37 2021 ../
24 -rw-r--r--   1 jacquesmenu  staff  10244 Jun 18 10:34:45 2021 .DS_Store
 0 drwxr-xr-x   8 jacquesmenu  staff    256 May 26 08:20:55 2021 msdl2braille/
 0 drwxr-xr-x   8 jacquesmenu  staff    256 May 26 08:20:55 2021 msdl2guido/
 0 drwxr-xr-x   8 jacquesmenu  staff    256 May 26 08:20:55 2021 msdl2lilypond/
 0 drwxr-xr-x   8 jacquesmenu  staff    256 May 26 08:20:55 2021 msdl2musicxml/
 0 drwxr-xr-x   8 jacquesmenu  staff    256 May 26 08:20:55 2021 msdlconverter/
 0 drwxr-xr-x   8 jacquesmenu  staff    256 May 26 08:20:55 2021 msr2braille/
 0 drwxr-xr-x   8 jacquesmenu  staff    256 May 26 08:20:55 2021 msr2guido/
 0 drwxr-xr-x   8 jacquesmenu  staff    256 May 26 08:20:55 2021 msr2lilypond/
 0 drwxr-xr-x   8 jacquesmenu  staff    256 May 26 08:20:55 2021 msr2musicxml/
 0 drwxr-xr-x   4 jacquesmenu  staff    128 May 26 08:20:55 2021 musicxml2braille/
 0 drwxr-xr-x   4 jacquesmenu  staff    128 May 26 08:20:55 2021 musicxml2guido/
 0 drwxr-xr-x   4 jacquesmenu  staff    128 May 26 08:20:55 2021 musicxml2lilypond/
 0 drwxr-xr-x   4 jacquesmenu  staff    128 May 26 08:20:55 2021 musicxml2musicxml/
\end{lstlisting}


% -------------------------------------------------------------------------
\section{Running a service}
% -------------------------------------------------------------------------

When a \mf\ \service\ is \MainIt{run} from the \CLI\ or through an \API\ function, an instance of \class{mfServiceRunData} is created.

This class is defined in \mflibraryBoth{mfServiceRunData} to hold data specific to the run. They are global data, but don't belong to the regular, invariant data contained in the library, such as the notes pitches in various languages: %%%JMI
\begin{lstlisting}[language=CPlusPlus]
class EXP mfServiceRunData : public smartable
{
  public:

    // creation
    // ------------------------------------------------------

    static SMARTP<mfServiceRunData> create (const string& serviceName);

    static SMARTP<mfServiceRunData> create (
                            const string& serviceName,
                            int           argc,
                            char*         argv[]);

    static SMARTP<mfServiceRunData> create (
                            const string&           serviceName,
                            mfOptionsAndArguments& optionsAndArguments);

  public:

    // constructors/destructor
    // ------------------------------------------------------

                          mfServiceRunData (const string& serviceName);

                          mfServiceRunData (
                            const string& serviceName,
                            int           argc,
                            char*         argv[]);

                          mfServiceRunData (
                            const string&           serviceName,
                            mfOptionsAndArguments& optionsAndArguments);

    virtual               ~mfServiceRunData ();

	// .. .. ..

  private:

    // private fields
    // ------------------------------------------------------

    // service name
    string                fServiceName;

    // conversion date
    string                fRunDateFull;
    string                fRunDateYYYYMMDD;

    // conversion command
    string                fCommandAsSupplied;

    string                fCommandWithLongOptionsNames;
    string                fCommandWithShortOptionsNames;

    // options and arguments
    mfOptionsAndArguments
                          fOptionsAndArguments;

    // command line
    string                fCommandLineAsSupplied;

    // input source
    string                fInputSourceName;
};
\end{lstlisting}

The various constructors are used depending on the way the service is run.

For example, if is created this way in \clisamples{xml2ly.cpp}:
\begin{lstlisting}[language=CPlusPlus]
int main (int argc, char* argv[])
{
  // setup signals catching
  // ------------------------------------------------------

// JMI	catchSignals ();

  // the service name
  // ------------------------------------------------------

  string serviceName = argv [0];

  // create the global output and log indented streams
  // ------------------------------------------------------

  createTheGlobalIndentedOstreams (cout, cerr);

  // create the global run data
  // ------------------------------------------------------

  gGlobalServiceRunData =
    mfServiceRunData::create (serviceName);

	// ... ... ...
}
\end{lstlisting}

Then the various run data can be accessed easily:
\begin{lstlisting}[language=CPlusPlus]
  string
    inputSourceName =
      gGlobalServiceRunData->getInputSourceName ();
\end{lstlisting}

The run date is used for example in \class{lpsrScore}, defined in \lpsrBoth{lpsrScores}:%%%JMI
\begin{lstlisting}[language=CPlusPlus]
lpsrScore::lpsrScore (
  int                 inputLineNumber,
  S_msrScore          theMsrScore,
  S_mfcMultiComponent multiComponent)
    : lpsrElement (inputLineNumber)
{
	// ...

  fMultiComponent = multiComponent;

  // should the initial comments about the service and the options used
  // be generated?
  if (gGlobalLpsr2lilypondOahGroup->getXml2lyInfos ()) {
    // create the 'generated by' comment
    {
      stringstream s;

      s <<
        "Generated by " <<
        gGlobalOahOahGroup->getOahOahGroupServiceName () <<
        ' ' <<
        getGlobalMusicFormatsVersionNumberAndDate () <<
        endl <<

        "% on " <<
        gGlobalServiceRunData->getRunDateFull () <<
        endl <<

        "% from ";

      string inputSourceName =
        gGlobalServiceRunData->getInputSourceName ();

      if (inputSourceName == "-") {
        s << "standard input";
      }
      else {
        s << "\"" << inputSourceName << "\"";
      }

      fInputSourceNameComment =
        lpsrComment::create (
          inputLineNumber,
          s.str (),
          lpsrComment::kGapAfterwardsYes);
    }
	}

	// ...
}
\end{lstlisting}

