% -------------------------------------------------------------------------
%  MusicFormats Library
%  Copyright (C) Jacques Menu 2016-2023

%  This Source Code Form is subject to the terms of the Mozilla Public
%  License, v. 2.0. If a copy of the MPL was not distributed with this
%  file, you can obtain one at http://mozilla.org/MPL/2.0/.

%  https://github.com/jacques-menu/musicformats
% -------------------------------------------------------------------------

% !TEX root = MusicFormatsMaintainanceGuide.tex

% -------------------------------------------------------------------------
\chapter{Figured bass handling}\label{Figured bass elements handling}
% -------------------------------------------------------------------------

Figured bass elements are presented at \sectionRef{Figured bass elements}.

The useful options here are:
\begin{itemize}
\item \optionBoth{trace-figured-bass}{tfigbass}
\item \optionBoth{display-msr1-skeleton}{dmsrskel}

\item \optionBoth{display-msr1}{dmsr1}
\item \optionBoth{display-msr1-full}{dmsr1full}

\item \optionBoth{display-msr2}{dmsr2}
\item \optionBoth{display-msr2-full}{dmsr2full}

\item \optionBoth{display-lpsr}{dlpsr}
\item \optionBoth{display-lpsr-full}{dlpsrfull}
\end{itemize}


% -------------------------------------------------------------------------
\section{Figured bass in MusicXML}
% -------------------------------------------------------------------------

In the \mxml\ view of figured bass, figured bass elements are simply \drawn\ at the current music position, so to say.
\begin{lstlisting}[language=Terminal]
<!--
	Figured bass elements take their position from the first
	regular note (not a grace note or chord note) that follows
	in score order. The optional duration element is used to
	indicate changes of figures under a note.

	Figures are ordered from top to bottom. A figure-number is
	a number. Values for prefix and suffix include plus and
	the accidental values sharp, flat, natural, double-sharp,
	flat-flat, and sharp-sharp. Suffixes include both symbols
	that come after the figure number and those that overstrike
	the figure number. The suffix values slash, back-slash, and
	vertical are used for slashed numbers indicating chromatic
	alteration. The orientation and display of the slash usually
	depends on the figure number. The prefix and suffix elements
	may contain additional values for symbols specific to
	particular figured bass styles. The value of parentheses
	is "no" if not present.
-->
<!ELEMENT figured-bass (figure+, duration?, %editorial;)>
<!ATTLIST figured-bass
    %print-style;
    %printout;
    parentheses %yes-no; #IMPLIED
    %optional-unique-id;
>
<!ELEMENT figure
	(prefix?, figure-number?, suffix?, extend?, %editorial;)>
<!ELEMENT prefix (#PCDATA)>
<!ATTLIST prefix
    %print-style;
>
<!ELEMENT figure-number (#PCDATA)>
<!ATTLIST figure-number
    %print-style;
>
<!ELEMENT suffix (#PCDATA)>
<!ATTLIST suffix
    %print-style;
>
\end{lstlisting}


% -------------------------------------------------------------------------
\section{Figured bass description}\label{Figured bass description}
% -------------------------------------------------------------------------

Figured bass is represented in \msrRepr\ by classes defined in \msrBoth{msrFiguredBasses}. There is \class{msrFiguredBass}:
\begin{lstlisting}[language=CPlusPlus]
class EXP msrFiguredBass : public msrMeasureElement
{
	// ... ... ...

  private:

    // private fields
    // ------------------------------------------------------

    // upLinks
    S_msrNote             fFiguredBassUpLinkToNote;
    S_msrVoice            fFiguredBassUpLinkToVoice; // for use in figured bass voices JMI

    mfRational              fFiguredBassDisplayWholeNotes;

    msrFiguredBassParenthesesKind
                          fFiguredBassParenthesesKind;

    std::list<S_msrBassFigure> fFiguredBassFiguresList;

    msrTupletFactor       fFiguredBassTupletFactor;
};
\end{lstlisting}

The figured bass figures are defined in:
\begin{lstlisting}[language=CPlusPlus]
class EXP msrBassFigure : public msrElement
{
	// ... ... ...

  private:

    // private fields
    // ------------------------------------------------------

    // upLinks
    S_msrPart             fFigureUpLinkToPart;

    msrBassFigurePrefixKind
                          fFigurePrefixKind;
    int                   fFigureNumber;
    msrBassFigureSuffixKind
                          fFigureSuffixKind;
};
\end{lstlisting}

Figured bass elements need special treatment since we need to determine their position in a figured bass voice. This is different than \mxml, where they are simply \drawn\ at the current music position, so to say.

They are handled this way:
\begin{itemize}
\item figured bass elements are stored in \class{msrNote}:
\item they are also stored in \class{msrPart} and \class{msrChord} and \class{msrTuplet} (\denorm);
\end{itemize}

In \class{msrNote}, there is:
\begin{lstlisting}[language=CPlusPlus]
    // figured bass
    void                  appendFiguredBassToNote (
                            const S_msrFiguredBass& figuredBass);

    const std::list<S_msrFiguredBass>&
                          getNoteFiguredBassesList () const
                              { return fNoteFiguredBassesList; }

		// ... ... ...

    // figured bass
    // ------------------------------------------------------

    std::list<S_msrFiguredBass>
                          fNoteFiguredBassesList;
\end{lstlisting}


% -------------------------------------------------------------------------
\section{Figured bass staves and voices}
% -------------------------------------------------------------------------

Every \class{msrVoice} instance in \mf\ belongs to an \class{msrStaff} instance. Staves are created specifically to hold figured bass voices, using specific numbers defined in \msr{msrParts.h}:
\begin{lstlisting}[language=CPlusPlus]
  public:

    // constants
    // ------------------------------------------------------

		// ... ... ...

    #define msrPart::K_PART_FIGURED_BASS_STAFF_NUMBER  20
    #define msrPart::K_PART_FIGURED_BASS_VOICE_NUMBER  21
\end{lstlisting}

In \class{msrStaff}, there is:
\begin{lstlisting}[language=CPlusPlus]
     void                  registerFiguredBassVoiceByItsNumber (
                            int        inputLineNumber,
                            const S_msrVoice& voice);
\end{lstlisting}

\Class{msrPart} also contains:
\begin{lstlisting}[language=CPlusPlus]
    // figured bass

    S_msrVoice            createPartFiguredBassVoice (
                            int    inputLineNumber,
                            std::string currentMeasureNumber);

    void                  appendFiguredBassToPart (
                            const S_msrVoice&             figuredBassSupplierVoice,
                           S_msrFiguredBass figuredBass);

    void                  appendFiguredBassToPartClone (
                            const S_msrVoice&              figuredBassSupplierVoice,
                            const S_msrFiguredBass& figuredBass);
\end{lstlisting}

\begin{lstlisting}[language=CPlusPlus]
    // figured bass

    S_msrStaff            fPartFiguredBassStaff;
    S_msrVoice            fPartFiguredBassVoice;
\end{lstlisting}


% -------------------------------------------------------------------------
\section{Figured bass staves creation}
% -------------------------------------------------------------------------

This is done in \mxsrToMsrBoth{mxsr2msrSkeletonBuilder.cpp}:
\begin{lstlisting}[language=CPlusPlus]
S_msrVoice mxsr2msrSkeletonBuilder::createPartFiguredBassVoiceIfNotYetDone (
  int        inputLineNumber,
  const S_msrPart&  part)
{
  // is the figured bass voice already present in part?
  S_msrVoice
    partFiguredBassVoice =
      part->
        getPartFiguredBassVoice ();

  if (! partFiguredBassVoice) {
    // create the figured bass voice and append it to the part
    partFiguredBassVoice =
      part->
        createPartFiguredBassVoice (
          inputLineNumber,
          fCurrentMeasureNumber);
  }

  return partFiguredBassVoice;
}
\end{lstlisting}

\Method{msrPart}{createPartFiguredBassVoice} creates the part figured bass staff and the part figured bass voice, and then registers the latter in the former:
\begin{lstlisting}[language=CPlusPlus]
S_msrVoice msrPart::createPartFiguredBassVoice (
  int    inputLineNumber,
  std::string currentMeasureNumber)
{
	// ... ... ...

  // create the part figured bass staff
  int partFiguredBassStaffNumber =
    msrPart::K_PART_FIGURED_BASS_STAFF_NUMBER;

	// ... ... ...

  fPartFiguredBassStaff =
    addHFiguredBassStaffToPart (
      inputLineNumber);

	// ... ... ...

  // create the figured bass voice
  int partFiguredBassVoiceNumber =
    msrPart::K_PART_FIGURED_BASS_VOICE_NUMBER;

	// ... ... ...

  fPartFiguredBassVoice =
    msrVoice::create (
      inputLineNumber,
      msrVoiceKind::kVoiceKindFiguredBass,
      partFiguredBassVoiceNumber,
      msrVoiceCreateInitialLastSegmentKind::kCreateInitialLastSegmentYes,
      fPartFiguredBassStaff);

  // register the figured bass voice in the part figured bass staff
  fPartFiguredBassStaff->
    registerVoiceInStaff (
      inputLineNumber,
      fPartFiguredBassVoice);

	// ... ... ...

  return fPartFiguredBassVoice;
}
\end{lstlisting}


% -------------------------------------------------------------------------
\section{Translating figured bass from MXSR to MSR}
% -------------------------------------------------------------------------

This is done in \mxsrToMsr{}, and this is where the \class{msrFiguredBass} instances are created.
There several methods for Figured bass elements creation:%%%JMI
\begin{lstlisting}[language=Terminal]
jacquesmenu@macmini:~/musicformats-git-dev/src/representations/msr > grep create msrFiguredBasses.h
    static SMARTP<msrBassFigure> create (
    SMARTP<msrBassFigure> createFigureNewbornClone (
    SMARTP<msrBassFigure> createFigureDeepClone ( // JMI ???
    static SMARTP<msrFiguredBass> create (
    static SMARTP<msrFiguredBass> create (
    SMARTP<msrFiguredBass> createFiguredBassNewbornClone (
    SMARTP<msrFiguredBass> createFiguredBassDeepClone ();
\end{lstlisting}

The MSR score skeleton created in \mxsrToMsrBoth{mxsr2msrSkeletonBuilder} contains the part groups, parts, staves and voices, as well as the number of measures. The voices do not contain any music elements yet.

A figured bass element belongs to \musicXmlMarkup{part} in \mxml, but we sometimes need to have it attached to a note.\\
Field{mxsr2msrSkeletonBuilder}{fThereAreFiguredBassToBeAttachedToCurrentNote} it used when visiting an \smartPointerType{S_FiguredBass} element to account for that:%%%JMI
\begin{lstlisting}[language=CPlusPlus]
void mxsr2msrSkeletonBuilder::visitStart ( S_figured_bass& elt )
{
#ifdef MF_TRACE_IS_ENABLED
  if (gGlobalMxsrOahGroup->getTraceMxsrVisitors ()) {
		std::stringstream ss;

		ss <<
      "--> Start visiting S_figured_bass" <<
      ", figuredBassVoicesCounter = " << fFiguredBassVoicesCounter <<
      ", line " << elt->getInputStartLineNumber ();

    gWaeHandler->waeTraceWithoutInputLocation (
      __FILE__, __LINE__,
      ss.str ());
  }
#endif // MF_TRACE_IS_ENABLED

  /* JMI
    several figured bass elements can be attached to a given note,
    leading to as many figured bass voices in the current part JMI TRUE???
  */

  // take figured bass voice into account
  ++fFiguredBassVoicesCounter;

  fThereAreFiguredBassToBeAttachedToCurrentNote = true;
}
\end{lstlisting}

Upon the second visit of \class{msrNote}, the part figured bass voice is created if figured bass elements are not to be ignored due to option \optionBoth{ignore-musicxml-figured-bass}{ofigbass} and it has not been created yet:
\begin{lstlisting}[language=CPlusPlus]
void mxsr2msrSkeletonBuilder::visitEnd ( S_note& elt )
{
	// ... ... ...

  // are there figured bass attached to the current note?
  if (fThereAreFiguredBassToBeAttachedToCurrentNote) {
    if (gGlobalMxsr2msrOahGroup->getIgnoreFiguredBasses ()) {
#ifdef MF_TRACE_IS_ENABLED
      if (gTraceOahGroup->getTraceFiguredBasses ()) {
        gLog <<
          "Ignoring the figured bass elements" <<
          ", line " <<
          inputLineNumber <<
          std::endl;

        gWaeHandler->waeTraceWithoutInputLocation (
          __FILE__, __LINE__,
          ss.str ());
      }
#endif // MF_TRACE_IS_ENABLED
    }
    else {
      // create the part figured bass voice if not yet done
      S_msrVoice
        partFiguredBassVoice =
          createPartFiguredBassVoiceIfNotYetDone (
            inputLineNumber,
            fCurrentPart);
    }

    fThereAreFiguredBassToBeAttachedToCurrentNote = false;

	// ... ... ...
}
\end{lstlisting}

Creating the part figured bass voice is delegated to the part:
\begin{lstlisting}[language=CPlusPlus]
S_msrVoice mxsr2msrSkeletonBuilder::createPartFiguredBassVoiceIfNotYetDone (
  int        inputLineNumber,
  const S_msrPart&  part)
{
  // is the figured bass voice already present in part?
  S_msrVoice
    partFiguredBassVoice =
      part->
        getPartFiguredBassVoice ();

  if (! partFiguredBassVoice) {
    // create the figured bass voice and append it to the part
    partFiguredBassVoice =
      part->
        createPartFiguredBassVoice (
          inputLineNumber,
          fCurrentMeasureNumber);
  }

  return partFiguredBassVoice;
}
\end{lstlisting}


% -------------------------------------------------------------------------
\subsection{First {\tt S_figured_bass} visit}\smartPointerTypeIndex{S_figured_bass}
% -------------------------------------------------------------------------

The first visit of \smartPointerType{S_figured_bass} initializes the fields storing values to be gathered visiting subelements:
\begin{lstlisting}[language=CPlusPlus]
void mxsr2msrTranslator::visitStart ( S_figured_bass& elt )
{
  int inputLineNumber =
    elt->getInputStartLineNumber ();

#ifdef MF_TRACE_IS_ENABLED
  if (gGlobalMxsrOahGroup->getTraceMxsrVisitors ()) {
		std::stringstream ss;

		ss <<
      "--> Start visiting S_figured_bass" <<
      ", line " << inputLineNumber;

    gWaeHandler->waeTraceWithoutInputLocation (
      __FILE__, __LINE__,
      ss.str ());
  }
#endif // MF_TRACE_IS_ENABLED

  ++fFiguredBassVoicesCounter;

  std::string parentheses = elt->getAttributeValue ("parentheses");

  fCurrentFiguredBassParenthesesKind =
    msrFiguredBassParenthesesKind::kFiguredBassParenthesesNo; // default value

  if (parentheses.size ()) {
    if (parentheses == "yes")
      fCurrentFiguredBassParenthesesKind =
        msrFiguredBassParenthesesKind::kFiguredBassParenthesesYes;

    else if (parentheses == "no")
     fCurrentFiguredBassParenthesesKind =
        msrFiguredBassParenthesesKind::kFiguredBassParenthesesNo;

    else {
      std::stringstream ss;

      ss <<
        "parentheses value " << parentheses <<
        " should be 'yes' or 'no'";

      musicxmlError (
        gServiceRunData->getInputSourceName (),
        inputLineNumber,
        __FILE__, __LINE__,
        ss.str ());
    }
  }

  fCurrentFiguredBassInputLineNumber   = -1;

  fCurrentFigureNumber = -1;

  fCurrentFigurePrefixKind = msrBassFigurePrefixKind::kBassFigurePrefix_UNKNOWN_;
  fCurrentFigureSuffixKind = msrBassFigureSuffixKind::kBassFigureSuffix_UNKNOWN_;

  fCurrentFiguredBassSoundingWholeNotes = mfRational (0, 1);
  fCurrentFiguredBassDisplayWholeNotes  = mfRational (0, 1);

  fOnGoingFiguredBass = true;
}
\end{lstlisting}


% -------------------------------------------------------------------------
\subsection{Second {\tt S_figured_bass} visit}\smartPointerTypeIndex{S_figured_bass}
% -------------------------------------------------------------------------

Upon the second visit of \smartPointerType{S_figured_bass}, the \class{msrFiguredBass} instance is created, populated and appended to \fieldName{mxsr2msrTranslator}{fPendingFiguredBassesList}:
\begin{lstlisting}[language=CPlusPlus]
void mxsr2msrTranslator::visitEnd ( S_figured_bass& elt )
{
  int inputLineNumber =
    elt->getInputStartLineNumber ();

#ifdef MF_TRACE_IS_ENABLED
  if (gGlobalMxsrOahGroup->getTraceMxsrVisitors ()) {
		std::stringstream ss;

		ss <<
      "--> End visiting S_figured_bass" <<
      ", line " << inputLineNumber;

    gWaeHandler->waeTraceWithoutInputLocation (
      __FILE__, __LINE__,
      ss.str ());
  }
#endif // MF_TRACE_IS_ENABLED

  // create the figured bass element
#ifdef MF_TRACE_IS_ENABLED
  if (gTraceOahGroup->getTraceFiguredBasses ()) {
		std::stringstream ss;

    ss <<
      "Creating a figured bass" <<
      ", line " << inputLineNumber << ":";

    gWaeHandler->waeTraceWithoutInputLocation (
      __FILE__, __LINE__,
      ss.str ());
  }
#endif // MF_TRACE_IS_ENABLED

  // create the figured bass element
  // if the sounding whole notes is 0/1 (no <duration /> was found), JMI ???
  // it will be set to the next note's sounding whole notes later
  S_msrFiguredBass
    figuredBass =
      msrFiguredBass::create (
        inputLineNumber,
  // JMI      fCurrentPart,
        fCurrentFiguredBassSoundingWholeNotes,
        fCurrentFiguredBassDisplayWholeNotes,
        fCurrentFiguredBassParenthesesKind,
        msrTupletFactor (1, 1));    // will be set upon next note handling

  // attach pending figures to the figured bass element
  if (! fPendingFiguredBassFiguresList.size ()) {
    musicxmlWarning (
      gServiceRunData->getInputSourceName (),
      inputLineNumber,
      "figured-bass has no figures contents, ignoring it");
  }
  else {
    // append the pending figures to the figured bass element
    for (S_msrBassFigure bassFigure : fPendingFiguredBassFiguresList) {
      figuredBass->
        appendFigureToFiguredBass (bassFigure);
    } // for

    // forget about those pending figures
    fPendingFiguredBassFiguresList.clear ();

    // append the figured bass element to the pending figured bass elements list
    fPendingFiguredBassesList.push_back (figuredBass);
  }

  fOnGoingFiguredBass = false;
}
\end{lstlisting}


% -------------------------------------------------------------------------
\subsection{Attaching {\tt msrFiguredBass} instances to notes}\className{msrFiguredBass}
% -------------------------------------------------------------------------

The contents of \fieldName{mxsr2msrTranslator}{fPendingFiguredBassesList} is attached to the \class{msrNote} instance in method\\
\method{mxsr2msrTranslator}{populateNote}:
\begin{lstlisting}[language=CPlusPlus]
void mxsr2msrTranslator::populateNote (
  int       inputLineNumber,
  const S_msrNote& newNote)
{
	// ... ... ...

  // handle the pending figured bass elements if any
  if (fPendingFiguredBassesList.size ()) {
    // get voice to insert figured bass elements into
    S_msrVoice
      voiceToInsertFiguredBassesInto =
        fCurrentPart->
          getPartFiguredBassVoice ();

		// ... ... ...

    handlePendingFiguredBasses (
      newNote,
      voiceToInsertFiguredBassesInto);

    // reset figured bass counter
    fFiguredBassVoicesCounter = 0;
  }
}
\end{lstlisting}


% -------------------------------------------------------------------------
\subsection{Populating {\tt msrFiguredBass} instances}
% -------------------------------------------------------------------------

In \msr{mxsr2msrTranslator.cpp}, the \class{msrFiguredBass} instances are populated further and attached to the note by \method{mxsr2msrTranslator}{handlePendingFiguredBasses}:
\begin{lstlisting}[language=CPlusPlus]
void mxsr2msrTranslator::handlePendingFiguredBasses (
  const S_msrNote&  newNote,
  const S_msrVoice& voiceToInsertInto)
{
	// ... ... ...

  mfRational
    newNoteSoundingWholeNotes =
      fCurrentNote->
        getSoundingWholeNotes (),
    newNoteDisplayWholeNotes =
      fCurrentNote->
        getNoteDisplayWholeNotes ();

  while (fPendingFiguredBassesList.size ()) { // recompute at each iteration
    S_msrFiguredBass
      figuredBass =
        fPendingFiguredBassesList.front ();

    /*
      Figured bass elements take their position from the first
      regular note (not a grace note or chord note) that follows
      in score order. The optional duration element is used to
      indicate changes of figures under a note.
    */

    // set the figured bass element's sounding whole notes
    figuredBass->
      setSoundingWholeNotes (
        newNoteSoundingWholeNotes,
        "handlePendingFiguredBasses()");

    // set the figured bass element's display whole notes JMI useless???
    figuredBass->
      setFiguredBassDisplayWholeNotes (
        newNoteDisplayWholeNotes);

    // set the figured bass element's tuplet factor
    figuredBass->
      setFiguredBassTupletFactor (
        msrTupletFactor (
          fCurrentNoteActualNotes,
          fCurrentNoteNormalNotes));

    // append the figured bass to newNote
    fCurrentNote->
      appendFiguredBassToNote (
        figuredBass);

/* JMI
    // get the figured bass voice for the current voice
    S_msrVoice
      voiceFiguredBassVoice =
        voiceToInsertInto->
          getRegularVoiceForwardLinkToFiguredBassVoice ();

#ifdef MF_SANITY_CHECKS_ARE_ENABLED
  // sanity check
    mfAssert (
      __FILE__, __LINE__,
      voiceFiguredBassVoice != nullptr,
      "voiceFiguredBassVoice is null");
#endif // MF_SANITY_CHECKS_ARE_ENABLED

    // set the figuredBass's voice upLink
    // only now that we know which figured bass voice will contain it
    figuredBass->
      setFiguredBassUpLinkToVoice (
        voiceFiguredBassVoice);

    // append the figured bass to the figured bass voice for the current voice
    voiceFiguredBassVoice->
      appendFiguredBassToVoice (
        figuredBass);
*/

    // don't append the figured bass to the part figured bass voice
    // before the note itself has been appended to the voice

    // remove the figured bass from the list
    fPendingFiguredBassesList.pop_front ();
  } // while
}
\end{lstlisting}

%%% JMI should be done for figured bass??? %%%JMI
%When a figured bass element is attached to a note that is a chord member, we have to attach it to the chord too, to facilitate setting its \pim\ when setting the chord's one.
%
\begin{lstlisting}[language=CPlusPlus]
%void mxsr2msrTranslator::copyNoteHarmoniesToChord (
%  S_msrNote note, S_msrChord chord)
%{
%  // copy note's harmony if any from the first note to chord
%
%  const std::list<S_msrHarmony>&
%    noteHarmoniesList =
%      note->getNoteHarmoniesList ();
%
%  if (noteHarmoniesList.size ()) {
%    std::list<S_msrHarmony>::const_iterator i;
%    for (i = noteHarmoniesList.begin (); i != noteHarmoniesList.end (); ++i) {
%      S_msrHarmony harmony = (*i);
%
%#ifdef MF_TRACE_IS_ENABLED
%      if (gTraceOahGroup->getTraceHarmonies ()) {
%        std::stringstream ss;
%
%        ss <<
%          "Copying harmony '" <<
%          harmony->asString () <<
%          "' from note " << note->asString () <<
%          " to chord '" << chord->asString () <<
%          '\'' <<
%          std::endl;
%
%      gWaeHandler->waeTraceWithoutInputLocation (
%        __FILE__, __LINE__,
%        ss.str ());
%      }
%#endif // MF_TRACE_IS_ENABLED
%
%      chord->
%        appendHarmonyToChord (harmony);
%
%    } // for
%  }
%}
%\end{lstlisting}


% -------------------------------------------------------------------------
\subsection{Inserting {\tt S_msrFiguredBass} instances in the part figured bass voice}\smartPointerTypeIndex{S_msrFiguredBass}
% -------------------------------------------------------------------------

\Method{msrVoice}{appendNoteToVoice} in \msr{msrNotes.cpp} inserts the figured bass elements in the part figured bass voice:
\begin{lstlisting}[language=CPlusPlus]
void msrVoice::appendNoteToVoice (const S_msrNote& note)
{
	// ... ... ...

  // are there figured bass elements attached to this note?
  const std::list<S_msrFiguredBass>&
    noteFiguredBassesList =
      note->
        getNoteFiguredBassesList ();

  if (noteFiguredBassesList.size ()) {
    // get the current part's figured bass voice
    S_msrVoice
      partFiguredBassVoice =
        part->
          getPartFiguredBassVoice ();

    for (S_msrFiguredBass figuredBass : noteFiguredBassesList) {
      // append the figured bass element to the part figured bass voice
      partFiguredBassVoice->
        appendFiguredBassToVoice (
          figuredBass);
    } // for
  }
};
\end{lstlisting}


% -------------------------------------------------------------------------
\section{Translating figured bass from MSR to MSR}
% -------------------------------------------------------------------------

In \msrToMsr{msr2msrTranslator.cpp}, a newborn clone of the figured bass element is created upon the first visit, stored in \fieldName{msr2msrTranslator}{fCurrentFiguredBassClone}, and appended to the current non grace note clone, the current chord clone or to the current voice clone, if the latter is a figured bass voice: %%%JMI
\begin{lstlisting}[language=CPlusPlus]
void msr2msrTranslator::visitStart (S_msrFiguredBass& elt)
{
#ifdef MF_TRACE_IS_ENABLED
  if (gMsrOahGroup->getTraceMsrVisitors ()) {
		std::stringstream ss;

    ss <<
      "--> Start visiting msrFiguredBass '" <<
      elt->asString () <<
      '\'' <<
      ", fOnGoingFiguredBassVoice = " << fOnGoingFiguredBassVoice <<
      ", line " << elt->getInputStartLineNumber ();

    gWaeHandler->waeTraceWithoutInputLocation (
      __FILE__, __LINE__,
      ss.str ());
  }
#endif // MF_TRACE_IS_ENABLED

  // create a figured bass element newborn clone
  fCurrentFiguredBassClone =
    elt->
      createFiguredBassNewbornClone (
        fCurrentVoiceClone);

  if (fOnGoingNonGraceNote) {
    // append the figured bass to the current non-grace note clone
    fCurrentNonGraceNoteClone->
      appendFiguredBassToNote (
      	fCurrentFiguredBassClone);

    // don't append the figured bass to the part figured bass,  JMI ???
    // this will be done below
  }

  /* JMI
  else if (fOnGoingChord) {
    // register the figured bass in the current chord clone
    fCurrentChordClone->
      setChordFiguredBass (fCurrentFiguredBassClone); // JMI
  }
  */

  else if (fOnGoingFiguredBassVoice) { // JMI
    /*
    // register the figured bass in the part clone figured bass
    fCurrentPartClone->
      appendFiguredBassToPartClone (
        fCurrentVoiceClone,
        fCurrentFiguredBassClone);
        */
    // append the figured bass to the current voice clone
    fCurrentVoiceClone->
      appendFiguredBassToVoiceClone (
        fCurrentFiguredBassClone);
  }

  else {
    std::stringstream ss;

    ss <<
      "figured bass is out of context, cannot be handled: " <<
      elt->asShortString ();

    msrInternalError (
      gServiceRunData->getInputSourceName (),
      elt->getInputStartLineNumber (),
      __FILE__, __LINE__,
      ss.str ());
  }
}
\end{lstlisting}

There are only fields updates upon the second visit:
\begin{lstlisting}[language=CPlusPlus]
void msr2msrTranslator::visitEnd (S_msrFiguredBass& elt)
{
#ifdef MF_TRACE_IS_ENABLED
  if (gMsrOahGroup->getTraceMsrVisitors ()) {
		std::stringstream ss;

    ss <<
      "--> End visiting msrFiguredBass '" <<
      elt->asString () <<
      '\'' <<
      ", line " << elt->getInputStartLineNumber ();

    gWaeHandler->waeTraceWithoutInputLocation (
      __FILE__, __LINE__,
      ss.str ());
  }
#endif // MF_TRACE_IS_ENABLED

  fCurrentFiguredBassClone = nullptr;
}
\end{lstlisting}


% -------------------------------------------------------------------------
\section{Translating figured bass from MSR to LPSR}
% -------------------------------------------------------------------------

The same occurs in \msrToLpsr{msr2lpsrTranslator.cpp}: a newborn clone of the figured bass element is created and appended to the current non grace note clone, the current chord clone or to the current voice clone, if the latter is a figured bass voice: %%%JMI
\begin{lstlisting}[language=CPlusPlus]
void msr2lpsrTranslator::visitStart (S_msrFiguredBass& elt)
{
#ifdef MF_TRACE_IS_ENABLED
  if (gMsrOahGroup->getTraceMsrVisitors ()) {
		std::stringstream ss;

    ss <<
      "--> Start visiting msrFiguredBass '" <<
      elt->asString () <<
      '\'' <<
      ", fOnGoingFiguredBassVoice = " << fOnGoingFiguredBassVoice <<
      ", line " << elt->getInputStartLineNumber ();

    gWaeHandler->waeTraceWithoutInputLocation (
      __FILE__, __LINE__,
      ss.str ());
  }
#endif // MF_TRACE_IS_ENABLED

  // create a figured bass newborn clone
  fCurrentFiguredBassClone =
    elt->
      createFiguredBassNewbornClone (
        fCurrentVoiceClone);

  if (fOnGoingNonGraceNote) {
    // append the figured bass to the current non-grace note clone
    fCurrentNonGraceNoteClone->
      appendFiguredBassToNote (
      	fCurrentFiguredBassClone);

    // don't append the figured bass to the part figured bass,  JMI ???
    // this will be done below
  }

  /* JMI
  else if (fOnGoingChord) {
    // register the figured bass in the current chord clone
    fCurrentChordClone->
      setChordFiguredBass (fCurrentFiguredBassClone); // JMI
  }
  */

  else if (fOnGoingFiguredBassVoice) { // JMI
    /*
    // register the figured bass in the part clone figured bass
    fCurrentPartClone->
      appendFiguredBassToPartClone (
        fCurrentVoiceClone,
        fCurrentFiguredBassClone);
        */
    // append the figured bass to the current voice clone
    fCurrentVoiceClone->
      appendFiguredBassToVoiceClone (
        fCurrentFiguredBassClone);
  }

  else {
    std::stringstream ss;

    ss <<
      "figured bass is out of context, cannot be handled: " <<
      elt->asShortString ();

    msrInternalError (
      gServiceRunData->getInputSourceName (),
      elt->getInputStartLineNumber (),
      __FILE__, __LINE__,
      ss.str ());
  }
}
\end{lstlisting}

Here too, there are only fields updates upon the second visit of \smartPointerType{S_msrFiguredBass} instances:
\begin{lstlisting}[language=CPlusPlus]
void msr2lpsrTranslator::visitEnd (S_msrFiguredBass& elt)
{
#ifdef MF_TRACE_IS_ENABLED
  if (gMsrOahGroup->getTraceMsrVisitors ()) {
		std::stringstream ss;

    ss <<
      "--> End visiting msrFiguredBass '" <<
      elt->asString () <<
      '\'' <<
      ", line " << elt->getInputStartLineNumber ();

    gWaeHandler->waeTraceWithoutInputLocation (
      __FILE__, __LINE__,
      ss.str ());
  }
#endif // MF_TRACE_IS_ENABLED

  fCurrentFiguredBassClone = nullptr;
}
\end{lstlisting}


% -------------------------------------------------------------------------
\section{Translating figured bass from LPSR to LilyPond}
% -------------------------------------------------------------------------

This is done in \lpsrToLilypond{}.

There is only one visit of \class{msrFiguredBass} instances in\\
\lpsrToLilypond{lpsr2lilypondTranslator.cpp}.

The \lily\ code is generated only if the figured bass element belongs to a figured bass voice: this is where \denorm\ ends in the workflow:%%%JMI
\begin{lstlisting}[language=CPlusPlus]
void msr2lpsrTranslator::visitStart (S_msrFiguredBass& elt)
{
#ifdef MF_TRACE_IS_ENABLED
  if (gMsrOahGroup->getTraceMsrVisitors ()) {
		std::stringstream ss;

    ss <<
      "--> Start visiting msrFiguredBass '" <<
      elt->asString () <<
      '\'' <<
      ", fOnGoingFiguredBassVoice = " << fOnGoingFiguredBassVoice <<
      ", line " << elt->getInputStartLineNumber ();

    gWaeHandler->waeTraceWithoutInputLocation (
      __FILE__, __LINE__,
      ss.str ());
  }
#endif // MF_TRACE_IS_ENABLED

  // create a figured bass newborn clone
  fCurrentFiguredBassClone =
    elt->
      createFiguredBassNewbornClone (
        fCurrentVoiceClone);

  if (fOnGoingNonGraceNote) {
    // append the figured bass to the current non-grace note clone
    fCurrentNonGraceNoteClone->
      appendFiguredBassToNote (
      	fCurrentFiguredBassClone);

    // don't append the figured bass to the part figured bass,  JMI ???
    // this will be done below
  }

  /* JMI
  else if (fOnGoingChord) {
    // register the figured bass in the current chord clone
    fCurrentChordClone->
      setChordFiguredBass (fCurrentFiguredBassClone); // JMI
  }
  */

  else if (fOnGoingFiguredBassVoice) { // JMI
    /*
    // register the figured bass in the part clone figured bass
    fCurrentPartClone->
      appendFiguredBassToPartClone (
        fCurrentVoiceClone,
        fCurrentFiguredBassClone);
        */
    // append the figured bass to the current voice clone
    fCurrentVoiceClone->
      appendFiguredBassToVoiceClone (
        fCurrentFiguredBassClone);
  }

  else {
    std::stringstream ss;

    ss <<
      "figured bass is out of context, cannot be handled: " <<
      elt->asShortString ();

    msrInternalError (
      gServiceRunData->getInputSourceName (),
      elt->getInputStartLineNumber (),
      __FILE__, __LINE__,
      ss.str ());
  }
}
\end{lstlisting}

