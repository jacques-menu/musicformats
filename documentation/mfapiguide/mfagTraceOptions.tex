% -------------------------------------------------------------------------
%  MusicFormats Library
%  Copyright (C) Jacques Menu 2016-2024

%  This Source Code Form is subject to the terms of the Mozilla Public
%  License, v. 2.0. If a copy of the MPL was not distributed with this
%  file, you can obtain one at http://mozilla.org/MPL/2.0/.

%  https://github.com/jacques-menu/musicformats
% -------------------------------------------------------------------------

% !TEX root = mfuserguide.tex

% -------------------------------------------------------------------------
\chapter{Trace options}
% -------------------------------------------------------------------------

\xmlToLy\ is equipped with a range of trace options, that are crucially needed by this author when testing and fine-tuning the code base.

The bulk of these options is placed in a group that is hidden by default:
\begin{lstlisting}[language=MusicXML]
  Trace (-ht, -help-trace) (hidden by default)
  --------------------------
\end{lstlisting}

The interested reader can see them with the \option{help-trace} group option:
\begin{lstlisting}[language=MusicXML]
menu@macbookprojm > xml2ly -help=trace

--- Help for group "Trace" ---

Trace (-ht, -help-trace) (hidden by default)
  There are trace options transversal to the successive passes,
  showing what's going on in the various translation activities.
  They're provided as a help for the maintainance of MusicFormats,
  as well as for the curious.
  The options in this group can be quite verbose, use them with small input data!
  All of them imply '-trace-passes, -tpasses'.
--------------------------
  Options handling trace          (-htoh, -help-trace-options-handling):
    -toah, -trace-oah
          Write a trace of options and help handling to standard error.
          This option should best appear first.
    -toahd, -trace-oah-details
          Write a trace of options and help handling with more details to standard error.
          This option should best appear first.
  Score to voices                 (-htstv, -help-trace-score-to-voices):
    -t<SHORT_NAME>, -trace<LONG_NAME>
          Trace SHORT_NAME/LONG_NAME in score to voices.
          The 9 known SHORT_NAMEs are:
            score, pgroups, pgroupsd, parts, staves, st, schanges, voices and voicesd.
          The 9 known LONG_NAMEs are:
            -score, -part-groups, -part-groups-details, -parts, -staves.
... ... ... ... ... ...
\end{lstlisting}

As can be seen, there are event options to trace the handling of options and help by \xmlToLy.

The source code contains many instances of trace code, such as:
\begin{lstlisting}[language=CPlusPlus]
#ifdef TRACE_OAH
  if (gTraceOah->fTraceVoices) {
    gLogOstream <<
      "Creating voice \"" << asString () << "\"";

    gWaeHandler->waeTraceWithoutInputLocation (
      __FILE__, __LINE__,
      ss.str ());
  }
#endif // TRACE_OAH
\end{lstlisting}

Building \xmlToLy\ with trace disabled only gains less than 5\% in speed, this is why trace is available by default.


