
% file to be included by LaTeX documents

% -------------------------------------------------------------------------
% lastpage
% -------------------------------------------------------------------------

\usepackage{lastpage} % the last page has number \pageref{LastPage}


% -------------------------------------------------------------------------
% layout
% -------------------------------------------------------------------------

\usepackage{fancyhdr}

\usepackage{etoolbox}

\pagestyle{fancy}

% redefine the plain style as being fancy
\fancypagestyle{plain}{}

% defeat headwidth not being equal to \textwidth %%%JMI should not be necessary ???
% the value depends on 11pt in the \documentclass !!!
\setlength{\headwidth}{1.425\textwidth}

\renewcommand{\headrulewidth}{0pt}
\renewcommand{\footrulewidth}{0pt}

% fancyhdr

\addtolength{\headheight}{\baselineskip}

%\fancyhfoffset{-0.5in}
%\setlength{\topmargin}{2in}
%\setlength{\headsep}{2in}

% headers

% the regular fancyhead will be set after the tables and lists,
% for which this setting is used:
\fancyhead[L]{}
\fancyhead[C]{\nouppercase\leftmark}
\fancyhead[R]{}
% then it will use the above again for the index

% footers

\fancyfoot[L]{}
\fancyfoot[C]{\thepage/\pageref{LastPage}}
\fancyfoot[R]{}


% resizing

\usepackage{adjustbox}


% -------------------------------------------------------------------------
% divisions
% -------------------------------------------------------------------------

%\newcommand{\Part}[1]{%Argument: name
%\part{#1}\label{#1}%
%}
%
%\newcommand{\Chapter}[1]{%Argument: name
%\chapter{#1}\label{#1}%
%}
%
%\newcommand{\Section}[1]{%Argument: name
%\section{#1}\label{#1}%
%}
%
%\newcommand{\SubSection}[1]{%Argument: name
%\subsection{#1}\label{#1}%
%}


% -------------------------------------------------------------------------
% tables and lists settings
% -------------------------------------------------------------------------

\makeatletter
%\renewcommand{\@pnumwidth}{10em}% default is 1.55em
\renewcommand{\@pnumwidth}{2.6em}
\makeatother

\usepackage{titletoc} % https://mirror.foobar.to/CTAN/macros/latex/contrib/titlesec/titlesec.pdf
%\renewcommand{\thechapter}{\Roman{chapter}}

\newlength{\ecartnumero}
\setlength{\ecartnumero}{0mm}

%\titlecontents{part}%
%  [\dimexpr 6.5mm+\ecartnumero]
%  {\vspace{4.2mm}\bfseries}
%  {\contentslabel{\dimexpr 10mm+\ecartnumero}}
%  {\hspace{\dimexpr -5.5mm-\ecartnumero}}
%  {\hfill\contentspage}
%  {\par}
%
%\dottedcontents{part}%
%  [\dimexpr 5mm+\ecartnumero]
%  {}
%  {\dimexpr 5mm+\ecartnumero}
%  {3.2mm}
%  {\par}

\titlecontents{chapter}%
  [\dimexpr 6.5mm+\ecartnumero]
  {\vspace{4.2mm}\bfseries}
  {\contentslabel{\dimexpr 10mm+\ecartnumero}}
  {\hspace{\dimexpr -5.5mm-\ecartnumero}}
  {\hfill\contentspage}

\titlecontents{section}%
  [\dimexpr 6.5mm+\ecartnumero]                % left
  {\vspace{4.2mm}\bfseries}                    % above code
  {\contentslabel{\dimexpr 10mm+\ecartnumero}} % numbered entry format
  {\hspace{\dimexpr -5.5mm-\ecartnumero}}      % numberless entry format
  {\hfill\contentspage}                        % filler page format
                                               % below code ??? JMI

\dottedcontents{chapter}%
  [\dimexpr 20mm+\ecartnumero] % left
  {}                           % above code
  {\dimexpr 15mm+\ecartnumero} % label width
  {3.2mm}                      % leader width

\dottedcontents{section}%
  [\dimexpr 25mm+\ecartnumero]
  {}
  {\dimexpr 15mm+\ecartnumero}
  {3.2mm}

\dottedcontents{subsection}%
  [\dimexpr 30mm+\ecartnumero]
  {}
  {\dimexpr 15mm+\ecartnumero}
  {3.2mm}

\dottedcontents{subsubsection}%
  [\dimexpr 35mm+\ecartnumero]
  {}
  {\dimexpr 15mm+\ecartnumero}
  {3.2mm}

\dottedcontents{figure}%
  [\dimexpr 15mm+\ecartnumero]
  {}
  {\dimexpr 15mm+\ecartnumero}
  {3.2mm}

\dottedcontents{lstlisting}%
  [\dimexpr 15mm+\ecartnumero]
  {}
  {\dimexpr 15mm+\ecartnumero}
  {3.2mm}


% -------------------------------------------------------------------------
% typesetting
% -------------------------------------------------------------------------

\usepackage{underscore}

% -------------------------------------------------------------------------
% layout
% -------------------------------------------------------------------------

\setlength{\parskip}{1.75ex plus \baselineskip minus 2pt}


% -------------------------------------------------------------------------
% general purpose
% -------------------------------------------------------------------------

\usepackage{underscore}

\usepackage{longtable}

\usepackage{bold-extra}

\usepackage{ifthen}

\usepackage{rotating}


% -------------------------------------------------------------------------
% FOO%%%JMI
% -------------------------------------------------------------------------

\usepackage{mfirstuc} % for \capitalisewords

% create an arrow
\newcommand{\arrowIn}{
\tikz \draw[-stealth] (-2pt,0) -- (2pt,0);
}

\global\def \tab {~~~}


\newcommand{\sep}{\vspace{1.5ex}}

\newcommand{\page}{\pagebreak}

\newcommand{\etc}{... ... ...}

\newcommand{\dx}[1][]{%
   \ifthenelse{ \equal{#1}{} }
      {\ensuremath{\;\mathrm{d}x}}
      {\ensuremath{\;\mathrm{d}#1}}
}
%$$\int x\dx$$
%$$\int t\dx[t]$$

\newcommand{\iBox}[2][]{%
   \ifthenelse { \equal{#2}{} }
      {
      	\hspace{#1}
      	\mbox{
      		\begin{minipage}[b]{0.925\textwidth}
      		#2
      		\end{minipage}
      	}
			}
      {
      	\hspace{#1}
      	\mbox{
      		\begin{minipage}[b]{0.925\textwidth}
      		#2
      		\end{minipage}
      	}
			}
}

\newcommand{\indentedBox}[2][0.05\textwidth]{
	\hspace{#1}
	\mbox{
		\begin{minipage}[b]{0.925\textwidth}
%		Mandatory arg: #2
%	  Optional arg : #1
		#2
		\end{minipage}
	}
}

\newcommand{\example}[2][YYY]{
	Mandatory arg: #2;
  Optional arg: #1.
}

%\example{BBB}
%\example[XXX]{AAA}

%This defines \example to be a command with two arguments,
%referred to as #1 and #2 in the {<definition>}--nothing new so far.
%But by adding a second optional argument to this \newcommand
%(the [YYY]) the first argument (#1) of the newly defined
%command \example is made optional with its default value being YYY.
%
%Thus the usage of \example is either:
%
%   \example{BBB}
%which prints:
%Mandatory arg: BBB; Optional arg: YYY.
%or:
%   \example[XXX]{AAA}
%which prints:
%Mandatory arg: AAA; Optional arg: XXX.


%\framebox[\textwidth][r]{
%	\parbox{0.9\textwidth}{
%		The useful options here are:
%  		\begin{itemize}
%  		\item \optionBoth{tharms}{trace-harmonies}
%  		\end{itemize}
%	}
%}\\

\usepackage{changepage} % for adjustwidth


% -------------------------------------------------------------------------
% tikz
% -------------------------------------------------------------------------

\usepackage{tikz}

\usetikzlibrary{math}

\usetikzlibrary{shapes.arrows}
\usetikzlibrary{shapes.multipart}

\usetikzlibrary{arrows}

\usetikzlibrary{calc,intersections}

\usetikzlibrary{decorations.pathreplacing,decorations.markings}


% -------------------------------------------------------------------------
% geometry
% -------------------------------------------------------------------------

\usepackage%
	[%
	paper=a4paper,%
	top=1.75cm,headheight=0.3cm,headsep=0.5cm,%
	bottom=2cm,footskip=1cm,%
	left=1.5cm,right=1.5cm%
	]%
	{geometry}


% -------------------------------------------------------------------------
% colors
% -------------------------------------------------------------------------

\usepackage{color}
\definecolor{verylightgray}{gray}{.9}
\definecolor{lightgray}{gray}{.8}
\definecolor{gray}{gray}{.7}
\definecolor{brown}{rgb}{0.74, 0.30, 0.12}
\definecolor{orange}{rgb}{1, 0.50, 0}
\definecolor{darkgreen}{rgb}{0, 0.9, 0}
\definecolor{lightblue}{rgb}{0.5, 0.5, 1}
\definecolor{bordeaux}{cmyk}{0, 0.735, 0.270, 0.257}


% -------------------------------------------------------------------------
% URLs
% -------------------------------------------------------------------------

\usepackage{url}


% -------------------------------------------------------------------------
% graphicx
% -------------------------------------------------------------------------

\usepackage{graphicx}
\graphicspath{{./}}


% -------------------------------------------------------------------------
% listings
% -------------------------------------------------------------------------

\usepackage{listings}
%\usepackage{textcomp} % to handle single quotes properly

\definecolor{codered}{rgb}{0.9,0,0}
\definecolor{codegreen}{rgb}{0,0.4,0}
\definecolor{codeblue}{rgb}{0,0,0.9}
\definecolor{codegray}{rgb}{0.5,0.5,0.5}
\definecolor{codepurple}{rgb}{0.88,0,0.62}
\definecolor{backcolour}{rgb}{0.95,0.95,0.92}
%\definecolor{backcolour}{rgb}{0.97,0.97,0.94}

\lstdefinestyle{mystyle}{
	frame=shadowbox, framesep=3pt, rulesep=2pt, rulesepcolor=\color{orange},
  %backgroundcolor=\color{backcolour},
  commentstyle=\color{codered},
  keywordstyle=\color{codeblue},
  basicstyle=\ttfamily\footnotesize,
  breakatwhitespace=false,
  breaklines=true,
  captionpos=t,      % top
  keepspaces=true,
	%
  %numbers=left,
  numberstyle=\tiny\color{codegray},
  stringstyle=\color{codepurple},
  numbers=left,
  numbersep=5pt,
  %
  showspaces=false,
  showstringspaces=false,
  showtabs=false,
  tabsize=2,
  %
  extendedchars=true,
  literate=
  {á}{{\'a}}1 {é}{{\'e}}1 {í}{{\'i}}1 {ó}{{\'o}}1 {ú}{{\'u}}1
  {Á}{{\'A}}1 {É}{{\'E}}1 {Í}{{\'I}}1 {Ó}{{\'O}}1 {Ú}{{\'U}}1
  {à}{{\`a}}1 {è}{{\`e}}1 {ì}{{\`i}}1 {ò}{{\`o}}1 {ù}{{\`u}}1
  {À}{{\`A}}1 {È}{{\'E}}1 {Ì}{{\`I}}1 {Ò}{{\`O}}1 {Ù}{{\`U}}1
  {ä}{{\"a}}1 {ë}{{\"e}}1 {ï}{{\"i}}1 {ö}{{\"o}}1 {ü}{{\"u}}1
  {Ä}{{\"A}}1 {Ë}{{\"E}}1 {Ï}{{\"I}}1 {Ö}{{\"O}}1 {Ü}{{\"U}}1
  {â}{{\^a}}1 {ê}{{\^e}}1 {î}{{\^i}}1 {ô}{{\^o}}1 {û}{{\^u}}1
  {Â}{{\^A}}1 {Ê}{{\^E}}1 {Î}{{\^I}}1 {Ô}{{\^O}}1 {Û}{{\^U}}1
  {œ}{{\oe}}1 {Œ}{{\OE}}1 {æ}{{\ae}}1 {Æ}{{\AE}}1 {ß}{{\ss}}1
  {ç}{{\c c}}1 {Ç}{{\c C}}1 {ø}{{\o}}1 {å}{{\r a}}1 {Å}{{\r A}}1
  {€}{{\EUR}}1 {£}{{\pounds}}1
  {©}{{\textcopyright}}1
  {®}{{\textregistered}}1
  {™}{{\texttrademark}}1
}

\lstset{style=mystyle}

\lstloadlanguages{[11]C++}

% LaTeX

% LaTeX definition; inspired from lstdrvrs.dtx
\lstdefinelanguage{Latex}%
  {%
  morekeywords={%
  	\document,%
  	\usepackage,%
  	\command,\newcommand,%
		\lstset,\lstdefinelanguage,%
	},%
  sensitive,%
  morestring=[b]",%
  morestring=[m]', % changed from `b' to `m'
}[keywords,comments,strings,directives]%

\lstdefinestyle{Latex}{
language=Latex,
basicstyle=\small\sffamily,
numbers=left,
numberstyle=\tiny,
frame=L,
columns=fullflexible,
showstringspaces=false,
}

\lstnewenvironment{Latex}{
\lstset{style=Latex}}{}


% Terminal

% Terminal definition; copied from lstdrvrs.dtx
\lstdefinelanguage{Terminal}%
{%
	morekeywords={%
		% sh
	  awk,break,case,cat,cd,continue,do,done,echo,elif,else,%
    env,esac,eval,exec,exit,export,expr,false,fi,for,function,getopts,%
    hash,history,if,in,kill,login,newgrp,nice,nohup,ps,pwd,read,%
    readonly,return,set,sed,shift,test,then,times,trap,true,type,%
    ulimit,umask,unset,until,wait,while,%
    % bash
		alias,bg,bind,builtin,caller,command,compgen,compopt,%
    complete,coproc,declare,disown,dirs,enable,fc,fg,help,history,%
    jobs,let,local,logout,mapfile,printf,pushd,popd,readarray,select,%
    set,suspend,shopt,source,times,type,typeset,ulimit,unalias,wait,%
    % our own
		cd,ls,ll,cat,more,%
		make,%
		RandomChords,RandomMusic,countnotes,%
		xml2guido,%
		xml2gmn,xml2ly,xml2brl,xml2xml,msdlcompiler%
		Mikrokosmos3Wandering,LilyPondIssue34,%
	},%
  % sensitive,%
  morecomment=[l]\#,%
  morestring=[d]",%
  morestring=[d]’%
}[keywords,comments,strings]%

\lstdefinestyle{Terminal}{
language=Terminal,
basicstyle=\small\sffamily,
numbers=left,
numberstyle=\tiny,
frame=L,
columns=fullflexible,
showstringspaces=false,
}

\lstnewenvironment{Terminal}{
\lstset{style=Terminal}}{}


% CPlusPlus

% modified C++ definition; copied from lstdrvrs.dtx
\lstdefinelanguage{CPlusPlus}[]{C++}%
{%
  morestring=[b]",%
  morestring=[m]',% changed from `b' to `m'
}[keywords,comments,strings,directives]%

\lstdefinestyle{CPlusPlus}{
language=CPlusPlus,
basicstyle=\small\sffamily,
numbers=left,
numberstyle=\tiny,
frame=L,
columns=fullflexible,
showstringspaces=false,
}

\lstnewenvironment{CPlusPlus}{
\lstset{style=CPlusPlus}}{}

 %\lstset{language=CPlusPlus, basicstyle=\tiny, emph={countnotes}, emphstyle=\color{red}}


% modified XML definition; copied from lstdrvrs.dtx
\lstdefinelanguage{MusicXML}%
{%
 	keywords={%
		% XML
		score,partwise,%
    identification,attributes,%
    work,identification,%
    part,measure%
    %
		,CDATA,DOCTYPE,ATTLIST,termdef,ELEMENT,EMPTY,ANY,ID,%
	  IDREF,IDREFS,ENTITY,ENTITIES,NMTOKEN,NMTOKENS,NOTATION,%
    INCLUDE,IGNORE,SYSTEM,PUBLIC,NDATA,PUBLIC,%
    PCDATA,REQUIRED,IMPLIED,FIXED,%%% preceded by #
    xml,xml:space,xml:lang,version,standalone,default,preserve,%
    %
    % our own
		xml2gmn,xml2ly,xml2brl,xml2xml,msdlcompiler%
   },%
 alsoother=$,%
 alsoletter=:,%
 tag=**[s]<>,%
 morestring=[d]",% ??? doubled
 morestring=[d]’,% ??? doubled
 MoreSelectCharTable=%
    \lst@CArgX--\relax\lst@DefDelimB{}{}%
        {\ifnum\lst@mode=\lst@tagmode\else
             \expandafter\@gobblethree
         \fi}%
        \lst@BeginComment\lst@commentmode{{\lst@commentstyle}}%
    \lst@CArgX--\relax\lst@DefDelimE{}{}{}%
        \lst@EndComment\lst@commentmode
    \lst@CArgX[CDATA[\relax\lst@CDef{}%
        {\ifnum\lst@mode=\lst@tagmode
             \expandafter\lst@BeginCDATA
         \else \expandafter\lst@CArgEmpty
         \fi}%
        \@empty
    \lst@CArgX]]\relax\lst@CDef{}%
        {\ifnum\lst@mode=\lst@GPmode
             \expandafter\lst@EndComment
         \else \expandafter\lst@CArgEmpty
         \fi}%
        \@empty
}[keywords,comments,strings,html]%

\lstdefinestyle{MusicXML}{
language=MusicXML,
basicstyle=\small\sffamily,
numbers=left,
numberstyle=\tiny,
frame=L,
columns=fullflexible,
showstringspaces=false,
}

\lstnewenvironment{MusicXML}{
\lstset{style=MusicXML}}{}


% LilyPond

% LilyPond definition; inspired from lstdrvrs.dtx
\lstdefinelanguage{Lilypond}%
  {%
  morekeywords={%
  	\version,%
  	\header,\paper,\score,\book,\layout,\midi,%
		\new,\context,\with,%
		\absolute,\relative,\fixed,%
		\language,\clef,\key,\time,%
	},%
  sensitive,%
  morestring=[b]",%
  morestring=[m]', % changed from `b' to `m'
}[keywords,comments,strings,directives]%

\lstdefinestyle{Lilypond}{
language=Lilypond,
basicstyle=\small\sffamily,
numbers=left,
numberstyle=\tiny,
frame=L,
columns=fullflexible,
showstringspaces=false,
}

\lstnewenvironment{Lilypond}{
\lstset{style=Lilypond}}{}


% Guido

% Guido definition; inspired from lstdrvrs.dtx
\lstdefinelanguage{Guido}%
  {%
  morekeywords={%
  	\staff,%
		\barFormat,\bar,%
  	\beamsOff,%
		\new,\context,\with,%
		\absolute,\relative,\fixed,%
		\language,\clef,\key,\time,%
	},%
  sensitive,%
  morestring=[b]",%
  morestring=[m]', % changed from `b' to `m'
}[keywords,comments,strings,directives]%

\lstdefinestyle{Guido}{
language=Guido,
basicstyle=\small\sffamily,
numbers=left,
numberstyle=\tiny,
frame=L,
columns=fullflexible,
showstringspaces=false,
}

\lstnewenvironment{Guido}{
\lstset{style=Guido}}{}


% MSDL

% MSDL definition; inspired from lstdrvrs.dtx
\lstdefinelanguage{MSDL}%
  {%
  morekeywords={%
  	title,titre,%
		composer,compositeur,%
		opus,%
		%
  	pitches,hauteurs,%
		octaves,%
		%
		anacrusis,anacrouse,%
		%
		book,livre,%
		score,partitions,%
		partGroup,groupeDeParties,%
		part,partie,%
  	music,musique,%
		fragment,%
		%
		clef,clé,%
		key,armure,%
		time,métrique,%
	},%
  sensitive,%
  morestring=[b]",%
  morestring=[m]', % changed from `b' to `m'
}[keywords,comments,strings,directives]%

\lstdefinestyle{MSDL}{
language=MSDL,
basicstyle=\small\sffamily,
numbers=left,
numberstyle=\tiny,
frame=L,
columns=fullflexible,
showstringspaces=false,
}

\lstnewenvironment{MSDL}{
\lstset{style=MSDL}}{}


% -------------------------------------------------------------------------
% fonts
% -------------------------------------------------------------------------

\newfont\pnt{pzdr at 24.88pt}
\newcommand{\hand}[1]{
  \makebox[0pt][r]{
  	\textcolor{bordeaux}{\raisebox{-.5ex}{\pnt\symbol{'345}}}\hspace{1em}
	}%
	\hfill%
	{%
	  \setlength{\fboxsep}{2ex}%
	  \colorbox{white}{
	  	\parbox{.8\textwidth}{\textcolor{bordeaux}{\textbf{#1}}}
	  }
		\hfill
	}
}


% -------------------------------------------------------------------------
% lengths
% -------------------------------------------------------------------------

\setlength{\parindent}{0mm}

\setlength{\parskip}{1.75ex plus \baselineskip minus 2pt}


% -------------------------------------------------------------------------
% indexing (BEFORE hyperref)
% -------------------------------------------------------------------------

\usepackage[]{imakeidx}
%\usepackage[splitindex]{imakeidx}
	% https://mirror.foobar.to/CTAN/macros/latex/contrib/imakeidx/imakeidx.pdf
	% https://www.tuteurs.ens.fr/logiciels/latex/makeindex.html
	% https://en.wikibooks.org/wiki/LaTeX/Indexing

% this necessary to have the subitems and subsubitems idented the same way
\usepackage[indentunit=0.75em, hangindent=1.5em, subindent=1.5em, subsubindent=1.5em]{idxlayout}
	% https://mirror.foobar.to/CTAN/macros/latex/contrib/idxlayout/idxlayout.pdf

\newcommand{\Main}[1]{%
#1\index[Main]{#1}%
}
\newcommand{\MainIt}[1]{%
{\it #1}\index[Main]{#1}%
}

\newcommand{\MainName}[1]{%
\index[Main]{#1}%
}

\newcommand{\code}[1]{%
{\tt #1}\index[Main]{{\tt #1}}%
}

%%%JMI\usepackage{cleveref}

%\usepackage{multido}%%%JMI
%
%\usepackage{multicol}
%
%\usepackage{chngpage}%for ``adjustwidth'' environment


% -------------------------------------------------------------------------
% hyperref (as LATE as possible)
% -------------------------------------------------------------------------

\usepackage[
  pdftex,pdfpagemode={FullScreen},
  pdfstartview={FitH},
  plainpages=false,
  pdfkeywords={latex,introduction UniBe},
  pdfauthor={Jacques Menu},
  colorlinks=true,linkcolor={blue},
  citecolor={red},
  urlcolor={red}
]{hyperref}

\newcommand{\mylabel}[1]{%Arguments: label
\hypertarget{#1}{\label{#1}}%
}

\newcommand{\mylink}[1]{%Arguments: label
\hyperlink{#1}{section~\ref{#1}, page~\pageref{#1}}%
}


% -------------------------------------------------------------------------
% LilyPond command
% -------------------------------------------------------------------------

\newcommand{\lilycmd}[1]{
{\tt \textbackslash #1 \textbraceleft...\textbraceright}
}


% -------------------------------------------------------------------------
% names
% -------------------------------------------------------------------------

\newcommand{\tikzpgf}{{\tt Ti{\kern-1pt\it k}Z/PGF}}

\newcommand{\libmusicxml}{{\tt libmusicxml2}\index[Main]{libmusicxml2}}
\newcommand{\fober}{Dominique Fober\index[Main]{Dominique Fober}}

\newcommand{\mf}{MusicFormats\index[Main]{MusicFormats}}

\newcommand{\dtd}{DTD\index[Main]{DTD}}

\newcommand{\pdf}{PDF\index[Main]{PDF}}

\newcommand{\oahRepr}{OAH\index[Main]{OAH}}

\newcommand{\mfcRepr}{MFC\index[Main]{MFC}}

\newcommand{\msrRepr}{MSR\index[Main]{MSR}}
\newcommand{\lpsrRepr}{LSPR\index[Main]{LSPR}}
\newcommand{\bsrRepr}{BSR\index[Main]{BSR}}

\newcommand{\mxsrRepr}{MXSR\index[Main]{MXSR}}

\newcommand{\mxml}{MusicXML\index[Main]{MusicXML}}
\newcommand{\mxmlName}{MusicXML}

\newcommand{\guido}{Guido\index[Main]{Guido}}

\newcommand{\msdlLang}{MSDL\index[Main]{MSDL}}
\newcommand{\msdLangComp}{MSDL converter}
\newcommand{\msdlconverter}{{\tt msdlconverter}}

\newcommand{\meiLang}{MEI\index[Main]{MEI}}

\newcommand{\bmmlLang}{BMML\index[Main]{BMML}}

\newcommand{\psu}{PhotoScore Ultimate\texttrademark~8.8.2\index[Main]{PhotoScore Ultimate}}

\newcommand{\fin}{Finale\texttrademark~2014\index[Main]{Finale}}
\newcommand{\sib}{Sibelius\texttrademark~7.1.3\index[Main]{Sibelius}}

\newcommand{\mscore}{{\tt MuseScore}\index[Main]{MuseScore}}
\newcommand{\muse}{MuseScore 3.3.4\index[Main]{MuseScore 3.3.4}}

\newcommand{\bmml}{BMML\index[Main]{BMML}}
\newcommand{\mei}{MEI\index[Main]{MEI}}
\newcommand{\midi}{MIDI\index[Main]{MIDI}}

\newcommand{\lily}{LilyPond\index[Main]{LilyPond}}

\newcommand{\jianpu}{Jianpu numeric notation\index[Main]{Jianpu numeric notation}}

\newcommand{\lilyJianpu}{{\tt lilypond-Jianpu}\index[Main]{lilypond-Jianpu}}
\newcommand{\lcg}{LilyPond code generated\index[Main]{LilyPond code generated}}


% -------------------------------------------------------------------------
% commands
% -------------------------------------------------------------------------

\newcommand{\mxmlToLy}{{\tt musicxml2ly}}

\newcommand{\msdlToMusicXML}{{\tt msdl -musicxml}}
\newcommand{\msdlToGuido}{{\tt msdl -guido}}
\newcommand{\msdlToLilyPond}{{\tt msdl -lilypond}}
\newcommand{\msdlToJianpuLilyPond}{{\tt msdl -lilypond -jianpu}}
\newcommand{\msdlToBraille}{{\tt msdl -braille}}

\newcommand{\xmlToXml}{{\tt xml2xml}}

\newcommand{\xmlToGuido}{{\tt xml2guido}}

\newcommand{\xmlToGmn}{{\tt xml2gmn}}
\newcommand{\xmlToLy}{{\tt xml2ly}}
\newcommand{\xmlToBrl}{{\tt xml2brl}}

\newcommand{\bmmlToXml}{{\tt bmml2xml}}

\newcommand{\bmmlToBmml}{{\tt bmml2bmml}}

\newcommand{\bmmlToGuido}{{\tt bmml2guido}}
\newcommand{\bmmlToLy}{{\tt bmml2ly}}
\newcommand{\bmmlToBrl}{{\tt bmml2brl}}

\newcommand{\bmmlToMei}{{\tt bmml2mei}}

\newcommand{\xmlToBmml}{{\tt xml2bmml}}

\newcommand{\meiToXml}{{\tt mei2xml}}

\newcommand{\meiToMei}{{\tt mei2mei}}

\newcommand{\meiToGuido}{{\tt mei2guido}}
\newcommand{\meiToLy}{{\tt mei2ly}}
\newcommand{\meiToBrl}{{\tt mei2brl}}

\newcommand{\xmlToMei}{{\tt xml2mei}}

\newcommand{\meiToBmml}{{\tt mei2bmml}}

\newcommand{\midiToLy}{{\tt midi2ly}}


% -------------------------------------------------------------------------
% references to labels
% -------------------------------------------------------------------------

\newcommand{\chapterRef}[1]{%
chapter~\ref{#1}, [#1], page~\pageref{#1}%
}

\newcommand{\sectionRef}[1]{%
section~\ref{#1}, [#1], page~\pageref{#1}%
}

\newcommand{\subSectionRef}[1]{%
subsection~\ref{#1}, [#1], page~\pageref{#1}%
}

\newcommand{\figureRef}[1]{%
figure~\ref{#1}, [#1], page~\pageref{#1}%
}

\newcommand{\listingRef}[1]{%
listing~\ref{#1}, [#1], page~\pageref{#1}%
}


% -------------------------------------------------------------------------
% short-cuts
% -------------------------------------------------------------------------

\newcommand{\OS}{operating system\index[Main]{operating system}}

\newcommand{\MacOS}{Mac OS\texttrademark\index[Main]{Mac OS\texttrademark}}
\newcommand{\Ubuntu}{Ubuntu\index[Main]{Ubuntu}}
\newcommand{\Linux}{Linux\index[Main]{Linux}}
\newcommand{\Windows}{Windows\texttrademark\index[Main]{Windows\texttrademark}}

\newcommand{\Gatekeeper}{Gatekeeper\index[Main]{Gatekeeper}}
\newcommand{\quarantine}{quarantine\index[Main]{quarantine}}

\newcommand{\CLI}{command line\index[Main]{command line}}
\newcommand{\GUI}{GUI\index[Main]{GUI}}
\newcommand{\API}{API\index[Main]{API}}
\newcommand{\Web}{Web\index[Main]{Web}}

\newcommand{\git}{git\index[Main]{git}}
\newcommand{\repo}{repository\index[Main]{repository}}

\newcommand{\argcargv}{
{\tt argc}/{\tt argv}\index[Main]{argc/argv}\index[ConstantsFunctionsAndVariables]{argc/argv}%
}

\newcommand{\quotes}{quotes\index[Main]{quotes}}
\newcommand{\Quotes}{Quotes\index[Main]{Quotes}}

\newcommand{\doubleQuotes}{double quotes\index[Main]{double quotes}}
\newcommand{\DoubleQuotes}{Double quotes\index[Main]{Double quotes}}

\newcommand{\denorm}{denormalization\index[Main]{denormalization}}

\newcommand{\format}{format\index[Main]{format}}
\newcommand{\Format}{Format\index[Main]{Format}}

\newcommand{\component}{component\index[Main]{component}}
\newcommand{\Component}{Component\index[Main]{Component}}

\newcommand{\representation}{representation\index[Main]{representation}}
\newcommand{\Representation}{Representation\index[Main]{Representation}}

\newcommand{\pass}{pass\index[Main]{pass}}
\newcommand{\Pass}{Pass\index[Main]{Pass}}

\newcommand{\multiPass}{multi-pass\index[Main]{multi-pass}}
\newcommand{\MultiPass}{Multi-pass\index[Main]{Multi-pass}}

\newcommand{\converter}{converter\index[Main]{converter}}
\newcommand{\Converter}{Converter\index[Main]{Converter}}

\newcommand{\generator}{generator\index[Main]{generator}}
\newcommand{\Generator}{Generator\index[Main]{Generator}}

\newcommand{\service}{service\index[Main]{service}}
\newcommand{\Service}{Service\index[Main]{Service}}

\newcommand{\earlyOption}{early option\index[Main]{early option}}
\newcommand{\EarlyOption}{Early option\index[Main]{Early option}}

\newcommand{\insider}{insider\index[Main]{insider}}
\newcommand{\Insider}{Insider\index[Main]{Insider}}
\newcommand{\regular}{regular\index[Main]{regular}}
\newcommand{\Regular}{Regular\index[Main]{Regular}}

\newcommand{\initialization}{initialization\index[Main]{initialization}}
\newcommand{\Initialization}{Initialization\index[Main]{Initialization}}

\newcommand{\cascade}{cascade\index[Main]{cascade}}
\newcommand{\cascading}{cascading\index[Main]{cascading}}
\newcommand{\Cascading}{Cascading\index[Main]{Cascading}}
\newcommand{\cascaded}{cascaded\index[Main]{cascaded}}
\newcommand{\Cascaded}{Cascaded\index[Main]{Cascaded}}

\newcommand{\braille}{Braille\index[Main]{Braille}}

\newcommand{\pim}{position in measure\index[Main]{position in measure}}

% capitalizing
%\newcommand{\Hafb}{\xmakefirstuc{\hafb}} % % only the first character in upper case
%\MakeUppercase{\hafb} % all the characters

\newcommand{\drawing}{%
{\tt drawing}\index[Main]{drawing}%
}
\newcommand{\drawn}{%
{\it drawn}\index[Main]{drawn}%
}

\newcommand{\musicXmlMarkup}[1]{%
{\tt <#1/>}\index[Main]{{\tt $<$#1 /$>$}}\index[MusicXML]{{\tt #1 $<$/$>$}}%
}
\newcommand{\musicXmlAttribute}[1]{%
{\tt "#1"}\index[Main]{{\tt $<$#1 /$>$}}\index[MusicXML]{{\tt #1 ""}}%
}

\newcommand{\file}[1]{%
file {\tt #1}\index[Main]{{\tt #1}}\index[Files]{{\tt #1}}%
}
\newcommand{\File}[1]{%
File {\tt #1}\index[Main]{{\tt #1}}\index[Files]{{\tt #1}}%
}

\newcommand{\fileName}[1]{%
{\tt #1}\index[Main]{{\tt #1}}\index[Files]{{\tt #1}}%
}
\newcommand{\fileNameBoth}[1]{%
{\textcolor{brown}{\tt *#1.h/.cpp}}\index[Main]{#1.h/.cpp@{{tt *#1.h/.cpp}}}\index[Files]{#1.h/.cpp@{{tt *#1.h/.cpp}}}%
}

\newcommand{\starFileName}[1]{%
{\tt #1}\index[Main]{#1@{\tt #1}}\index[Files]{#1@{\tt #1}}%
}
\newcommand{\starFileNameBoth}[1]{%
{\textcolor{brown}{\tt *#1.h/.cpp}}\index[Main]{#1.h/.cpp@{{tt *#1.h/.cpp}}}\index[Files]{#1.h/.cpp@{{tt *#1.h/.cpp}}}%
}

\newcommand{\type}[1]{%
type {\tt #1}\index[Main]{{\tt #1}}\index[Main]{enumeration type}\index[Types]{{\tt #1}}%
}
\newcommand{\Type}[1]{%
Type {\tt #1}\index[Main]{{\tt #1}}\index[Main]{enumeration type}\index[Types]{{\tt #1}}%
}

\newcommand{\enumType}[1]{%
enumeration type {\tt #1}\index[Main]{{\tt #1}}\index[Main]{enumeration type}\index[Types]{{\tt #1}}%
}
\newcommand{\EnumType}[1]{%
Enumeration type {\tt #1}\index[Main]{{\tt #1}}\index[Main]{enumeration type}\index[Types]{{\tt #1}}%
}

\newcommand{\class}[1]{%
class {\tt #1}\index[Main]{{\tt #1}}\index[Types]{{\tt #1}}%
}
\newcommand{\Class}[1]{%
Class {\tt #1}\index[Main]{{\tt #1}}\index[Types]{{\tt #1}}%
}
\newcommand{\className}[1]{%
{\tt #1}\index[Main]{{\tt #1}}\index[Types]{{\tt #1}}%
}

\newcommand{\purelyVirtualClass}[1]{%
purely virtual class {\tt #1}\index[Main]{{\tt #1}}\index[Types]{{\tt #1}}%
}
\newcommand{\PurelyVirtualClass}[1]{%
Purely virtual class {\tt #1}\index[Main]{{\tt #1}}\index[Types]{{\tt #1}}%
}
\newcommand{\purelyVirtualClasses}[1]{%
purely virtual classes {\tt #1}\index[Main]{{\tt #1}}\index[Types]{{\tt #1}}%
}
\newcommand{\PurelyVirtualClasses}[1]{%
Purely virtual classes {\tt #1}\index[Main]{{\tt #1}}\index[Types]{{\tt #1}}%
}

\newcommand{\smart}{smart pointer\index[Main]{smart pointer}}
\newcommand{\smartPointerType}[1]{%
{\tt #1}\index[Main]{{\tt #1}}\index[Types]{{\tt #1}}
}
\newcommand{\smartPointerTypeIndex}[1]{%
\index[Main]{{\tt #1}}\index[Types]{{\tt #1}}
}

\newcommand{\constant}[2]{%
constant {\tt #1::#2}\index[Main]{{\tt #1}!{\tt #2}}\index[Main]{{\tt #2}}\index[ConstantsFunctionsAndVariables]{{\tt #2}}%
}
\newcommand{\Constant}[2]{%
Constant {\tt #1::#2}\index[Main]{{\tt #1}!{\tt #2}}\index[Main]{{\tt #2}}\index[ConstantsFunctionsAndVariables]{{\tt #2}}%
}
\newcommand{\constantName}[2]{%
{\tt #1::#2}\index[Main]{{\tt #1}!{\tt #2}}\index[Main]{{\tt #2}}\index[ConstantsFunctionsAndVariables]{{\tt #2}}%
}

\newcommand{\functionPrivateVariable}[1]{%
function private variable {\tt #1}\index[Main]{{\tt #1}}\index[ConstantsFunctionsAndVariables]{{\tt #1}}%
}
\newcommand{\FunctionPrivateVariable}[1]{%
Function private variable {\tt #1}\index[Main]{{\tt #1}}\index[ConstantsFunctionsAndVariables]{{\tt #1}}%
}

\newcommand{\globalVariable}[1]{%
global variable {\tt #1}\index[Main]{{\tt #1}}\index[ConstantsFunctionsAndVariables]{{\tt #1}}%
}
\newcommand{\GlobalVariable}[1]{%
Global variable {\tt #1}\index[Main]{{\tt #1}}\index[ConstantsFunctionsAndVariables]{{\tt #1}}%
}

\newcommand{\constructor}[1]{%
constructor {\tt #1::#1~()}\index[Main]{{\tt #1}!{\tt #1}~()}\index[Main]{{\tt #1}~()}\index[MethodsAndFields]{{\tt #1}!{\tt #1}~()}%
}
\newcommand{\Constructor}[1]{%
Constructor {\tt #1::#1~()}\index[Main]{{\tt #1}!{\tt #1}~()}\index[Main]{{\tt #1}~()}\index[MethodsAndFields]{{\tt #1}!{\tt #1}~()}%
}
\newcommand{\constructorName}[1]{%
{\tt #1::#1~()}\index[Main]{{\tt #1}!{\tt #1}~()}\index[Main]{{\tt #1}~()}\index[MethodsAndFields]{{\tt #1}!{\tt #1}~()}%
}

\newcommand{\field}[2]{%
field {\tt #1::#2}\index[Main]{{\tt #1}!{\tt #2}}\index[Main]{{\tt #2}}\index[MethodsAndFields]{{\tt #1}!{\tt #2}}%
}
\newcommand{\Field}[2]{%
Field {\tt #1::#2}\index[Main]{{\tt #1}!{\tt #2}}\index[Main]{{\tt #2}}\index[MethodsAndFields]{{\tt #1}!{\tt #2}}%
}
\newcommand{\fieldIndex}[2]{%
\index[Main]{{\tt #1}!{\tt #2}}\index[Main]{{\tt #2}}\index[MethodsAndFields]{{\tt #1}!{\tt #2}}%
}

\newcommand{\method}[2]{%
method {\tt #1::#2~()}\index[Main]{{\tt #1}!{\tt #2}~()}\index[Main]{{\tt #2}~()}\index[MethodsAndFields]{{\tt #1}!{\tt #2}~()}%
}
\newcommand{\Method}[2]{%
Method {\tt #1::#2~()}\index[Main]{{\tt #1}!{\tt #2}~()}\index[Main]{{\tt #2}~()}\index[MethodsAndFields]{{\tt #1}!{\tt #2}~()}%
}

\newcommand{\starMethodNameTwo}[2]{%
{\tt *#1::#2~()}\index[Main]{#1@{\tt #1}!{\tt #2}~()}\index[Main]{#1@{\tt #2}~()}\index[MethodsAndFields]{{\tt #1}!{\tt #2}~()}%
}

\newcommand{\methodName}[1]{%
{\tt #1~()}\index[Main]{{\tt #1}~()}\index[MethodsAndFields]{{\tt #1}~()}%
}
\newcommand{\starMethodName}[1]{%
{\tt *#1~()}\index[Main]{#1~()@{\tt *#1}}\index[MethodsAndFields]{*#1~()@{\tt *#1~()}}%
}

\newcommand{\visitorMethod}[2]{%
visitor method {\tt #1::#2}\index[Main]{{\tt #1}!{\tt #2}}\index[Main]{{\tt #2}}\index[MethodsAndFields]{{\tt #1}!{\tt #2}}%
}
\newcommand{\VisitorMethod}[2]{%
vVsitor method {\tt #1::#2}\index[Main]{{\tt #1}!#2}\index[Main]{{\tt #2}}\index[MethodsAndFields]{{\tt #1}!{\tt #2}}%
}

\newcommand{\classMethod}[1]{%
class method {\tt #1~()}\index[Main]{{\tt #1}~()}\index[MethodsAndFields]{{\tt #1}~()}%
}
\newcommand{\classMethods}[1]{%
class methods {\tt {\tt #1}~()}\index[Main]{{\tt #1}~()}\index[MethodsAndFields]{{\tt #1}~()}%
}

\newcommand{\regularMethod}[1]{%
regular method {\tt {\tt #1}~()}\index[Main]{{\tt #1}~()}\index[MethodsAndFields]{{\tt #1}~()}%
}
\newcommand{\regularMethods}[1]{%
regular methods {\tt {\tt #1}~()}\index[Main]{{\tt #1}~()}\index[MethodsAndFields]{{\tt #1}~()}%
}

\newcommand{\virtualMethod}[1]{%
virtual method {\tt {\tt #1}~()}\index[Main]{{\tt #1}~()}\index[MethodsAndFields]{{\tt #1}~()}%
}
\newcommand{\virtualMethods}[1]{%
virtual methods {\tt {\tt #1}~()}\index[Main]{{\tt #1}~()}\index[MethodsAndFields]{{\tt #1}~()}%
}

\newcommand{\purelyVirtualMethod}[1]{%
purely virtual method {\tt {\tt #1}~()}\index[Main]{{\tt #1}~()}\index[MethodsAndFields]{{\tt #1}~()}%
}
\newcommand{\purelyVirtualMethods}[1]{%
purely virtual methods {\tt {\tt #1}~()}\index[Main]{{\tt #1}~()}\index[MethodsAndFields]{{\tt #1}~()}%
}

\newcommand{\fieldName}[1]{%
{\tt #1}\index[Main]{{\tt #1}}\index[MethodsAndFields]{{\tt #1}}%
}
\newcommand{\starFieldName}[1]{%
{\tt *#1}\index[Main]{#1@{\tt *#1}}\index[MethodsAndFields]{*#1@{\tt *#1}}%
}

\newcommand{\function}[1]{%
function {\tt #1~()}\index[Main]{{\tt #1}~()}\index[ConstantsFunctionsAndVariables]{{\tt #1}~()}%
}
\newcommand{\Function}[1]{%
Functions {\tt #1~()}\index[Main]{{\tt #1}~()}\index[ConstantsFunctionsAndVariables]{{\tt #1}~()}%
}

\newcommand{\functionName}[1]{%
{\tt #1~()}\index[Main]{{\tt #1}~()}\index[ConstantsFunctionsAndVariables]{{\tt #1}~()}%
}

\newcommand{\functions}{%
functions\index[Main]{functions}%
}

\newcommand{\mainFunction}{%
{\tt main ()}\index[Main]{main ()}\index[Files]{main ()}%
}

\newcommand{\interfaceFunction}[1]{%
interface function {\tt #1~()}\index[Main]{{\tt #1}~()}\index[ConstantsFunctionsAndVariables]{{\tt #1}~()}%
}
\newcommand{\InterfaceFunction}[1]{%
Interface function {\tt #1~()}\index[Main]{{\tt #1}~()}\index[ConstantsFunctionsAndVariables]{{\tt #1}~()}%
}


\usepackage{xparse}
% https://mirror.foobar.to/CTAN/macros/latex/contrib/l3packages/xparse.pdf
% https://mirror.foobar.to/CTAN/macros/latex/required/tools/xspace.pdf
% https://mirror.foobar.to/CTAN/macros/latex/contrib/siunitx/siunitx.pdf
% https://mirror.foobar.to/CTAN/macros/generic/xstring/xstring-en.pdf
% https://mirror.foobar.to/CTAN/macros/latex/contrib/stringstrings/stringstrings.pdf

\ExplSyntaxOn
\NewDocumentCommand{\Index}{om}
 {
  \str_set:Nn \l_tmpa_str { #2 }
  \str_if_in:NnTF \l_tmpa_str { - }
   {
    \str_set_eq:NN \l_tmpb_str \l_tmpa_str
    \str_remove_all:Nn \l_tmpb_str { - }
    \use:x
     {
      \exp_not:N \index
      \IfValueT { #1 } { [ #1 ] }
      { \str_use:N \l_tmpb_str @ \str_use:N \l_tmpa_str }
     }
   }
   {
    \use:x
     {
      \exp_not:N \index
      \IfValueT { #1 } { [ #1 ] }
      { \str_use:N \l_tmpa_str }
     }
   }
 }
\ExplSyntaxOff

\usepackage{stringstrings}

\newcommand{\option}[1]{%
%option {\tt-#1}'\index[Main]{\tt #1}\index[Options]{{\tt \gobblechar[v]{#1}}@#1}%
option {\tt -#1}\index[Main]{\tt -#1}\index[Options]{\tt -#1}%
}
\newcommand{\Option}[1]{%
Option {\tt -#1}\index[Main]{\tt -#1}\index[Options]{\tt -#1}%
}
\newcommand{\optionName}[1]{%
%option {\tt-#1}'\index[Main]{\tt #1}\index[Options]{{\tt \gobblechar[v]{#1}}@#1}%
{\tt -#1}\index[Main]{\tt -#1}\index[Options]{\tt -#1}%
}

\newcommand{\optionBoth}[2]{%
%option {\tt -#1, -#2}\index[Main]{\tt -#1, -#2}\index[Options]{{\tt \gobblechar[v]{#1}}@#1}%
option {\tt -#1, -#2}\index[Main]{\tt -#1, -#2}\index[Options]{\tt -#1, -#2}%
}
\newcommand{\OptionBoth}[2]{%
Option {\tt -#1, -#2}\index[Main]{\tt -#1, -#2}\index[Options]{\tt -#1, -#2}%
}
\newcommand{\optionNameBoth}[2]{%
{\tt -#1, -#2}\index[Main]{\tt -#1, -#2}\index[Options]{\tt -#1, -#2}%
}


% -------------------------------------------------------------------------
% files commands
% -------------------------------------------------------------------------

% GitHub

\newcommand{\mxmlfile}[1]{%
\href{https://github.com/jacques-menu/musicformats/tree/master/files/musicxml/#1}{{\tt #1}}%
}

\newcommand{\docpdf}[2]{%
\href{https://github.com/jacques-menu/musicformats/tree/dev/documentation/#1/#2}{{\tt #2}}%
}
%%%JMI https://github.com/jacques-menu/musicformats/blob/dev/documentation/MusicFormatsAPIUserGuide/MusicFormatsAPIUserGuide.pdf

\newcommand{\schemafile}[1]{%
\href{https://github.com/jacques-menu/musicformats/tree/master/schemas/MusicXML/dtds/3.1/#1}{{\tt #1}}%
}

\newcommand{\subdir}[1]{%
\href{https://github.com/jacques-menu/musicformats/tree/master/#1}{{\tt #1}}%
}

\newcommand{\cppsamplefile}[1]{%
\href{https://github.com/jacques-menu/musicformats/tree/master/samples/#1}{{\tt #1}}%
}

\newcommand{\passfile}[1]{%
\href{https://github.com/jacques-menu/musicformats/tree/dev/src/passes/#1}{{\tt passes/#1}}%
}

% libmusicxml2 access path

\newcommand{\elements}[1]{%
{\textcolor{brown}{\tt libmusicxml/src/elements!#1}}%
}
\newcommand{\elementsBoth}[1]{%
{\textcolor{brown}{\tt libmusicxml/src/elements!#1.h/.cpp}}%
}

\newcommand{\templates}[1]{%
{\textcolor{brown}{\tt libmusicxml/src/elements/templates!#1}}%
}
\newcommand{\templatesBoth}[1]{%
{\textcolor{brown}{\tt libmusicxml/src/elements/templates!#1.h/.cpp}}%
}

\newcommand{\factory}[1]{%
{\textcolor{brown}{\tt libmusicxml/src/factory!#1}}%
}
\newcommand{\factoryBoth}[1]{%
{\textcolor{brown}{\tt libmusicxml/src/factory!#1.h/.cpp}}%
}

\newcommand{\libFiles}[1]{%
{\textcolor{brown}{\tt libmusicxml/src/files/#1}}\index[Main]{#1}\index[Files]{libmusicxml/src/files!#1}%
}
\newcommand{\libFilesBoth}[1]{%
{\textcolor{brown}{\tt libmusicxml/src/files/#1.h/.cpp}}\index[Main]{#1.h/.cpp}\index[Files]{libmusicxml/src/files!#1.h/.cpp}%
}

\newcommand{\visitors}[1]{%
{\textcolor{brown}{\tt libmusicxml/src/visitors!#1}}%
}
\newcommand{\visitorsBoth}[1]{%
{\textcolor{brown}{\tt libmusicxml/src/visitors!#1.h/.cpp}}%
}

\newcommand{\lib}[1]{%
{\textcolor{brown}{\tt libmusicxml/src/lib!#1}}%
}
\newcommand{\libBoth}[1]{%
{\textcolor{brown}{\tt libmusicxml/src/lib!#1.h/.cpp}}%
}

\newcommand{\libmusicxmlsamples}[1]{%
{\textcolor{brown}{\tt libmusicxml/samples/#1}}\index[Main]{#1}\index[Files]{libmusicxml/samples!#1}%
}

% MusicFormats folders

\newcommand{\doc}[1]{%
{\textcolor{brown}{\tt documentation/#1}}\index[Files]{documentation!#1}%
}

\newcommand{\common}[1]{%
{\textcolor{brown}{\tt documentation/common/#1}}\index[Files]{documentation/common!#1}%
}

\newcommand{\src}[1]{%
{\textcolor{brown}{\tt src/#1}}\index[Files]{src!#1}%
}

\newcommand{\build}[1]{%
{\textcolor{brown}{\tt build/#1}}\index[Files]{build!#1}%
}

\newcommand{\mfc}[1]{%
{\textcolor{brown}{\tt src/components/#1}}\index[Main]{#1}\index[Files]{src/components!#1}%
}
\newcommand{\mfcBoth}[1]{%
{\textcolor{brown}{\tt src/utilities/#1.h/.cpp}}\index[Main]{#1.h/.cpp}\index[Files]{src/utilities!#1.h/.cpp}%
}

\newcommand{\mfutilities}[1]{%
{\textcolor{brown}{\tt src/mfutilities/#1}}\index[Main]{#1}\index[Files]{src/mfutilities!#1}%
}
\newcommand{\mfutilitiesBoth}[1]{%
{\textcolor{brown}{\tt src/utilities/#1.h/.cpp}}\index[Main]{#1.h/.cpp}\index[Files]{src/utilities!#1.h/.cpp}%
}

% MusicFormats wae access paths

\newcommand{\wae}[1]{%
{\textcolor{brown}{\tt src/wae/#1}}\index[Main]{#1}\index[Files]{src/wae!#1}%
}
\newcommand{\waeBoth}[1]{%
{\textcolor{brown}{\tt src/wae/#1.h/.cpp}}\index[Files]{src/wae!#1.h/.cpp}%
}

% MusicFormats oah access paths

\newcommand{\oah}[1]{%
{\textcolor{brown}{\tt src/oah/#1}}\index[Main]{#1}\index[Files]{src/oah!#1}%
}
\newcommand{\oahBoth}[1]{%
{\textcolor{brown}{\tt src/oah/#1.h/.cpp}}\index[Files]{src/oah!#1.h/.cpp}%
}

% MusicFormats CLI samples access paths

\newcommand{\clisamples}[1]{%
{\textcolor{brown}{\tt src/clisamples/#1}}\index[Main]{#1}\index[Files]{src/clisamples!#1}\index[Files]{clisamples!#1}%
}

% MusicFormats formats access paths

\newcommand{\mflibrary}[1]{%
{\textcolor{brown}{\tt src/mflibrary#1}}\index[Main]{#1}\index[Files]{src/mflibrary!#1}%
}
\newcommand{\mflibraryBoth}[1]{%
{\textcolor{brown}{\tt src/mflibrary/#1.h/.cpp}}\index[Files]{src/mflibrary!#1.h/.cpp}%
}

\newcommand{\mxsr}[1]{%
{\textcolor{brown}{\tt src/formats/mxsr/#1}}\index[Main]{#1}\index[Files]{src/formats/mxsr!#1}%
}
\newcommand{\mxsrBoth}[1]{%
{\textcolor{brown}{\tt src/formats/mxsr/#1.h/.cpp}}\index[Files]{src/formats/mxsr!#1.h/.cpp}%
}

\newcommand{\msr}[1]{%
{\textcolor{brown}{\tt src/formats/msr/#1}}\index[Main]{#1}\index[Files]{src/formats/msr!#1}%
}
\newcommand{\msrBoth}[1]{%
{\textcolor{brown}{\tt src/formats/msr/#1.h/.cpp}}\index[Files]{src/formats/msr!#1.h/.cpp}%
}

\newcommand{\bsr}[1]{%
{\textcolor{brown}{\tt src/formats/bsr/#1}}\index[Main]{#1}\index[Files]{src/formats/bsr!#1}%
}
\newcommand{\bsrBoth}[1]{%
{\textcolor{brown}{\tt src/formats/bsr/#1.h/.cpp}}\index[Main]{#1}\index[Files]{src/formats/bsr!#1.h/.cpp}%
}

\newcommand{\lpsr}[1]{%
{\textcolor{brown}{\tt src/formats/lpsr//#1}}\index[Main]{#1}\index[Files]{src/formats/lpsr!#1}%
}
\newcommand{\lpsrBoth}[1]{%
{\textcolor{brown}{\tt src/formats/lpsr//#1.h/.cpp}}\index[Files]{src/formats/lpsr!#1.h/.cpp}%
}

\newcommand{\msdl}[1]{%
{\textcolor{brown}{\tt src/formats/msdl/#1}}\index[Main]{#1}\index[Files]{src/formats/msdl!#1}%
}
\newcommand{\msdlBoth}[1]{%
{\textcolor{brown}{\tt src/formats/msdl/#1.h/.cpp}}\index[Files]{src/formats/msdl!#1.h/.cpp}%
}

% MusicFormats passes access paths

\newcommand{\msdlToMsr}[1]{%
{\textcolor{brown}{\tt src/passes/msdl2msr/#1}}%
}
\newcommand{\msdlToMsrBoth}[1]{%
{\textcolor{brown}{\tt src/passes/msdl2msr/#1.h/.cpp}}%
}

\newcommand{\bsrToBraille}[1]{%
{\textcolor{brown}{\tt src/passes/bsr2braille/#1}}%
}
\newcommand{\bsrToBrailleBoth}[1]{%
{\textcolor{brown}{\tt src/passes/bsr2braille/#1.h/.cpp}}%
}

\newcommand{\mxsrToMsr}[1]{%
{\textcolor{brown}{\tt src/passes/mxsr2msr/#1}}%
}
\newcommand{\mxsrToMsrBoth}[1]{%
{\textcolor{brown}{\tt src/passes/mxsr2msr/#1.h/.cpp}}%
}

\newcommand{\msrToMsr}[1]{%
{\textcolor{brown}{\tt src/passes/msr2msr/#1}}%
}
\newcommand{\msrToMsrBoth}[1]{%
{\textcolor{brown}{\tt src/passes/msr2msr/#1.h/.cpp}}%
}

\newcommand{\msrToLpsr}[1]{%
{\textcolor{brown}{\tt src/passes/msr2lpsr/#1}}%
}
\newcommand{\msrToLpsrBoth}[1]{%
{\textcolor{brown}{\tt src/passes/msr2lpsr/#1.h/.cpp}}%
}

\newcommand{\lpsrToLilypond}[1]{%
{\textcolor{brown}{\tt src/passes/lpsr2lilypond/#1}}%
}
\newcommand{\lpsrToLilypondBoth}[1]{%
{\textcolor{brown}{\tt src/passes/lpsr2lilypond/#1.h/.cpp}}%
}

\newcommand{\msrToBsr}[1]{%
{\textcolor{brown}{\tt src/passes/msr2bsr/#1}}%
}
\newcommand{\msrToBsrBoth}[1]{%
{\textcolor{brown}{\tt src/passes/msr2bsr/#1.h/.cpp}}%
}

\newcommand{\msrToMxsr}[1]{%
{\textcolor{brown}{\tt src/passes/msr2mxsr/#1}}%
}
\newcommand{\msrToMxsrBoth}[1]{%
{\textcolor{brown}{\tt src/passes/msr2mxsr/#1.h/.cpp}}%
}

\newcommand{\mxsrToMusicxml}[1]{%
{\textcolor{brown}{\tt src/passes/mxsr2musicxml/#1}}%
}
\newcommand{\mxsrToMusicxmlBoth}[1]{%
{\textcolor{brown}{\tt src/passes/mxsr2musicxml/#1.h/.cpp}}%
}

% MusicFormats formats generation access paths

\newcommand{\lilypondGeneration}[1]{%
{\textcolor{brown}{\tt src/formatsgeneration/lilypondGeneration/#1}}%
}
\newcommand{\lilypondGenerationBoth}[1]{%
{\textcolor{brown}{\tt src/formatsgeneration/lilypondGeneration/#1.h/.cpp}}%
}

\newcommand{\brailleGeneration}[1]{%
{\textcolor{brown}{\tt src/formatsgeneration/brailleGeneration/#1}}%
}
\newcommand{\brailleGenerationBoth}[1]{%
{\textcolor{brown}{\tt src/formatsgeneration/brailleGeneration/#1.h/.cpp}}%
}

\newcommand{\mxsrGeneration}[1]{%
{\textcolor{brown}{\tt src/formatsgeneration/mxsrGeneration/#1}}%
}
\newcommand{\mxsrGenerationBoth}[1]{%
{\textcolor{brown}{\tt src/formatsgeneration/mxsrGeneration/#1.h/.cpp}}%
}

\newcommand{\guidoGeneration}[1]{%
{\textcolor{brown}{\tt src/formatsgeneration/guidoGeneration/#1}}%
}
\newcommand{\guidoGenerationBoth}[1]{%
{\textcolor{brown}{\tt src/formatsgeneration/guidoGeneration/#1.h/.cpp}}%
}

\newcommand{\multiGeneration}[1]{%
{\textcolor{brown}{\tt src/formatsgeneration/multiGeneration/#1}}%
}
\newcommand{\multiGenerationBoth}[1]{%
{\textcolor{brown}{\tt src/formatsgeneration/multiGeneration/#1.h/.cpp}}%
}

% MusicFormats converters access paths

\newcommand{\converters}[1]{%
{\textcolor{brown}{\tt src/converters/#1}}%
}
\newcommand{\convertersBoth}[1]{%
{\textcolor{brown}{\tt src/converters/#1.h/.cpp}}%
}

\newcommand{\musicxmlToLilypond}[1]{%
{\textcolor{brown}{\tt src/converters/musicxml2lilypond/#1}}%
}
\newcommand{\musicxmlToLilypondBoth}[1]{%
{\textcolor{brown}{\tt src/converters/musicxml2lilypond/#1.h/.cpp}}%
}

\newcommand{\musicxmlToBraille}[1]{%
{\textcolor{brown}{\tt src/converters/musicxml2braille/#1}}%
}
\newcommand{\musicxmlToBrailleBoth}[1]{%
{\textcolor{brown}{\tt src/converters/musicxml2braille/#1.h/.cpp}}%
}

\newcommand{\musicxmlToMusicxml}[1]{%
{\textcolor{brown}{\tt src/converters/musicxml2musicxml/#1}}%
}
\newcommand{\musicxmlToMusicxmlBoth}[1]{%
{\textcolor{brown}{\tt src/converters/musicxml2musicxml/#1.h/.cpp}}%
}

\newcommand{\musicxmlToGuido}[1]{%
{\textcolor{brown}{\tt src/converters/musicxml2guido/#1}}%
}
\newcommand{\musicxmlToGuidoBoth}[1]{%
{\textcolor{brown}{\tt src/converters/musicxml2guido/#1.h/.cpp}}%
}

\newcommand{\msdlToLilypond}[1]{%
{\textcolor{brown}{\tt src/converters/msdl2lilypond/#1}}%
}
\newcommand{\msdlToLilypondBoth}[1]{%
{\textcolor{brown}{\tt src/converters/msdl2lilypond/#1.h/.cpp}}%
}

% MusicFormats interfaces access paths

%\newcommand{\interfaces}[1]{%
%{\textcolor{brown}{\tt src/interfaces/#1}}%
%}
%\newcommand{\interfacesBoth}[1]{%
%{\textcolor{brown}{\tt src/interfaces/#1.h/.cpp}}%
%}
%
%\newcommand{\libraryInterfaces}[1]{%
%{\textcolor{brown}{\tt src/interfaces/libraryinterfaces/#1}}%
%}
%\newcommand{\libraryInterfacesBoth}[1]{%
%{\textcolor{brown}{\tt src/interfaces/libraryinterfaces/#1.h/.cpp}}%
%}

% MusicFormats generators access paths

\newcommand{\generators}[1]{%
{\textcolor{brown}{\tt src/generators/#1}}%
}
\newcommand{\generatorsBoth}[1]{%
{\textcolor{brown}{\tt src/generators/#1.h/.cpp}}%
}

\newcommand{\xmlToXmlTool}[1]{%
{\textcolor{brown}{\tt src/clisamples/xml2xml/#1}}%
}
\newcommand{\xmlToXmlToolBoth}[1]{%
{\textcolor{brown}{\tt src/clisamples/xml2xml/#1.h/.cpp}}%
}

% MusicFormats manpage sgeneration access paths JMI???

\newcommand{\formatsgeneration}[1]{%
{\textcolor{brown}{\tt src/formatsgeneration/#1}}%
}
\newcommand{\formatsgenerationBoth}[1]{%
{\textcolor{brown}{\tt src/formatsgeneration/#1.h/.cpp}}%
}


% -------------------------------------------------------------------------
% music notation
% -------------------------------------------------------------------------

\newcommand{\Flat}{\raisebox{0.75ex}{$\flat$}}
\newcommand{\Natural}{\raisebox{0.75ex}{$\natural$}}
\newcommand{\Sharp}{\raisebox{0.75ex}{$\sharp$}}

\newcommand{\DoubleFlat}{\raisebox{0.75ex}{$\flat\flat$}}
\newcommand{\DoubleSharp}{\raisebox{0.75ex}{$\sharp\sharp$}}


%% -------------------------------------------------------------------------
%% hyperref
%% -------------------------------------------------------------------------
%
%\def\@StructureLinksColor{\ifx\isundefined \StructureLinksColor darkgray \else \StructureLinksColor\fi}
%
%\def\@URLColor{\ifx\isundefined \URLColor blue \else \URLColor\fi}
%
%\def\@PageLinksColor{\ifx\isundefined \PageLinksColor darkgray \else \PageLinksColor\fi}
%
%\def\@ContentsLinksColor{\ifx\isundefined \ContentsLinksColor purple \else \ContentsLinksColor\fi}
%
%\def\@TOCLinksColor{\ifx\isundefined \TOCLinksColor purple \else \TOCLinksColor\fi}
%
%
%\hypersetup{
%	colorlinks=true,
%	breaklinks=true,
%	linkcolor=\@StructureLinksColor,
%	urlcolor=\@URLColor,
%	pagecolor=\@PageLinksColor
%	}
%
%\newcommand{\ContentsLabel}[1]%
%	% Arguments: etiquette
%	{%
%	\hypertarget{#1}{\label{#1}}%
%	}
%
%\newcommand{\ContentsLink}[2]%
%	% Arguments: etiquette texteDuLien
%	{%
%	\hypersetup{linkcolor=\@ContentsLinksColor}%
%	\hyperlink{#1}{#2, voir page~\pageref{#1}~\includegraphics{MarquePourLiens}}%
%	\hypersetup{linkcolor=\@StructureLinksColor}%
%	}
%
%\newcommand{\ContentsPageLink}[2]%
%	% Arguments: etiquette texteDuLien
%	{%
%	\hypersetup{linkcolor=\@PageLinksColor}%
%	\hyperlink{#1}{#2 page~\pageref{#1}~\includegraphics{MarquePourLiens}}%
%	\hypersetup{linkcolor=\@StructureLinksColor}%
%	}
%
%\newcommand{\URL}[1]%
%	% Argument: l'URL
%	{{\footnotesize \url{#1}}}
%
%\newcommand{\RFC}[1]%
%	% Argument: numero de RFC
%	{%
%	\href{http://www.ietf.org/rfc/rfc#1.txt}{RFC~#1~ \includegraphics{MarquePourLiens}}%
%	}

