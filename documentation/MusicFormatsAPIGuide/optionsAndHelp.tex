% !TEX root = MusicFormatsUserGuide.tex

% -------------------------------------------------------------------------
\chapter{Options and help (OAH)}
% -------------------------------------------------------------------------

\mf\ having many \service s with many options makes options and help handling a challenge.\\
This is why \mf\ provides OAH (Options And Help), a full-fledged object-oriented options and help management infrastructure.

\oahRepr\ (Options And Help) is supposed to be pronounced something close to "whaaaah!"
    The intonation is left to the speaker, though...
    And as the saying goes: "\oahRepr? why not!"

OAH organizes the options and the corresponding help in a hierarchy of groups, sub-groups and so-called atoms. OAH is introspective, thus help can be obtained for every group, sub-group or atom at will.

Each pass supplies a OAH group, containing its own options and help. The converters then aggregate the OAH groups of the passes they are composed of to offer their options and help to the user.

\mf\ is equipped with a full-fledged set of options with the corresponding help. Since there are many options and the translation work is done in successive passes, the help is organized in a hierarchy of groups, each containing sub-groups of individual options called \MainIt{atoms}.


% -------------------------------------------------------------------------
\section{Options basics}
% -------------------------------------------------------------------------

THe \oahRepr\ options are very easy to use. They are inspired by \Main{GNU} options, with more power and flexibility:
\begin{itemize}
\item the options can be supplied in the \CLI\ as usual;

  \item they can also be supplied in a call to an \API\ \Main{function} such as \functionName{musicxmlfile2lilypond}, in an options and arguments argument. \\
    		See \chapterRef{API principles} for the details;

\item options are introduced either by \code{-} or \code{--}, which can be used at will. Both ways are equivalent;

\item all options have a long name, and some have a complementary short name.
 The latter is not provided if the long name is short enough, such as \optionName{jianpu}, \optionName{cubase}, \optionName{ambitus} or \optionName{custos}.\\
    Short and long names can be used and mixed at will in the \CLI\ and in option vectors (\API);

\item some short option names are supplied as is usually expected for open sotware, such as \optionName{h} (help), \optionName{a} (about), and \optionName{o} (output file name):
\begin{lstlisting}[language=Terminal]
jacquesmenu@macmini > xml2ly -query output-file-name
--- Help for atom "output-file-name" in subgroup "Files"
    -output-file-name, -o FILENAME
          Write output to file FILENAME instead of standard output.
\end{lstlisting}

\item options and arguments such as file names can be intermixed at will. Thus:
\begin{lstlisting}[language=Terminal]
xml2ly --cpu basic/HelloWorld.xml
\end{lstlisting}
and
\begin{lstlisting}[language=Terminal]
xml2ly basic/HelloWorld.xml -cpu
\end{lstlisting}

produce the exact same result;

\item some options names, either long or short, share a common \Main{prefix}. This allows them to be \MainIt{contracted}, as in \code{-h=rests,notes}, which is equivalent to \code{-hrests, -hnotes}, and \code{-trace=voices,notes}, equivalent to \code{-trace-voices, -trace-notes};

\item the single-character options can be \MainIt{clustered}: \optionName{vac} is equivalent to: \code{-v, -a, -c}.

\end{itemize}


% -------------------------------------------------------------------------
\section{Displaying help about options usage}
% -------------------------------------------------------------------------

\begin{lstlisting}[language=Terminal]
jacquesmenu@macmini > xml2ly -help-options-usage
xml2ly options usage:
  In xml2ly, '-' as an argument, represents standard input.

  Most options have a short and a long name for commodity.
  The long name may be empty if the short name is explicit enough.

  The options are organized in a group-subgroup-atom hierarchy.
  Help can be obtained for groups or subgroups at will,
  as well as for any option with the '-name-help, -nh' option.

  A subgroup can be showm as a header only, in which case its description is printed
  only when the corresponding short or long names are used.

  Both '-' and '--' can be used to introduce options,
  even though the help facility only shows them with '-'.

  There some prefixes to allow for shortcuts,
  such as '-t=voices,meas' for '-tvoices, -tmeas'.

  The options can be placed in any order,
  provided the values immediately follow the atoms that need them.

  Using options that attempt to create files, such as '-o, -output-file-name',
  leads to an error if the environment is read-only access,
  as is the case of https://libmusicxml.grame.fr .
\end{lstlisting}


% -------------------------------------------------------------------------
\section{The {\tt -insider} option}
% -------------------------------------------------------------------------

As mentioned above, the \mf\ library components, i.e. \representation s, \pass es, \converter s and \generator s, have options and help attached to them. There are also other 'global' sets of options, independently of the individual components themselves.

\mf\ has to 'modes' for options and help handling:
\begin{itemize}
\item in \MainIt{regular} mode, the default, the options are grouped by subject, such as tuplets or chords. In other words, there are grouped in a user-oriented way;
\item in \MainIt{insider} mode, they are grouped as there are used internally by MusicFormats behind the scenes, in an implementation-oriented way, hence the name.
\end{itemize}

Switching from the default regular mode to the insider mode is done with the \optionNameBoth{insider}{ins} option:
\begin{lstlisting}[language=Terminal]
jacquesmenu@macmini > xml2ly -query insider
--- Help for atom "insider" in subgroup "Options and help"
    -insider, -ins
          Use the 'insider' mode for the options and help,
          in which the options are grouped as they are used internally by MusicFormats.
          In the 'regular' defaut mode, they are grouped by user-oriented topics,
          such a slurs, tuplets and figured bass.
\end{lstlisting}

In regular mode, the options are displayed in subgroups only. The groups containing them are not displayed for simplicity, because a three-level options hierarchy is not what users expect and are used to.

For example, the \optionNameBoth{ignore-musicxml-ornaments}{oorns} option is displayed this way in regular mode:
\begin{lstlisting}[language=Terminal]
jacquesmenu@macmini > xml2ly -query ignore-musicxml-ornaments
--- Help for atom "ignore-musicxml-ornaments" in subgroup "Ornaments"
    -ignore-musicxml-ornaments, -oorns
          Ignore ornaments in MusicXML data.
\end{lstlisting}

In insider mode, on the contrary, the full group-subgroup-atom hierarchy is visible, as well as the attachment of the options to the groups managed internally by \mf:
\begin{lstlisting}[language=Terminal]
jacquesmenu@macmini > xml2ly -query ignore-musicxml-ornaments -insider
--- Help for atom "ignore-musicxml-ornaments" in subgroup "Notes" of group "mxsr2msr" ---
    -ignore-musicxml-ornaments, -oorns
          Ignore ornaments in MusicXML data.
\end{lstlisting}

To summarize things up, it can be said that the regular mode offers a user-oriented \MainIt{view} of the options available in the insider mode.

%For example, the \mf\ \service s need options to control their output file name. There is thus:
%
\begin{lstlisting}[language=Terminal]
%jacquesmenu@macmini: ~ > xml2ly -help-output-file-group -insider
%--- Help for group "OutputFile" ---
%  OutputFile (-help-output-file-group, -hofg):
%  --------------------------
%    Output file    (-help-output-file, -hof):
%      -output-file-name, -o FILENAME
%            Write output to file FILENAME instead of standard output.
%      -auto-output-file-name, -aofn
%            This option can only be used when reading from a file.
%            Write MusicXML code to a file in the current working directory.
%            The file name is derived from that of the input file,
%            replacing any suffix after the '.' by 'xml'
%            or adding '.xml' if none is present.
%\end{lstlisting}



% -------------------------------------------------------------------------
\section{Displaying a help summary}
% -------------------------------------------------------------------------

This can be done with the \optionNameBoth{help-summary}{hs} option:
\begin{lstlisting}[language=Terminal]
jacquesmenu@macmini > xml2ly -query help-summary
--- Help for atom "help-summary" in subgroup "Options and help"
    -help-summary, -hs
          Display xml2ly's help summary.
\end{lstlisting}

% -------------------------------------------------------------------------
\section{Restricting help to a given group or subgroup}
% -------------------------------------------------------------------------


% -------------------------------------------------------------------------
\section{Options flavours}
% -------------------------------------------------------------------------

There are various options in \mr\ for various needs:
\begin{itemize}
\item oahAtomExpectingAValue
oahAtomImplicitlyStoringAValue
oahAtomStoringAValue
oahPureHelpAtomWithoutAValue
oahPureHelpAtomExpectingAValue

class EXP oahAtomAlias : public oahAtom
class EXP oahMacroAtom : public oahAtom
class EXP oahOptionsUsageAtom : public oahPureHelpAtomWithoutAValue
class EXP oahHelpAtom : public oahPureHelpAtomWithoutAValue
class EXP oahHelpSummaryAtom : public oahPureHelpAtomWithoutAValue
class EXP oahAboutAtom : public oahPureHelpAtomWithoutAValue
class EXP oahVersionAtom : public oahPureHelpAtomWithoutAValue
class EXP oahLibraryVersionAtom : public oahPureHelpAtomWithoutAValue
class EXP oahHistoryAtom : public oahPureHelpAtomWithoutAValue
class EXP oahLibraryHistoryAtom : public oahPureHelpAtomWithoutAValue
class EXP oahContactAtom : public oahPureHelpAtomWithoutAValue
class EXP oahBooleanAtom : public oahAtom
class EXP oahTwoBooleansAtom : public oahBooleanAtom
class EXP oahThreeBooleansAtom : public oahBooleanAtom
class EXP oahCombinedBooleansAtom : public oahAtom
class EXP oahCommonPrefixBooleansAtom : public oahAtom
class EXP oahIntegerAtom : public oahAtomStoringAValue
class EXP oahTwoIntegersAtom : public oahIntegerAtom
class EXP oahFloatAtom : public oahAtomStoringAValue
class EXP oahStringAtom : public oahAtomStoringAValue
class EXP oahFactorizedStringAtom : public oahAtom
class EXP oahStringWithDefaultValueAtom : public oahStringAtom
class EXP oahStringWithRegexAtom : public oahStringAtom
class EXP oahRationalAtom : public oahAtomStoringAValue
class EXP oahNaturalNumbersSetElementAtom : public oahAtomStoringAValue
class EXP oahRGBColorAtom : public oahAtomStoringAValue
class EXP oahIntSetElementAtom : public oahAtomStoringAValue
class EXP oahStringSetElementAtom : public oahAtomStoringAValue
class EXP oahStringToIntMapElementAtom : public oahAtomStoringAValue
class EXP oahStringAndIntegerAtom : public oahAtomStoringAValue
class EXP oahStringAndTwoIntegersAtom : public oahAtomStoringAValue
class EXP oahLengthUnitKindAtom : public oahAtomStoringAValue
class EXP oahLengthAtom : public oahAtomStoringAValue
class EXP oahMidiTempoAtom : public oahAtomStoringAValue
class EXP oahOptionNameHelpAtom : public oahStringWithDefaultValueAtom
class EXP oahQueryOptionNameAtom : public oahPureHelpAtomExpectingAValue
class EXP oahFindStringAtom : public oahPureHelpAtomExpectingAValue


\end{itemize}


% -------------------------------------------------------------------------
\section{Displaying the options values}
% -------------------------------------------------------------------------

This can be done with the \optionNameBoth{display-options-values}{dov} option:
\begin{lstlisting}[language=Terminal]
jacquesmenu@macmini > xml2ly -query display-options-values
--- Help for atom "display-options-values" in subgroup "Options and help"
    -display-options-values, -dov
          Write the chosen options values to standard error.
\end{lstlisting}


% -------------------------------------------------------------------------
\section{Querying about options by name}
% -------------------------------------------------------------------------

One can obtain help on any specific group, sub-group or atom with the \optionNameBoth{display-options-values}{dov} option, such as:
\begin{lstlisting}[language=Terminal]
menu@macbookprojm > xml2ly -query ambitus

--- Help for option 'ambitus' in subgroup "Engravers" of group "LilyPond" ---

LilyPond (-hlily, -help-lilypond):
  These lilypond control which LilyPond code is generated.

--------------------------
  Engravers (-hlpe, -help-lilypond-engravers):

    -ambitus
          Generate an ambitus range at the beginning of the staves/voices.

\end{lstlisting}

Some options have an optional value such as \optionName{query}, whose default value is\dots \code{query}:
\begin{lstlisting}[language=Terminal]
menu@macbookprojm > xml2ly -query

--- Help for option 'onh' in subgroup "Options help" of group "Options and help" ---

Options and help (-hoah, -help-options-and-help):
--------------------------
  Options help (-hoh, -help-options-help):

    -onh, -query[=OPTION_NAME]
          Print help about OPTION_NAME.
          OPTION_NAME is optional, and the default value is 'onh'.
\end{lstlisting}


% -------------------------------------------------------------------------
\section{Searching the help for a string}
% -------------------------------------------------------------------------


% -------------------------------------------------------------------------
\section{Displaying MusicFormats representations}
% -------------------------------------------------------------------------


% -------------------------------------------------------------------------
\section{Including options from a file}
% -------------------------------------------------------------------------



