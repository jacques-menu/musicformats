% !TEX root = MusicFormatsAPIGuide.tex

% -------------------------------------------------------------------------
\chapter{\xmlToGmn\ }
% -------------------------------------------------------------------------

\subimport{./}{xml2gmnArchitecturePicture}


% -------------------------------------------------------------------------
\section{Why \xmlToGmn?}
% -------------------------------------------------------------------------

\libmusicxml\ comes with \xmlToGuido, a \converter\ of \mxml\ files to \guido\ syntax, which has some limitations. It is supplied as as sample of the library's use.

\xmlToGmn\ has been designed to complement \libmusicxml\ features: it provides the same translation as \xmlToGuido, with more options for flexibility.

Work remains to be done in the conversion of \msrRepr\ to MXSR, but \xmlToGmn\ is already fairly complete.


% -------------------------------------------------------------------------
\section{What \xmlToGmn\ does}
% -------------------------------------------------------------------------

\xmlToGmn\ performs the 5 steps from \mxml\ to LilyPond\ to translate the former into the latter, as shown in \figureRef{xmlToLyArchitecture}. Converting from MXSR to MSR\ is done in two sub-phases for implementation reasons.

The '{\tt -about}' option to \xmlToGmn\ details that somewhat:
\begin{lstlisting}[language=MusicXML]
jacquesmenu@macmini > xml2xml -about
What xml2xml does:

    This multi-pass converter basically performs 6 passes:
        Pass 1:  reads the contents of MusicXMLFile or stdin ('-')
                 and converts it to a MusicXML tree;
        Pass 2a: converts that MusicXML tree into
                 a first Music Score Representation (MSR) skeleton;
        Pass 2b: populates the MSR skeleton from the MusicXML tree
                 to get a full MSR;
        Pass 3:  converts the first MSR into a second MSR, to apply options;
        Pass 4:  converts the second MSR into a second MusicXML tree;
        Pass 5:  converts the second MusicXML tree to MusicXML code
                 and writes it to standard output.

    Other passes are performed according to the options, such as
    displaying views of the internal data or printing a summary of the score.

    The activity log and warning/error messages go to standard error.
\end{lstlisting}
