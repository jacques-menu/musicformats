% !TEX root = MusicFormatsUserGuide.tex

% -------------------------------------------------------------------------
\chapter{Introduction}
% -------------------------------------------------------------------------

\mf\ is open source software, available with source code and documentation at \url{https://github.com/jacques-menu/musicformats}. It is written in \CPlusplus\ and provides a set of music scores representations and converters between various textual music scores formats. Building it only requires a \CPlusplus\ compiler and \cmake.

This document shows how to use the \mf\ library, both from the \CLI\ and from within applications. It is part of the \mf\ documentation, and can be found at \gitdocpdf{MusicFormatsUserGuide}{MusicFormatsUserGuide.pdf}.

\mf\ can be used from the \CLI\ on Linux, Windows and Mac OS. The \API\ also allows it to be used from applications, including in \Web\ sites.

The \mf\ repository contains several versions:
\begin{itemize}
\item the \default \master\ version, to be found at \url{https://github.com/jacques-menu/musicformats}, is where changes are pushed by the maintainers of \mf. It is the most up to date;
\item the \code{v.\dots} versions are the \master\ versions \frozen\ at some time.
\end{itemize}

This document mentions sample files in various formats. They can be seen online on the \code{dev} version, whose name and URL never change, at \url{https://github.com/jacques-menu/musicformats/tree/dev/files}.

These examples are mentioned in this document as they appear in the \mf\ repository, in subdirectories of the \fileName{files} directory. Currently, there are:
\begin{itemize}
\item \fileName{musicxmlfiles} for \mxml\ files;
\item \fileName{msdlfiles} for \msdlLang\ files.
%\item \fileName{meifiles} for \meiLang\ files;
%\item \fileName{bmmlfiles} for \bmmlLang\ files;
%\item \fileName{midifiles} for \MIDI\ files.
\end{itemize}

A typical example is \gitmxmlfile{basic/HelloWorld.xml}, which stands for the following, with the \code{dev} cloned locally in the \fileName{musicformats-git-dev} directory:
\begin{lstlisting}[language=Terminal]
jacquesmenu@macmini: ~/musicformats-git-dev > ls -sal files/musicxmlfiles/basic/HelloWorld.xml
8 -rw-r--r--@ 1 jacquesmenu  staff  1266 Apr 22  2021 files/musicxmlfiles/basic/HelloWorld.xml
\end{lstlisting}

This document is organized in four parts:
\begin{itemize}
\item the first part presents an overview of the library, as wall as its architecture;
\item then the options and help provided by the library are presented;
\item the third part is dedicated to the handling or warnings and errors;
\item the fourth part shows the use of \mf\ from the command line;
\item and finally, the fifth part shows how to use \mf\ through its \API. This is intended for users who create applications such as \Web\ sites that do not use command line commands, but call functions\index{functions} provided by the library instead. The exact same functionality is available this way.
\end{itemize}

