% MusicFormats CLI user's guide
% Jacques Menu, 2021-2022

\documentclass[11pt,a4paper]{report}


% -------------------------------------------------------------------------
% import common LaTeX settings
% -------------------------------------------------------------------------

\usepackage{import}

\subimport{../CommonLaTeXFiles}{MusicFormatsLaTeXSettings}


% -------------------------------------------------------------------------
% indexing (specific to this document)
% -------------------------------------------------------------------------

\makeindex[name=Main, options= -s ../CommonLaTeXFiles/MusicFormats.ist, columns=2, title=Main index, intoc]

% JMI don't create too many indexes, this causes idxlayout not to produce any indexes at all!

\makeindex[name=Files, options= -s ../CommonLaTeXFiles/MusicFormats.ist, columns=2, title=Files index, intoc]
%
%\makeindex[name=Types, options= -s ../CommonLaTeXFiles/MusicFormats.ist, columns=2, title=Types index, intoc]
%
%\makeindex[name=MethodsAndFields, options= -s ../CommonLaTeXFiles/MusicFormats.ist, columns=2, title=Methods and fields index, intoc]
%
%\makeindex[name=ConstantsFunctionsAndVariables, options= -s ../CommonLaTeXFiles/MusicFormats.ist, columns=2, title={Constants, functions and variables index}, intoc]
%
\makeindex[name=Options, options= -s ../CommonLaTeXFiles/MusicFormats.ist, columns=2, title=Options, intoc]
%
\makeindex[name=MusicXML, options= -s ../CommonLaTeXFiles/MusicFormats.ist, columns=2, title=MusicXML index, intoc]

\indexsetup{level=\section}


%\usepackage{cleveref}
	% http://tug.ctan.org/tex-archive/macros/latex/contrib/cleveref/cleveref.pdf


% -------------------------------------------------------------------------
\begin{document}
% -------------------------------------------------------------------------

\begin{titlepage}
  \begin{center}
    \vspace*{2cm}

    \textbf{
      \LARGE{\mf\ command user guide} \\[10pt]
			\Large{\url{https://github.com/jacques-menu/musicformats}}
		}

    \vspace{0.25cm}

    \large{v\input{../MusicFormatsVersionNumber.txt} (\today)}
%    \large{v\input{../../src/mflibrary/MusicFormatsVersionNumber.txt} (\today)}

    \vspace{0.75cm}

    \large{\textbf{Jacques Menu}}
  \end{center}

  \vspace{1cm}


\mf\ is open source software, available with source code and documentation at \url{https://github.com/jacques-menu/musicformats}. It is written in C++11 and provides a set of music scores representations and converters between various textual music scores formats. Building it only requires a C++11 compiler and {\tt cmake}.

This document shows how to use the \mf\ library, both from the \CLI\ and from within applications. It is part of the \mf\ documentation, and can be found at \docpdf{MusicFormatsCLIUserGuide}{MusicFormatsCLIUserGuide.pdf}.

\mf\ can be used from the \CLI\ on Linux, Windows and Mac OS. The \API\ also allows it to be used from applications, including in \Web\ sites.


  \begin{center}
\begin{turn}{1}
\maxsizebox{0.9\linewidth}{5cm}{
\begin{minipage}{\linewidth}
\begin{lstlisting}[language=Terminal]
jacquesmenu@macmini > xml2ly -about
What xml2ly does:

    This multi-pass converter basically performs 5 passes:
        Pass 1:  reads the contents of MusicXMLFile or stdin ('-')
                 and converts it to a MusicXML tree;
        Pass 2a: converts that MusicXML tree into
                 a first Music Score Representation (MSR) skeleton;
        Pass 2b: populates the first MSR skeleton from the MusicXML tree
                 to get a full MSR;
        Pass 3:  converts the first MSR into a second MSR to apply options
        Pass 4:  converts the second MSR into a
                 LilyPond Score Representation (LPSR);
        Pass 5:  converts the LPSR to LilyPond code
                 and writes it to standard output.

    Other passes are performed according to the options, such as
    displaying views of the internal data or printing a summary of the score.

    The activity log and warning/error messages go to standard error.
}\end{lstlisting} % no line break here
\end{minipage}
}
\end{turn}

    \vspace{.5cm}

\includegraphics[scale=1.0]{../graphics/MinimalScore.png}

  \vfill

  \end{center}
\end{titlepage}


% -------------------------------------------------------------------------
{ % reduce vertical size of tables and lists
  \setlength {\parskip} {0.3ex plus \baselineskip minus 2pt}

  \tableofcontents

  \listoffigures
}


%% -------------------------------------------------------------------------
\part{Preamble}
%% -------------------------------------------------------------------------

\subimport{./}{acknowledgements}

% now we can set the regular fancyhead
\fancyhead[L]{\nouppercase\leftmark}
\fancyhead[C]{}
\fancyhead[R]{\nouppercase\rightmark}

\subimport{./}{about}


% -------------------------------------------------------------------------
\part{Discovering MusicFormats}
% -------------------------------------------------------------------------

\subimport{./}{architecture}

\subimport{./}{aFirstExample}

\subimport{./}{repository}

\subimport{./}{libraryComponents}


% -------------------------------------------------------------------------
\part{Shell basics}
% -------------------------------------------------------------------------

\subimport{./}{shellBasics}


% -------------------------------------------------------------------------
\part{Installing MusicFormats}
% -------------------------------------------------------------------------

\subimport{./}{installing}


% -------------------------------------------------------------------------
\part{Options and help (OAH)}
% -------------------------------------------------------------------------

\subimport{./}{optionsAndHelp}

\subimport{./}{nonMusicalOptions}

\subimport{./}{traceOptions}

%\subimport{./}{inputAndOutput}%%%JMI


% -------------------------------------------------------------------------
\part{Warnings and errors (WAE)}
% -------------------------------------------------------------------------

\subimport{./}{warningsAndErrors}


% -------------------------------------------------------------------------
\part{Multiple languages support}
% -------------------------------------------------------------------------

\subimport{./}{multipleLanguagesSupport}


% -------------------------------------------------------------------------
\part{{\tt xml2ly}}
% -------------------------------------------------------------------------

\subimport{./}{xml2ly}


% -------------------------------------------------------------------------
\part{{\tt xml2brl}}
% -------------------------------------------------------------------------

\subimport{./}{xml2brl}


% -------------------------------------------------------------------------
\part{{\tt xml2xml}}
% -------------------------------------------------------------------------

\subimport{./}{xml2xml}


% -------------------------------------------------------------------------
\part{{\tt xml2gmn}}
% -------------------------------------------------------------------------

\subimport{./}{xml2gmn}


% -------------------------------------------------------------------------
% postamble
% -------------------------------------------------------------------------

% -------------------------------------------------------------------------
\part{Indexes}
% -------------------------------------------------------------------------

% back to the lists fancyhead settings:
\fancyhead[L]{}
\fancyhead[C]{\nouppercase\leftmark}
\fancyhead[R]{}

\printindex[Files]

\printindex[Options]

\printindex[MusicXML]

\printindex[Main]


% -------------------------------------------------------------------------
\end{document}
% -------------------------------------------------------------------------
