% !TEX root = MusicFormatsCLIUserGuide.tex

% -------------------------------------------------------------------------
\chapter{About this document}
% -------------------------------------------------------------------------

This document is organized in four parts:
\begin{itemize}
\item the part II lets the user discover the library, as well as its architecture, see \sectionRef{The MusicFormats architecture};
\item then the options and help so-called \oahRepr\ infrastructure provided by the library is presented in part III;
\item the part IV is dedicated to the handling or warnings and errors;
\item the part V presents the multiple languages support provided by \mf;
\item parts VI to IX show the specific features of the various converter;
\item and finally, there is a comprehensive set of indexes.
\end{itemize}

The use of \mf\ through its \API s is described in a specific documentation, to be found at \gitdocpdf{MusicFormatsAPIUserGuide}{MusicFormatsAPIUserGuide.pdf}. This is intended for users who create applications such as \Web\ sites that do not use command line commands, but call functions\index{functions} provided by the library instead. The exact same functionality is available this way.

In fact, the \CLI\ versions of the \service s merely use these \API\ functions.
