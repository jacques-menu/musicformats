% !TEX root = MusicFormatsCLIUserGuide.tex

\begin{adjustwidth}{0cm}{0cm}
\begin{figure}[ht]
\caption {\xmlToBrl\ architecture}
\label{xml2brlArchitecture}
\begin{center}

% The sizes
% ------------------------------------------------

\def \bendAngle {60}

\def \minimumRectangleHeight {10 mm}
\def \minimumRectangleWidth  {30 mm}

\def \separatorWidth {\minimumRectangleWidth}

\def \roundedCorners {10pt}


% -------------------------------------------------------------------------
\begin{tikzpicture}[
	%
	scale=0.42,
	%
	>=open triangle 45,
	]
% -------------------------------------------------------------------------

% elements positions
% ------------------------------------------------

\tikzmath{
	% internal diameter
	\interalDiameter = 9;
	%
	% external diameter
	\externalDiameter = 18;
	%
	% the sectors
	\sectorsNumber = 20; % 4 times as many radiuses are used, for fine placement
	\sectorAngle   = 360 / \sectorsNumber;
	\origingAngle  = 90;
	%
  % MSR
  \MSRAbs = 0;
  \MSROrd = 0;
  %
  % UpperSpace
  \UpperSpaceAbs = 0;
  \UpperSpaceOrd = \externalDiameter * 1.1;
	%
  % Mxsr
  \MxsrAngle = \origingAngle + 0 * \sectorAngle;
  \MxsrAbs = \interalDiameter*cos(\MxsrAngle);
  \MxsrOrd = \interalDiameter*sin(\MxsrAngle);
  %
  % MusicXML
  \MusicXMLAngle = \MxsrAngle;
  \MusicXMLAbs = \externalDiameter*cos(\MusicXMLAngle);
  \MusicXMLOrd = \externalDiameter*sin(\MusicXMLAngle);
  %
  % Guido
  \GuidoAngle = \origingAngle + 2 * \sectorAngle;
  \GuidoAbs = \externalDiameter*cos(\GuidoAngle);
  \GuidoOrd = \externalDiameter*sin(\GuidoAngle);
  %
  % MDSR
  \MDSRAngle = \origingAngle + 4 * \sectorAngle;
  \MDSRAbs = \interalDiameter*cos(\MDSRAngle);
  \MDSROrd = \interalDiameter*sin(\MDSRAngle);
  %
  % MIDI
  \MIDIAngle = \MDSRAngle;
  \MIDIAbs      = \externalDiameter*cos(\MIDIAngle);
  \MIDIOrd      = \externalDiameter*sin(\MIDIAngle);
  %
  % LPSR
  \LPSRAngle = \origingAngle + 10 * \sectorAngle;
  \LPSRAbs      = \interalDiameter*cos(\LPSRAngle);
  \LPSROrd      = \interalDiameter*sin(\LPSRAngle);
  %
  % LilyPond
%  \LilyPondAngle = \LPSRAngle + 17.5;
  \LilyPondAngle = \origingAngle + 9 * \sectorAngle;
  \LilyPondAbs   = \externalDiameter*cos(\LilyPondAngle);
  \LilyPondOrd   = \externalDiameter*sin(\LilyPondAngle);
  %
  % Jianpu LilyPond
%  \JianpuLilyPondAngle = \LPSRAngle - 17.5;
  \JianpuLilyPondAngle = \origingAngle + 11 * \sectorAngle;
  \JianpuLilyPondAbs   = \externalDiameter*cos(\JianpuLilyPondAngle);
  \JianpuLilyPondOrd   = \externalDiameter*sin(\JianpuLilyPondAngle);
  %
  % BSR
  \BSRAngle = \origingAngle + 14 * \sectorAngle;
  \BSRAbs      = \interalDiameter*cos(\BSRAngle);
  \BSROrd      = \interalDiameter*sin(\BSRAngle);
  %
  % Braille
  \BrailleAngle = \BSRAngle;
  \BrailleAbs   = \externalDiameter*cos(\BrailleAngle);
  \BrailleOrd   = \externalDiameter*sin(\BrailleAngle);
  %
  % MSDL
  \MSDLAngle = \origingAngle + 17 * \sectorAngle;
  \MSDLAbs   = \externalDiameter*cos(\MSDLAngle);
  \MSDLOrd   = \externalDiameter*sin(\MSDLAngle);
  %
  %
  %
  % RandomMusic
  \RandomMusicStartAbs = -1.55;
  \RandomMusicStartOrd = \MxsrOrd + 0.35;
  \RandomMusicInterAbs = \MxsrAbs - 0.10;
  \RandomMusicInterOrd = \MxsrOrd - 0.10;
  \RandomMusicEndAbs   = \MusicXMLAbs - 0.20;
  \RandomMusicEndOrd   = \MusicXMLOrd - 0.15;
  %
  % textOutput
  \textOutputAngle = 135;
  \textOutputRadius = 1.6;
  \textOutputAbs = cos(\textOutputAngle) * \textOutputRadius;
  \textOutputOrd = sin(\textOutputAngle) * \textOutputRadius;
  %
  % tools
  \toolsStartAbs = \MusicXMLAbs - 0.125;
  \toolsStartOrd = \MusicXMLOrd - 0.15;
  \toolsInterAbs = \MxsrAbs - 0.05;
  \toolsInterOrd = \MxsrOrd - 0.1;
  \toolsEndAbs   = \textOutputAbs + 0.15;
  \toolsEndOrd   = \textOutputOrd + 0.0;
  %
  % xmlToGuido
  \xmlToGuidoInterangle = (\MxsrAngle + \GuidoAngle) / 2;
  \xmlToGuidoInterRadius = 0.6;
  \xmlToGuidoStartAbs = \MusicXMLAbs + 0.15;
  \xmlToGuidoStartOrd = \MusicXMLOrd - 0.15;
  \xmlToGuidoInterAbs = cos(\xmlToGuidoInterangle) * \xmlToGuidoInterRadius;
  \xmlToGuidoInterOrd = sin(\xmlToGuidoInterangle) * \xmlToGuidoInterRadius;
  \xmlToGuidoEndAbs   = \GuidoAbs - 0.15;
  \xmlToGuidoEndOrd   = \GuidoOrd - 0.15;
  %
  % xml2ly
  \xmlToLyInterIAngle  = mod((\MxsrAngle + \LPSRAngle) * 2 / 3, 360);
  \xmlToLyInterIIAngle = mod((\MxsrAngle + \LPSRAngle) * 1 / 3, 360);
  \xmlToLyInterRadius = 0.10;
  \xmlToLyStartAbs = \MusicXMLAbs + 0.05;
  \xmlToLyStartOrd = \MusicXMLOrd - 0.15;
  \xmlToLyInterIAbs = cos(\xmlToLyInterIAngle) * \xmlToLyInterRadius;
  \xmlToLyInterIOrd = sin(\xmlToLyInterIAngle) * \xmlToLyInterRadius;
  \xmlToLyInterIIAbs = cos(\xmlToLyInterIIAngle) * \xmlToLyInterRadius;
  \xmlToLyInterIIOrd = sin(\xmlToLyInterIIAngle) * \xmlToLyInterRadius;
  \xmlToLyEndAbs   = \LilyPondAbs - 0.045;
  \xmlToLyEndOrd   = \LilyPondOrd + 0.15;
  %
  % xml2brl
  \xmlToBrlInterIAngle  = mod((\MxsrAngle + \LPSRAngle) * 2 / 3, 360);
  \xmlToBrlInterIIAngle = mod((\MxsrAngle + \LPSRAngle) * 1 / 3, 360);
  \xmlToBrlInterRadius = 0.2;
  \xmlToBrlStartAbs = \MusicXMLAbs + 0.10;
  \xmlToBrlStartOrd = \MusicXMLOrd - 0.15;
  \xmlToBrlInterIAbs = cos(\xmlToBrlInterIAngle) * \xmlToBrlInterRadius;
  \xmlToBrlInterIOrd = sin(\xmlToBrlInterIAngle) * \xmlToBrlInterRadius;
  \xmlToBrlInterIIAbs = cos(\xmlToBrlInterIIAngle) * \xmlToBrlInterRadius;
  \xmlToBrlInterIIOrd = sin(\xmlToBrlInterIIAngle) * \xmlToBrlInterRadius;
  \xmlToBrlEndAbs   = \BrailleAbs - 0.05;
  \xmlToBrlEndOrd   = \BrailleOrd + 0.15;
  %
  %
  %
  % toBeWrittenCommon
  \toBeWrittenCommonStartAbs = -1.15;
  \toBeWrittenCommonStartOrd = \MDSROrd + 0.65;
  \toBeWrittenCommonEndAbs = \toBeWrittenCommonStartAbs + 0.65;
  \toBeWrittenCommonEndOrd = \toBeWrittenCommonStartOrd - 0.225;
  %
  % toBeWrittenToMusicXML
  \toBeWrittenToMusicXMLInterIAngle  = mod((\MxsrAngle + \MDSRAngle) * -0.1, 360);
  \toBeWrittenToMusicXMLInterIIAngle = mod((\MxsrAngle + \MDSRAngle) * 0.375, 360);
  \toBeWrittenToMusicXMLInterRadius = -0.20;
  \toBeWrittenToMusicXMLStartAbs = \toBeWrittenCommonEndAbs;
  \toBeWrittenToMusicXMLStartOrd = \toBeWrittenCommonEndOrd;
  \toBeWrittenToMusicXMLInterIAbs = cos(\toBeWrittenToMusicXMLInterIAngle) * \toBeWrittenToMusicXMLInterRadius;
  \toBeWrittenToMusicXMLInterIOrd = sin(\toBeWrittenToMusicXMLInterIAngle) * \toBeWrittenToMusicXMLInterRadius;
  \toBeWrittenToMusicXMLInterIIAbs = cos(\toBeWrittenToMusicXMLInterIIAngle) * \toBeWrittenToMusicXMLInterRadius;
  \toBeWrittenToMusicXMLInterIIOrd = sin(\toBeWrittenToMusicXMLInterIIAngle) * \toBeWrittenToMusicXMLInterRadius;
  \toBeWrittenToMusicXMLEndAbs   = \MusicXMLAbs - 0.05;
  \toBeWrittenToMusicXMLEndOrd   = \MusicXMLOrd - 0.15;
	%
  % toBeWrittenToLilyPond
  \toBeWrittenToLilyPondInterIAngle  = mod((\LPSRAngle + \MDSRAngle) * -0.1, 360);
  \toBeWrittenToLilyPondInterIIAngle = mod((\LPSRAngle + \MDSRAngle) * 0.375, 360);
  \toBeWrittenToLilyPondInterRadius = 0.30;
  \toBeWrittenToLilyPondStartAbs = \toBeWrittenCommonEndAbs5;
  \toBeWrittenToLilyPondStartOrd = \toBeWrittenCommonEndOrd;
  \toBeWrittenToLilyPondInterIAbs = cos(\toBeWrittenToLilyPondInterIAngle) * \toBeWrittenToLilyPondInterRadius;
  \toBeWrittenToLilyPondInterIOrd = sin(\toBeWrittenToLilyPondInterIAngle) * \toBeWrittenToLilyPondInterRadius;
  \toBeWrittenToLilyPondInterIIAbs = cos(\toBeWrittenToLilyPondInterIIAngle) * \toBeWrittenToLilyPondInterRadius;
  \toBeWrittenToLilyPondInterIIOrd = sin(\toBeWrittenToLilyPondInterIIAngle) * \toBeWrittenToLilyPondInterRadius;
  \toBeWrittenToLilyPondEndAbs   = \LilyPondAbs - 0.15;
  \toBeWrittenToLilyPondEndOrd   = \LilyPondOrd - 0.15;
	%
  % toBeWrittenToBraille
  \toBeWrittenToBrailleInterIAngle  = mod((\LPSRAngle + \MDSRAngle) * -0.1, 360);
  \toBeWrittenToBrailleInterIIAngle = mod((\LPSRAngle + \MDSRAngle) * 0.375, 360);
  \toBeWrittenToBrailleInterRadius = 0.30;
  \toBeWrittenToBrailleStartAbs = \toBeWrittenCommonEndAbs5;
  \toBeWrittenToBrailleStartOrd = \toBeWrittenCommonEndOrd;
  \toBeWrittenToBrailleInterIAbs = cos(\toBeWrittenToBrailleInterIAngle) * \toBeWrittenToBrailleInterRadius;
  \toBeWrittenToBrailleInterIOrd = sin(\toBeWrittenToBrailleInterIAngle) * \toBeWrittenToBrailleInterRadius;
  \toBeWrittenToBrailleInterIIAbs = cos(\toBeWrittenToBrailleInterIIAngle) * \toBeWrittenToBrailleInterRadius;
  \toBeWrittenToBrailleInterIIOrd = sin(\toBeWrittenToBrailleInterIIAngle) * \toBeWrittenToBrailleInterRadius;
  \toBeWrittenToBrailleEndAbs   = \BrailleAbs - 0.15;
  \toBeWrittenToBrailleEndOrd   = \BrailleOrd - 0.15;
}


\footnotesize


% The coordinates
% ------------------------------------------------

\coordinate (MSR) at (\MSRAbs,\MSROrd);

\coordinate (UpperSpace) at (\UpperSpaceAbs,\UpperSpaceOrd);

\coordinate (Mxsr) at (\MxsrAbs,\MxsrOrd);
\coordinate (MusicXML) at (\MusicXMLAbs,\MusicXMLOrd);

\coordinate (Guido) at (\GuidoAbs,\GuidoOrd);

\coordinate (LPSR) at (\LPSRAbs, \LPSROrd);
\coordinate (LilyPond) at (\LilyPondAbs, \LilyPondOrd);
\coordinate (JianpuLilyPond) at (\JianpuLilyPondAbs, \JianpuLilyPondOrd);

\coordinate (Braille) at (\BrailleAbs, \BrailleOrd);
\coordinate (BSR) at (\BSRAbs, \BSROrd);

\coordinate (MSDL) at (\MSDLAbs, \MSDLOrd);

\coordinate (MIDI) at (\MIDIAbs, \MIDIOrd);
\coordinate (MDSR) at (\MDSRAbs, \MDSROrd);

\coordinate (textOutput) at (\textOutputAbs, \textOutputOrd);


% Draw the elements
% ------------------------------------------------

\def \minimumCircleSize {85}
\def \minimumSquareSize {75}

\def \MusicXMLColor {red!15.5}

\def \GuidoColor {brown!20}

\def \LilyPondColor {orange!20}

\def \BrailleColor {purple!7.5}

\def \MSDLColor {purple!7.5}

\def \BMMLColor {blue!7.5}

\def \MEIColor {blue!7.5}

\def \MIDIColor {black!3.75}

\def \roundedCorners {10pt}

% MSR
\node[align=center,style={circle,minimum size=100,fill=green!20}]
  (MSR) at (MSR) {{\bf MSR}\\(graph)};

% UpperSpace
\node[align=center,style={rectangle,minimum height=50,fill=white}]
  (UpperSpace) at (UpperSpace) {~};

% MusicXML
\node[align=center,style={rectangle, minimum size=\minimumSquareSize,fill=\MusicXMLColor}]
  (MusicXML)
  at (MusicXML) {{\bf MusicXML}\\(text)};

% Mxsr
\node[align=center,style={circle,minimum size=\minimumCircleSize,fill=\MusicXMLColor}]
  (Mxsr)
  at (Mxsr) {{\bf MXSR}\\(tree)};

% Braille
\node[align=center,style={rectangle,rounded corners=\roundedCorners,minimum size=\minimumSquareSize,fill=\BrailleColor}]
  (Braille)
  at (Braille) {{\bf Braille}\\(text)};
\node[align=center,style={circle,minimum size=\minimumCircleSize,fill=\BrailleColor}]
  (BSR)
  at (BSR) {{\bf BSR}\\(graph)};


% Draw the arcs between the elements
% ------------------------------------------------

\def \bendAngle {20}

\def \arcsColor {blue}
\def \dashedArcsColor {gray!50}
\def \loopsColor {red}

% MusicXML <-> Mxsr
\path [->, sloped, bend right=\bendAngle, below, thick, \arcsColor]
(MusicXML)
edge node {\rotatebox{-90}{1}}
(Mxsr);

% Mxsr <-> MSR
\path [->, sloped, bend right=\bendAngle, above, thick, \arcsColor]
(Mxsr)
edge node {\rotatebox{-82}{2}}
(MSR);

% MSR -> MSR
\path [->, sloped, below, thick, \loopsColor]
(MSR)
edge [out=205, in=225, loop] node {\rotatebox{55}{3}}
(MSR);

% MSR <-> BSR
\path [->, sloped, bend right=\bendAngle, below, thick, \arcsColor]
(MSR)
edge node {\rotatebox{10}{4}}
(BSR);

% BSR -> BSR
\path [->, sloped, below, thick, \loopsColor]
(BSR)
edge [out=-110, in=-90, loop] node {\rotatebox{10}{5}}
(BSR);

% BSR -> Braille
\path [->, sloped, above, thick, \arcsColor]
(BSR)
edge node {\rotatebox{17.5}{6}}
(Braille);


% -------------------------------------------------------------------------
\end{tikzpicture}
 % -------------------------------------------------------------------------

\end{center}

\end{figure}
\end{adjustwidth}

